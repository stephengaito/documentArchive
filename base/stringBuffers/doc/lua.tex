% A ConTeXt document [master document: stringBuffers.tex]

\section[title=Lua functions]

\component gitVersion-c

\startCCode
static int lua_stringBuffers_getGitVersion (lua_State *lstate) {
  const char* aKey   = lua_tostring(lstate, 1);
  const char* aValue = "no valid key provided";
  if (aKey) aValue = getGitVersion(gitVersionKeyValues, aKey);
  lua_pushstring(lstate, aValue);
  return 1;
}

static const struct luaL_Reg lua_stringBuffers [] = {
  {"gitVersion", lua_stringBuffers_getGitVersion},
  {NULL, NULL}
};

int luaopen_joylol_stringBuffers (lua_State *lstate) {
  getJoyLoLInterpInto(lstate, jInterp);
  registerStringBuffers(jInterp);
  luaL_newlib(lstate, lua_stringBuffers);
  return 1;
}
\stopCCode

In some instances, such as the typical \type{CTest} program 
\type{allCTests}, this Lua module (which can be \type{require}d as a 
shared library) is actually statically linked into the executable. In 
these cases we need the ability to mimic the standard Lua \type{require} 
process. The following \type{requireStaticallyLinkedStringBuffers} does just 
this. 

\startCHeader
Boolean requireStaticallyLinkedStringBuffers(
  lua_State *lstate
);
\stopCHeader

\startCCode
Boolean requireStaticallyLinkedStringBuffers(
  lua_State *lstate
) {
  lua_getglobal(lstate, "package");
  lua_getfield(lstate, -1, "loaded");
  luaopen_joylol_stringBuffers(lstate);
  lua_setfield(lstate, -2, "joylol.stringBuffers");
  lua_setfield(lstate, -2, "loaded");
  lua_pop(lstate, 1);
  return TRUE;
}
\stopCCode
