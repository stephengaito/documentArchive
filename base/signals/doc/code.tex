% A ConTeXt document [master document: signals.tex]

\section[title=Code]
\setCHeaderStream{public}

\dependsOn[jInterps]
%\dependsOn[context]

\component gitVersion-c

\startTestSuite[newSignal]

\startCHeader

#define SIGNAL_END_OF_TEXT 1

typedef JObj* (NewSignal)(
  JoyLoLInterp*,
  size_t
);

#define newSignal(jInterp, aSignal)       \
  (                                       \
    assert(getSignalsClass(jInterp)       \
      ->newSignalFunc),                   \
    (getSignalsClass(jInterp)             \
      ->newSignalFunc(jInterp, aSignal))  \
  )

#define asSignal(aLoL) ((aLoL)->flags)
\stopCHeader

\setCHeaderStream{private}
\startCHeader
extern JObj* newSignalImpl(
  JoyLoLInterp *jInterp,
  size_t        aSignal
);
\stopCHeader
\setCHeaderStream{public}

\startCCode
JObj* newSignalImpl(
  JoyLoLInterp *jInterp,
  size_t        aSignal
) {
  assert(jInterp);
  assert(jInterp->coAlgs);
  
  JObj* result = newObject(jInterp, SignalsTag);
  assert(result);
  
  result->type  = jInterp->coAlgs[SignalsTag];
  result->flags = aSignal;
  return result;
}
\stopCCode

\startTestCase[should create a new signal]

\startCTest
  AssertPtrNotNull(jInterp);

  JObj* aNewSignal = newSignal(jInterp, 12);
  AssertPtrNotNull(aNewSignal);
  AssertPtrNotNull(asType(aNewSignal));
  AssertIntEquals(asTag(aNewSignal), SignalsTag);
  AssertIntEquals(asFlags(aNewSignal), 12);
  AssertIntTrue(isAtom(aNewSignal));
  AssertIntTrue(isSignal(aNewSignal));
  AssertIntFalse(isPair(aNewSignal));
\stopCTest
\stopTestCase

\startTestCase[print Signal]
\startCTest
  AssertPtrNotNull(jInterp);

  StringBufferObj *aStrBuf = newStringBuffer(jInterp);
  AssertPtrNotNull(aStrBuf);

  JObj* aLoL = newSignal(jInterp, 12);
  AssertPtrNotNull(aLoL);
  printLoL(jInterp, aStrBuf, aLoL);
  AssertStrEquals(getCString(jInterp, aStrBuf), "signal:12 ");
  strBufClose(jInterp, aStrBuf);
\stopCTest
\stopTestCase

\stopTestSuite

\startTestSuite[isSignal]

\startCHeader
#define isSignal(aLoL)            \
  (                               \
    (                             \
      (aLoL) &&                   \
      asType(aLoL) &&             \
      (asTag(aLoL) == SignalsTag) \
    ) ?                           \
      TRUE :                      \
      FALSE                       \
  )
\stopCHeader

\startTestCase[should return appropriate signal values]

\startCTest
  JObj *aSignal = newSignal(jInterp, 12);
  AssertPtrNotNull(aSignal);
  AssertPtrNotNull(asType(aSignal));
  AssertIntEquals(asTag(aSignal), SignalsTag);
  AssertIntEquals(asSignal(aSignal), 12);
\stopCTest
\stopTestCase
\stopTestSuite

\setCHeaderStream{private}
\startCHeader
extern Boolean equalityBoolCoAlg(
  JoyLoLInterp *jInterp,
  JObj     *lolA,
  JObj     *lolB
);
\stopCHeader
\setCHeaderStream{public}

\startCCode
Boolean equalityBoolCoAlg(
  JoyLoLInterp *jInterp,
  JObj     *lolA,
  JObj     *lolB
) {
  DEBUG(jInterp, "boolCoAlg-equal a:%p b:%p\n", lolA, lolB);
  if (!lolA && !lolB) return TRUE;
  if (!lolA && lolB)  return FALSE;
  if (lolA  && !lolB) return FALSE;
  if (asType(lolA) != asType(lolB)) return FALSE;
  if (!asType(lolA)) return FALSE;
  if (asTag(lolA)  != SignalsTag) return FALSE;
  if (asSignal(lolA) != asSignal(lolB)) return FALSE;
  return TRUE;
}
\stopCCode

\startTestSuite[printing signals]

\setCHeaderStream{private}
\startCHeader
extern Boolean printBoolCoAlg(
  JoyLoLInterp    *jInterp,
  StringBufferObj *aStrBuf,
  JObj        *aLoL
);
\stopCHeader
\setCHeaderStream{public}

\startCCode
Boolean printBoolCoAlg(
  JoyLoLInterp    *jInterp,
  StringBufferObj *aStrBuf,
  JObj            *aLoL
) {
  assert(aLoL);
  assert(asTag(aLoL) == SignalsTag);

  strBufPrintf(jInterp, aStrBuf, "signal:%zu ", asSignal(aLoL));
  return TRUE;
}
\stopCCode

\startTestCase[should print signals]

\startCTest
  AssertPtrNotNull(jInterp);
  AssertPtrNotNull(jInterp->coAlgs[SignalsTag]);

  StringBufferObj *aStrBuf = newStringBuffer(jInterp);
  AssertPtrNotNull(aStrBuf);
  
  JObj* aNewSignal = newSignal(jInterp, 12);
  AssertPtrNotNull(aNewSignal);
  printLoL(jInterp, aStrBuf, aNewSignal);
  AssertStrEquals(getCString(jInterp, aStrBuf), "signal:12 ");
  strBufClose(jInterp, aStrBuf);
\stopCTest
\stopTestCase
\stopTestSuite

\startTestSuite[registerSignals]

\startCHeader
typedef struct signals_class_struct {
  JClass     super;
  NewSignal *newSignalFunc;
} SignalsClass;

\stopCHeader

\startCCode
static Boolean initializeSignals(
  JoyLoLInterp *jInterp,
  JClass       *aJClass
) {
  assert(jInterp);
  assert(aJClass);
  return TRUE;
}
\stopCCode

\setCHeaderStream{private}
\startCHeader
extern Boolean registerSignals(JoyLoLInterp *jInterp);
\stopCHeader
\setCHeaderStream{public}

\startCCode
Boolean registerSignals(JoyLoLInterp *jInterp) {
  assert(jInterp);
  assert(jInterp->coAlgs);
  
  SignalsClass* theCoAlg
    = joyLoLCalloc(1, SignalsClass);
  assert(theCoAlg);
  
  theCoAlg->super.name           = SignalsName;
  theCoAlg->super.objectSize     = sizeof(JObj);
  theCoAlg->super.initializeFunc = initializeSignals;
  theCoAlg->super.registerFunc   = registerSignalWords;
  theCoAlg->super.equalityFunc   = equalityBoolCoAlg;
  theCoAlg->super.printFunc      = printBoolCoAlg;
  theCoAlg->newSignalFunc        = newSignalImpl;
  size_t tag =
    registerJClass(jInterp, (JClass*)theCoAlg);
  
  // do a sanity check...
  assert(tag == SignalsTag);
  assert(jInterp->coAlgs[tag]);
   
  return TRUE;
}
\stopCCode

\startTestCase[should register the Signals coAlg]

\startCTest
  // CTestsSetup has already created a jInterp
  // and run registerSignals
  AssertPtrNotNull(jInterp);
  AssertPtrNotNull(jInterp->coAlgs);
  AssertPtrNotNull(getSignalsClass(jInterp));
  SignalsClass *coAlg = getSignalsClass(jInterp);
  registerSignals(jInterp);
  AssertPtrNotNull(getSignalsClass(jInterp));
  AssertPtrEquals(getSignalsClass(jInterp), coAlg);
  AssertIntEquals(
    getSignalsClass(jInterp)->super.objectSize,
    sizeof(JObj)
  )
\stopCTest
\stopTestCase
\stopTestSuite
