% A ConTeXt document [master document: dictionaries.tex]

\section[title=Code]
\setCHeaderStream{public}

\dependsOn[jInterps]
%\dependsOn[context]

\component gitVersion-c

\startCHeader
typedef struct dictionary_object_struct {
  JObj          super;
  JoyLoLInterp *jInterp;
  DictObj      *parent;
  DictNodeObj  *root;
  DictNodeObj  *firstSymbol;
} DictObj;

#define asCFunc(aLoL) (((CFunctionObj*)(aLoL))->func)
\stopCHeader

\startTestSuite[newDictionary]

\startCHeader
typedef DictObj* (NewDictionary)(
  JoyLoLInterp *jInterp,
  DictObj      *parent
);

#define newDictionary(jInterp, parent)      \
  (                                         \
    assert(getDictionariesClass(jInterp)    \
      ->newDictionaryFunc),                 \
    (getDictionariesClass(jInterp)          \
      ->newDictionaryFunc(jInterp, parent)) \
  )
\stopCHeader

\setCHeaderStream{private}
\startCHeader
extern DictObj* newDictionaryImpl(
  JoyLoLInterp *jInterp,
  DictObj      *parent
);
\stopCHeader
\setCHeaderStream{public}

\startCCode
DictObj* newDictionaryImpl(
  JoyLoLInterp *jInterp,
  DictObj      *parent
) {
  assert(jInterp);
  assert(jInterp->coAlgs);
  
  DictObj* result =
    (DictObj*)newObject(jInterp, DictionariesTag);
  result->jInterp     = jInterp;
  result->parent      = parent;
  result->root        = NULL;
  result->firstSymbol = NULL;
  assert(result);
  
  result->super.type  = jInterp->coAlgs[DictionariesTag];
 
  return result;
}
\stopCCode

\startTestCase[should create a new Dictionary]

\startCTest
  AssertPtrNotNull(jInterp);

  JObj* aNewDictionary = newDictionary(jInterp, NULL);
  AssertPtrNotNull(aNewDictionary);
  AssertPtrEquals(aNewDictionary->jInterp, jInterp);
  AssertPtrNull(aNewDictionary->parent);
  AssertPtrNull(aNewDictionary->root);
  AssertPrtNull(aNewDictionary->firstSymbol);
  AssertPtrNotNull(asType(aNewDictionary));
  AssertIntEquals(asTag(aNewDictionary), DictionariesTag);
  AssertIntTrue(isAtom(aNewDictionary));
  AssertIntTrue(isDictionary(aNewDictionary));
  AssertIntFalse(isPair(aNewDictionary));
\stopCTest
\stopTestCase

\startTestSuite[findSymbol]

\startCHeader
typedef DictNodeObj *(FindSymbol)(
  DictObj *aDict,
  Symbol  *aSymbol
);

#define findSymbol(aDict, aSymbol)              \
  (                                             \
    assert(aDict->jInterp),                     \
    assert(getDictionariesClass(aDict->jInterp) \
      ->findSymbolFunc),                        \
    (getDictionariesClass(aDict->jInterp)       \
      ->findSymbolFunc(aDict, aSymbol))         \
  )
\stopCHeader

\setCHeaderStream{private}
\startCHeader
extern DictNodeObj* findSymbolInThisDictionary(
  DictObj *aDict,
  Symbol  *aSymbol
);

extern DictNodeObj* findSymbolImpl(
  DictObj *aDict,
  Symbol  *aSymbol
);
\stopCHeader
\setCHeaderStream{public}

\startCCode
DictNodeObj* findSymbolInThisDictionary(
  DictObj *aDict,
  Symbol  *aSymbol
) {
  if (!aSymbol) return NULL;
  assert(aDict);
  return findSymbolRecurse(aDict, aDict->root, aSymbol);
}

DictNodeObj* findSymbolImpl(
  DictObj *aDict,
  Symbol  *aSymbol
) {
  if (!aSymbol) return NULL;
  while (aDict) {
    //
    // Look for this symbol in this naming scope
    //
    DictNodeObj *aDictNode = 
      findSymbolRecurse(aDict, aDict->root, aSymbol);
    if (aDictNode) return aDictNode;
    //
    // We have not found this symbol in this naming scope
    // so look in the parent's naming scope
    //
    aDict = aDict->parent;
  }
  //
  // Alas, we have not found this symbol in any naming scope
  //
  return NULL;
}
\stopCCode

\startTestCase[should find symbols in parent dictionary]
\startCTest
  AssertPtrNotNull(jInterp);

  DictObj* parentDict = newDictionary(jInterp, NULL);
  AssertPtrNotNull(parentDict);
  AssertPtrNull(parentDict->parent);
  DictObj* childDict  = newDictionary(jInterp, parentDict);
  AssertPtrNotNull(childDict);
  AssertPtrNotNull(childDict->parent);
  AssertPtrEquals(childDict->parent, parentDict);
  //
  DictNodeObj* parentSym = findSymbol(childDict, "test");
  AssertPtrNull(parentSym);
  //
  parentSym = 
    createSymbolInThisDictionary(parentDict, "test");
  AssertPtrNotNull(parentSym);
  AssertStrEquals(parentSym->symbol, "test");
  //
  DictNodeObj* testSym = findSymbol(childDict, "test");
  AssertPtrNotNull(testSym);
  AssertPtrEquals(testSym, parentSym);
  //
  DictNodeObj* childSym =
    createSymbolInThisDictionary(childDict, "test");
  AssertPtrNotNull(childSym);
  AssertPtrNotEquals(parentSym, childSym);
  AssertStrEquals(childSym->symbol, "test");
  //
  testSym = findSymbol(childDict, "test");
  AssertPtrNotNull(testSym);
  AssertPtrEquals(testSym, childSym);
  AssertPtrNotEquals(testSym, parentSym);
  //
  testSym = findSymbol(parentDict, "test");
  AssertPtrNotNull(testSym);
  AssertPtrNotEquals(testSym, childSym);
  AssertPtrEquals(testSym, parentSym);
\stopCTest
\stopTestCase
\stopTestSuite

\startTestSuite[insertSymbol]

\startCHeader
typedef DictNodeObj *(InsertSymbol)(
  DictObj *aDict,
  Symbol  *aSymbol
);

#define insertSymbol(aDict, aSymbol)            \
  (                                             \
    assert(aDict),                              \
    assert(getDictionariesClass(aDict->jInterp) \
      ->insertSymbolFunc),                      \
    (getDictionariesClass(aDict->jInterp)       \
      ->insertSymbolFunc(aDict, aSymbol))       \
  )
\stopCHeader

\setCHeaderStream{private}
\startCHeader
extern DictNodeObj* insertSymbolImpl(
  DictObj *aDict,
  Symbol  *aSymbol
);
\stopCHeader
\setCHeaderStream{public}

\startCCode
DictNodeObj* insertSymbolImpl(
  DictObj *aDict,
  Symbol  *aSymbol
) {
  assert(aDict);
  assert(aSymbol);

  // lazy initialization
  if (!aDict->root) {
    assert(aDict->jInterp);
    DictNodeObj* firstNode = newDictNode(aDict->jInterp, aSymbol);
    aDict->root            = firstNode;
    aDict->firstSymbol     = firstNode;
    return firstNode;
  }

  return insertSymbolRecurse(aDict, aDict->root, aSymbol);
}
\stopCCode
\stopTestSuite

\startTestSuite[deleteSymbolFromThisDictionary]

\startCHeader
typedef void (DeleteSymbolFromThisDictionary)(
  DictObj *aDict,
  Symbol  *aSymbol
);

#define deleteSymbolFromThisDictionary(aDict, aSymbol)      \
  (                                                         \
    assert(aDict->jInterp),                                 \
    assert(getDictionariesClass(aDict->jInterp)             \
      ->deleteSymbolFromThisDictionaryFunc),                \
    (getDictionariesClass(aDict->jInterp)                   \
      ->deleteSymbolFromThisDictionaryFunc(aDict, aSymbol)) \
  )
\stopCHeader

\setCHeaderStream{private}
\startCHeader
extern void deleteSymbolFromThisDictionaryImpl(
  DictObj *aDict,
  Symbol  *aSymbol
);
\stopCHeader
\setCHeaderStream{public}

\startCCode
void deleteSymbolFromThisDictionaryImpl(
  DictObj *aDict,
  Symbol  *aSymbol
) {
  if (!aSymbol) return;
  assert(aDict);
  aDict->root = deleteSymbolRecurse(aDict, aDict->root, aSymbol);
}
\stopCCode

\startTestCase[should delete random symbols from a randomly created dictionary]
\startCTest
  srand(time(NULL));

  AssertPtrNotNull(jInterp);
  DictObj *aDict = newDictionary(jInterp, NULL);

  for (int i = 0; i < 1000; i++) {
    char itoa[100];
    sprintf(itoa, "%03d", rand() % 100);
    createSymbolInThisDictionary(aDict, itoa);
  }
  createSymbolInThisDictionary(aDict, "000");
  createSymbolInThisDictionary(aDict, "100");

  DictNodeObj *aNode = findSymbol(aDict, "000");
  AssertPtrNotNull(aNode);
  AssertStrEquals(aNode->symbol, "000");
  AssertPtrEquals(aDict->firstSymbol, aNode);
  
  aNode = findSymbol(aDict, "100");
  AssertPtrNotNull(aNode);
  AssertStrEquals(aNode->symbol, "100");
  aNode = aDict->firstSymbol;
  while (aNode->next) aNode = aNode->next;
  AssertStrEquals(aNode->symbol, "100");
  
  for (int i = 0; i < 1000; i++) {
    char itoa[100];
    int randNum = rand() % 100;
    //
    if (randNum == 0 || randNum == 100) continue;
    //
    sprintf(itoa, "%03d", randNum);
    deleteSymbolFromThisDictionary(aDict, itoa);
    DictNodeObj *aNode = aDict->firstSymbol;
    AssertStrEquals(aNode->symbol, "000");
    while (aNode->next) {
      AssertStrNotEquals(aNode->symbol, itoa);
      aNode = aNode->next;
    }
    AssertStrEquals(aNode->symbol, "100");   
    while (aNode->previous) {
      AssertStrNotEquals(aNode->symbol, itoa);
      aNode = aNode->previous;
    }
    AssertStrEquals(aNode->symbol, "000");
  }  
\stopCTest
\stopTestCase
\stopTestSuite

\startTestSuite[findLUBSymbol]
\startCHeader
typedef DictNodeObj *(FindLUBSymbol)(
  DictObj *aDict,
  Symbol  *aSymbol
);

#define findLUBSymbol(aDict, aSymbol)           \
  (                                             \
    assert(aDict),                              \
    assert(getDictionariesClass(aDict->jInterp) \
      ->findLUBSymbolFunc),                     \
    (getDictionariesClass(aDict->jInterp)       \
      ->findLUBSymbolFunc(aDict, aSymbol))      \
  )
\stopCHeader

\setCHeaderStream{private}
\startCHeader
extern DictNodeObj* findLUBSymbolImpl(
  DictObj *aDict,
  Symbol  *aSymbol
);
\stopCHeader
\setCHeaderStream{public}

\startCCode
DictNodeObj* findLUBSymbolImpl(
  DictObj *aDict,
  Symbol  *aSymbol
) {
  assert(aDict);
  if (!aSymbol) return aDict->firstSymbol;
  return findLUBSymbolRecurse(aDict, aDict->root, aSymbol);
}
\stopCCode
\stopTestSuite

\startTestSuite[createSymbol]

\startCHeader
typedef DictNodeObj *(CreateSymbolInThisDictionary)(
  DictObj *aDict,
  Symbol  *aSymbol
);

#define createSymbolInThisDictionary(aDict, aSymbol)      \
        (                                                 \
    assert(aDict),                                        \
    assert(getDictionariesClass(aDict->jInterp)           \
      ->createSymbolInThisDictionaryFunc),                \
    (getDictionariesClass(aDict->jInterp)                 \
      ->createSymbolInThisDictionaryFunc(aDict, aSymbol)) \
  )
\stopCHeader

\setCHeaderStream{private}
\startCHeader
extern DictNodeObj* createSymbolInThisDictionaryImpl(
  DictObj *aDict,
  Symbol  *aSymbol
);
\stopCHeader
\setCHeaderStream{public}

\startCCode
DictNodeObj* createSymbolInThisDictionaryImpl(
  DictObj *aDict,
  Symbol  *aSymbol
) {
  assert(aDict);
  if (!aSymbol) return NULL;
  DictNodeObj* aSym =
    findSymbolInThisDictionary(aDict, aSymbol);
  if (!aSym) {
    aDict->root = insertSymbol(aDict, aSymbol);
    aSym = findSymbolInThisDictionary(aDict, aSymbol);
  }
  return aSym;
}
\stopCCode
\stopTestSuite

\startTestSuite[getSymbol]

\startCHeader
typedef JObj *(GetSymbol)(
  DictObj *aDict,
  Symbol  *aSymbol
);

#define getSymbol(aDict, aSymbol)               \
  (                                             \
    assert(aDict),                              \
    assert(getDictionariesClass(aDict->jInterp) \
      ->getSymbolFunc),                         \
    (getDictionariesClass(aDict->jInterp)       \
      ->getSymbolFunc(aDict, aSymbol))          \
  )
\stopCHeader

\setCHeaderStream{private}
\startCHeader
extern JObj* getSymbolImpl(
  DictObj *aDict,
  Symbol  *aSymbol
);
\stopCHeader
\setCHeaderStream{public}

\startCCode
JObj* getSymbolImpl(
  DictObj *aDict,
  Symbol  *aSymbol
) {
  assert(aDict);
  DictNodeObj* aSym = findSymbol(aDict, aSymbol);
  if (!aSym) {
    aSym = createSymbolInThisDictionary(aDict, aSymbol);
  }
  return newSymbol(aDict->jInterp, aSym->symbol);
}
\stopCCode
\stopTestSuite

\startTestSuite[listDefinitions]

\startCHeader
typedef void (ListDefinitions)(
  DictObj         *aDict,
  StringBufferObj *aStrBuf
);

#define listDefinitions(aDict, aStrBuf)         \
  (                                             \
    assert(aDict),                              \
    assert(getDictionariesClass(aDict->jInterp) \
      ->listDefinitionsFunc),                   \
    (getDictionariesClass(aDict->jInterp)       \
      ->listDefinitionsFunc(aDict, aStrBuf))    \
  )
\stopCHeader

\setCHeaderStream{private}
\startCHeader
extern void listDefinitionsImpl(
  DictObj         *aDict,
  StringBufferObj *aStrBuf
);
\stopCHeader
\setCHeaderStream{public}

\startCCode
void listDefinitionsImpl(
  DictObj         *aDict,
  StringBufferObj *aStrBuf
) {
  assert(aDict);
  DictNodeObj* curNode = aDict->firstSymbol;
  while(curNode) {
    if (curNode->value) {
      strBufPrintf(aDict->jInterp, aStrBuf,"%s == ", curNode->symbol);
      printLoL(aDict->jInterp, aStrBuf, curNode->value);
      strBufPrintf(aDict->jInterp, aStrBuf,"\n");
    }
    curNode = curNode->next;
  }
}
\stopCCode

\startTestCase[print Dictionary]
\startCTest
  AssertPtrNotNull(jInterp);

  StringBufferObj *aStrBuf = newStringBuffer(jInterp);
  AssertPtrNotNull(aStrBuf);

  DictObj* aLoL = newDictionary(jInterp, NULL);
  AssertPtrNotNull(aLoL);
  printLoL(jInterp, aStrBuf, (JObj*)aLoL);
  AssertStrEquals(getCString(jInterp, aStrBuf), "dictionary ");
  strBufClose(jInterp, aStrBuf);
\stopCTest
\stopTestCase
\stopTestSuite

\startTestSuite[isDictionary]
\startCHeader
#define isDictionary(aLoL)              \
  (                                     \
    (                                   \
      (aLoL) &&                         \
      asType(aLoL) &&                   \
      (asTag(aLoL) == DictionariesTag)  \
    ) ?                                 \
      TRUE :                            \
      FALSE                             \
  )
\stopCHeader
\stopTestSuite

\setCHeaderStream{private}
\startCHeader
extern Boolean equalityDictionaryCoAlg(
  JoyLoLInterp *jInterp,
  JObj     *lolA,
  JObj     *lolB
);
\stopCHeader
\setCHeaderStream{public}

\startCCode
Boolean equalityDictionaryCoAlg(
  JoyLoLInterp *jInterp,
  JObj     *lolA,
  JObj     *lolB
) {
  DEBUG(jInterp, "dictionaryCoAlg-equal a:%p b:%p\n", lolA, lolB);
  if (!lolA && !lolB) return TRUE;
  if (!lolA && lolB)  return FALSE;
  if (lolA  && !lolB) return FALSE;
  if (asType(lolA) != asType(lolB)) return FALSE;
  if (!asType(lolA)) return FALSE;
  if (asTag(lolA)  != DictionariesTag) return FALSE;
  if (lolA != lolB) return FALSE;
  return TRUE;
}
\stopCCode

\startTestSuite[printing dictionaries]

\setCHeaderStream{private}
\startCHeader
extern Boolean printDictionaryCoAlg(
  JoyLoLInterp    *jInterp,
  StringBufferObj *aStrBuf,
  JObj        *aLoL
);
\stopCHeader
\setCHeaderStream{public}

\startCCode
Boolean printDictionaryCoAlg(
  JoyLoLInterp    *jInterp,
  StringBufferObj *aStrBuf,
  JObj        *aLoL
) {
  assert(aLoL);
  assert(asTag(aLoL) == DictionariesTag);

  strBufPrintf(jInterp, aStrBuf, "dictionary ");
  return TRUE;
}
\stopCCode

\startTestCase[should print dictionaries]

\startCTest
  AssertPtrNotNull(jInterp);
  AssertPtrNotNull(jInterp->coAlgs[DictionariesTag]);

  StringBufferObj *aStrBuf = newStringBuffer(jInterp);
  AssertPtrNotNull(aStrBuf);
  
  DictObj* aNewDictionary = newDictionary(jInterp, NULL);
  AssertPtrNotNull(aNewDictionary);
  printLoL(jInterp, aStrBuf, (JObj*)aNewDictionary);
  AssertStrEquals(getCString(jInterp, aStrBuf), "dictionary ");
  strBufClose(jInterp, aStrBuf);
\stopCTest
\stopTestCase
\stopTestSuite

\startTestSuite[registerDictionaries]

\startCHeader
typedef struct dictionaries_class_struct {
  JClass           super;
  NewDictionary   *newDictionaryFunc;
  CreateSymbolInThisDictionary
                  *createSymbolInThisDictionaryFunc;
  DeleteSymbolFromThisDictionary
                  *deleteSymbolFromThisDictionaryFunc;
  GetSymbol       *getSymbolFunc;
  FindSymbol      *findSymbolFunc;
  InsertSymbol    *insertSymbolFunc;
  FindLUBSymbol   *findLUBSymbolFunc;
  ListDefinitions *listDefinitionsFunc;  
} DictionariesClass;

\stopCHeader

\startCCode
static Boolean initializeDictionaries(
  JoyLoLInterp *jInterp,
  JClass       *aJClass
) {
  assert(jInterp);
  assert(aJClass);
  return TRUE;
}
\stopCCode

\setCHeaderStream{private}
\startCHeader
extern Boolean registerDictionaries(JoyLoLInterp *jInterp);
\stopCHeader
\setCHeaderStream{public}

\startCCode
Boolean registerDictionaries(JoyLoLInterp *jInterp) {
  assert(jInterp);
  assert(jInterp->coAlgs);
  
  DictionariesClass* theCoAlg
    = joyLoLCalloc(1, DictionariesClass);
  assert(theCoAlg);
  
  theCoAlg->super.name           = DictionariesName;
  theCoAlg->super.objectSize     = sizeof(DictObj);
  theCoAlg->super.initializeFunc = initializeDictionaries;
  theCoAlg->super.registerFunc   = registerDictionaryWords;
  theCoAlg->super.equalityFunc   = equalityDictionaryCoAlg;
  theCoAlg->super.printFunc      = printDictionaryCoAlg;
  theCoAlg->newDictionaryFunc    = newDictionaryImpl;
  theCoAlg->createSymbolInThisDictionaryFunc =
    createSymbolInThisDictionaryImpl;
  theCoAlg->deleteSymbolFromThisDictionaryFunc =
    deleteSymbolFromThisDictionaryImpl;
  theCoAlg->getSymbolFunc        = getSymbolImpl;
  theCoAlg->findSymbolFunc       = findSymbolImpl;
  theCoAlg->insertSymbolFunc     = insertSymbolImpl;
  theCoAlg->findLUBSymbolFunc    = findLUBSymbolImpl;
  theCoAlg->listDefinitionsFunc  = listDefinitionsImpl;  
  size_t tag =
    registerJClass(jInterp, (JClass*)theCoAlg);
  
  // do a sanity check...
  assert(tag == DictionariesTag);
  assert(jInterp->coAlgs[tag]);
   
  return TRUE;
}
\stopCCode

\startTestCase[should register the Dictionaries coAlg]

\startCTest
  // CTestsSetup has already created a jInterp
  // and run registerDictionaries
  AssertPtrNotNull(jInterp);
  AssertPtrNotNull(jInterp->coAlgs);
  AssertPtrNotNull(getDictionariesClass(jInterp));
  DictionariesClass *coAlg = getDictionariesClass(jInterp);
  registerDictionaries(jInterp);
  AssertPtrNotNull(getDictionariesClass(jInterp));
  AssertPtrEquals(getDictionariesClass(jInterp), coAlg);
  AssertIntEquals(
    getDictionariesClass(jInterp)->super.objectSize,
    sizeof(DictObj)
  )
\stopCTest
\stopTestCase
\stopTestSuite
