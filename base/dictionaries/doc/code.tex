% A ConTeXt document [master document: dictionaries.tex]

\section[title=Code]
\setCHeaderStream{public}

\dependsOn[jInterps]
\dependsOn[symbols]

\startCHeader
typedef struct dictionaries_object_struct DictObj;

typedef struct dictionaries_object_struct {
  CoAlgObj  super;
  Symbol*   symbol;
  CoAlgObj* preObs;
  CoAlgObj* value;
  CoAlgObj* postObs;
  DictObj*  left;
  DictObj*  right;
  DictObj*  previous;
  DictObj*  next;
  size_t    height;
  long      balance;
} DictObj;
\stopCHeader

\startTestSuite[newDict]

\startCHeader
extern DictObj *newDict(JoyLoLInterp *jInterp, Symbol *aSym);
\stopCHeader

\setCHeaderStream{private}
\startCHeader
#define asSymbol(aNode)   ((DictObj*)(aNode))->symbol
#define asPreObs(aNode)   ((DictObj*)(aNode))->preObs
#define asValue(aNode)    ((DictObj*)(aNode))->value
#define asPostObs(aNode)  ((DictObj*)(aNode))->postObs
#define asLeft(aNode)     ((DictObj*)(aNode))->left
#define asRight(aNode)    ((DictObj*)(aNode))->right
#define asPrevious(aNode) ((DictObj*)(aNode))->previous
#define asNext(aNode)     ((DictObj*)(aNode))->next
#define asHeight(aNode)   ((DictObj*)(aNode))->height
#define asBalance(aNode)  ((DictObj*)(aNode))->balance
\stopCHeader
\setCHeaderStream{public}

\startCCode
// We implement our dictionary as an AVL binary tree using AVLNodes.
//
// Our implementation is inspired by:
// The Crazy Programmer's "Program for AVL Tree in C" (Neeraj Mishra)
// http://www.thecrazyprogrammer.com/2014/03/c-program-for-avl-tree-implementation.html
// and by:
// Jianye Hao's CSC2100B Tutorial 4 "Binary and AVL Trees in C"
// https://www.cse.cuhk.edu.hk/irwin.king/_media/teaching/csc2100b/tu4.pdf
//
// At the moment we only insert and search (we never delete).
//
// ANY AVLTree node can be the root of a new dictionary.
//

DictObj *newDict(JoyLoLInterp *jInterp, Symbol *aSym) {
  assert(jInterp);
  assert(jInterp->coAlgs);
  assert(DictionariesTag < jInterp->numCoAlgs);
  assert(jInterp->coAlgs[DictionariesTag].sClass);
  
  assert(aSym);
  
  DEBUG(FALSE, "newAVLNode [%s]\n", aSym);
  CoAlgObj* newNode   = newObject(jInterp, DictionariesTag);
  asSymbol(newNode)   = strdup(aSym);
  asPreObs(newNode)   = NULL;
  asValue(newNode)    = NULL;
  asPostObs(newNode)  = NULL;
  asLeft(newNode)     = NULL;
  asRight(newNode)    = NULL;
  asPrevious(newNode) = NULL;
  asNext(newNode)     = NULL;
  asHeight(newNode)   = 1;
  asBalance(newNode)  = 0;
  return (DictObj*)newNode;
}
\stopCCode

\startTestCase[should add a new dict object (AVL node)]

\startCTest
  AssertPtrNotNull(jInterp);
  AssertPtrNotNull(jInterp->coAlgs[DictionariesTag].sClass);

  DictObj* aNode = newDict(jInterp, "aNodeSymbol");
  AssertPtrNotNull(aNode);
  AssertPtrNotNull(asSymbol(aNode));
  AssertStrEquals(asSymbol(aNode), "aNodeSymbol");
  AssertPtrNull(asValue(aNode));
  AssertPtrNull(asLeft(aNode));
  AssertPtrNull(asRight(aNode));
  AssertPtrNull(asPrevious(aNode));
  AssertPtrNull(asNext(aNode));
  AssertIntEquals(asHeight(aNode), 1);
  AssertIntEquals(asBalance(aNode), 0);
\stopCTest
\stopTestCase
\stopTestSuite

\startTestSuite[createSymbol]

\startCHeader
DictObj* createSymbol(JoyLoLInterp *jInterp, Symbol* aSymbol);
\stopCHeader

\startCCode
DictObj* createSymbol(JoyLoLInterp *jInterp, Symbol* aSymbol) {
  assert(jInterp);
  if (!aSymbol) return NULL;
  DictObj* aSym = findSymbol(jInterp, aSymbol);
  if (!aSym) {
    jInterp->dict.root = insertSymbol(jInterp, aSymbol);
    aSym = findSymbol(jInterp, aSymbol);
  }
  return aSym;
}
\stopCCode
\stopTestSuite

\startCHeader
extern CoAlgObj* getSymbol(JoyLoLInterp* jInterp,
                           Symbol* aSymbol);
\stopCHeader

\startCCode
CoAlgObj* getSymbol(JoyLoLInterp* jInterp,
                    Symbol* aSymbol) {
  DictObj* aSym = createSymbol(jInterp, aSymbol);
  return newSymbol(jInterp, aSym->symbol);
}
\stopCCode

\startCHeader
extern void listDefinitions(JoyLoLInterp* jInterp, FILE* outFile);
\stopCHeader

\startCCode
void listDefinitions(JoyLoLInterp* jInterp, FILE* outFile) {
  DictObj* curNode = jInterp->dict.firstSymbol;
  while(curNode) {
    if (curNode->value) {
      fprintf(outFile,"%s == %s\n",
              curNode->symbol, printLoL(curNode->value));
    }
    curNode = curNode->next;
  }
}
\stopCCode

\setCHeaderStream{private}
\startCHeader
size_t printDictionaryCoAlgObjSize(CoAlgObj* anObj, Boolean debugFlag);
size_t printDicSize(DictObj* anAVLNode, Boolean debugFlag);
\stopCHeader

\startCCode
size_t printDictionaryCoAlgObjSize(CoAlgObj* anObj, Boolean debugFlag) {
  if (!anObj) return 0;
  if (anObj->tag != DictionariesTag) return 0;
  return printDicSize((DictObj*)anObj, debugFlag);
}

size_t printDicSize(DictObj* anAVLNode, Boolean debugFlag) {
  if (!anAVLNode) return 0;
  return printDicSize(anAVLNode->left, debugFlag)
    + printDicSize(anAVLNode->right, debugFlag)
    + strlen(anAVLNode->symbol) + 20;
}
\stopCCode

\startCHeader
extern Boolean printDicionaryCoAlgObjInto(
  CoAlgObj* anAVLNode, char* buffer, size_t bufferSize
);
extern Boolean printDicInto(
  DictObj* anAVLNode, char* buffer, size_t bufferSize
);
\stopCHeader

\startCCode
Boolean printDictionaryCoAlgObjInto(
  CoAlgObj* anAVLNode, char* buffer, size_t bufferSize
) {
  if (!anAVLNode) return FALSE;
  if (anAVLNode->tag != DictionariesTag) return FALSE;
  return printDicInto((DictObj*)anAVLNode, buffer, bufferSize);
}

Boolean printDicInto(DictObj* anAVLNode, char* buffer, size_t bufferSize) {
  if (!anAVLNode) return TRUE;
  strcat(buffer, "[");
  strcat(buffer, anAVLNode->symbol);
  strcat(buffer, "] l:( ");
  printDicInto(anAVLNode->left, buffer, bufferSize);
  strcat(buffer, " ) r:( ");
  printDicInto(anAVLNode->right, buffer, bufferSize);
  strcat(buffer, " ) ");
  return TRUE;
}
\stopCCode

\startCHeader
char* printDictionaryDebug(DictObj* anAVLNode, Boolean debugFlag);
#define printDictionary(anAVLNode) \
  printDictionaryDebug(anAVLNode, FALSE)
\stopCHeader

\startCCode
char* printDictionaryDebug(DictObj* anAVLNode, Boolean debugFlag) {
  size_t bufferSize = printDicSize(anAVLNode, debugFlag) + 10;
  char* buffer = (char*) calloc(bufferSize, sizeof(char));
  assert(buffer);
  printDicInto(anAVLNode, buffer, bufferSize);
  buffer[strlen(buffer)-1] = 0;
  return buffer;
}
\stopCCode

\startTestSuite[registerDictionaries]

\startCHeader
Boolean registerDictionaries(JoyLoLInterp *jInterp);
\stopCHeader

\startCCode
Boolean registerDictionaries(JoyLoLInterp *jInterp) {
  assert(jInterp);
  
  CoAlgebra* theCoAlg    = (CoAlgebra*) calloc(1, sizeof(CoAlgebra));
  assert(theCoAlg);
  
  theCoAlg->name         = "Dictionaries";
  theCoAlg->objectSize   = sizeof(DictObj);
  theCoAlg->registerFunc = registerDictionaries;
  theCoAlg->equalityFunc = NULL;
  theCoAlg->printSize    = printDictionaryCoAlgObjSize;
  theCoAlg->printStr     = printDictionaryCoAlgObjInto;
  size_t tag = registerCoAlgebra(jInterp, theCoAlg);
  
  // do a sanity check...
  assert(tag == DictionariesTag);
  assert(jInterp->coAlgs[tag].sClass);
  
  registerDictionaryWords(jInterp);
  
  return TRUE;
}
\stopCCode

\startTestCase[should register the Dictionaries coAlg]

\startCTest
  // CTestsSetup has already created a jInterp
  // and run regiserDictionaries
  AssertPtrNotNull(jInterp);
  AssertPtrNotNull(jInterp->coAlgs[DictionariesTag].sClass);
  CoAlgebra *coAlg = jInterp->coAlgs[DictionariesTag].sClass;
  AssertIntTrue(registerDictionaries(jInterp));
  AssertPtrNotNull(jInterp->coAlgs[DictionariesTag].sClass);
  AssertPtrEquals(jInterp->coAlgs[DictionariesTag].sClass, coAlg);
  AssertIntEquals(
    jInterp->coAlgs[DictionariesTag].sClass->objectSize, 
    sizeof(DictObj)
  );
\stopCTest
\stopTestCase
\stopTestSuite

\setCHeaderStream{public}
