% A ConTeXt document [master document: dictionaries.tex]

\section[title=Code]
\setCHeaderStream{public}

\dependsOn[jInterps]
%\dependsOn[context]

\component gitVersion-c

\startCHeader
typedef struct dictionary_object_struct {
  JObj          super;
  JoyLoLInterp *jInterp;
  Symbol       *name;
  DictObj      *parent;
  DictNodeObj  *root;
  DictNodeObj  *firstSymbol;
} DictObj;

#define asCFunc(aLoL) (((CFunctionObj*)(aLoL))->func)
\stopCHeader

\startTestSuite[newDictionary]

\startCHeader
typedef DictObj* (NewDictionary)(
  JoyLoLInterp *jInterp,
  Symbol       *name,
  DictObj      *parent
);

#define newDictionary(jInterp, name, parent)      \
  (                                               \
    assert(getDictionariesClass(jInterp)          \
      ->newDictionaryFunc),                       \
    (getDictionariesClass(jInterp)                \
      ->newDictionaryFunc(jInterp, name, parent)) \
  )
\stopCHeader

\setCHeaderStream{private}
\startCHeader
extern DictObj* newDictionaryImpl(
  JoyLoLInterp *jInterp,
  Symbol       *name,
  DictObj      *parent
);
\stopCHeader
\setCHeaderStream{public}

\startCCode
DictObj* newDictionaryImpl(
  JoyLoLInterp *jInterp,
  Symbol       *name,
  DictObj      *parent
) {
  assert(jInterp);
  assert(jInterp->coAlgs);
  
  DictObj* result =
    (DictObj*)newObject(jInterp, DictionariesTag);
  result->jInterp     = jInterp;
  result->name        = strdup(name);
  result->parent      = parent;
  result->root        = NULL;
  result->firstSymbol = NULL;
  assert(result);
  
  result->super.type  = jInterp->coAlgs[DictionariesTag];
 
  return result;
}
\stopCCode

\startTestCase[should create a new Dictionary]

\startCTest
  AssertPtrNotNull(jInterp);

  JObj* aNewDictionary = newDictionary(jInterp, "aName", NULL);
  AssertPtrNotNull(aNewDictionary);
  AssertPtrEquals(aNewDictionary->jInterp, jInterp);
  AssertStrEquals(aNewDictionary->name, "aName");
  AssertPtrNull(aNewDictionary->parent);
  AssertPtrNull(aNewDictionary->root);
  AssertPrtNull(aNewDictionary->firstSymbol);
  AssertPtrNotNull(asType(aNewDictionary));
  AssertIntEquals(asTag(aNewDictionary), DictionariesTag);
  AssertIntTrue(isAtom(aNewDictionary));
  AssertIntTrue(isDictionary(aNewDictionary));
  AssertIntFalse(isPair(aNewDictionary));
\stopCTest
\stopTestCase

\startCHeader
#define isDict(aLoL)                    \
  (                                     \
    (                                   \
      (aLoL) &&                         \
      asType(aLoL) &&                   \
      (asTag(aLoL) == DictionariesTag)  \
    ) ?                                 \
      TRUE :                            \
      FALSE                             \
  )
\stopCHeader

\startTestSuite[findSymbol]

\setCHeaderStream{private}
\startCHeader
typedef DictNodeObj* (FindSymbol)(
  DictObj *aDict,
  Symbol  *aSymbol
);

extern DictNodeObj* findSymbolInThisDictionary(
  DictObj *aDict,
  Symbol  *aSymbol
);

extern DictNodeObj* findSymbol(
  DictObj *aDict,
  Symbol  *aSymbol
);
\stopCHeader
\setCHeaderStream{public}

\startCCode
DictNodeObj* findSymbolInThisDictionary(
  DictObj *aDict,
  Symbol  *aSymbol
) {
  if (!aSymbol) return NULL;
  assert(aDict);
  return findSymbolRecurse(aDict, aDict->root, aSymbol);
}

DictNodeObj* findSymbol(
  DictObj *aDict,
  Symbol  *aSymbol
) {
  if (!aSymbol) return NULL;
  while (aDict) {
    //
    // Look for this symbol in this naming scope
    //
    DictNodeObj *aDictNode = 
      findSymbolRecurse(aDict, aDict->root, aSymbol);
    if (aDictNode) return aDictNode;
    //
    // We have not found this symbol in this naming scope
    // so look in the parent's naming scope
    //
    aDict = aDict->parent;
  }
  //
  // Alas, we have not found this symbol in any naming scope
  //
  return NULL;
}
\stopCCode

\startTestCase[should find symbols in parent dictionary]
\startCTest
  AssertPtrNotNull(jInterp);

  DictObj* parentDict = newDictionary(jInterp, "parent", NULL);
  AssertPtrNotNull(parentDict);
  AssertPtrNull(parentDict->parent);
  DictObj* childDict  = newDictionary(jInterp, "child", parentDict);
  AssertPtrNotNull(childDict);
  AssertPtrNotNull(childDict->parent);
  AssertPtrEquals(childDict->parent, parentDict);
  //
  DictNodeObj* parentSym = findSymbol(childDict, "test");
  AssertPtrNull(parentSym);
  //
  parentSym = 
    createSymbolInThisDictionary(parentDict, "test");
  AssertPtrNotNull(parentSym);
  AssertStrEquals(parentSym->symbol, "test");
  //
  DictNodeObj* testSym = findSymbol(childDict, "test");
  AssertPtrNotNull(testSym);
  AssertPtrEquals(testSym, parentSym);
  //
  DictNodeObj* childSym =
    createSymbolInThisDictionary(childDict, "test");
  AssertPtrNotNull(childSym);
  AssertPtrNotEquals(parentSym, childSym);
  AssertStrEquals(childSym->symbol, "test");
  //
  testSym = findSymbol(childDict, "test");
  AssertPtrNotNull(testSym);
  AssertPtrEquals(testSym, childSym);
  AssertPtrNotEquals(testSym, parentSym);
  //
  testSym = findSymbol(parentDict, "test");
  AssertPtrNotNull(testSym);
  AssertPtrNotEquals(testSym, childSym);
  AssertPtrEquals(testSym, parentSym);
  //
  StringBufferObj *aStrBuf = newStringBuffer(jInterp->rootCtx);
  AssertPtrNotNull(aStrBuf);
  printLoL(aStrBuf, (JObj*)childDict);
  AssertStrEquals(getCString(aStrBuf), "dict:parent.child ");
  strBufClose(aStrBuf);  
\stopCTest
\stopTestCase
\stopTestSuite

\startTestSuite[insertSymbol]

\startCCode
static DictNodeObj* insertSymbol(
  DictObj *aDict,
  Symbol  *aSymbol
) {
  assert(aDict);
  assert(aSymbol);

  // lazy initialization
  if (!aDict->root) {
    assert(aDict->jInterp);
    DictNodeObj* firstNode = newDictNode(aDict->jInterp, aSymbol);
    aDict->root            = firstNode;
    aDict->firstSymbol     = firstNode;
    return firstNode;
  }

  return insertSymbolRecurse(aDict, aDict->root, aSymbol);
}
\stopCCode
\stopTestSuite

\startTestSuite[deleteSymbol]

\startCHeader
typedef void (DeleteSymbol)(
  DictObj *aDict,
  Symbol  *aSymbol
);

#define deleteSymbol(aDict, aSymbol)            \
  (                                             \
    assert(aDict->jInterp),                     \
    assert(getDictionariesClass(aDict->jInterp) \
      ->deleteSymbolFunc),                      \
    (getDictionariesClass(aDict->jInterp)       \
      ->deleteSymbolFunc(aDict, aSymbol))       \
  )
\stopCHeader

\setCHeaderStream{private}
\startCHeader
extern void deleteSymbolImpl(
  DictObj *aDict,
  Symbol  *aSymbol
);
\stopCHeader
\setCHeaderStream{public}

\startCCode
void deleteSymbolImpl(
  DictObj *aDict,
  Symbol  *aSymbol
) {
  if (!aSymbol) return;
  assert(aDict);
  //
  // Look for the first dot
  //
  char* symPrefix = strdup(aSymbol);
  char* restOfSym = strchr(symPrefix, '.');
  //
  if (!restOfSym) {
    //
    // no dot found so this is the end of the recursion
    // just delete the symbol
    //
    aDict->root = 
      deleteSymbolRecurse(aDict, aDict->root, aSymbol);
    free(symPrefix); // this is *our* duplicate of aSymbol
    return;
  }
  //
  // a dot has been found...
  // so split the symbol into two a the first dot...
  //
  *restOfSym = 0; // terminate the symPrefix at the dot
  restOfSym++;    // move onto the next char
  //
  // now look for the prefix dictionary
  // start by looking in the dictionary provided
  //
  // there is NO dot in the prefix...
  // so just simply find the symbol
  //
  DictNodeObj* aDictSym = findSymbol(aDict, symPrefix);
  if (!aDictSym || !isDictionary(aDictSym->value)) {
    //
    // there is no prefix dictionary
    // there is nothing to delete...
    // so free our duplicate of aSymbol
    // and return
    //
    free(symPrefix);
    return;
  }
  //
  // a prefix dictionary HAS been found
  // so setup the prefix dictionary
  //
  DictObj* prefixDict = (DictObj*)(aDictSym->value);
  //
  // and delete the restOfSym from the prefixDict
  //
  deleteSymbolImpl(prefixDict, restOfSym);
  //
  free(symPrefix); // this is *our* duplicate of aSymbol
}
\stopCCode

\startTestCase[should delete random symbols from a randomly created dictionary]
\startCTest
  srand(time(NULL));

  AssertPtrNotNull(jInterp);
  DictObj *aDict = newDictionary(jInterp, "tests", NULL);

  for (int i = 0; i < 1000; i++) {
    char itoa[100];
    sprintf(itoa, "%03d", rand() % 100);
    createSymbolInThisDictionary(aDict, itoa);
  }
  createSymbolInThisDictionary(aDict, "000");
  createSymbolInThisDictionary(aDict, "100");

  DictNodeObj *aNode = findSymbol(aDict, "000");
  AssertPtrNotNull(aNode);
  AssertStrEquals(aNode->symbol, "000");
  AssertPtrEquals(aDict->firstSymbol, aNode);
  
  aNode = findSymbol(aDict, "100");
  AssertPtrNotNull(aNode);
  AssertStrEquals(aNode->symbol, "100");
  aNode = aDict->firstSymbol;
  while (aNode->next) aNode = aNode->next;
  AssertStrEquals(aNode->symbol, "100");
  
  for (int i = 0; i < 1000; i++) {
    char itoa[100];
    int randNum = rand() % 100;
    //
    if (randNum == 0 || randNum == 100) continue;
    //
    sprintf(itoa, "%03d", randNum);
    deleteSymbol(aDict, itoa);
    DictNodeObj *aNode = aDict->firstSymbol;
    AssertStrEquals(aNode->symbol, "000");
    while (aNode->next) {
      AssertStrNotEquals(aNode->symbol, itoa);
      aNode = aNode->next;
    }
    AssertStrEquals(aNode->symbol, "100");   
    while (aNode->previous) {
      AssertStrNotEquals(aNode->symbol, itoa);
      aNode = aNode->previous;
    }
    AssertStrEquals(aNode->symbol, "000");
  }  
\stopCTest
\stopTestCase
\stopTestSuite

\startTestSuite[findLUBSymbol]
\startCHeader
typedef DictNodeObj *(FindLUBSymbol)(
  DictObj *aDict,
  Symbol  *aSymbol
);

#define findLUBSymbol(aDict, aSymbol)           \
  (                                             \
    assert(aDict),                              \
    assert(getDictionariesClass(aDict->jInterp) \
      ->findLUBSymbolFunc),                     \
    (getDictionariesClass(aDict->jInterp)       \
      ->findLUBSymbolFunc(aDict, aSymbol))      \
  )
\stopCHeader

\setCHeaderStream{private}
\startCHeader
extern DictNodeObj* findLUBSymbolImpl(
  DictObj *aDict,
  Symbol  *aSymbol
);
\stopCHeader
\setCHeaderStream{public}

\startCCode
DictNodeObj* findLUBSymbolImpl(
  DictObj *aDict,
  Symbol  *aSymbol
) {
  assert(aDict);
  if (!aSymbol) return aDict->firstSymbol;
  return findLUBSymbolRecurse(aDict, aDict->root, aSymbol);
}
\stopCCode
\stopTestSuite

\startTestSuite[createSymbolInThisDictionary]

\setCHeaderStream{private}
\startCHeader
extern DictNodeObj* createSymbolInThisDictionary(
  DictObj *aDict,
  Symbol  *aSymbol
);
\stopCHeader
\setCHeaderStream{public}

\startCCode
DictNodeObj* createSymbolInThisDictionary(
  DictObj *aDict,
  Symbol  *aSymbol
) {
  assert(aDict);
  if (!aSymbol) return NULL;
  DictNodeObj* aSym =
    findSymbolInThisDictionary(aDict, aSymbol);
  if (!aSym) {
    aDict->root = insertSymbol(aDict, aSymbol);
    aSym = findSymbolInThisDictionary(aDict, aSymbol);
  }
  return aSym;
}
\stopCCode
\stopTestSuite

\startTestSuite[getSymbolEntry]

\startCCode
DictNodeObj* walkEntryPath(
  DictObj    *aDict,
  Symbol     *aSymbol,
  FindSymbol *symbolFinder
) {
  assert(aDict);
  assert(aSymbol);
  DEBUG(aDict->jInterp,
    "walkEntryPath: [%s](%p) [%s] %p\n",
    aDict->name, aDict, aSymbol, symbolFinder
  );
  //
  // Look for the first dot
  //
  char* symPrefix = strdup(aSymbol);
  char* restOfSym = strchr(symPrefix, '.');
  //
  if (!restOfSym) {
    DEBUG(aDict->jInterp, "no dot found in [%s]\n", aSymbol);
    //
    // no dot found so this is the end of the recursion
    // just find the symbol or create one
    //
    DictNodeObj* aSym = symbolFinder(aDict, aSymbol);
    if (!aSym) {
      DEBUG(aDict->jInterp,
        "creating new entry in [%s](%p) using [%s]\n",
        aDict->name, aDict, aSymbol
      );
      aSym = createSymbolInThisDictionary(aDict, aSymbol);
    }
    free(symPrefix); // this is *our* duplicate of aSymbol
    DEBUG(aDict->jInterp,
      "using symbol [%s](%p) value: %p\n",
      (aSym ? aSym->symbol : ""),
      aSym,
      (aSym ? aSym->value : NULL)
    );
    return aSym;
  }
  //
  // a dot has been found...
  // so split the symbol into two at the first dot...
  //
  *restOfSym = 0; // terminate the symPrefix at the dot
  restOfSym++;    // move onto the next char
  DEBUG(aDict->jInterp,
    "dot found in [%s] splitting into [%s] and [%s]\n",
    aSymbol, symPrefix, restOfSym
  );

  //
  // setup the prefix dictionary
  //
  DictObj* prefixDict = NULL;
  //
  // now look for the prefix dictionary
  // start by looking in the dictionary provided
  //
  DictNodeObj *aDictSym =
    walkEntryPath(aDict, symPrefix, symbolFinder);
  assert(aDictSym);
  //
  if (!aDictSym->value) {
    //
    // no dictionary found...
    // create a new one...
    // and use it as the prefix Dictionary
    //
    // WHICH PARENT DICTIONARY SHOULD WE USE?
    //
    prefixDict = 
      newDictionary(aDict->jInterp, symPrefix, aDict);
    aDictSym->value = (JObj*)prefixDict;
    DEBUG(aDict->jInterp,
      "no dictionary found in [%s](%p) created new dictionary: [%s](%p)\n",
      symPrefix, aDict, prefixDict->name, prefixDict
    );
    //
  } else {
    //
    // The prefix entry has been found...
    //
    if (isDictionary(aDictSym->value)) {
      //
      // the symbol IS a dictionary
      // so use it as the prefix dictionary
      //
      prefixDict = (DictObj*)(aDictSym->value);
      DEBUG(aDict->jInterp,
        "prefix dictionary found: [%s](%p)\n",
        prefixDict->name, prefixDict
      );
      //
    } else {
      //
      // the symbol is NOT a dictionary...
      // so replace the dot with an '_'
      restOfSym--;
      *restOfSym = '_';
      // and retry...
      //
      // use the original dictionary
      //
      prefixDict = aDict;
      //
      // and use the modified symbol
      //
      restOfSym = symPrefix;
      DEBUG(aDict->jInterp,
        "non-dictionary found for prefix [%s]\n", symPrefix
      );
      DEBUG(aDict->jInterp,
        "re-trying with dictionary: [%s](%p) and symbol [%s]\n",
        prefixDict->name, prefixDict, restOfSym
      );
    }
  }
  assert(prefixDict);
  //
  DictNodeObj* aSym =
    walkEntryPath(prefixDict, restOfSym, symbolFinder);
  //
  free(symPrefix); // this is *our* duplicate of aSymbol
  DEBUG(aDict->jInterp,
    "using symbol [%s](%p) value: %p\n",
    (aSym ? aSym->symbol : ""),
    aSym,
    (aSym ? aSym->value : NULL)
  );

  return aSym;
}
\stopCCode

\startCHeader
typedef DictNodeObj *(GetSymbolEntry)(
  DictObj *aDict,
  Symbol  *aSymbol
);

#define getSymbolEntry(aDict, aSymbol)          \
  (                                             \
    assert(aDict),                              \
    assert(getDictionariesClass(aDict->jInterp) \
      ->getSymbolEntryFunc),                    \
    (getDictionariesClass(aDict->jInterp)       \
      ->getSymbolEntryFunc(aDict, aSymbol))     \
  )

#define getSymbolEntryInChild(aDict, aSymbol)       \
  (                                                 \
    assert(aDict),                                  \
    assert(getDictionariesClass(aDict->jInterp)     \
      ->getSymbolEntryFunc),                        \
    (getDictionariesClass(aDict->jInterp)           \
      ->getSymbolEntryInChildFunc(aDict, aSymbol))  \
  )
\stopCHeader

\setCHeaderStream{private}
\startCHeader
extern DictNodeObj* getSymbolEntryImpl(
  DictObj *aDict,
  Symbol  *aSymbol
);

extern DictNodeObj* getSymbolEntryInChildImpl(
  DictObj *aDict,
  Symbol  *aSymbol
);
\stopCHeader
\setCHeaderStream{public}

\startCCode
DictNodeObj* getSymbolEntryImpl(
  DictObj *aDict,
  Symbol  *aSymbol
) {
  assert(aDict);
  assert(aSymbol);
  DEBUG(aDict->jInterp,
    "getSymbolEntry: %p [%s]\n", aDict, aSymbol
  );
  return walkEntryPath(
    aDict,
    aSymbol,
    findSymbol
  );
}

DictNodeObj* getSymbolEntryInChildImpl(
  DictObj *aDict,
  Symbol  *aSymbol
) {
  assert(aDict);
  assert(aSymbol);
  DEBUG(aDict->jInterp,
    "getSymbolEntryInChild: %p [%s]\n", aDict, aSymbol
  );
  return walkEntryPath(
    aDict,
    aSymbol,
    findSymbolInThisDictionary
  );
}
\stopCCode

\startTestCase[should get and delete dotted symbols]
\startCTest
  AssertPtrNotNull(jInterp);

  DictObj* parentDict =
    newDictionary(jInterp, "parent", NULL);
  AssertPtrNotNull(parentDict);
  AssertPtrNull(parentDict->parent);
  DictNodeObj* parentEntry =
    createSymbolInThisDictionary(parentDict, "parent");
  AssertPtrNotNull(parentEntry);
  parentEntry->value = (JObj*)parentDict;
  
  DictObj* childDict =
    newDictionary(jInterp, "child", parentDict);
  AssertPtrNotNull(childDict);
  AssertPtrNotNull(childDict->parent);
  AssertPtrEquals(childDict->parent, parentDict);
  DictNodeObj* childEntry =
    createSymbolInThisDictionary(parentDict, "child");
  AssertPtrNotNull(childEntry);
  childEntry->value = (JObj*)childDict;
  
  DictNodeObj* testSym = 
    createSymbolInThisDictionary(childDict, "test");
  AssertPtrNotNull(testSym);
  AssertStrEquals(testSym->symbol, "test");
  
  DictNodeObj* foundSym =
    getSymbolEntry(childDict, "test");
  AssertPtrNotNull(foundSym);
  AssertPtrEquals(foundSym, testSym);

  foundSym = getSymbolEntry(parentDict, "child.test");
  AssertPtrNotNull(foundSym);
  AssertPtrEquals(foundSym, testSym);
  
  foundSym = getSymbolEntry(childDict, "child.test");
  AssertPtrNotNull(foundSym);
  AssertPtrEquals(foundSym, testSym);
  
  DictNodeObj* aTestSym =
    getSymbolEntry(parentDict, "assert.test");
  AssertPtrNotNull(aTestSym);
  AssertPtrNotEquals(aTestSym, testSym);
  AssertStrEquals(aTestSym->symbol, "test");
  
  DictNodeObj* aDictSym =
    getSymbolEntry(childDict, "assert");
  AssertPtrNotNull(aDictSym);
  AssertPtrNotNull(aDictSym->value);
  AssertIntTrue(isDictionary(aDictSym->value));
  DictObj* aDict = (DictObj*)(aDictSym->value);
  
  foundSym = getSymbolEntry(aDict, "test");
  AssertPtrNotNull(foundSym);
  AssertPtrEquals(foundSym, aTestSym);
  
  deleteSymbol(childDict, "assert.test");
  foundSym = findSymbol(childDict, "assert");
  AssertPtrNotNull(foundSym);
  AssertPtrEquals(foundSym, aDictSym);
  AssertPtrNotNull(foundSym->value);
  AssertIntTrue(isDictionary(foundSym->value));
  AssertPtrEquals(foundSym->value, (JObj*)aDict);
  
  foundSym = findSymbol(aDict, "test");
  AssertPtrNull(foundSym);
  
  foundSym = findSymbol(childDict, "test");
  AssertPtrNotNull(foundSym);
\stopCTest
\stopTestCase
\stopTestSuite

\startTestSuite[getAsSymbol]

\startCHeader
typedef JObj *(GetAsSymbol)(
  DictObj *aDict,
  Symbol  *aSymbol,
  Symbol  *fileName,
  size_t   line
);

#define getAsSymbol(aDict, aSymbol, fileName, line)       \
  (                                                       \
    assert(aDict),                                        \
    assert(getDictionariesClass(aDict->jInterp)           \
      ->getAsSymbolFunc),                                 \
    (getDictionariesClass(aDict->jInterp)                 \
      ->getAsSymbolFunc(aDict, aSymbol, fileName, line))  \
  )
\stopCHeader

\setCHeaderStream{private}
\startCHeader
extern JObj* getAsSymbolImpl(
  DictObj *aDict,
  Symbol  *aSymbol,
  Symbol  *fileName,
  size_t   line
);
\stopCHeader
\setCHeaderStream{public}

\startCCode
JObj* getAsSymbolImpl(
  DictObj *aDict,
  Symbol  *aSymbol,
  Symbol  *fileName,
  size_t   line
) {
  assert(aDict);
  DictNodeObj* aSym = getSymbolEntryImpl(aDict, aSymbol);
  assert(aSym);
  return newSymbol(aDict->jInterp, aSym->symbol, fileName, 0);
}
\stopCCode
\stopTestSuite

\startTestSuite[listDefinitions]

\startCHeader
typedef void (ListDefinitions)(
  DictObj         *aDict,
  StringBufferObj *aStrBuf
);

#define listDefinitions(aDict, aStrBuf)         \
  (                                             \
    assert(aDict),                              \
    assert(getDictionariesClass(aDict->jInterp) \
      ->listDefinitionsFunc),                   \
    (getDictionariesClass(aDict->jInterp)       \
      ->listDefinitionsFunc(aDict, aStrBuf))    \
  )
\stopCHeader

\setCHeaderStream{private}
\startCHeader
extern void listDefinitionsImpl(
  DictObj         *aDict,
  StringBufferObj *aStrBuf
);
\stopCHeader
\setCHeaderStream{public}

\startCCode
void listDefinitionsImpl(
  DictObj         *aDict,
  StringBufferObj *aStrBuf
) {
  assert(aDict);
  DictNodeObj* curNode = aDict->firstSymbol;
  while(curNode) {
    if (curNode->value) {
      strBufPrintf(aStrBuf,"%s == ", curNode->symbol);
      printLoL(aStrBuf, curNode->value);
      strBufPrintf(aStrBuf,"\n");
    }
    curNode = curNode->next;
  }
}
\stopCCode

\startTestCase[print Dictionary]
\startCTest
  AssertPtrNotNull(jInterp);

  StringBufferObj *aStrBuf = newStringBuffer(jInterp->rootCtx);
  AssertPtrNotNull(aStrBuf);

  DictObj* aLoL = newDictionary(jInterp, "tests", NULL);
  AssertPtrNotNull(aLoL);
  printLoL(aStrBuf, (JObj*)aLoL);
  AssertStrEquals(getCString(aStrBuf), "dict:tests ");
  strBufClose(aStrBuf);
\stopCTest
\stopTestCase
\stopTestSuite

\startTestSuite[isDictionary]
\startCHeader
#define isDictionary(aLoL)              \
  (                                     \
    (                                   \
      (aLoL) &&                         \
      asType(aLoL) &&                   \
      (asTag(aLoL) == DictionariesTag)  \
    ) ?                                 \
      TRUE :                            \
      FALSE                             \
  )
\stopCHeader
\stopTestSuite

\setCHeaderStream{private}
\startCHeader
extern Boolean equalityDictionaryCoAlg(
  JoyLoLInterp *jInterp,
  JObj         *lolA,
  JObj         *lolB,
  size_t        timeToLive
);
\stopCHeader
\setCHeaderStream{public}

\startCCode
Boolean equalityDictionaryCoAlg(
  JoyLoLInterp *jInterp,
  JObj         *lolA,
  JObj         *lolB,
  size_t        timeToLive
) {
  DEBUG(jInterp, "dictionaryCoAlg-equal a:%p b:%p\n", lolA, lolB);
  if (!lolA && !lolB) return TRUE;
  if (!lolA && lolB)  return FALSE;
  if (lolA  && !lolB) return FALSE;
  if (asType(lolA) != asType(lolB)) return FALSE;
  if (!asType(lolA)) return FALSE;
  if (asTag(lolA)  != DictionariesTag) return FALSE;
  if (lolA != lolB) return FALSE;
  return TRUE;
}
\stopCCode

\startTestSuite[printing dictionaries]

\setCHeaderStream{private}
\startCHeader
extern Boolean printDictionaryCoAlg(
  StringBufferObj *aStrBuf,
  JObj            *aLoL,
  size_t           timeToLive
);
\stopCHeader
\setCHeaderStream{public}

\startCCode
static void printDictionaryName(
  StringBufferObj *aStrBuf,
  DictObj         *aDict
) {
  assert(aDict);
  if (aDict->parent) {
    printDictionaryName(aStrBuf, aDict->parent);
    strBufPrintf(aStrBuf, ".");
  } else {
    strBufPrintf(aStrBuf, "dict:");
  }
  strBufPrintf(aStrBuf, aDict->name);
}

Boolean printDictionaryCoAlg(
  StringBufferObj *aStrBuf,
  JObj            *aLoL,
  size_t           timeToLive
) {
  assert(aLoL);
  assert(asTag(aLoL) == DictionariesTag);
  
  DictObj* theDict = (DictObj*)aLoL;
  
  printDictionaryName(aStrBuf, theDict);
  strBufPrintf(aStrBuf, " ");
  
  return TRUE;
}
\stopCCode

\startTestCase[should print dictionaries]

\startCTest
  AssertPtrNotNull(jInterp);
  AssertPtrNotNull(jInterp->coAlgs[DictionariesTag]);

  StringBufferObj *aStrBuf = newStringBuffer(jInterp->rootCtx);
  AssertPtrNotNull(aStrBuf);
  
  DictObj* aNewDictionary = newDictionary(jInterp, "tests", NULL);
  AssertPtrNotNull(aNewDictionary);
  printLoL(aStrBuf, (JObj*)aNewDictionary);
  AssertStrEquals(getCString(aStrBuf), "dict:tests ");
  strBufClose(aStrBuf);
\stopCTest
\stopTestCase
\stopTestSuite

\startTestSuite[registerDictionaries]

\startCHeader
typedef struct dictionaries_class_struct {
  JClass           super;
  NewDictionary   *newDictionaryFunc;
  GetSymbolEntry  *getSymbolEntryFunc;
  GetSymbolEntry  *getSymbolEntryInChildFunc;
  GetAsSymbol     *getAsSymbolFunc;
  DeleteSymbol    *deleteSymbolFunc;
  FindLUBSymbol   *findLUBSymbolFunc;
  ListDefinitions *listDefinitionsFunc;  
} DictionariesClass;

\stopCHeader

\startCCode
static Boolean initializeDictionaries(
  JoyLoLInterp *jInterp,
  JClass       *aJClass
) {
  assert(jInterp);
  assert(aJClass);
  return TRUE;
}
\stopCCode

\setCHeaderStream{private}
\startCHeader
extern Boolean registerDictionaries(JoyLoLInterp *jInterp);
\stopCHeader
\setCHeaderStream{public}

\startCCode
Boolean registerDictionaries(JoyLoLInterp *jInterp) {
  assert(jInterp);
  assert(jInterp->coAlgs);
  
  DictionariesClass* theCoAlg
    = joyLoLCalloc(1, DictionariesClass);
  assert(theCoAlg);
  
  theCoAlg->super.name                = DictionariesName;
  theCoAlg->super.objectSize          = sizeof(DictObj);
  theCoAlg->super.initializeFunc      = initializeDictionaries;
  theCoAlg->super.registerFunc        = registerDictionaryWords;
  theCoAlg->super.equalityFunc        = equalityDictionaryCoAlg;
  theCoAlg->super.printFunc           = printDictionaryCoAlg;
  theCoAlg->newDictionaryFunc         = newDictionaryImpl;
  theCoAlg->getSymbolEntryFunc        = getSymbolEntryImpl;
  theCoAlg->getSymbolEntryInChildFunc = getSymbolEntryInChildImpl;
  theCoAlg->getAsSymbolFunc           = getAsSymbolImpl;
  theCoAlg->deleteSymbolFunc          = deleteSymbolImpl;
  theCoAlg->findLUBSymbolFunc         = findLUBSymbolImpl;
  theCoAlg->listDefinitionsFunc       = listDefinitionsImpl;  
  size_t tag =
    registerJClass(jInterp, (JClass*)theCoAlg);
  
  // do a sanity check...
  assert(tag == DictionariesTag);
  assert(jInterp->coAlgs[tag]);
   
  return TRUE;
}
\stopCCode

\startTestCase[should register the Dictionaries coAlg]

\startCTest
  // CTestsSetup has already created a jInterp
  // and run registerDictionaries
  AssertPtrNotNull(jInterp);
  AssertPtrNotNull(jInterp->coAlgs);
  AssertPtrNotNull(getDictionariesClass(jInterp));
  DictionariesClass *coAlg = getDictionariesClass(jInterp);
  registerDictionaries(jInterp);
  AssertPtrNotNull(getDictionariesClass(jInterp));
  AssertPtrEquals(getDictionariesClass(jInterp), coAlg);
  AssertIntEquals(
    getDictionariesClass(jInterp)->super.objectSize,
    sizeof(DictObj)
  )
\stopCTest
\stopTestCase
\stopTestSuite
