% A ConTeXt document [master document: booleans.tex]

\section[title=Code]

\dependsOn[jInterps]

\component gitVersion-c

\starttyping
//dictionary.c
#include <stdlib.h>
#include <string.h>
#include <assert.h>

#include "joyLoL/macros.h"
#include "joyLoL/coAlg/coAlgs.h"
#include "joyLoL/lists.h"
#include "joyLoL/dictionary.h"
#include "joyLoL/dictionary_private.h"
#include "joyLoL/printer.h"

// We implement our dictionary as an AVL binary tree using AVLNodes.
//
// Our implementation is inspired by:
// The Crazy Programmer's "Program for AVL Tree in C" (Neeraj Mishra)
// http://www.thecrazyprogrammer.com/2014/03/c-program-for-avl-tree-implementation.html
// and by:
// Jianye Hao's CSC2100B Tutorial 4 "Binary and AVL Trees in C"
// https://www.cse.cuhk.edu.hk/irwin.king/_media/teaching/csc2100b/tu4.pdf
//
// At the moment we only insert and search (we never delete).
//
// ANY AVLTree node can be the root of a new dictionary.
//

Dictionary* createDictionary(void) {
  Dictionary* aDic = (Dictionary*) calloc(1, sizeof(Dictionary));
  assert(aDic);

  // we now initialize lazily in the insertSymbol method
  aDic->dicRoot     = NULL;
  aDic->firstSymbol = NULL;

  return aDic;
}

AVLNode* newAVLNode(const char* aSymbol) {
  assert(aSymbol);
  DEBUG(FALSE, "newAVLNode [%s]\n", aSymbol);
  AVLNode* newNode  = (AVLNode*) calloc(1, sizeof(AVLNode));
  newNode->symbol   = strdup(aSymbol);
  newNode->preObs   = NULL;
  newNode->value    = NULL;
  newNode->postObs  = NULL;
  newNode->left     = NULL;
  newNode->right    = NULL;
  newNode->previous = NULL;
  newNode->next     = NULL;
  newNode->height   = 1;
  newNode->balance  = 0;
  return newNode;
}

AVLNode* createSymbol(Dictionary* aDic, const char* aSymbol) {
  assert(aDic);
  if (!aSymbol) return NULL;
  AVLNode* aSym = findSymbol(aDic, aSymbol);
  if (!aSym) {
    aDic->dicRoot = insertSymbol(aDic, aSymbol);
    aSym = findSymbol(aDic, aSymbol);
  }
  return aSym;
}

PairAtom* getSymbol(CoAlgebras* coAlgs,
                    Dictionary* aDic,
                    const char* aSymbol) {
  AVLNode* aSym = createSymbol(aDic, aSymbol);
  return createAtom(coAlgs, aSym);
}

PairAtom* createAtom(CoAlgebras* coAlgs, AVLNode* anAVLNode) {
  assert(anAVLNode);
  assert(coAlgs);
  return newSymbol(coAlgs, anAVLNode->symbol);
}

void reCalculateAVLNodeHeightBalance(AVLNode* anAVLNode) {
  if (!anAVLNode) return;

  if (!anAVLNode->left && !anAVLNode->right) {
    anAVLNode->height  = 1;
    anAVLNode->balance = 0;
  } else if (!anAVLNode->left) {
    anAVLNode->height  =  1 + anAVLNode->right->height;
    anAVLNode->balance = -1 - anAVLNode->right->height;
  } else if (!anAVLNode->right) {
    anAVLNode->height  = 1 + anAVLNode->left->height;
    anAVLNode->balance = 1 + anAVLNode->left->height;
  } else if (anAVLNode->left->height < anAVLNode->right->height) {
    anAVLNode->height  = 1 + anAVLNode->right->height;
    anAVLNode->balance = anAVLNode->left->height - anAVLNode->right->height;
  } else {
    anAVLNode->height  = 1 + anAVLNode->left->height;
    anAVLNode->balance = anAVLNode->left->height - anAVLNode->right->height;
  }
}

size_t deepCalculateAVLNodeHeight(AVLNode* anAVLNode) {
  if (!anAVLNode) return 0;

  size_t leftHeight = 1 + deepCalculateAVLNodeHeight(anAVLNode->left);
  size_t rightHeight = 1 + deepCalculateAVLNodeHeight(anAVLNode->right);

  if (leftHeight > rightHeight) return leftHeight;
  return rightHeight;
}

void listDefinitions(Dictionary* aDic, FILE* outFile) {
  AVLNode* curNode = aDic->firstSymbol;
  while(curNode) {
    if (curNode->value) {
      fprintf(outFile,"%s == %s\n",
              curNode->symbol, printLoL(curNode->value));
    }
    curNode = curNode->next;
  }
}

\stoptyping

\starttyping
//check
#include <stdlib.h>
#include <string.h>
#include <assert.h>

#include "joyLoL/macros.h"
#include "joyLoL/coAlg/coAlgs.h"
#include "joyLoL/dictionary.h"
#include "joyLoL/dictionary_private.h"

// We implement our dictionary as an AVL binary tree using PairAtoms.
//
// Our implementation is inspired by:
// The Crazy Programmer's "Program for AVL Tree in C" (Neeraj Mishra)
// http://www.thecrazyprogrammer.com/2014/03/c-program-for-avl-tree-implementation.html
// and by:
// Jianye Hao's CSC2100B Tutorial 4 "Binary and AVL Trees in C"
// https://www.cse.cuhk.edu.hk/irwin.king/_media/teaching/csc2100b/tu4.pdf
//
// At the moment we only insert and search (we never delete).
//
// ANY AVLTree node can be the root of a new dictionary.
//

size_t checkAVLNode(AVLNode* anAVLNode, size_t debugFlag) {
  if (!anAVLNode) return TRUE;

  DEBUG(debugFlag, "checkAVLNode %p %ld:%zu=%zu %s\n",
        anAVLNode, anAVLNode->balance, anAVLNode->height,
        deepCalculateAVLNodeHeight(anAVLNode),
        printDictionary(anAVLNode));

  if (anAVLNode->left) {
    DEBUG(debugFlag, "car>-checkAVLNode %p\n", anAVLNode);
    checkAVLNode(anAVLNode->left, debugFlag);
    assert(0 < strcmp(anAVLNode->symbol,
                      anAVLNode->left->symbol));
    DEBUG(debugFlag, "car<-checkAVLNode %p\n", anAVLNode);
  }

  if (anAVLNode->right) {
    DEBUG(debugFlag, "cdr>-checkAVLNode %p\n", anAVLNode);
    checkAVLNode(anAVLNode->right, debugFlag);
    assert(strcmp(anAVLNode->symbol,
                  anAVLNode->right->symbol) < 0);
    DEBUG(debugFlag, "cdr<-checkAVLNode %p\n", anAVLNode);
  }

  assert(anAVLNode->height == deepCalculateAVLNodeHeight(anAVLNode));

  return TRUE;
}

size_t printDicSize(AVLNode* anAVLNode) {
  if (!anAVLNode) return 0;
  return printDicSize(anAVLNode->left)
    + printDicSize(anAVLNode->right)
    + strlen(anAVLNode->symbol) + 20;
}

void printDicInto(AVLNode* anAVLNode, char* buffer, size_t bufferSize) {
  if (!anAVLNode) return;
  strcat(buffer, "[");
  strcat(buffer, anAVLNode->symbol);
  strcat(buffer, "] l:( ");
  printDicInto(anAVLNode->left, buffer, bufferSize);
  strcat(buffer, " ) r:( ");
  printDicInto(anAVLNode->right, buffer, bufferSize);
  strcat(buffer, " ) ");
}

char* printDictionary(AVLNode* anAVLNode) {
  size_t bufferSize = printDicSize(anAVLNode) + 10;
  char* buffer = (char*) calloc(bufferSize, sizeof(char));
  assert(buffer);
  printDicInto(anAVLNode, buffer, bufferSize);
  buffer[strlen(buffer)-1] = 0;
  return buffer;
}
\stoptyping

\starttyping
// find.c
#include <stdlib.h>
#include <string.h>
#include <assert.h>

#include "joyLoL/macros.h"
#include "joyLoL/coAlg/coAlgs.h"
#include "joyLoL/lists.h"
#include "joyLoL/dictionary.h"
#include "joyLoL/dictionary_private.h"

// We implement our dictionary as an AVL binary tree using PairAtoms.
//
// Our implementation is inspired by:
// The Crazy Programmer's "Program for AVL Tree in C" (Neeraj Mishra)
// http://www.thecrazyprogrammer.com/2014/03/c-program-for-avl-tree-implementation.html
// and by:
// Jianye Hao's CSC2100B Tutorial 4 "Binary and AVL Trees in C"
// https://www.cse.cuhk.edu.hk/irwin.king/_media/teaching/csc2100b/tu4.pdf
//
// At the moment we only insert and search (we never delete).
//
// ANY AVLTree node can be the root of a new dictionary.
//

AVLNode* findSymbolRecurse(Dictionary* aDic,
                           AVLNode* anAVLNode,
                           const char* aSymbol) {
  if (!anAVLNode) return NULL;
  int aStrCmp = strcmp(aSymbol, anAVLNode->symbol);
  if (aStrCmp < 0) {
    // aSymbol < anAVLNode->symbol // search the LEFT subtree
    return findSymbolRecurse(aDic, anAVLNode->left, aSymbol);
  } else if (0 < aStrCmp) {
    // aSymbol > anAVLNode->symbol // search the RIGHT subtree
    return findSymbolRecurse(aDic, anAVLNode->right, aSymbol);
  } else {
    // aSymbol == anAVLNode->symbol // return this association pair
    return anAVLNode;
  }
  return NULL;
}

AVLNode* findSymbol(Dictionary* aDic, const char* aSymbol) {
  if (!aSymbol) return NULL;
  assert(aDic);
  return findSymbolRecurse(aDic, aDic->dicRoot, aSymbol);
}

AVLNode* findLUBSymbolRecurse(Dictionary* aDic,
                              AVLNode* anAVLNode,
                              const char* aSymbol) {
  assert(aDic);
  if (!anAVLNode) return aDic->firstSymbol;

  DEBUG(FALSE, "findLUBSymbol %p {%s}[%s] %p %p\n",
        anAVLNode, anAVLNode->symbol, aSymbol,
        anAVLNode->left, anAVLNode->right);

  int aStrCmp = strcmp(aSymbol, anAVLNode->symbol);
  DEBUG(FALSE, "findLUBSymbol cmp: %d\n", aStrCmp);
  if (aStrCmp < 0) {
    // aSymbol < anAVLNode->symbol
    // the current anAVLNode->symbol is an upper bound
    // search the LEFT subtree for a smaller upper bound
    if (anAVLNode->left) {
      AVLNode* aNode = findLUBSymbolRecurse(aDic, anAVLNode->left, aSymbol);
      if (!aNode) {
        // there is nothing in the LEFT subtree which is an upper bound
        // so return this node.
        return anAVLNode;
      }
      // we have found a smaller upper bound... so return it
      return aNode;
    }
    // there is nothing less than this node so return this node
    return anAVLNode;
    //
  } else if (0 < aStrCmp) {
    // anAVLNode->symbol < symbol
    // the current anAVLNode->symbol is a lower bound
    // search the RIGHT subtree for any upper bounds
    if (anAVLNode->right) {
      return findLUBSymbolRecurse(aDic, anAVLNode->right, aSymbol);
    }
    // there is nothing greater than this node so return NULL to signal failure
    return NULL;
    //
  } else {
    // aSymbol == anAVLNode->symbol
    // the current anAVLNode->symbol is the lowest possible upper bound
    // return it
    return anAVLNode;
    //
  }
  return aDic->firstSymbol;
}

AVLNode* findLUBSymbol(Dictionary* aDic, const char* aSymbol) {
  assert(aDic);
  if (!aSymbol) return aDic->firstSymbol;
  return findLUBSymbolRecurse(aDic, aDic->dicRoot, aSymbol);
}

\stoptyping

\starttyping
//insert.c
#include <stdlib.h>
#include <string.h>
#include <assert.h>

#include "joyLoL/macros.h"
#include "joyLoL/coAlg/coAlgs.h"
#include "joyLoL/lists.h"
#include "joyLoL/dictionary.h"
#include "joyLoL/dictionary_private.h"

// We implement our dictionary as an AVL binary tree using AVLNodes.
//
// Our implementation is inspired by:
// The Crazy Programmer's "Program for AVL Tree in C" (Neeraj Mishra)
// http://www.thecrazyprogrammer.com/2014/03/c-program-for-avl-tree-implementation.html
// and by:
// Jianye Hao's CSC2100B Tutorial 4 "Binary and AVL Trees in C"
// https://www.cse.cuhk.edu.hk/irwin.king/_media/teaching/csc2100b/tu4.pdf
//
// At the moment we only insert and search (we never delete).
//
// ANY AVLTree node can be the root of a new dictionary.
//

AVLNode* insertSymbolRecurse(Dictionary* aDic,
                             AVLNode* anAVLNode,
                             const char* aSymbol) {
  if (!anAVLNode) return newAVLNode(aSymbol);

  assert(aDic);

  DEBUG(FALSE, "\ninsertSymbol %p <%s>[%s] %ld:%zu\n",
        anAVLNode, anAVLNode->symbol, aSymbol,
        anAVLNode->balance, anAVLNode->height);

  DEBUG(FALSE, "insertSymbol strncmp %d\n",
        strcmp(aSymbol, anAVLNode->symbol));

  int aStrCmp = strcmp(aSymbol, anAVLNode->symbol);
  if (aStrCmp < 0) {
    // aSymbol < anAVLNode->symbol // insert in LEFT subtree
    DEBUG(FALSE, ">-insert LEFT subtree %p [%s] %ld:%zu=%zu %s\n",
          anAVLNode, aSymbol, anAVLNode->balance,
          anAVLNode->height, deepCalculateAVLNodeHeight(anAVLNode),
          printDictionary(anAVLNode));
    AVLNode* leftResult =
      insertSymbolRecurse(aDic, anAVLNode->left, aSymbol);
    assert(leftResult);
    if (!anAVLNode->left) {
      // we have inserted a new node ...
      // ... insert this new node into the doubly linked list
      //
      AVLNode* oldPrevious               = anAVLNode->previous;
      assert(aDic->firstSymbol);
      if (oldPrevious) oldPrevious->next = leftResult;
      else aDic->firstSymbol             = leftResult;
      leftResult->next                   = anAVLNode;
      leftResult->previous               = oldPrevious;
      anAVLNode->previous                = leftResult;
      //
    }
    anAVLNode->left = leftResult;
    reCalculateAVLNodeHeightBalance(anAVLNode);
    DEBUG(FALSE, "<-insert LEFT subtree %p [%s] %ld:%zu=%zu %s\n",
          anAVLNode, aSymbol, anAVLNode->balance,
          anAVLNode->height, deepCalculateAVLNodeHeight(anAVLNode),
          printDictionary(anAVLNode));
    //
    if (2 < anAVLNode->balance) {
      assert(anAVLNode->left);
      if (strcmp(aSymbol, anAVLNode->left->symbol) < 0) {
        anAVLNode = rotateLeftLeft(anAVLNode);
      } else {
        anAVLNode = rotateLeftRight(anAVLNode);
      }
    }
  } else if (0 < aStrCmp) {
    // aSymbol > anAVLNode->symbol // insert in RIGHT subtree
    DEBUG(FALSE, ">-insert RIGHT subtree %p [%s] %ld:%zu=%zu %s\n",
        anAVLNode, aSymbol, anAVLNode->balance,
        anAVLNode->height, deepCalculateAVLNodeHeight(anAVLNode),
        printDictionary(anAVLNode));
    AVLNode* rightResult =
      insertSymbolRecurse(aDic, anAVLNode->right, aSymbol);
    if (!anAVLNode->right) {
      // we have inserted a new node ...
      // ... insert this new node into the doubly linked list
      //
      AVLNode* oldNext               = anAVLNode->next;
      if (oldNext) oldNext->previous = rightResult;
      rightResult->previous          = anAVLNode;
      rightResult->next              = oldNext;
      anAVLNode->next                = rightResult;
      //
    }
    anAVLNode->right = rightResult;
    reCalculateAVLNodeHeightBalance(anAVLNode);
    DEBUG(FALSE, "<-insert RIGHT subtree %p [%s] %ld:%zu=%zu %s\n",
        anAVLNode, aSymbol, anAVLNode->balance,
        anAVLNode->height, deepCalculateAVLNodeHeight(anAVLNode),
        printDictionary(anAVLNode));
    //
    if (anAVLNode->balance < -2) {
      assert(anAVLNode->right);
      if (strcmp(aSymbol, anAVLNode->right->symbol) > 0) {
        anAVLNode = rotateRightRight(anAVLNode);
      } else {
        anAVLNode = rotateRightLeft(anAVLNode);
      }
    }
  } else {
    // aSymbol == anAVLNode->symbol // nothing to do...
    DEBUG(FALSE,"symols equal <%s>[%s]\n",
          anAVLNode->symbol, aSymbol);
  }

  reCalculateAVLNodeHeightBalance(anAVLNode);
  return anAVLNode;
}

AVLNode* insertSymbol(Dictionary* aDic, const char* aSymbol) {
  assert(aSymbol);
  assert(aDic);

  // lazy initialization
  if (!aDic->dicRoot) {
    AVLNode* firstNode = newAVLNode(aSymbol);
    aDic->dicRoot     = firstNode;
    aDic->firstSymbol = firstNode;
    return firstNode;
  }

  return insertSymbolRecurse(aDic, aDic->dicRoot, aSymbol);
}

\stoptyping

\starttyping
//rotate.c
#include <assert.h>

#include "joyLoL/macros.h"
#include "joyLoL/dictionary.h"
#include "joyLoL/dictionary_private.h"

// We implement our dictionary as an AVL binary tree using AVLNodes.
//
// Our implementation is inspired by:
// The Crazy Programmer's "Program for AVL Tree in C" (Neeraj Mishra)
// http://www.thecrazyprogrammer.com/2014/03/c-program-for-avl-tree-implementation.html
// and by:
// Jianye Hao's CSC2100B Tutorial 4 "Binary and AVL Trees in C"
// https://www.cse.cuhk.edu.hk/irwin.king/_media/teaching/csc2100b/tu4.pdf
//
// At the moment we only insert and search (we never delete).
//
// ANY AVLTree node can be the root of a new dictionary.
//

AVLNode* rotateLeft(AVLNode* anAVLNode) {
  DEBUG(FALSE, ">-rotateLeft %p %ld:%zu=%zu %s\n",
        anAVLNode, anAVLNode->balance, anAVLNode->height,
        deepCalculateAVLNodeHeight(anAVLNode),
        printDictionary(anAVLNode));
  assert(anAVLNode->right);

  AVLNode* newRoot = anAVLNode->right;
  anAVLNode->right = newRoot->left;
  newRoot->left    = anAVLNode;

  reCalculateAVLNodeHeightBalance(anAVLNode);
  reCalculateAVLNodeHeightBalance(newRoot);

  DEBUG(FALSE, "<o-rotateLeft %p %ld:%zu=%zu %s\n",
        anAVLNode, anAVLNode->balance, anAVLNode->height,
        deepCalculateAVLNodeHeight(anAVLNode),
        printDictionary(anAVLNode));
  DEBUG(FALSE, "<n-rotateLeft %p %ld:%zu=%zu %s\n",
        newRoot, newRoot->balance, newRoot->height,
        deepCalculateAVLNodeHeight(newRoot),
        printDictionary(newRoot));
  assert(anAVLNode->height == deepCalculateAVLNodeHeight(anAVLNode));
  assert(newRoot->height == deepCalculateAVLNodeHeight(newRoot));
  return newRoot;
}

AVLNode* rotateRight(AVLNode* anAVLNode) {
  DEBUG(FALSE, ">-rotateRight %p %ld:%zu=%zu %s\n",
        anAVLNode, anAVLNode->balance, anAVLNode->height,
        deepCalculateAVLNodeHeight(anAVLNode),
        printDictionary(anAVLNode));
  assert(anAVLNode->left);

  AVLNode* newRoot = anAVLNode->left;
  anAVLNode->left = newRoot->right;
  newRoot->right  = anAVLNode;

  reCalculateAVLNodeHeightBalance(anAVLNode);
  reCalculateAVLNodeHeightBalance(newRoot);

  DEBUG(FALSE, "<o-rotateRight %p %ld:%zu=%zu %s\n",
        anAVLNode, anAVLNode->balance, anAVLNode->height,
        deepCalculateAVLNodeHeight(anAVLNode),
        printDictionary(anAVLNode));
  DEBUG(FALSE, "<n-rotateRight %p %ld:%zu=%zu %s\n",
        newRoot, newRoot->balance, newRoot->height,
        deepCalculateAVLNodeHeight(newRoot),
        printDictionary(newRoot));
  assert(anAVLNode->height == deepCalculateAVLNodeHeight(anAVLNode));
  assert(newRoot->height == deepCalculateAVLNodeHeight(newRoot));
  return newRoot;
}

AVLNode* rotateLeftLeft(AVLNode* anAVLNode) {
  DEBUG(FALSE, "LL %p %s\n", anAVLNode, printDictionary(anAVLNode));
  return rotateRight(anAVLNode);
}

AVLNode* rotateLeftRight(AVLNode* anAVLNode) {
  DEBUG(FALSE, "0-LR %p %s\n", anAVLNode, printDictionary(anAVLNode));
  anAVLNode->left = rotateLeft(anAVLNode->left);
  DEBUG(FALSE, "1-LR %p %s\n", anAVLNode, printDictionary(anAVLNode));
  return rotateRight(anAVLNode);
}

AVLNode* rotateRightLeft(AVLNode* anAVLNode) {
  DEBUG(FALSE, "0-RL %p %s\n", anAVLNode, printDictionary(anAVLNode));
  anAVLNode->right = rotateRight(anAVLNode->right);
  DEBUG(FALSE, "1-RL %p %s\n", anAVLNode, printDictionary(anAVLNode));
  return rotateLeft(anAVLNode);
}

AVLNode* rotateRightRight(AVLNode* anAVLNode) {
  DEBUG(FALSE, "RR %p %s\n", anAVLNode, printDictionary(anAVLNode));
  return rotateLeft(anAVLNode);
}

\stoptyping