% A ConTeXt document [master document: joyLoL.tex ]

\chapter[title=Compiler/Interpreter]

\section[title=Overview]

The \joylol\ compiler/interpreter is a fairly standard multi-component 
compiler. However instead of compiling to machine code, it actually 
transpiles core functions to ANSI-C and interprets non-core functions by 
calling the core functions in order from the process stack. 

\placefigure[][fig:compiler]
{A high-level structure of the \joylol\ compiler/interpreter} 
\bgroup\startMPcode
input hatching.mp;
picture dots; dots := dashpattern(on 0.1mm off 0.5mm);

path parser;
parser := (4.5cm, 15.2cm)  -- (13.5cm, 15.2cm) --
          (13.5cm, 11.8cm) -- (4.5cm, 11.8cm)  -- cycle;
draw image (
  hatchfill parser
    withcolor (45,2mm,-.5bp);
) dashed dots withcolor darkgray;
draw parser
  withpen pencircle scaled 1mm
  withcolor lightgray;
label.llft("Parser", (13cm, 15cm));

path verifier;
verifier := (4.5cm, 11.2cm) -- (13.5cm, 11.2cm) --
            (13.5cm, 7.8cm) -- (4.5cm, 7.8cm)   -- cycle;
draw image (
  hatchfill verifier
    withcolor (45,2mm,-.5bp);
) dashed dots withcolor darkgray;
draw verifier
  withpen pencircle scaled 1mm
  withcolor lightgray;
label.llft("Verifier", (13cm, 11cm));

path interpreter;
interpreter := (4.5cm, 7.2cm)  -- (13.5cm, 7.2cm) --
               (13.5cm, 1.8cm) -- (4.5cm, 1.8cm)  -- cycle;
draw image (
  hatchfill interpreter
    withcolor (45,2mm,-.5bp);
) dashed dots withcolor darkgray;
draw interpreter
  withpen pencircle scaled 1mm
  withcolor lightgray;
label.llft("Interpreter", (13cm, 7cm));


draw unitsquare xscaled 3cm yscaled 1cm shifted (5cm, 16cm);
label("Code Text", (6.5cm,16.5cm));

drawarrow (6.5cm, 16cm) -- (6.5cm, 15cm);
label.rt("Character stream", (6.5cm, 15.5cm));

draw unitsquare xscaled 3cm yscaled 1cm shifted (5cm, 14cm);
label("Lexer", (6.5cm,14.5cm));

drawarrow (6.5cm, 14cm) -- (6.5cm, 13cm);
label.rt("Lexeme stream", (6.5cm, 13.5cm));

draw unitsquare xscaled 3cm yscaled 1cm shifted (5cm, 12cm);
label("Parser", (6.5cm,12.5cm));

drawarrow (6.5cm, 12cm) -- (6.5cm, 11cm);
label.rt("Abstract Syntax Tree (AST)", (6.5cm, 11.5cm));

draw unitsquare xscaled 3cm yscaled 1cm shifted (5cm, 10cm);
label("Type Inference", (6.5cm,10.5cm));

drawarrow (6.5cm, 10cm) -- (6.5cm, 9cm);
label.rt("Abstract Syntax Tree (AST)", (6.5cm, 9.5cm));

draw unitsquare xscaled 3cm yscaled 1cm shifted (5cm, 8cm);
label("Correctness", (6.5cm,8.75cm));
label("Verifier", (6.5cm,8.25cm));

drawarrow (6.5cm, 8cm)  -- (6.5cm, 7cm);
label.rt("Abstract Syntax Tree (AST)", (6.5cm, 7.5cm));

draw unitsquare xscaled 3cm yscaled 1cm shifted (5cm, 6cm);
label("Code Generator", (6.5cm,6.5cm));

drawarrow (7cm, 6cm)    -- (10.5cm, 5cm);
label.urt("(Restricted) ANSI-C code", (8.75cm, 5.5cm));

draw unitsquare xscaled 3cm yscaled 1cm shifted (9cm, 4cm);
label("ANSI-C compiler", (10.5cm,4.5cm));

drawarrow (10.5cm, 4cm) -- (7cm, 3cm);
label.lrt("Dynamically linked library", (8.75cm,3.5cm));

drawarrow (6.5cm, 6cm)  -- (6.5cm, 3cm);
label.rt("Process stack", (6.5cm, 4.5cm));

draw unitsquare xscaled 3cm yscaled 1cm shifted (5cm, 2cm);
label("Interpreter", (6.5cm,2.5cm));

draw unitsquare xscaled 3cm yscaled 1cm shifted (1cm, 0cm);
label("Input", (2.5cm,0.5cm));
drawarrow (2.5cm, 1cm)  -- (5cm, 2.5cm);

draw unitsquare xscaled 3cm yscaled 1cm shifted (9cm, 0cm);
label("Output", (10.5cm,0.5cm));
drawarrow (8cm, 2.5cm)  -- (10.5cm, 1cm);

draw unitsquare xscaled 3cm yscaled 1cm shifted (1cm, 9cm);
label("Symbol Table", (2.5cm,9.5cm));

drawdblarrow (2.5cm, 10cm) -- (5cm, 14.5cm);
drawdblarrow (3cm, 10cm)   -- (5cm, 12.5cm);

drawdblarrow (4cm, 9.75cm) -- (5cm, 10.5cm);
drawdblarrow (4cm, 9.25cm) -- (5cm, 8.5cm);

drawdblarrow (3cm, 9cm)    -- (5cm, 6.5cm);
drawdblarrow (2.5cm, 9cm)  -- (6cm, 3cm);
\stopMPcode\egroup

As can be seen in figure \in[fig:compiler], the compiler/interpreter 
consists of three main parts, a parser, a (correctness) verifier, and an 
interpreter, each with additional sub-structure. 

\startitemize[1]

\item \bold{Parser}

The parser is composed of a lexer, parser pair. Both the lexer and parser 
will use a push down automata composed by a \joylol\ Parsing Expression 
Grammar, jPeg. Using a push down automata means that, technically, the 
lexer is powerful enough to directly parse \joylol\ code. We use a 
standard lexer, parser pair to allow the lexer and parser's grammars to be 
kept simpler, both to understand as well as to be more performant. 

\startitemize[2]

\item \bold{Lexer}

The lexer extracts lexemes consisting of operators, identifiers, strings, 
comments, and numbers, from the character stream. Each lexeme is either 
categorized by or inserted into the symbol table. Any categorizations will 
be used by the parser. 

\item \bold{Parser}

The parser builds an abstract syntax tree (AST) using the categorized 
lexemes extracted by the lexer. 

\stopitemize

\item \bold{Verifier}

The verifier consists of a type inferencer and a correctness verifier. 

\startitemize[2]

\item \bold{Type inferencer}

The type inferencer ensures that all data on the stack as well as all 
variables in the register machine code has a definite type. 

\item \bold{Correctness verifier}

The correctness verifier checks all implicit and explicit post conditions 
imply the corresponding pre conditions conditions in both the \joylol\ and 
register machine code. The invariants associated with each data type 
contribute implicit pre and post conditions for each \joylol\ statement. 

\stopitemize

\item \bold{Interpreter}

The interpreter consists of the code generator, ANSI-C compiler and the 
\joylol\ interpreter.

\startitemize[2]

\item \bold{Code generator}

The code generator determines whether any particular code fragment will be 
interpreted as a list of \joylol\ words, or as ANSI-C compiled code. If it 
decides that a code fragment can be compiled as ANSI-C code, the code 
generator assigns variables and optimizes the control flow graph to use 
ANSI-C if-then-else or for statements as appropriate. 

Code which will be interpreted as a list of \joylol\ words are placed 
directly on the process stack using the entries in the symbol table. This 
ensures that the interpreter does not usually need to query the symbol 
table as it is interpreting. 

\item \bold{ANSI-C compiler}

The ANSI-C compiler compiles a restricted form of ANSI-C code generated by 
the code generator. These restricted forms ensure that only (relatively) 
well understood ANSI-C code is compiled, in order to maintain the 
correctness of the over all code. We ensure that \quote{risky} C code is 
not used in implementing \joylol. 

While there are a large number of ANSI-C compilers that could be used in 
any project, the typical compilers will be Gnu's gcc and LLVM's clang. If 
LLVM's clang is used, it is potentially possible to dynamically compile 
and load any resulting code. However, the whole system is designed to use 
a completely external compiler in a multi-phased compilation. 

\item\bold{Interpreter}

The interpreter calls the individual \joylol\ words taken from the process 
stack. Each call may alter either the data or process (or both) stacks as 
the result of its action. 

\stopitemize

\stopitemize


