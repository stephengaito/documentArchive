% A ConTeXt document [master document: joyLoL.tex ]

\chapter[title=Compiler/Interpreter]

\section[title=Overview]

The \joylol\ compiler/interpreter is a fairly standard multi-component 
compiler. However instead of compiling to machine code, it actually 
transpiles core functions to ANSI-C and interprets non-core functions by 
calling the core functions in order from the process stack. 

%%%%%%%%%%%%%%%%%%%%%%%%%%%%%%%%%%%%%%%%%%%%%%%%%%%%%%%%%%%%%%%
\placefigure[here][fig:compiler]
{A high-level structure of the \joylol\ compiler/interpreter} 
\bgroup\startMPcode
draw unitsquare xscaled 3cm yscaled 1cm shifted (5cm, 8cm);
label("Code text", (6.5cm,8.5cm));

drawarrow (6.5cm, 8cm)  -- (6.5cm, 7cm);
label.rt("Character stream", (6.5cm, 7.5cm));

draw unitsquare xscaled 3cm yscaled 1cm shifted (5cm, 6cm);
label("Parser", (6.5cm,6.5cm));

drawarrow (6.5cm, 6cm)    -- (6.5cm, 5cm);
label.rt("Abstract Syntax Tree (AST)", (6.5cm, 5.5cm));

draw unitsquare xscaled 3cm yscaled 1cm shifted (5cm, 4cm);
label("Verifier", (6.5cm,4.5cm));

drawarrow (6.5cm, 4cm) -- (6.5cm, 3cm);
label.rt("Abstract Syntax Tree (AST)", (6.5cm,3.5cm));

draw unitsquare xscaled 3cm yscaled 1cm shifted (5cm, 2cm);
label("Interpreter", (6.5cm,2.5cm));

drawarrow (8cm, 2.75cm){dir 0} .. (11.5cm, 3.5cm) ..
  (11.5cm, 7.5cm).. {dir 180}(8cm, 8.25cm);

draw unitsquare xscaled 3cm yscaled 1cm shifted (1cm, 1cm);
label("Input", (2.5cm,1.5cm));
drawarrow (4cm, 1.75cm)  -- (5cm, 2.25cm);

draw unitsquare xscaled 3cm yscaled 1cm shifted (9cm, 1cm);
label("Output", (10.5cm,1.5cm));
drawarrow (8cm, 2.25cm)  -- (9cm, 1.75cm);

draw unitsquare xscaled 3cm yscaled 1cm shifted (1cm, 5cm);
label("Symbol Table", (2.5cm,5.5cm));

drawdblarrow (4cm, 5.75cm) -- (5cm, 6.5cm);
drawdblarrow (4cm, 5.25cm) -- (5cm, 4.5cm);

drawdblarrow (2.5cm, 5cm)  -- (5cm, 2.75cm);
\stopMPcode\egroup
%%%%%%%%%%%%%%%%%%%%%%%%%%%%%%%%%%%%%%%%%%%%%%%%%%%%%%%%%%%%%%%

As can be seen in figure \in[fig:compiler], the compiler/interpreter 
consists of three main parts, a parser, a (correctness) verifier, and an 
interpreter, each with additional sub-structure. 

\startitemize[1]

\item \bold{Parser}

%%%%%%%%%%%%%%%%%%%%%%%%%%%%%%%%%%%%%%%%%%%%%%%%%%%%%%%%%%%%%%%%
\placefigure[here][fig:parser]
{A high-level structure of the \joylol\ parser} 
\bgroup\startMPcode
input hatching.mp;
picture dots; dots := dashpattern(on 0.1mm off 0.5mm);

path parser;
parser := (4.5cm, 15.2cm)  -- (13.5cm, 15.2cm) --
          (13.5cm, 11.8cm) -- (4.5cm, 11.8cm)  -- cycle;
draw image (
  hatchfill parser
    withcolor (45,2mm,-.5bp);
) dashed dots withcolor darkgray;
draw parser
  withpen pencircle scaled 1mm
  withcolor lightgray;
label.llft("Parser", (13cm, 15cm));

draw unitsquare xscaled 3cm yscaled 1cm shifted (5cm, 16cm);
label("Code Text", (6.5cm,16.5cm));

drawarrow (6.5cm, 16cm) -- (6.5cm, 15cm);
label.rt("Character stream", (6.5cm, 15.5cm));

draw unitsquare xscaled 3cm yscaled 1cm shifted (5cm, 14cm);
label("Lexer", (6.5cm,14.5cm));

drawarrow (6.5cm, 14cm) -- (6.5cm, 13cm);
label.rt("Lexeme stream", (6.5cm, 13.5cm));

draw unitsquare xscaled 3cm yscaled 1cm shifted (5cm, 12cm);
label("Parser", (6.5cm,12.5cm));

drawarrow (6.5cm, 12cm) -- (6.5cm, 11cm);
label.rt("Abstract Syntax Tree (AST)", (6.5cm, 11.5cm));

draw unitsquare xscaled 3cm yscaled 1cm shifted (5cm, 10cm);
label("Verifier", (6.5cm,10.5cm));

draw unitsquare xscaled 3cm yscaled 1cm shifted (1cm, 13cm);
label("Symbol Table", (2.5cm,13.5cm));

drawdblarrow (4cm, 13.75cm) -- (5cm, 14.5cm);
drawdblarrow (4cm, 13.25cm) -- (5cm, 12.5cm);

drawdblarrow (2.5cm, 13cm)  -- (5cm, 10.5cm);
\stopMPcode\egroup
%%%%%%%%%%%%%%%%%%%%%%%%%%%%%%%%%%%%%%%%%%%%%%%%%%%%%%%%%%%%%%%%

As can be seen in figure \in[fig:parser], the parser is composed of a 
lexer, parser pair. Both the lexer and parser will use a push down 
automata composed by a \joylol\ Parsing Expression Grammar, jPeg. Using a 
push down automata means that, technically, the lexer is powerful enough 
to directly parse \joylol\ code. We use a standard lexer, parser pair to 
allow the lexer and parser's grammars to be kept simpler, both to 
understand as well as to be more performant. 

\startitemize[2]

\item \bold{Lexer}

The lexer extracts lexemes consisting of operators, identifiers, strings, 
comments, and numbers, from the character stream. Each lexeme is either 
categorized by or inserted into the symbol table. Any categorizations will 
be used by the parser. 

\item \bold{Parser}

The parser builds an abstract syntax tree (AST) using the categorized 
lexemes extracted by the lexer. 

\stopitemize

\item \bold{Verifier}

%%%%%%%%%%%%%%%%%%%%%%%%%%%%%%%%%%%%%%%%%%%%%%%%%%%%%%%%%%%%%%%%
\placefigure[here][fig:verifier]
{A high-level structure of the \joylol\ verifier} 
\bgroup\startMPcode
input hatching.mp;
picture dots; dots := dashpattern(on 0.1mm off 0.5mm);

path verifier;
verifier := (4.5cm, 11.2cm) -- (13.5cm, 11.2cm) --
            (13.5cm, 7.8cm) -- (4.5cm, 7.8cm)   -- cycle;
draw image (
  hatchfill verifier
    withcolor (45,2mm,-.5bp);
) dashed dots withcolor darkgray;
draw verifier
  withpen pencircle scaled 1mm
  withcolor lightgray;
label.llft("Verifier", (13cm, 11cm));

draw unitsquare xscaled 3cm yscaled 1cm shifted (5cm, 12cm);
label("Parser", (6.5cm,12.5cm));

drawarrow (6.5cm, 12cm) -- (6.5cm, 11cm);
label.rt("Abstract Syntax Tree (AST)", (6.5cm, 11.5cm));

draw unitsquare xscaled 3cm yscaled 1cm shifted (5cm, 10cm);
label("Type Inference", (6.5cm,10.5cm));

drawarrow (6.5cm, 10cm) -- (6.5cm, 9cm);
label.rt("Abstract Syntax Tree (AST)", (6.5cm, 9.5cm));

draw unitsquare xscaled 3cm yscaled 1cm shifted (5cm, 8cm);
label("Correctness", (6.5cm,8.75cm));
label("Verifier", (6.5cm,8.25cm));

drawarrow (6.5cm, 8cm)  -- (6.5cm, 7cm);
label.rt("Abstract Syntax Tree (AST)", (6.5cm, 7.5cm));

draw unitsquare xscaled 3cm yscaled 1cm shifted (5cm, 6cm);
label("Interpreter", (6.5cm,6.5cm));

draw unitsquare xscaled 3cm yscaled 1cm shifted (1cm, 9cm);
label("Symbol Table", (2.5cm,9.5cm));

drawdblarrow (2.5cm, 10cm) -- (5cm, 12.5cm);

drawdblarrow (4cm, 9.75cm) -- (5cm, 10.5cm);
drawdblarrow (4cm, 9.25cm) -- (5cm, 8.5cm);

drawdblarrow (2.5cm, 9cm)  -- (5cm, 6.5cm);
\stopMPcode\egroup
%%%%%%%%%%%%%%%%%%%%%%%%%%%%%%%%%%%%%%%%%%%%%%%%%%%%%%%%%%%%%%%%

As can be seen in figure \in[fig:verifier], the verifier consists of a 
type inferencer and a correctness verifier. 

\startitemize[2]

\item \bold{Type inferencer}

The type inferencer ensures that all data on the stack as well as all 
variables in the register machine code has a definite type. 

\item \bold{Correctness verifier}

The correctness verifier checks all implicit and explicit post conditions 
imply the corresponding pre conditions conditions in both the \joylol\ and 
register machine code. The invariants associated with each data type 
contribute implicit pre and post conditions for each \joylol\ statement. 

\stopitemize

\item \bold{Interpreter}

%%%%%%%%%%%%%%%%%%%%%%%%%%%%%%%%%%%%%%%%%%%%%%%%%%%%%%%%%%%%%%%%
\placefigure[here][fig:interpreter]
{A high-level structure of the \joylol\ interpreter} 
\bgroup\startMPcode
input hatching.mp;
picture dots; dots := dashpattern(on 0.1mm off 0.5mm);

path interpreter;
interpreter := (4.5cm, 7.2cm)  -- (13.5cm, 7.2cm) --
               (13.5cm, 1.8cm) -- (4.5cm, 1.8cm)  -- cycle;
draw image (
  hatchfill interpreter
    withcolor (45,2mm,-.5bp);
) dashed dots withcolor darkgray;
draw interpreter
  withpen pencircle scaled 1mm
  withcolor lightgray;
label.llft("Interpreter", (13cm, 7cm));

draw unitsquare xscaled 3cm yscaled 1cm shifted (5cm, 8cm);
label("Verifier", (6.5cm,8.5cm));

drawarrow (6.5cm, 8cm)  -- (6.5cm, 7cm);
label.rt("Abstract Syntax Tree (AST)", (6.5cm, 7.5cm));

draw unitsquare xscaled 3cm yscaled 1cm shifted (5cm, 6cm);
label("Code Generator", (6.5cm,6.5cm));

drawarrow (7cm, 6cm)    -- (10.5cm, 5cm);
label.urt("(Restricted) ANSI-C code", (8.75cm, 5.5cm));

draw unitsquare xscaled 3cm yscaled 1cm shifted (9cm, 4cm);
label("ANSI-C compiler", (10.5cm,4.5cm));

drawarrow (10.5cm, 4cm) -- (7cm, 3cm);
label.lrt("Dynamically linked library", (8.75cm,3.5cm));

drawarrow (6.5cm, 6cm)  -- (6.5cm, 3cm);
label.rt("Process stack", (6.5cm, 4.5cm));

draw unitsquare xscaled 3cm yscaled 1cm shifted (5cm, 2cm);
label("Interpreter", (6.5cm,2.5cm));

draw unitsquare xscaled 3cm yscaled 1cm shifted (1cm, 0.5cm);
label("Input", (2.5cm,1cm));
drawarrow (4cm, 1.25cm)  -- (5cm, 2.25cm);

draw unitsquare xscaled 3cm yscaled 1cm shifted (9cm, 0.5cm);
label("Output", (10.5cm,1cm));
drawarrow (8cm, 2.25cm)  -- (9cm, 1.25cm);

draw unitsquare xscaled 3cm yscaled 1cm shifted (1cm, 4cm);
label("Symbol Table", (2.5cm,4.5cm));

drawdblarrow (2.5cm, 5cm)  -- (5cm, 8.5cm);

drawdblarrow (3cm, 5cm) -- (5cm, 6.5cm);
drawdblarrow (3cm, 4cm) -- (5.5cm, 3cm);
\stopMPcode\egroup
%%%%%%%%%%%%%%%%%%%%%%%%%%%%%%%%%%%%%%%%%%%%%%%%%%%%%%%%%%%%%%%%

As can be seen in figure \in[fig:interpreter], the interpreter consists of 
the code generator, ANSI-C compiler and the \joylol\ interpreter. 

\startitemize[2]

\item \bold{Code generator}

The code generator determines whether any particular code fragment will be 
interpreted as a list of \joylol\ words, or as ANSI-C compiled code. If it 
decides that a code fragment can be compiled as ANSI-C code, the code 
generator assigns variables and optimizes the control flow graph to use 
ANSI-C if-then-else or for statements as appropriate. 

Code which will be interpreted as a list of \joylol\ words are placed 
directly on the process stack using the entries in the symbol table. This 
ensures that the interpreter does not usually need to query the symbol 
table as it is interpreting. 

\item \bold{ANSI-C compiler}

The ANSI-C compiler compiles a restricted form of ANSI-C code generated by 
the code generator. These restricted forms ensure that only (relatively) 
well understood ANSI-C code is compiled, in order to maintain the 
correctness of the over all code. We ensure that \quote{risky} C code is 
not used in implementing \joylol. 

While there are a large number of ANSI-C compilers that could be used in 
any project, the typical compilers will be Gnu's gcc and LLVM's clang. If 
LLVM's clang is used, it is potentially possible to dynamically compile 
and load any resulting code. However, the whole system is designed to use 
a completely external compiler in a multi-phased compilation. 

\item\bold{Interpreter}

The interpreter calls the individual \joylol\ words taken from the process 
stack. Each call may alter either the data or process (or both) stacks as 
the result of its action. 

\stopitemize

\stopitemize

\section[title=Challenges]

\subsection[title=Sublanguages]

Since our \joylol\ language is actually best structured as a pair of 
sublanguages, we need a way of \quote{switching} between these 
sublanguages \emph{inside} a given \joylol\ text. Our overall strategy is 
to construct the compiler as a sequence of co-routines, which each 
consume, transform and the generate \quote{objects} along the sequence. To 
deal with multiple sublanguages we need the ability to deal with a 
\quote{sequence} of co-routines which does not have a \emph{linear} 
topology. That is for some sections of the co-routine structure there will 
be multiple paths. To deal with this we need multiplexer co-routine which 
listens for a small collection of tokens and then chooses which 
sub-sequence along which to send all subsequence tokens. Conversely we 
need a reverse-multiplexer to fold tokens from multiple streams back into 
one stream. The reverse-multiplexer will \quote{listen} on all of its 
\quote{input} streams waiting for the next token\footnote{This 
architecture seems to assume that we have an inherently sequential, 
non-parallel, computational system with only one co-routine 
\quote{running} at any one time.}. 

A subsequent question now becomes, where do we locate the multiplexer and 
reverse-multiplexer? For our purposes, locating the multiplexer depends 
upon whether or not the sublanguages (as well as the multiplexer) can 
share the \quote{lexer}\footnote{Note that, since the lexer will take care 
of rather large structures such as quoted strings and comments, sharing 
lexers \emph{means} that the sublanguages \emph{must} also share the 
syntax of these \quote{large} structures.}. If they can all share the same 
lexer then the multiplexer should be located between the single lexer and 
the multiple parsers\footnote{Alternatively the composite parser includes 
the multiplexer and sub-language parsers.}. 

\subsection[title=Co-routines]

Since co-routines will be an important way to structure complex 
mathematical computational proofs, we need to ensure switching between 
co-routines can be compiled into performant (ANSI-C) idioms. 

\section[title=Syntax/Semantics]

\subsection[title=The JoyLoL Configuration Language]

\bold{Syntax} 

\starttyping
configuration = systemConfig 
  defaults^0 
  clusters^0 
  externals^0 
  generation
  end ("--" "system" systemName)^0
systemConfig = "system" systemName
systemName   = name

defaults = "default" optionClause (";" optionClause)^0
options  = "option" optionClause (";" optionClause)^0
optionClause = optionTag optionMark^0 targetList^0
optionTag    = classTag + systemTag + freeTag
classTag     = "assertion" + "debug" + "optimize" + "trace"
systemTag    = "collect"
freeTag      = name
optionMark   = "(" optionValue")"
optionValue  = standardValue + classValue + freeValue
standardValue = "yes" + "no" + "all"
classValue    = "require" + "ensure" + "invariant" + "loop" + "check"
freeValue     = fileName + directoryName + name

clusters = "cluster" clusterClause (";" clusterClause)^0
clusterClause = clusterTag^0 directoryName clusterProperties^0
clusterTag    = clusterName ":"
clusterName   = name
clusterProperties = use^0
  include^0
  exclude^0
  nameAdaptation^0
  defaults^0
  options^0
  visibility^0
use = "use" fileName
include = "include" fileList
exclude = "exclude" fileList
fileList = fileName (";" fileName)^0
fileName = name

nameAdaptation = "adapt" clusterAdaptationList
clusterAdaptationList = clusterAdaptation (";" clusterAdaptation)^0
clusterAdaptation = clusterIgnore + clusterRenameClause
clusterIgnore = clusterName ":" "ignore"
clusterRenameClause = clusterName ":" "rename" classRenameList
classRenameList = classRenamePair ("," classRenamePair)^0
classRenamePair = className "as" className
className = name

externals = "external" languageContribution (";" languageContributions)^0
languageContribution = language ":" fileList
language = "JoyLoL" +
  "C" + "C++" +
  "Executable" + "SharedLibrary" + 
  "Object" +
  "Make" +
  "ConTeXt" +
  name

generation = "generate" languageGeneration (";" languageGeneration)^0
languageGeneration = language generateOption^0 ":" target
generateOption = "(" generateOptionValue ")"
generateOptionValue = "yes" + "no"
target = directoryPath + fileName




  
\stoptyping

\subsection[title=The JoyLoL Language]

