% A ConTeXt document [master document: joyLoLMinus.tex ]

\section[title=Objects]

\setCHeaderStream{public}
\startCHeader

#define JObjTag         0
#define PairTag         1
#define ContextTag      2
#define CFunctionTag    3
#define CtxCFunctionTag 4
#define LAST_JOBJ_TAG   5

#define JObjName          "JObj"
#define PairName          "Pair"
#define ContextName       "Context"
#define CFunctionName     "CFunction"
#define CtxCFunctionName  "CtxCFunction"

typedef struct jObjectTagName_struct {
  size_t tag;
  const char *name;
} JObjTagName;

extern JObjTagName jObjTagNameMap[];
\stopCHeader

\startCHeader
typedef struct jObject_struct {
  size_t tag;
} JObj;

JObj *newJObject(void);
void printJObj(JObj *anObj, FILE *aFile, const char *indent);
#define asJObj(anObj)  ((JObj*)(anObj))
#define getJObjTag(anObj)  (((JObj*)(anObj))->tag)
#define getJObjName(anObj) \
  ((anObj) ? (jObjTagNameMap[getJObjTag(anObj)].name) : "null")
\stopCHeader

\startCCode
JObj *newJObject(void) {
  JObj *newObj = (JObj*)calloc(1, sizeof(JObj));
  newObj->tag = JObjTag;
  return newObj;
}

JObjTagName jObjTagNameMap[] = {
  {JObjTag,         JObjName},
  {PairTag,         PairName},
  {ContextTag,      ContextName},
  {CFunctionTag,    CFunctionName},
  {CtxCFunctionTag, CtxCFunctionName},
  {0,               NULL}
};

void printJObj(JObj *anObj, FILE *aFile, const char *indent) {
  if (!anObj) {
    fprintf(aFile, "%s<null>: null\n", indent);
    return;
  }
  
  assert(anObj->tag < LAST_JOBJ_TAG);
  
  fprintf(
    aFile, 
    "%s<%s>: %p\n",
    indent,
    getJObjName(anObj),
    anObj
  );
}
\stopCCode

\startCHeader
typedef struct pair_struct {
  size_t tag;
  JObj  *car;
  JObj  *cdr;
} Pair;

Pair *newPair(JObj* car, JObj *cdr);
#define asPair(anObj) ((Pair*)(anObj))
\stopCHeader

\startCCode
Pair *newPair(JObj *car, JObj *cdr) {
  Pair *newObj = (Pair*)calloc(1, sizeof(Pair));
  newObj->tag = PairTag;
  newObj->car = car;
  newObj->cdr = cdr;
  return newObj;
}
\stopCCode

\startCHeader
typedef struct context_struct {
  size_t tag;
  JObj  *data;
  JObj  *process;
} Context;

Context *newContext(JObj* data, JObj *process);
#define asContext(anObj) ((Context*)(anObj))
\stopCHeader

\startCCode
Context *newContext(JObj *data, JObj *process) {
  Context *newObj = (Context*)calloc(1, sizeof(Context));
  newObj->tag     = ContextTag;
  newObj->data    = data;
  newObj->process = process;
  return newObj;
}
\stopCCode

\startCHeader
typedef void (CFuncImp)(Context*);

typedef struct cFunction_struct {
  size_t    tag;
  CFuncImp *func;
} CFunction;

CFunction *newCFunction(CFuncImp* func);
#define asCFunction(anObj) ((CFunction*)(anObj))
\stopCHeader

\startCCode
CFunction *newCFunction(CFuncImp *func) {
  CFunction *newObj = (CFunction*)calloc(1, sizeof(CFunction));
  newObj->tag  = CFunctionTag;
  newObj->func = func;
  return newObj;
}
\stopCCode

\startCHeader
typedef Context* (CtxCFuncImp)(Context*);

typedef struct ctxCFunction_struct {
  size_t       tag;
  CtxCFuncImp *func;
} CtxCFunction;

CtxCFunction *newCtxCFunction(CtxCFuncImp* func);
#define asCtxCFunction(anObj) ((CtxCFunction*)(anObj))
\stopCHeader

\startCCode
CtxCFunction *newCtxCFunction(CtxCFuncImp *func) {
  CtxCFunction *newObj = (CtxCFunction*)calloc(1, sizeof(CtxCFunction));
  newObj->tag  = CtxCFunctionTag;
  newObj->func = func;
  return newObj;
}
\stopCCode
