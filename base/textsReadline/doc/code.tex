% A ConTeXt document [master document: textsReadline.tex]

\section[title=Code]
\setCHeaderStream{public}

\dependsOn[jInterps]
\dependsOn[texts]

\startCCode
// Texts are a collection of characters, which are used by the Parser
// to extract successive symbols.
//
// Texts are created on one of three backing suppliers of characters:
// 1. a single string
// 2. a NULL terminated array of strings
// 3. an external file
// 4. a readline interaction with a user
//
// In all four cases, the Parser's nextSymbol method requests successive
// **lines** of characters (deliminated by new-line-characters).
//
// It is critical, for correct interaction with the user via readline,
// that the initial line is NOT obtained until actually requested by
// the parser's nextSymbol method.
//
// It is also critical that once completed, none of the sources, get
// asked for subsequent lines.
//
// When the text has been completed, the nextLine function ensures
// that aText->curLine is NULL.
\stopCCode

\startCCode
static void clearReadlinePrompt(TextObj* aText) {
  aText->curPrompt = aText->newPrompt;
}

static void setReadlinePrompts(TextObj* aText,
                        const char* newPrompt,
                        const char* continuePrompt
) {
  if (newPrompt) aText->newPrompt = newPrompt;
  else           aText->newPrompt = ">";

  if (continuePrompt) aText->continuePrompt = continuePrompt;
  else                aText->continuePrompt = ":";

  clearReadlinePrompt(aText);
}
\stopCCode


\startCHeader
extern void saveReadlineHistory(TextObj* aText);
\stopCHeader

\startCCode
void saveReadlineHistory(TextObj* aText) {
  write_history(".joyLoL-history");
}
\stopCCode

\startCCode
static void nextLineFromReadline(TextObj* aText) {
  assert(aText);
  DEBUG(aText->debugFlag, "->nextLineFromReadline %s\n", "");
  if (!aText) return; // there is nothing we can do!

  if (aText->completed) {
    // we have reached the end of the interaction with the user
    aText->completed = TRUE;
    aText->curLine   = NULL;
    aText->curChar   = NULL;
    aText->lastChar  = NULL;
    return;
  }

  // readline returns alloc'ed memory so we free it here.
  if (aText->curLine) free((void*)aText->curLine);

  aText->curLine  = NULL;
  aText->curChar  = NULL;
  aText->lastChar = NULL;

  aText->curLine = readline (aText->curPrompt);

  if (!aText->curLine) {
    aText->completed = TRUE;
    return;
  }

  if (*aText->curLine) {
    history_set_pos(0); // start at the beginning
    if (0 <= history_search (aText->curLine, -1)) { // search backwards
      remove_history(where_history());
    } else if (0 <= history_search (aText->curLine, 1)) { // search forwards
      remove_history(where_history());
    }
    history_set_pos(0);
    add_history (aText->curLine);
  }

  aText->curChar  = aText->curLine;
  aText->lastChar = aText->curChar + strlen(aText->curChar);

  aText->curPrompt = aText->continuePrompt;
}
\stopCCode

\startCCode
static TextObj* currentReadLineText = NULL;
\stopCCode

\startCCode
static char* dictionaryWalker(const char* text, int state) {
  if (!currentReadLineText) return NULL;
  if (!state) {

    assert(currentReadLineText->jInterp);

    currentReadLineText->curNode =
      findLUBSymbol(currentReadLineText->jInterp, text);
    DEBUG(FALSE, "dictionaryWalker-start %p\n",
          currentReadLineText->curNode);
  }
  DictObj* curNode = currentReadLineText->curNode;

  if (!curNode) return NULL;

  if (strncmp(curNode->symbol, text, strlen(text)) == 0) {
    DEBUG(FALSE, "dictionaryWalker %p {%s}[%s]\n",
          curNode, text, curNode->symbol);
    currentReadLineText->curNode = curNode->next;
    return strdup(curNode->symbol);
  }

  currentReadLineText->curNode = NULL;
  return NULL;
}

static char** dictionaryCompletion(const char* text, int start, int end) {
 return rl_completion_matches(text, dictionaryWalker);
}
\stopCCode

\startCCode
TextObj* createTextFromReadline(JoyLoLInterp *jInterp) {
  assert(jInterp);
  assert(readlineTag);
  
  TextObj* aText = (TextObj*)newObject(jInterp, readlineTag);
  assert(aText);
  //
  // readline specific initializations
  //
  using_history();
  read_history(".joyLoL-history");
  rl_readline_name = "joyLoL";
  rl_attempted_completion_function = dictionaryCompletion;
  setReadlinePrompts(aText, NULL, NULL);
  aText->curNode = NULL;
  aText->curCompletionText = NULL;
  aText->curCompletionLen  = 0;
  aText->nextLine = nextLineFromReadline;
  //
  // general initialization
  //

  aText->jInterp    = jInterp;
  aText->completed  = FALSE;
  aText->sym        = NULL;
  aText->curLine    = NULL;
  aText->curChar    = NULL;
  aText->lastChar   = NULL;
  //
  aText->inputFile = NULL;
  //
  aText->textLines  = NULL;
  aText->curLineNum = 0;

  assert(!currentReadLineText); // there should not be more than one
  currentReadLineText = aText;

  return aText;
}
\stopCCode

\startTestSuite[registerTextsReadline]

\startCHeader
extern Boolean registerTextsReadline(JoyLoLInterp *jInterp);
extern size_t readlineTag;
\stopCHeader

\startCCode
size_t   readlineTag = 0;

size_t registerTextsReadline(JoyLoLInterp *jInterp) {
  if (!readlineTag) {
    // WE CAN ONLY register this coAlg ONCE
    // (since readline itself can only be loaded once)
    //
    CoAlgebra* theCoAlg    = (CoAlgebra*) calloc(1, sizeof(CoAlgebra));
    theCoAlg->name         = "TextsReadline";
    theCoAlg->objectSize   = sizeof(TextObj);
    theCoAlg->registerFunc = registerTextsReadline;
    theCoAlg->equalityFunc = NULL;
    theCoAlg->printStr     = NULL;
    readlineTag = registerCoAlgebra(jInterp, theCoAlg);
  }
  
  // do a sanity check...
  assert(readlineTag != UnusedTag);
  assert(jInterp->coAlgs[readlineTag].sClass);
  
  registerTextReadlineWords(jInterp);

  return TRUE;
}
\stopCCode

\startTestCase[should register the Texts coAlg]

\startCTest
  // CTestsSetup has already created a jInterp
  // and run registerTextsReadline
  
  AssertPtrNotNull(jInterp);
  AssertPtrNotNull(jInterp->coAlgs);
  AssertPtrNotNull(jInterp->coAlgs[readlineTag].sClass);
  CoAlgebra *coAlg = jInterp->coAlgs[readlineTag].sClass;
//  AssertIntTrue(registerTextsReadline(jInterp));
  AssertPtrNotNull(jInterp->coAlgs[readlineTag].sClass);
  AssertPtrEquals(jInterp->coAlgs[readlineTag].sClass, coAlg);
  AssertIntEquals(
    jInterp->coAlgs[readlineTag].sClass->objectSize,
    sizeof(TextObj)
  )
\stopCTest
\stopTestCase
\stopTestSuite
