% A ConTeXt document [master document: coAlgs.tex]

\section[title=Class Code]

\setCHeaderStream{public}
\setCCodeStream{class}
\setCTestStream{class}
\addCTestInclude{<coAlgs.h>}

In this section we concentrate on the C code required to implement the 
kernel JoyLoL CoAlgebra. The CoAlgebra, \type{CoAlg}, is the base of the 
JoyLoL interpreter. This CoAlgebra has two parts, a \quote{class} part, 
and an \quote{instance} part. This class part manages the loading, 
registration and listing of new \quote{external} CoAlgebras. The instance 
part, discussed in the next section, manages the creation and removal of 
CoAlgebra \quote{values} (more commonly know as \quote{object} instances). 

We begin by setting up some standard C header defines, the most important 
of which deal with memory allocation and alignment as well as a simple 
\type{DEBUG} system. 

\startCHeader
#include <stdlib.h>
#include <string.h>

#ifndef JOYLOL_SYSTEM_CONFIG_PATH
#define JOYLOL_SYSTEM_CONFIG_PATH "/usr/local/etc/joyLoL"
#endif

#ifndef JOYLOL_SYSTEM_PLUGINS_PATH
#define JOYLOL_SYSTEM_PLUGINS_PATH "/usr/local/lib/joyLoL"
#endif

#ifndef JOYLOL_COALGS_INCREMENT
#define JOYLOL_COALGS_INCREMENT 10
#endif
 
#ifndef NULL
#define NULL (void*)0
#endif

#ifndef TRUE
#define TRUE 1
#endif
#ifndef FALSE
#define FALSE 0
#endif

#define MEM_ALIGNMENT ((size_t)0x7)
#define IS_MEM_ALIGNED(someMem) (!( ((size_t)(someMem)) & (MEM_ALIGNMENT) ))

extern void* joyLoLCalloc0(size_t numItems, size_t sizeOfItem);
#define joyLoLCalloc(numItems, itemType) (itemType*)joyLoLCalloc0((numItems), sizeof(itemType))

#ifndef NDEBUG
#include <stdio.h>
#define DEBUG(debugFlag, format, ... ) \
  if (debugFlag) { printf( format, __VA_ARGS__ ); fflush(stdout); }
#else
#define DEBUG(...)
#endif
\stopCHeader

The ANSI-C implementation of a CoAlgebra is as a simple \type{struct} 
which contains the following items:

\startitemize

\item a \type{size_t} value which is unique for each registered CoAlgebra 
and hence acting as a test for identity, 

\item a \type{Symbol} value which provides a human readable \emph{name} 
for the CoAlgebra, 

\item a \type{C-function} which (recursively) tests for equality of two 
CoAlgebra instances of given type of CoAlgebra.

\stopitemize

\startCHeader
typedef struct coalgebra_struct         CoAlgebra;
typedef struct coalgebra_handle_struct  CoAlgHandle;
typedef const char Symbol;

typedef size_t (CoAlgEquality)(CoAlgebra*, CoAlgHandle*, CoAlgHandle*, size_t);
//typedef size_t (CoAlgPrintSize)(CoAlgHandle*, size_t);
//typedef size_t (CoAlgPrintStr)(CoAlgHandle*, char*, size_t);

typedef struct coalgebra_struct {
  size_t          isA;
  Symbol*         name;
  CoAlgEquality*  equality;
//  CoAlgPrintSize* printSize;
//  CoAlgPrintStr*  printStr;
} CoAlgebra;

typedef struct coalgebras_struct {
  size_t numCoAlgs;
  size_t maxNumCoAlgs;
  CoAlgebra* theCoAlgs;
} CoAlgebras;
\stopCHeader

\startTestSuite[registerCoAlgebra]

\startCHeader
extern CoAlgebras *registerCoAlgebra(
  CoAlgebras *theCoAlgs,
  Symbol *name,
  CoAlgEquality *equalityFunc
);
\stopCHeader

\startCCode
CoAlgebras *registerCoAlgebra(
  CoAlgebras *coAlgs,
  Symbol *name,
  CoAlgEquality *equalityFunc
) {
  // we follow a policy of lazy management of the coAlgs
  // if coAlgs is NULL, we create a new one
  // if coAlgs is too small we expand it
  if (!coAlgs) {
    // we need to create a new coAlgs structure...
    //
    coAlgs = (CoAlgebras*) calloc(1, sizeof(CoAlgebras));
    assert(coAlgs);
    coAlgs->numCoAlgs    = 0;
    coAlgs->maxNumCoAlgs = 0;
    coAlgs->theCoAlgs    = NULL;
  }
  if (coAlgs->maxNumCoAlgs <= coAlgs->numCoAlgs) {
    // we need to expand the existing theCoAlgs vector...
    //
    CoAlgebra *oldCoAlgs = coAlgs->theCoAlgs;
    size_t oldCoAlgsSize = coAlgs->maxNumCoAlgs * sizeof(CoAlgebra);
    coAlgs->maxNumCoAlgs += JOYLOL_COALGS_INCREMENT;
    coAlgs->theCoAlgs = calloc(coAlgs->maxNumCoAlgs, sizeof(CoAlgebra));
    if (oldCoAlgs) {
      memcpy(coAlgs->theCoAlgs, oldCoAlgs, oldCoAlgsSize);
      free(oldCoAlgs);
    }
  }
  CoAlgebra *theCoAlgs = coAlgs->theCoAlgs;
  size_t newCoAlg = coAlgs->numCoAlgs;
  theCoAlgs[newCoAlg].isA       = newCoAlg;
  theCoAlgs[newCoAlg].name      = strdup(name);
  theCoAlgs[newCoAlg].equality  = equalityFunc;
  //  theCoAlgs[newCoAlg].printSize = NULL;
  //  theCoAlgs[newCoAlg].printStr  = NULL;
  coAlgs->numCoAlgs++;
  return coAlgs;
}
\stopCCode

\startTestCase[should create a new coAlgs]

\startCTest
  CoAlgebras *someCoAlgs = registerCoAlgebra(NULL, "newCoAlg", NULL);
  AssertPtrNotNull(someCoAlgs);
\stopCTest
\stopTestCase
\stopTestSuite

\subsection[title=createCoAlgebras]

\startCHeader
extern CoAlgebras* createCoAlgebras(void);
\stopCHeader
\startCCode
CoAlgebras* createCoAlgebras(void) {
//  CoAlgebras* coAlgs = (CoAlgebras*) calloc(1, sizeof(CoAlgebras));
  CoAlgebras* coAlgs = NULL;
//  assert(coAlgs);


  // create all the known coAlgebras
//  coAlgs->assertions    = createAssertionsCoAlgebra();
//  coAlgs->booleans      = createBooleansCoAlgebra();
//  coAlgs->contexts      = createContextsCoAlgebra();
//  coAlgs->descriptions  = createDescriptionsCoAlgebra();
//  coAlgs->functions     = createFunctionsCoAlgebra();
//  coAlgs->naturals      = createNaturalsCoAlgebra();
//  coAlgs->observations  = createObservationsCoAlgebra();
//  coAlgs->pairs         = createPairsCoAlgebra();
//  coAlgs->proofs        = createProofsCoAlgebra();
//  coAlgs->symbols       = createSymbolsCoAlgebra();

  // initialize all the known coAlgebras
//  initPairsCoAlgebra(coAlgs); // need to initialize this first for listMemory

//  initAssertionsCoAlgebra(coAlgs);
//  initBooleansCoAlgebra(coAlgs);
//  initContextsCoAlgebra(coAlgs);
//  initDescriptionsCoAlgebra(coAlgs);
//  initFunctionsCoAlgebra(coAlgs);
//  initNaturalsCoAlgebra(coAlgs);
//  initObservationsCoAlgebra(coAlgs);
//  initProofsCoAlgebra(coAlgs);
//  initSymbolsCoAlgebra(coAlgs);

  return coAlgs;
}
\stopCCode

\startCHeader
//#include "joyLoL/coAlg/assertions.h"
//#include "joyLoL/coAlg/booleans.h"
//#include "joyLoL/coAlg/contexts.h"
//#include "joyLoL/coAlg/descriptions.h"
//#include "joyLoL/coAlg/functions.h"
//#include "joyLoL/coAlg/naturals.h"
//#include "joyLoL/coAlg/observations.h"
//#include "joyLoL/coAlg/pairs.h"
//#include "joyLoL/coAlg/proofs.h"
//#include "joyLoL/coAlg/symbols.h"

#define ASSERTION_COALG    1
#define BOOLEAN_COALG      2
#define CONTEXT_COALG      3
#define DESCRIPTION_COALG  4
#define FUNCTION_COALG     5
#define NATURAL_COALG      6
#define OBSERVATION_COALG  7
#define PAIR_COALG         8
#define PROOF_COALG        9
#define SYMBOL_COALG       10
\stopCHeader

\subsection[title=Old tests]

\starttyping
#include <stdio.h>
#include <stdlib.h>
#include <string.h>

#include "CuTest.h"

#include "joyLoL/macros.h"
#include "joyLoL/coAlg/coAlgs.h"
#include "joyLoL/lists.h"
#include "joyLoL/dictionary.h"
#include "joyLoL/dictionary_private.h"
#include "joyLoL/text.h"
#include "joyLoL/parser.h"
#include "joyLoL/printer.h"
#include "joyLoL/eval.h"
#include "joyLoL/eval_private.h"

// suiteName: - Base CoAlgebra tests -

void Test_reportSizes(CuTest* tc) {
  printf("\nStructure Sizes\n");
  printf("              void* = %zu bytes\n", sizeof(void*));
  printf("                int = %zu bytes\n", sizeof(int));
  printf("           long int = %zu bytes\n", sizeof(long int));
  printf("\n");

  PairAtom aPairAtom;
  printf("           PairAtom = %zu bytes\n", sizeof(PairAtom));
  printf("     PairAtom.coAlg = %zu bytes\n", sizeof(aPairAtom.coAlg));
  printf("       PairAtom.tag = %zu bytes\n", sizeof(aPairAtom.tag));
  printf("      PairAtom.pair = %zu bytes\n", sizeof(aPairAtom.pair));
  printf("  PairAtom.pair.car = %zu bytes\n", sizeof(aPairAtom.pair.car));
  printf("  PairAtom.pair.cdr = %zu bytes\n", sizeof(aPairAtom.pair.cdr));
  printf("   PairAtom.boolean = %zu bytes\n", sizeof(aPairAtom.boolean));
  printf("    PairAtom.symbol = %zu bytes\n", sizeof(aPairAtom.symbol));
  printf("      PairAtom.func = %zu bytes\n", sizeof(aPairAtom.func));
  printf("   PairAtom.natural = %zu bytes\n", sizeof(aPairAtom.natural));
  printf("\n");

  ListBlock aListBlock;
  printf("          ListBlock = %zu bytes\n", sizeof(ListBlock));
  printf("    ListBlock.block = %zu bytes\n", sizeof(aListBlock.block));
  printf("ListBlock.nextBlock = %zu bytes\n", sizeof(aListBlock.nextBlock));
  printf("\n");
}

void Test_createBaseCoAlgebras(CuTest* tc) {
  CoAlgebras *coAlgs = createCoAlgebras();
  CuAssertPtrNotNull(tc, coAlgs);
  CuAssertPtrNotNull(tc, coAlgs->booleans);
  CuAssertPtrNotNull(tc, coAlgs->contexts);
  CuAssertPtrNotNull(tc, coAlgs->functions);
  CuAssertPtrNotNull(tc, coAlgs->naturals);
  CuAssertPtrNotNull(tc, coAlgs->pairs);
  CuAssertPtrNotNull(tc, coAlgs->symbols);
}
\stoptyping

%\startsyntax

%\stopsyntax 

%\startinitialization

%\stopinitialization
