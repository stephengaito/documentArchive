% A ConTeXt document [master document: symbols.tex]

\section[title=Code]
\setCHeaderStream{public}

\dependsOn[jInterps]
%\dependsOn[context]

\startCHeader
typedef struct symbol_object_struct {
  JObj super;
  Symbol *sym;
  Symbol *file;
  size_t  line;
} SymbolObj;

#define LOOKUP_SYMBOL_FLAG 0x8L

#define asSymbol(aObj) (((SymbolObj*)(aObj))->sym)
#define asFile(aObj)   (((SymbolObj*)(aObj))->file)
#define asLine(aObj)   (((SymbolObj*)(aObj))->line)
\stopCHeader

\startTestSuite[newString]

\stopTestSuite

\startTestSuite[newSymbol]

\startCHeader
typedef JObj *(NewSymbol)(
  JoyLoLInterp *jInterp,
  Symbol       *theSymbol,
  Symbol       *fileName,
  size_t        line,
  Boolean       isALookupSymbol
);

#define newSymbol(jInterp, aSymbol, fileName, line)             \
  (                                                             \
    assert(getSymbolsClass(jInterp)                             \
      ->newSymbolFunc),                                         \
    (getSymbolsClass(jInterp)                                   \
      ->newSymbolFunc(jInterp, aSymbol, fileName, line, FALSE)) \
  )
\stopCHeader

\setCHeaderStream{private}
\startCHeader
extern JObj *newSymbolImpl(
  JoyLoLInterp *jInterp,
  Symbol       *aSymbol,
  Symbol       *fileName,
  size_t        line,
  Boolean       isALookupSymbol
);
\stopCHeader
\setCHeaderStream{public}

\startCCode
JObj* newSymbolImpl(
  JoyLoLInterp *jInterp,
  Symbol       *aSymbol,
  Symbol       *fileName,
  size_t        line,
  Boolean       isALookupSymbol
) {
  assert(aSymbol);
  assert(jInterp);
  if (!fileName) fileName = "unknown(symbol)";
  JObj* result = newObject(jInterp, SymbolsTag);
  assert(result);
  asSymbol(result) = strdup(aSymbol);
  asFile(result)   = strdup(fileName);
  asLine(result)   = line;
  if (isALookupSymbol) {
    result->flags |= LOOKUP_SYMBOL_FLAG;
  }
  return result;
}
\stopCCode

\startTestCase[should create some new symbols]

\startCTest
  AssertPtrNotNull(jInterp);
  AssertPtrNotNull(jInterp->coAlgs);

  const char* testStr = "test string";
  JObj* aNewSymbol = newSymbol(jInterp, testStr, "testStr", 11);
  AssertPtrNotNull(aNewSymbol);
  AssertPtrNotNull(asType(aNewSymbol));
  AssertIntEquals(asTag(aNewSymbol), SymbolsTag);
  AssertPtrNotNull(asSymbol(aNewSymbol));
  AssertPtrNotEquals(asSymbol(aNewSymbol), testStr);
  AssertIntEquals(strcmp(asSymbol(aNewSymbol), testStr), 0);
  AssertStrEquals(asFile(aNewSymbol), "testStr");
  AssertIntEquals(asLine(aNewSymbol), 11);
  AssertIntTrue(isSymbol(aNewSymbol));
  AssertIntTrue(isAtom(aNewSymbol));
  AssertIntFalse(isPair(aNewSymbol));
  
  aNewSymbol = newSymbol(jInterp, testStr, NULL, 0);
  AssertPtrNotNull(aNewSymbol);
  AssertStrEquals(asFile(aNewSymbol), "unknown(symbol)");
  
\stopCTest
\stopTestCase
\stopTestSuite

\startTestSuite[isSymbol and symbolIs]

\startCHeader
#define isSymbol(aLoL)            \
  (                               \
    (                             \
      (aLoL) &&                   \
      asType(aLoL) &&             \
      (asTag(aLoL) == SymbolsTag) \
    ) ?                           \
      TRUE :                      \
      FALSE                       \
  )

#define isLookupSymbol(aLoL)                  \
  (                                           \
    (                                         \
      (aLoL) &&                               \
      asType(aLoL) &&                         \
      (asTag(aLoL) == SymbolsTag)             \
      isFlagSet(aLoL, LOOKUP_SYMBOL_FLAG) &&  \
    ) ?                                       \
      TRUE :                                  \
      FALSE                                   \
  )\stopCHeader

\startCHeader
typedef Boolean (SymbolIs)(
  JObj *aLoL,
  Symbol   *aSymbol
);

#define symbolIs(jInterp, aLoL, aSymbol)  \
  (                                       \
    assert(getSymbolsClass(jInterp)       \
      ->symbolIsFunc),                    \
    (getSymbolsClass(jInterp)             \
      ->symbolIsFunc(aLoL, aSymbol))      \
  )
\stopCHeader

\setCHeaderStream{private}
\startCHeader
extern Boolean symbolIsImpl(
  JObj *aLoL, 
  Symbol *aSymbol
);
\stopCHeader
\setCHeaderStream{public}

\startCCode
Boolean symbolIsImpl(
  JObj *aLoL,
  Symbol *aSymbol
) {
  if (isSymbol(aLoL) &&
      (strcmp(asSymbol(aLoL), aSymbol) == 0)) {
    return TRUE;
  }
  return FALSE;
}
\stopCCode

\startTestCase[should return true if a symbol]

\startCTest
  JObj *aSym = newSymbol(jInterp, "this is a test", NULL, 0);
  AssertIntTrue(isSymbol(aSym));
  AssertIntTrue(symbolIs(jInterp, aSym, "this is a test"));
\stopCTest
\stopTestCase

\startTestCase[should return false if not a symbol]
\startCTest
  AssertIntFalse(isSymbol(NULL));
  AssertIntFalse(symbolIs(jInterp, NULL, "this is NOT a test"));
  JObj *aObj = newObject(jInterp, BooleansTag);
  AssertIntFalse(isSymbol(aObj));
  AssertIntFalse(symbolIs(jInterp, aObj, "this is NOT a test"));
  JObj *aSym = newSymbol(jInterp, "this is a test", NULL, 0);
  AssertIntFalse(symbolIs(jInterp, aSym, "this is NOT a test"));
\stopCTest
\stopTestCase
\stopTestSuite

\startTestSuite[symbol equality]

\setCHeaderStream{private}
\startCHeader
Boolean symbolsEqual(
  JoyLoLInterp *jInterp,
  JObj         *lolA,
  JObj         *lolB,
  size_t        timeToLive
);
\stopCHeader
\setCHeaderStream{public}

\startCCode
Boolean symbolsEqual(
  JoyLoLInterp *jInterp,
  JObj         *lolA,
  JObj         *lolB,
  size_t        timeToLive
) {
  DEBUG(jInterp, "symbolsEqual a:%p b:%p\n", lolA, lolB);
  if (!lolA && !lolB) return TRUE;
  if (!lolA || !lolB) return FALSE;
  if (asType(lolA) != asType(lolB)) return FALSE;
  if (!asType(lolA)) return FALSE;
  if (asTag(lolA) != SymbolsTag) return FALSE;
  if (strcmp(asSymbol(lolA), asSymbol(lolB)) != 0) return FALSE;
  return TRUE;
}
\stopCCode

\startTestCase[should return true if symbols are equal]

\startCTest
  AssertIntTrue(symbolsEqual(jInterp, NULL, NULL, 10));
  JObj *symA = newSymbol(jInterp, "the same text", NULL, 0);
  JObj *symB = newSymbol(jInterp, "the same text", NULL, 0);
  AssertIntTrue(symbolsEqual(jInterp, symA, symB, 10));
\stopCTest
\stopTestCase

\startTestCase[should return false if symbols are not equal]

\startCTest
  JObj *symA = newSymbol(jInterp, "text A", NULL, 0);
  JObj *symB = newSymbol(jInterp, "text B", NULL, 0);
  AssertIntFalse(symbolsEqual(jInterp, NULL, symB, 10));
  AssertIntFalse(symbolsEqual(jInterp, symA, NULL, 10));
  AssertIntFalse(symbolsEqual(jInterp, symA, symB, 10));
\stopCTest
\stopTestCase
\stopTestSuite


\startTestSuite[printing symbols]

\setCHeaderStream{private}
\startCHeader
extern size_t printSymbolCoAlg(
  StringBufferObj *aStrBuf,
  JObj            *aLoL,
  size_t           timeToLive
);
\stopCHeader
\setCHeaderStream{public}

\startCCode
Boolean printSymbolCoAlg(
  StringBufferObj *aStrBuf,
  JObj            *aLoL,
  size_t           timeToLive
) {
  assert(aLoL);
  assert(asTag(aLoL) == SymbolsTag);

  if (isFlagSet(aLoL, LOOKUP_SYMBOL_FLAG)) {
    ContextObj  *aCtx    = aStrBuf->theCtx;
    assert(aCtx);
    DictObj     *theDict = aCtx->dict;
    assert(theDict);
    DictNodeObj *assoc   =
      getSymbolEntry(theDict, asSymbol(aLoL));
  
    if (assoc && assoc->value) {
      printLoLTTL(aStrBuf, assoc->value, timeToLive-1);
      return TRUE;
    }
  }
  
  strBufPrintf(aStrBuf, "%s ", asSymbol(aLoL));
  return TRUE;
}
\stopCCode

\startTestCase[should print symbols]

\startCTest
  AssertPtrNotNull(jInterp);
  AssertPtrNotNull(jInterp->coAlgs[SymbolsTag]);

  JObj* aNewSymbol = newSymbol(jInterp, "test string", NULL, 0);
  AssertPtrNotNull(aNewSymbol);
  
  StringBufferObj* aStrBuf = newStringBuffer(jInterp->rootCtx);
  
  printLoL(aStrBuf, aNewSymbol);
  AssertStrEquals(getCString(aStrBuf),
    "test string ");
  strBufClose(aStrBuf);
\stopCTest
\stopTestCase
\stopTestSuite

\startTestSuite[registerSymbols]

\startCHeader
typedef struct symbols_class_struct {
  JClass  super;
  NewSymbol  *newSymbolFunc;
  SymbolIs   *symbolIsFunc;
} SymbolsClass;
\stopCHeader

\startCCode
static Boolean initializeSymbols(
  JoyLoLInterp *jInterp,
  JClass   *aJClass
) {
  assert(jInterp);
  assert(aJClass);
  return TRUE;
}
\stopCCode

\setCHeaderStream{private}
\startCHeader
extern Boolean registerSymbols(JoyLoLInterp* jInterp);
\stopCHeader
\setCHeaderStream{public}

\startCCode
Boolean registerSymbols(JoyLoLInterp* jInterp) {
  assert(jInterp);
  assert(jInterp->coAlgs);
  
  SymbolsClass* theCoAlg = joyLoLCalloc(1, SymbolsClass);
  theCoAlg->super.name           = SymbolsName;
  theCoAlg->super.objectSize     = sizeof(SymbolObj);
  theCoAlg->super.initializeFunc = initializeSymbols;
  theCoAlg->super.registerFunc   = registerSymbolWords;
  theCoAlg->super.equalityFunc   = symbolsEqual;
  theCoAlg->super.printFunc      = printSymbolCoAlg;
  theCoAlg->newSymbolFunc        = newSymbolImpl;
  theCoAlg->symbolIsFunc         = symbolIsImpl;

  size_t tag =
    registerJClass(jInterp, (JClass*)theCoAlg);
  
  // sanity check...
  assert(tag == SymbolsTag);
  assert(jInterp->coAlgs[tag]);

  return TRUE;
}
\stopCCode

\startTestCase[should register the Symbols coAlg]

\startCTest
  // CTestsSetup has already created a jInterp
  // and run registerSymbols
  AssertPtrNotNull(jInterp);
  AssertPtrNotNull(jInterp->coAlgs);
  AssertPtrNotNull(getSymbolsClass(jInterp));
  SymbolsClass *coAlg = getSymbolsClass(jInterp);
  AssertIntTrue(registerSymbols(jInterp));
  AssertPtrNotNull(getSymbolsClass(jInterp));
  AssertPtrEquals(getSymbolsClass(jInterp), coAlg);
  AssertIntEquals(
    getSymbolsClass(jInterp)->super.objectSize,
    sizeof(SymbolObj)
  )
\stopCTest
\stopTestCase
\stopTestSuite
