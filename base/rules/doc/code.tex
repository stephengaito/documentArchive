% A ConTeXt document [master document: rules.tex]

\section[title=Code]
\setCHeaderStream{public}

\dependsOn[jInterps]
%\dependsOn[context]

\component gitVersion-c

\startCHeader
typedef struct rule_object_struct {
  JObj    super;
  Symbol *name;
  Symbol *body;
} RuleObj;
\stopCHeader

\startTestSuite[newRule]

\startCHeader
typedef RuleObj* (NewRule)(
  JoyLoLInterp *jInterp,
  Symbol       *aName,
  Symbol       *aBody
);

#define newRule(jInterp, aName, aBody)      \
  (                                                   \
    assert(getRulesClass(jInterp)           \
      ->newRuleFunc),                       \
    (getRulesClass(jInterp)                 \
      ->newRuleFunc(jInterp, aName, aBody)) \
  )

//#define asRule(aLoL) (((aLoL)->flags) & BOOLEAN_FLAG_MASK)
\stopCHeader

\setCHeaderStream{private}
\startCHeader
extern RuleObj* newRuleImpl(
  JoyLoLInterp *jInterp,
  Symbol       *aName,
  Symbol       *aBody
);
\stopCHeader
\setCHeaderStream{public}

\startCCode
RuleObj* newRuleImpl(
  JoyLoLInterp *jInterp,
  Symbol       *aName,
  Symbol       *aBody
) {
  assert(jInterp);
  assert(jInterp->coAlgs);
  
  RuleObj *result =
    (RuleObj*)newObject(jInterp, RulesTag);
  assert(result);
  
//  result->super.type  = jInterp->coAlgs[RulesTag];
  result->name        = strdup(aName);
  result->body        = strdup(aBody);
 
  return result;
}
\stopCCode

\startTestCase[should create a new Rule]

\startCTest
  AssertPtrNotNull(jInterp);

  RuleObj* aNewRule =
    newRule(jInterp, "ansiC", "a rule body");
  AssertPtrNotNull(aNewRule);
  AssertPtrNotNull(asType(aNewRule));
  AssertIntEquals(asTag(aNewRule), RulesTag);
  AssertIntTrue(isAtom(aNewRule));
  AssertIntTrue(isRule(aNewRule));
  AssertIntFalse(isPair(aNewRule));
\stopCTest
\stopTestCase

\startTestCase[print Rule]
\startCTest
  AssertPtrNotNull(jInterp);

  StringBufferObj *aStrBuf = newStringBuffer(jInterp);
  AssertPtrNotNull(aStrBuf);

  RuleObj* aLoL =
    newRule(jInterp, "ansiC", "a rule body");
  AssertPtrNotNull(aLoL);
  printLoL(jInterp, aStrBuf, (JObj*)aLoL);
  AssertStrEquals(getCString(jInterp, aStrBuf), "rule ");
  strBufClose(jInterp, aStrBuf);
\stopCTest
\stopTestCase
\stopTestSuite

\startTestSuite[isRule]

\startCHeader
#define isRule(aLoL)            \
  (                                       \
    (                                     \
      (aLoL) &&                           \
      asType(aLoL) &&                     \
      (asTag(aLoL) == RulesTag) \
    ) ?                                   \
      TRUE :                              \
      FALSE                               \
  )
\stopCHeader

\setCHeaderStream{private}
\startCHeader
extern Boolean equalityRuleCoAlg(
  JoyLoLInterp *jInterp,
  JObj         *lolA,
  JObj         *lolB,
  size_t        timeToLive
);
\stopCHeader
\setCHeaderStream{public}

\startCCode
Boolean equalityRuleCoAlg(
  JoyLoLInterp *jInterp,
  JObj         *lolA,
  JObj         *lolB,
  size_t        timeToLive
) {
  DEBUG(jInterp, "ruleCoAlg-equal a:%p b:%p\n", lolA, lolB);
  if (!lolA && !lolB) return TRUE;
  if (!lolA && lolB)  return FALSE;
  if (lolA  && !lolB) return FALSE;
  if (asType(lolA) != asType(lolB)) return FALSE;
  if (!asType(lolA)) return FALSE;
  if (asTag(lolA)  != RulesTag) return FALSE;
  if (lolA != lolB) return FALSE;
  return TRUE;
}
\stopCCode

\startTestSuite[printing rules]

\setCHeaderStream{private}
\startCHeader
extern Boolean printRuleCoAlg(
  JoyLoLInterp    *jInterp,
  StringBufferObj *aStrBuf,
  JObj            *aLoL,
  size_t           timeToLive
);
\stopCHeader
\setCHeaderStream{public}

\startCCode
Boolean printRuleCoAlg(
  JoyLoLInterp    *jInterp,
  StringBufferObj *aStrBuf,
  JObj            *aLoL,
  size_t           timeToLive
) {
  assert(aLoL);
  assert(asTag(aLoL) == RulesTag);

  strBufPrintf(jInterp, aStrBuf, "rule ");
  return TRUE;
}
\stopCCode

\startTestCase[should print rules]

\startCTest
  AssertPtrNotNull(jInterp);
  AssertPtrNotNull(jInterp->coAlgs[RulesTag]);

  StringBufferObj *aStrBuf = newStringBuffer(jInterp);
  AssertPtrNotNull(aStrBuf);
  
  RuleObj* aNewRule =
    newRule(jInterp, "ansiC", "a rule body");
  AssertPtrNotNull(aNewRule);
  printLoL(jInterp, aStrBuf, (JObj*)aNewRule);
  AssertStrEquals(getCString(jInterp, aStrBuf), "rule ");
  strBufClose(jInterp, aStrBuf);
\stopCTest
\stopTestCase
\stopTestSuite

\startTestSuite[registerRules]

\startCHeader
typedef struct rules_class_struct {
  JClass       super;
  NewRule      *newRuleFunc;
} RulesClass;

\stopCHeader

\startCCode
static Boolean initializeRules(
  JoyLoLInterp *jInterp,
  JClass   *aJClass
) {
  assert(jInterp);
  assert(aJClass);
  return TRUE;
}
\stopCCode

\setCHeaderStream{private}
\startCHeader
extern Boolean registerRules(JoyLoLInterp *jInterp);
\stopCHeader
\setCHeaderStream{public}

\startCCode
Boolean registerRules(JoyLoLInterp *jInterp) {
  assert(jInterp);
  assert(jInterp->coAlgs);
  
  RulesClass* theCoAlg
    = joyLoLCalloc(1, RulesClass);
  assert(theCoAlg);
  
  theCoAlg->super.name           = RulesName;
  theCoAlg->super.objectSize     = sizeof(RuleObj);
  theCoAlg->super.initializeFunc = initializeRules;
  theCoAlg->super.registerFunc   = registerRuleWords;
  theCoAlg->super.equalityFunc   = equalityRuleCoAlg;
  theCoAlg->super.printFunc      = printRuleCoAlg;
  theCoAlg->newRuleFunc = newRuleImpl;
  size_t tag =
    registerJClass(jInterp, (JClass*)theCoAlg);
  
  // do a sanity check...
  assert(tag == RulesTag);
  assert(jInterp->coAlgs[tag]);
   
  return TRUE;
}
\stopCCode

\startTestCase[should register the Rules coAlg]

\startCTest
  // CTestsSetup has already created a jInterp
  // and run registerRules
  AssertPtrNotNull(jInterp);
  AssertPtrNotNull(jInterp->coAlgs);
  AssertPtrNotNull(getRulesClass(jInterp));
  RulesClass *coAlg = getRulesClass(jInterp);
  registerRules(jInterp);
  AssertPtrNotNull(getRulesClass(jInterp));
  AssertPtrEquals(getRulesClass(jInterp), coAlg);
  AssertIntEquals(
    getRulesClass(jInterp)->super.objectSize,
    sizeof(RuleObj)
  )
\stopCTest
\stopTestCase
\stopTestSuite
