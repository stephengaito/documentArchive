% A ConTeXt document [master document: dictNodes.tex]

\section[title=Code]
\setCHeaderStream{public}

\dependsOn[jInterps]
\dependsOn[symbols]
\dependsOn[stringBuffers]

\startCHeader
typedef struct dictNode_object_struct DictNodeObj;

typedef struct dictNode_object_struct {
  JObj  super;
  Symbol*   symbol;
  JObj* preObs;
  JObj* value;
  JObj* postObs;
  DictNodeObj*  left;
  DictNodeObj*  right;
  DictNodeObj*  previous;
  DictNodeObj*  next;
  size_t    height;
  long      balance;
} DictNodeObj;
\stopCHeader

\startTestSuite[newDict]

\startCHeader
typedef DictNodeObj *(NewDictNode)(
  JoyLoLInterp *jInterp,
  Symbol       *aSym
);

#define newDictNode(jInterp, aSym)      \
  (                                     \
    assert(getDictNodesClass(jInterp)   \
      ->newDictNodeFunc),               \
    (getDictNodesClass(jInterp)         \
      ->newDictNodeFunc(jInterp, aSym)) \
  )
\stopCHeader

\setCHeaderStream{private}
\startCHeader
extern DictNodeObj *newDictNodeImpl(
  JoyLoLInterp *jInterp,
  Symbol       *aSym
);

#define dictAsSymbol(aNode)   ((DictNodeObj*)(aNode))->symbol
#define dictAsPreObs(aNode)   ((DictNodeObj*)(aNode))->preObs
#define dictAsValue(aNode)    ((DictNodeObj*)(aNode))->value
#define dictAsPostObs(aNode)  ((DictNodeObj*)(aNode))->postObs
#define dictAsLeft(aNode)     ((DictNodeObj*)(aNode))->left
#define dictAsRight(aNode)    ((DictNodeObj*)(aNode))->right
#define dictAsPrevious(aNode) ((DictNodeObj*)(aNode))->previous
#define dictAsNext(aNode)     ((DictNodeObj*)(aNode))->next
#define dictAsHeight(aNode)   ((DictNodeObj*)(aNode))->height
#define dictAsBalance(aNode)  ((DictNodeObj*)(aNode))->balance
\stopCHeader
\setCHeaderStream{public}

\startCCode
// We implement our dictionary as an AVL binary tree using AVLNodes.
//
// Our implementation is inspired by:
// The Crazy Programmer's "Program for AVL Tree in C" (Neeraj Mishra)
// http://www.thecrazyprogrammer.com/2014/03/c-program-for-avl-tree-implementation.html
// and by:
// Jianye Hao's CSC2100B Tutorial 4 "Binary and AVL Trees in C"
// https://www.cse.cuhk.edu.hk/irwin.king/_media/teaching/csc2100b/tu4.pdf
//
// At the moment we only insert and search (we never delete).
//
// ANY AVLTree node can be the root of a new dictionary.
//

DictNodeObj *newDictNodeImpl(
  JoyLoLInterp *jInterp,
  Symbol       *aSym
) {
  assert(jInterp);
  assert(jInterp->coAlgs);
  assert(DictNodesTag < jInterp->numCoAlgs);
  assert(jInterp->coAlgs[DictNodesTag]);
  
  assert(aSym);
  
  DEBUG(jInterp, "newDictNode [%s]\n", aSym);
  JObj* newNode   = newObject(jInterp, DictNodesTag);
  dictAsSymbol(newNode)   = strdup(aSym);
  dictAsPreObs(newNode)   = NULL;
  dictAsValue(newNode)    = NULL;
  dictAsPostObs(newNode)  = NULL;
  dictAsLeft(newNode)     = NULL;
  dictAsRight(newNode)    = NULL;
  dictAsPrevious(newNode) = NULL;
  dictAsNext(newNode)     = NULL;
  dictAsHeight(newNode)   = 1;
  dictAsBalance(newNode)  = 0;
  return (DictNodeObj*)newNode;
}
\stopCCode

\startTestCase[should add a new dictNode object (AVL node)]

\startCTest
  AssertPtrNotNull(jInterp);
  AssertPtrNotNull(jInterp->coAlgs[DictNodesTag]);

  DictNodeObj* aNode = newDictNode(jInterp, "aNodeSymbol");
  AssertPtrNotNull(aNode);
  AssertPtrNotNull(asType(aNode));
  AssertIntEquals(asTag(aNode), DictNodesTag)
  AssertPtrNotNull(dictAsSymbol(aNode));
  AssertStrEquals(dictAsSymbol(aNode), "aNodeSymbol");
  AssertPtrNull(dictAsValue(aNode));
  AssertPtrNull(dictAsLeft(aNode));
  AssertPtrNull(dictAsRight(aNode));
  AssertPtrNull(dictAsPrevious(aNode));
  AssertPtrNull(dictAsNext(aNode));
  AssertIntEquals(dictAsHeight(aNode), 1);
  AssertIntEquals(dictAsBalance(aNode), 0);
\stopCTest
\stopTestCase
\stopTestSuite

\setCHeaderStream{private}
\startCHeader
extern Boolean printDicionaryJObjInto(
  JoyLoLInterp    *jInterp,
  StringBufferObj *aStrBuf,
  JObj        *anAVLNode
);
extern Boolean printDicInto(
  JoyLoLInterp    *jInterp,
  StringBufferObj *aStrBuf,
  DictNodeObj         *anAVLNode
);
\stopCHeader
\setCHeaderStream{public}

\startCCode
Boolean printDictionaryJObjInto(
  JoyLoLInterp    *jInterp,
  StringBufferObj *aStrBuf,
  JObj        *anAVLNode
) {
  if (!anAVLNode) return FALSE;
  if (asTag(anAVLNode) != DictNodesTag) return FALSE;
  return printDicInto(jInterp, aStrBuf, (DictNodeObj*)anAVLNode);
}

Boolean printDicInto(
  JoyLoLInterp    *jInterp,
  StringBufferObj *aStrBuf,
  DictNodeObj         *anAVLNode
) {
  if (!anAVLNode) return TRUE;
  strBufPrintf(jInterp, aStrBuf, "[%s] l:( ", anAVLNode->symbol);
  printDicInto(jInterp, aStrBuf, anAVLNode->left);
  strBufPrintf(jInterp, aStrBuf, " ) r:( ");
  printDicInto(jInterp, aStrBuf, anAVLNode->right);
  strBufPrintf(jInterp, aStrBuf, " ) ");
  return TRUE;
}
\stopCCode

\startTestSuite[registerDictNodes]

\startCHeader
typedef struct dictNodes_class_struct {
  JClass super;
  NewDictNode          *newDictNodeFunc;
  FindSymbolRecurse    *findSymbolRecurseFunc;
  InsertSymbolRecurse  *insertSymbolRecurseFunc;
  FindLUBSymbolRecurse *findLUBSymbolRecurseFunc;
} DictNodesClass;
\stopCHeader

\startCCode
static Boolean initializeDictNodes(
  JoyLoLInterp *jInterp,
  JClass   *aJClass
) {
  assert(jInterp);
  assert(aJClass);
  return TRUE;
}
\stopCCode

\setCHeaderStream{private}
\startCHeader
extern Boolean registerDictNodes(JoyLoLInterp *jInterp);
\stopCHeader
\setCHeaderStream{public}

\startCCode
Boolean registerDictNodes(JoyLoLInterp *jInterp) {
  assert(jInterp);
  
  DictNodesClass* theCoAlg =
    joyLoLCalloc(1, DictNodesClass);
  assert(theCoAlg);
  
  theCoAlg->super.name            = DictNodesName;
  theCoAlg->super.objectSize      = sizeof(DictNodeObj);
  theCoAlg->super.initializeFunc  = initializeDictNodes;
  theCoAlg->super.registerFunc    = registerDictNodeWords;
  theCoAlg->super.equalityFunc    = NULL;
  theCoAlg->super.printFunc       = printDictionaryJObjInto;
  theCoAlg->newDictNodeFunc       = newDictNodeImpl;
  theCoAlg->findSymbolRecurseFunc = findSymbolRecurseImpl;
  theCoAlg->insertSymbolRecurseFunc =
    insertSymbolRecurseImpl;
  theCoAlg->findLUBSymbolRecurseFunc =
    findLUBSymbolRecurseImpl;

  size_t tag =
    registerJClass(jInterp, (JClass*)theCoAlg);
  
  // do a sanity check...
  assert(tag == DictNodesTag);
  assert(jInterp->coAlgs[tag]);
    
  return TRUE;
}
\stopCCode

\startTestCase[should register the DictNodes coAlg]

\startCTest
  // CTestsSetup has already created a jInterp
  // and run regiserDictNodes
  AssertPtrNotNull(jInterp);
  AssertPtrNotNull(getDictNodesClass(jInterp));
  DictNodesClass *coAlg =
    getDictNodesClass(jInterp);
  AssertIntTrue(registerDictNodes(jInterp));
  AssertPtrNotNull(getDictNodesClass(jInterp));
  AssertPtrEquals(getDictNodesClass(jInterp), coAlg);
  AssertIntEquals(
    getDictNodesClass(jInterp)->super.objectSize, 
    sizeof(DictNodeObj)
  );
\stopCTest
\stopTestCase
\stopTestSuite

\setCHeaderStream{public}
