% A ConTeXt document [master document: fragments.tex]

\section[title=Code]
\setCHeaderStream{public}

\dependsOn[jInterps]
%\dependsOn[context]

\component gitVersion-c

\startCHeader
typedef struct fragment_object_struct {
  JObj    super;
  Symbol *name;
  Symbol *body;
} FragmentObj;
\stopCHeader

\startTestSuite[newFragment]

\startCHeader
typedef FragmentObj* (NewFragment)(
  JoyLoLInterp *jInterp,
  Symbol       *aName,
  Symbol       *aBody
);

#define newFragment(jInterp, aName, aBody)      \
  (                                             \
    assert(getFragmentsClass(jInterp)           \
      ->newFragmentFunc),                       \
    (getFragmentsClass(jInterp)                 \
      ->newFragmentFunc(jInterp, aName, aBody)) \
  )

//#define asFragment(aLoL) (((aLoL)->flags) & BOOLEAN_FLAG_MASK)
\stopCHeader

\setCHeaderStream{private}
\startCHeader
extern FragmentObj* newFragmentImpl(
  JoyLoLInterp *jInterp,
  Symbol       *aName,
  Symbol       *aBody
);
\stopCHeader
\setCHeaderStream{public}

\startCCode
FragmentObj* newFragmentImpl(
  JoyLoLInterp *jInterp,
  Symbol       *aName,
  Symbol       *aBody
) {
  assert(jInterp);
  assert(jInterp->coAlgs);
  
  FragmentObj *result =
    (FragmentObj*)newObject(jInterp, FragmentsTag);
  assert(result);
  
//  result->super.type  = jInterp->coAlgs[FragmentsTag];
  result->name        = strdup(aName);
  result->body        = strdup(aBody);
 
  return result;
}
\stopCCode

\startTestCase[should create a new fragment]

\startCTest
  AssertPtrNotNull(jInterp);

  FragmentObj* aNewFragment =
    newFragment(jInterp, "ansiC", "a fragment body");
  AssertPtrNotNull(aNewFragment);
  AssertPtrNotNull(asType(aNewFragment));
  AssertIntEquals(asTag(aNewFragment), FragmentsTag);
  AssertIntTrue(isAtom(aNewFragment));
  AssertIntTrue(isFragment(aNewFragment));
  AssertIntFalse(isPair(aNewFragment));
\stopCTest
\stopTestCase

\startTestCase[print Fragment]
\startCTest
  AssertPtrNotNull(jInterp);

  StringBufferObj *aStrBuf = newStringBuffer(jInterp);
  AssertPtrNotNull(aStrBuf);

  FragmentObj* aLoL =
    newFragment(jInterp, "ansiC", "a fragment body");
  AssertPtrNotNull(aLoL);
  printLoL(jInterp, aStrBuf, (JObj*)aLoL);
  AssertStrEquals(getCString(jInterp, aStrBuf), "fragment ");
  strBufClose(jInterp, aStrBuf);
\stopCTest
\stopTestCase
\stopTestSuite

\startTestSuite[isFragment]

\startCHeader
#define isFragment(aLoL)             \
  (                                 \
    (                               \
      (aLoL) &&                     \
      asType(aLoL) &&               \
      (asTag(aLoL) == FragmentsTag)  \
    ) ?                             \
      TRUE :                        \
      FALSE                         \
  )
\stopCHeader

\setCHeaderStream{private}
\startCHeader
extern Boolean equalityFragmentCoAlg(
  JoyLoLInterp *jInterp,
  JObj         *lolA,
  JObj         *lolB,
  size_t        timeToLive
);
\stopCHeader
\setCHeaderStream{public}

\startCCode
Boolean equalityFragmentCoAlg(
  JoyLoLInterp *jInterp,
  JObj         *lolA,
  JObj         *lolB,
  size_t        timeToLive
) {
  DEBUG(jInterp, "fragmentCoAlg-equal a:%p b:%p\n", lolA, lolB);
  if (!lolA && !lolB) return TRUE;
  if (!lolA && lolB)  return FALSE;
  if (lolA  && !lolB) return FALSE;
  if (asType(lolA) != asType(lolB)) return FALSE;
  if (!asType(lolA)) return FALSE;
  if (asTag(lolA)  != FragmentsTag) return FALSE;
  if (lolA != lolB) return FALSE;
  return TRUE;
}
\stopCCode

\startTestSuite[printing fragments]

\setCHeaderStream{private}
\startCHeader
extern Boolean printFragmentCoAlg(
  JoyLoLInterp    *jInterp,
  StringBufferObj *aStrBuf,
  JObj            *aLoL,
  size_t           timeToLive
);
\stopCHeader
\setCHeaderStream{public}

\startCCode
Boolean printFragmentCoAlg(
  JoyLoLInterp    *jInterp,
  StringBufferObj *aStrBuf,
  JObj            *aLoL,
  size_t           timeToLive
) {
  assert(aLoL);
  assert(asTag(aLoL) == FragmentsTag);

  strBufPrintf(jInterp, aStrBuf, "fragment ");
  return TRUE;
}
\stopCCode

\startTestCase[should print fragments]

\startCTest
  AssertPtrNotNull(jInterp);
  AssertPtrNotNull(jInterp->coAlgs[FragmentsTag]);

  StringBufferObj *aStrBuf = newStringBuffer(jInterp);
  AssertPtrNotNull(aStrBuf);
  
  FragmentObj* aNewFragment =
    newFragment(jInterp, "ansiC", "a fragment body");
  AssertPtrNotNull(aNewFragment);
  printLoL(jInterp, aStrBuf, (JObj*)aNewFragment);
  AssertStrEquals(getCString(jInterp, aStrBuf), "fragment ");
  strBufClose(jInterp, aStrBuf);
\stopCTest
\stopTestCase
\stopTestSuite

\startTestSuite[registerFragments]

\startCHeader
typedef struct fragments_class_struct {
  JClass       super;
  NewFragment      *newFragmentFunc;
} FragmentsClass;

\stopCHeader

\startCCode
static Boolean initializeFragments(
  JoyLoLInterp *jInterp,
  JClass   *aJClass
) {
  assert(jInterp);
  assert(aJClass);
  return TRUE;
}
\stopCCode

\setCHeaderStream{private}
\startCHeader
extern Boolean registerFragments(JoyLoLInterp *jInterp);
\stopCHeader
\setCHeaderStream{public}

\startCCode
Boolean registerFragments(JoyLoLInterp *jInterp) {
  assert(jInterp);
  assert(jInterp->coAlgs);
  
  FragmentsClass* theCoAlg
    = joyLoLCalloc(1, FragmentsClass);
  assert(theCoAlg);
  
  theCoAlg->super.name           = FragmentsName;
  theCoAlg->super.objectSize     = sizeof(FragmentObj);
  theCoAlg->super.initializeFunc = initializeFragments;
  theCoAlg->super.registerFunc   = registerFragmentWords;
  theCoAlg->super.equalityFunc   = equalityFragmentCoAlg;
  theCoAlg->super.printFunc      = printFragmentCoAlg;
  theCoAlg->newFragmentFunc = newFragmentImpl;
  size_t tag =
    registerJClass(jInterp, (JClass*)theCoAlg);
  
  // do a sanity check...
  assert(tag == FragmentsTag);
  assert(jInterp->coAlgs[tag]);
   
  return TRUE;
}
\stopCCode

\startTestCase[should register the Fragments coAlg]

\startCTest
  // CTestsSetup has already created a jInterp
  // and run registerFragments
  AssertPtrNotNull(jInterp);
  AssertPtrNotNull(jInterp->coAlgs);
  AssertPtrNotNull(getFragmentsClass(jInterp));
  FragmentsClass *coAlg = getFragmentsClass(jInterp);
  registerFragments(jInterp);
  AssertPtrNotNull(getFragmentsClass(jInterp));
  AssertPtrEquals(getFragmentsClass(jInterp), coAlg);
  AssertIntEquals(
    getFragmentsClass(jInterp)->super.objectSize,
    sizeof(FragmentObj)
  )
\stopCTest
\stopTestCase
\stopTestSuite
