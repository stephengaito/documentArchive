% A ConTeXt document [master document: luaFunctions.tex]

\section[title=Code]
\setCHeaderStream{public}

\dependsOn[jInterps]
%\dependsOn[context]
%\dependsOn[boolean]

\starttyping
\startCHeader

typedef void (CFunction)(Context*);

typedef struct luaFunction_struct {
  CoAlgebras super;
  CFunction  theFunction;
} CFunction;

\stopCHeader

\startJoyLoLWord[isCFunction]

\preProcessStack[aCFunc][]
\postDataStack[isBoolean]

\startCCode

if (!aCFunc) then
  pushDataFalse(aCtx);
if (aCFunc->isA == CFUNCTIONS_COALG) then
  pushDataTrue(aCtx);
pushDataFalse(aCtx);

\stopCCode

\stopJoyLoLWord

\startJoyLoLWord[executeCFunc]

\preProcessStack[aCFunc][]

\startCCode

if (!aCFunc) then
  raiseException(aCtx, "no atom (CFunction)");
if (aCFunc->super.isA != CFUNCTIONS_COALG) then
  raiseException(aCtx, "not a CFunction");
if (!aCFunc->theFunction) then
  raiseException(aCtx, "no CFunction found");

(aCFunc->theFunction)(aCtx);

\stopCCode

\stopJoyLoLWord
\stoptyping

\starttyping
#ifndef JOYLOL_COALG_FUNCTIONS_H
#define JOYLOL_COALG_FUNCTIONS_H

typedef struct context_struct Context;
typedef void (JoyLoLFunction)(Context*);

typedef struct functions_struct {
  CoAlgebra super;
  // other things
} Functions;

extern Functions* createFunctionsCoAlgebra(void);
extern void initFunctionsCoAlgebra(CoAlgebras* coAlgs);

extern PairAtom* newJoyLoLFunc(CoAlgebras* coAlgs, JoyLoLFunction *aJoyLoLFunc);

extern size_t isFunction(PairAtom* aLoL);

#endif
\stoptyping

\starttyping
#include <stdlib.h>
#include <string.h>
#include <assert.h>

#include "joyLoL/macros.h"
#include "joyLoL/coAlg/coAlgs.h"
#include "joyLoL/lists.h"
#include "joyLoL/printer.h"

static size_t equalityFuncCoAlg(CoAlgebra* klass,
                                PairAtom* lolA, PairAtom* lolB,
                                size_t debugFlag) {
  DEBUG(debugFlag, "funcCoAlg-equal klass:%p a:%p b:%p\n", klass, lolA, lolB);
  if (!lolA && !lolB) return TRUE;
  if (!lolA && lolB)  return FALSE;
  if (lolA  && !lolB) return FALSE;
  if (lolA->coAlg != klass) return FALSE;
  if (lolB->coAlg != klass) return FALSE;
  if (lolA->func != lolB->func) return FALSE;
  return TRUE;
}


static size_t printSizeFuncCoAlg(PairAtom* aLoL, size_t debugFlag) {
  DEBUG(debugFlag, "funcCoAlg-printSize: %p\n", aLoL);
  assert(aLoL);
  assert(aLoL->coAlg);
  assert(aLoL->coAlg->isA == FUNCTION_COALG);
  DEBUG(debugFlag, "funcCoAlg-printSize: func<%p> %p\n", aLoL->func, aLoL);
  return 15;
}

static size_t printStrFuncCoAlg(PairAtom* aLoL,
                               char* buffer, size_t bufferSize) {
  assert(aLoL);
  assert(aLoL->coAlg);
  assert(aLoL->coAlg->isA == FUNCTION_COALG);

  char ptoa[100];
  sprintf(ptoa, "<%p> ", aLoL->func);
  strcat(buffer, ptoa);
  return TRUE;
}

Functions* createFunctionsCoAlgebra(void) {
  Functions* funcs = (Functions*) calloc(1, sizeof(Functions));
  initACoAlgebra((CoAlgebra*)funcs);

  funcs->super.isA       = FUNCTION_COALG;
  funcs->super.name      = "function";
  funcs->super.equality  = equalityFuncCoAlg;
  funcs->super.printSize = printSizeFuncCoAlg;
  funcs->super.printStr  = printStrFuncCoAlg;
  return funcs;
}

PairAtom* newJoyLoLFunc(CoAlgebras* coAlgs, JoyLoLFunction *aFunc) {
  assert(coAlgs);
  PairAtom* aNewFunc = newPairAtom(coAlgs);
  assert(aNewFunc);
  aNewFunc->coAlg = (CoAlgebra*)coAlgs->functions;
  aNewFunc->tag   = 0;
  aNewFunc->func  = aFunc;
  return aNewFunc;
}

size_t isFunction(PairAtom* aLoL) {
  if (!aLoL) return FALSE;
  if (aLoL->coAlg && (aLoL->coAlg->isA == FUNCTION_COALG)) return TRUE;
  return FALSE;
}

static void isFunctionAP(Context* aCtx) {
  assert(aCtx);
  assert(aCtx->coAlgebras);
  popCtxDataInto(aCtx, top);
  PairAtom* result = NULL;
  if (isFunction(top)) result = newBoolean(aCtx->coAlgebras, TRUE);
  else                 result = newBoolean(aCtx->coAlgebras, FALSE);
  pushCtxData(aCtx, result);
}

void initFunctionsCoAlgebra(CoAlgebras* coAlgs) {
  extendJoyLoL(coAlgs, "isFunction", isFunctionAP);
}
\stoptyping

\starttyping
#include <stdio.h>
#include <stdlib.h>
#include <string.h>

#include "CuTest.h"

#include "joyLoL/macros.h"
#include "joyLoL/coAlg/coAlgs.h"
#include "joyLoL/dictionary.h"
#include "joyLoL/lists.h"
#include "joyLoL/printer.h"

// suiteName: - Functions CoAlgebra tests -

void Test_createFunctionsCoAlgebra(CuTest* tc) {
  CoAlgebras *coAlgs = createCoAlgebras();
  CuAssertPtrNotNull(tc, coAlgs);
  CuAssertPtrNotNull(tc, coAlgs->functions);
}

void testJoyLoLFunction(Context* aCtx) { }


void Test_newJoyLoLFunction(CuTest* tc) {
  CoAlgebras* coAlgs = createCoAlgebras();
  CuAssertPtrNotNull(tc, coAlgs);
  CuAssertPtrNotNull(tc, coAlgs->functions);

  PairAtom* aNewFunc = newJoyLoLFunc(coAlgs, testJoyLoLFunction);
  CuAssertPtrNotNull(tc, aNewFunc);
  CuAssertPtrNotEquals(tc, aNewFunc, testJoyLoLFunction);
  CuAssertPtrNotNull(tc, aNewFunc->coAlg);
  CuAssertIntEquals(tc, aNewFunc->coAlg->isA, FUNCTION_COALG);
  CuAssertPtrEquals(tc, aNewFunc->func, testJoyLoLFunction);
  CuAssertTrue(tc, isFunction(aNewFunc));
  CuAssertTrue(tc, isAtom(aNewFunc));
  CuAssertFalse(tc, isBoolean(aNewFunc));
  CuAssertFalse(tc, isContext(aNewFunc));
  CuAssertFalse(tc, isNatural(aNewFunc));
  CuAssertFalse(tc, isPair(aNewFunc));
  CuAssertFalse(tc, isSymbol(aNewFunc));
}

void Test_printJoyLoLFuncion(CuTest* tc) {
  CoAlgebras* coAlgs = createCoAlgebras();
  CuAssertPtrNotNull(tc, coAlgs);
  CuAssertPtrNotNull(tc, coAlgs->functions);

  char buffer[100];
  buffer[0] = 0;
  sprintf((char*)&buffer, "<%p>", testJoyLoLFunction);

  PairAtom* aNewFunc = newJoyLoLFunc(coAlgs, testJoyLoLFunction);
  CuAssertPtrNotNull(tc, aNewFunc);
  CuAssertIntEquals(tc, printSizeDebug(aNewFunc, FALSE), 15);
  CuAssertStrEquals(tc, printLoLDebug(aNewFunc, FALSE), buffer);
}

void Test_checkFunctionsSymbols(CuTest* tc) {
  CoAlgebras* coAlgs = createCoAlgebras();
  CuAssertPtrNotNull(tc, coAlgs);

  Symbols* symbols = coAlgs->symbols;
  CuAssertPtrNotNull(tc, symbols);
  Dictionary* mainDic = symbols->dictionary;
  CuAssertPtrNotNull(tc, mainDic);

  AVLNode* aNode = findSymbol(mainDic, "isFunction");
  CuAssertPtrNotNull(tc, aNode);
  CuAssertStrEquals(tc, aNode->symbol, "isFunction");
  CuAssertPtrNotNull(tc, aNode->value);
  CuAssertTrue(tc, isFunction(aNode->value));
}

\stoptyping

\startTestSuite[registerLuaFunctions]

\startCHeader
typedef struct luaFunctions_class_struct {
  CoAlgClass super;
} LuaFunctionsClass;
\stopCHeader

\setCHeaderStream{private}
\startCHeader
extern Boolean registerLuaFunctions(JoyLoLInterp *jInterp);
\stopCHeader
\setCHeaderStream{public}

\startCCode
Boolean registerLuaFunctions(JoyLoLInterp *jInterp) {
  assert(jInterp);
  
  LuaFunctionsClass* theCoAlg =
    joyLoLCalloc(1, LuaFunctionsClass);
  theCoAlg->super.name         = LuaFunctionsName;
  theCoAlg->super.objectSize   = sizeof(CoAlgObj);
  theCoAlg->super.registerFunc = registerLuaFunctions;
  theCoAlg->super.equalityFunc = NULL;
  theCoAlg->super.printFunc    = NULL;
  
  size_t tag =
    registerCoAlgClass(jInterp, (CoAlgClass*)theCoAlg);

  // do a sanity check...
  assert(tag == LuaFunctionsTag);
  assert(jInterp->coAlgs[tag].sClass);
  
  registerLuaFunctionWords(jInterp);
  
  return TRUE;
}
\stopCCode

\startTestCase[should register the LuaFunctions coAlg]

\startCTest
  // CTestsSetup has already created a jInterp
  // and run registerLuaFunctions
  
  AssertPtrNotNull(jInterp);
  AssertPtrNotNull(jInterp->coAlgs);
  AssertPtrNotNull(getLuaFunctionsClass(jInterp));
  LuaFunctionsClass *coAlg =
    getLuaFunctionsClass(jInterp);
  AssertIntTrue(registerLuaFunctions(jInterp));
  AssertPtrNotNull(getLuaFunctionsClass(jInterp));
  AssertPtrEquals(getLuaFunctionsClass(jInterp), coAlg);
  AssertIntEquals(
    getLuaFunctionsClass(jInterp)->super.objectSize,
    sizeof(CoAlgObj)
  )
\stopCTest
\stopTestCase
\stopTestSuite
