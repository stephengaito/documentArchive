% A ConTeXt document [master document: jInterps.tex]

\section[title=Lua interface functions]
\setCCodeStream{lua}

\startTestSuite[ set, add and get LuaPath]

\startCHeader
extern Boolean setLuaPath(
  JoyLoLInterp *jInterp,
  Symbol *aLuaPath,
  Boolean addToCPath
);

extern Boolean addLuaPath(
  JoyLoLInterp *jInterp,
  Symbol *aLuaPath,
  Boolean addToCPath
);

extern Symbol *getLuaPath(
  JoyLoLInterp *jInterp,
  Boolean getCPath
);
\stopCHeader

\startCCode
Boolean setLuaPath(
  JoyLoLInterp *jInterp,
  Symbol *aLuaPath,
  Boolean addToCPath
) {
  assert(jInterp);
  assert(jInterp->lstate);

  char* path = "path";
  if (addToCPath) {
    path = "cpath";
  }
  
  lua_State *lstate = jInterp->lstate;
  lua_getglobal(lstate, "package");
  lua_pushstring(lstate, aLuaPath);
  lua_setfield(lstate, -2, path);
  lua_pop(lstate, 1);
  return TRUE;
}

Boolean addLuaPath(
  JoyLoLInterp *jInterp,
  Symbol *aLuaPath,
  Boolean addToCPath
) {
  assert(jInterp);
  assert(jInterp->lstate);

  char* path = "path";
  if (addToCPath) {
    path = "cpath";
  }
  
  lua_State *lstate = jInterp->lstate;
  luaL_Buffer lBuf;
  luaL_buffinit(lstate, &lBuf);
  luaL_addstring(&lBuf, aLuaPath);
  luaL_addstring(&lBuf, ";");
  lua_getglobal(lstate, "package");
  lua_getfield(lstate, -1, path);
  luaL_addvalue (&lBuf);
  luaL_pushresult(&lBuf);
  lua_setfield(lstate, -2, path);
  lua_pop(lstate, 1);
  return TRUE;
}

Symbol *getLuaPath(
  JoyLoLInterp *jInterp,
  Boolean getCPath
) {
  assert(jInterp);
  assert(jInterp->lstate);

  char* path = "path";
  if (getCPath) {
    path = "cpath";
  }
  
  lua_State *lstate = jInterp->lstate;
  lua_getglobal(lstate, "package");
  lua_getfield(lstate, -1, path);
  Symbol *result = strdup(lua_tostring(lstate, -1));
  lua_pop(lstate, 2);
  return result;
}
\stopCCode

\startTestCase[should set, add and get package.path]

\startCTest
  Symbol *origPath = getLuaPath(jInterp, FALSE);
  AssertPtrNotNull(origPath);
  AssertPtrNotNull(strstr(origPath, "?.lua"));

  addLuaPath(jInterp, "tests", FALSE);
  Symbol *testPath = getLuaPath(jInterp, FALSE);
  AssertPtrNotNull(testPath);

  AssertIntZero(strncmp(testPath, "tests;", strlen("tests;")));
  
  setLuaPath(jInterp, origPath, FALSE);
  free((void*)testPath);
  testPath = getLuaPath(jInterp, FALSE);
  AssertPtrNotNull(testPath);
  AssertIntZero(strcmp(origPath, testPath));

  free((void*)testPath);
  free((void*)origPath);
\stopCTest
\stopTestCase

\startTestCase[should set, add and get package.cpath]

\startCTest
  Symbol *origPath = getLuaPath(jInterp, TRUE);
  AssertPtrNotNull(origPath);
  AssertPtrNotNull(strstr(origPath, "?.so"));

  addLuaPath(jInterp, "tests", TRUE);
  Symbol *testPath = getLuaPath(jInterp, TRUE);
  AssertPtrNotNull(testPath);

  AssertIntZero(strncmp(testPath, "tests;", strlen("tests;")));
  
  setLuaPath(jInterp, origPath, TRUE);
  free((void*)testPath);
  testPath = getLuaPath(jInterp, TRUE);
  AssertPtrNotNull(testPath);
  AssertIntZero(strcmp(origPath, testPath));
  
  free((void*)testPath);
  free((void*)origPath);
\stopCTest
\stopTestCase
\stopTestSuite

\startTestSuite[require lua modules]

\startCHeader
extern Symbol *requireLuaModule(
  JoyLoLInterp *jInterp,
  Symbol *aLuaModule
);
\stopCHeader

\startCCode
Symbol *requireLuaModule(
  JoyLoLInterp *jInterp,
  Symbol *aLuaModule
) {
  assert(jInterp);
  assert(jInterp->lstate);

  Symbol *result = NULL;
  lua_State *lstate = jInterp->lstate;
  lua_getglobal(lstate, "require");
  lua_pushstring(lstate, aLuaModule);
  if (lua_pcall(lstate, 1, 1, 0)) {
    // there was an error...
    // so return a copy of the error message
    result = strdup(lua_tostring(lstate, -1));
  }
  lua_pop(lstate, 1);
  return result;
}
\stopCCode

\startTestCase[should not be able to load 'noLuaModule']

\startCTest
  Symbol *result = requireLuaModule(jInterp, "noLuaModule");
  AssertPtrNotNull(result);
  Symbol *msg = "module 'noLuaModule' not found";
  AssertIntZero(strncmp(result, msg, strlen(msg)));
  free((void*)result);
\stopCTest

\stopTestCase
\stopTestSuite

\component gitVersion-c

\startCCode
static int lua_jInterps_getGitVersion (lua_State *lstate) {
  const char* aKey   = lua_tostring(lstate, 1);
  const char* aValue = "no valid key provided";
  if (aKey) aValue = getGitVersion(gitVersionKeyValues, aKey);
  lua_pushstring(lstate, aValue);
  return 1;
}
\stopCCode

\startCCode
static const struct luaL_Reg lua_jInterps [] = {
  {"gitVersion", lua_jInterps_getGitVersion},
  {NULL, NULL}
};
\stopCCode

\startCCode
static void registerJInterpsLuaInterface(lua_State *lstate) {
  assert(lstate);
  luaL_newlib(lstate, lua_jInterps);
}
\stopCCode

Now each shared library needs to implement the following code to provide 
an interface between the generic C-implementation and the Lua module for 
the shared library. 

\startCCode
int luaopen_joylol_jInterps (lua_State *lstate) {
  getJoyLoLInterpInto(lstate, jInterp);
  registerJInterps(jInterp);
  registerJInterpsLuaInterface(lstate);
  return 1;
}
\stopCCode

In some instances, such as the typical \type{CTest} program 
\type{allCTests}, this Lua module (which can be \type{require}d as a 
shared library) is actually statically linked into the executable. In 
these cases we need the ability to mimic the standard Lua \type{require} 
process. The following \type{requireStaticallyLinkedJInterps} does just this. 

\startCHeader
Boolean requireStaticallyLinkedJInterps(
  lua_State *lstate
);
\stopCHeader

\startCCode
Boolean requireStaticallyLinkedJInterps(
  lua_State *lstate
) {
  lua_getglobal(lstate, "package");
  lua_getfield(lstate, -1, "loaded");
  luaopen_joylol_jInterps(lstate);
  lua_setfield(lstate, -2, "joylol.jInterps");
  lua_setfield(lstate, -2, "loaded");
  lua_pop(lstate, 1);
  return TRUE;
}
\stopCCode
