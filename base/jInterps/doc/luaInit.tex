% A ConTeXt document [master document: jInterps.tex]

\section[title=Lua initialization functions]
\setCCodeStream{lua}

\startTestSuite[ set, add and get LuaPath]

\startCHeader
#define setLuaPath(lstate, aLuaPath, addToCPath)  \
  lua_getglobal(lstate, "package");               \
  lua_pushstring(lstate, aLuaPath);               \
  lua_setfield(lstate, -2,                        \
    (addToCPath ? "cpath" : "path"));             \
  lua_pop(lstate, 1)

#define addLuaPath(lstate, aLuaPath, addToCPath)              \
for(size_t iAddLuaPath = 0; iAddLuaPath < 1; iAddLuaPath++) { \
  luaL_Buffer lBuf;                                           \
  luaL_buffinit(lstate, &lBuf);                               \
  luaL_addstring(&lBuf, aLuaPath);                            \
  luaL_addstring(&lBuf, ";");                                 \
  lua_getglobal(lstate, "package");                           \
  lua_getfield(lstate, -1,                                    \
    (addToCPath ? "cpath" : "path"));                         \
  luaL_addvalue (&lBuf);                                      \
  luaL_pushresult(&lBuf);                                     \
  lua_setfield(lstate, -2,                                    \
    (addToCPath ? "cpath" : "path"));                         \
  lua_pop(lstate, 1);                                         \
}

#define getLuaPathInto(lstate, result, getCPath)  \
  lua_getglobal(lstate, "package");               \
  lua_getfield(lstate, -1,                        \
    (getCPath ? "cpath" : "path"));               \
  Symbol *result =                                \
    strdup(lua_tostring(lstate, -1));             \
  lua_pop(lstate, 2)

#define PATH_BUFFER_SIZE  8000

#define buildCWDPathInto(buffer, aDir)                  \
  char *buffer = (char*)calloc(1, PATH_BUFFER_SIZE);    \
  if (!getcwd(buffer, PATH_BUFFER_SIZE)) {              \
    /*return*/ luaL_error(lstate, "%s%s%s",             \
      "\nERROR:\n",                                     \
      "  Path buffer size too small\n",                 \
      "  while building CWD path\n\n");                 \
  }                                                     \
  strncat(buffer, "/",                                  \
    PATH_BUFFER_SIZE - strlen(buffer) - 1);             \
  strncat(buffer, aDir,                                 \
    PATH_BUFFER_SIZE - strlen(buffer) - 1)

#define addBuildLuaPath(lstate)            \
  buildCWDPathInto(buildLuaPath,           \
    "buildDir/?.lua");                     \
  addLuaPath(lstate, buildLuaPath, FALSE); \
  free(buildLuaPath);                      \
  buildCWDPathInto(buildCPath,             \
    "buildDir/?.so");                      \
  addLuaPath(lstate, buildCPath, TRUE);    \
  free(buildCPath)
  
#define buildHomePathInto(buffer, homeBuffer, aDir)     \
  char *buffer = (char*)calloc(1, PATH_BUFFER_SIZE);    \
  char *homeBuffer = getenv("HOME");                    \
  if (!homeBuffer) {                                    \
    /*return*/ luaL_error(lstate, "%s",                 \
      "\nERROR:\n",                                     \
      "  Environment variable 'HOME'\n",                \
      "  is empty!\n\n");                               \
  }                                                     \
  strncat(buffer, homeBuffer,                           \
    PATH_BUFFER_SIZE - strlen(buffer) - 1);             \
  strncat(buffer, "/",                                  \
    PATH_BUFFER_SIZE - strlen(buffer) - 1);             \
  strncat(buffer, aDir,                                 \
    PATH_BUFFER_SIZE - strlen(buffer) - 1)

#define addJoyLoLLuaPath(lstate)                 \
  buildHomePathInto(joyLoLLuaPath, homeLuaPath,  \
    ".joylol/?.lua");                            \
  addLuaPath(lstate, joyLoLLuaPath, FALSE);      \
  free(joyLoLLuaPath);                           \
  buildHomePathInto(joyLoLCPath, homeCPath,      \
    ".joylol/?.so");                             \
  addLuaPath(lstate, joyLoLCPath, TRUE);         \
  free(joyLoLCPath)
\stopCHeader

\startTestCase[should set, add and get package.path]

\startCTest
  getLuaPathInto(lstate, origPath, FALSE);
  AssertPtrNotNull(origPath);
  AssertPtrNotNull(strstr(origPath, "?.lua"));

  addLuaPath(lstate, "tests", FALSE);
  getLuaPathInto(lstate, testPath, FALSE);
  AssertPtrNotNull(testPath);

  AssertIntZero(strncmp(testPath, "tests;", strlen("tests;")));
  
  setLuaPath(lstate, origPath, FALSE);
  getLuaPathInto(lstate, newTestPath, FALSE);
  AssertPtrNotNull(newTestPath);
  AssertIntZero(strcmp(origPath, newTestPath));

  free((void*)testPath);
  free((void*)newTestPath);
  free((void*)origPath);
\stopCTest
\stopTestCase

\startTestCase[should set, add and get package.cpath]

\startCTest
  getLuaPathInto(lstate, origPath, TRUE);
  AssertPtrNotNull(origPath);
  AssertPtrNotNull(strstr(origPath, "?.so"));

  addLuaPath(lstate, "tests", TRUE);
  getLuaPathInto(lstate, testPath, TRUE);
  AssertPtrNotNull(testPath);

  AssertIntZero(strncmp(testPath, "tests;", strlen("tests;")));
  
  setLuaPath(lstate, origPath, TRUE);
  getLuaPathInto(lstate, newTestPath, TRUE);
  AssertPtrNotNull(newTestPath);
  AssertIntZero(strcmp(origPath, newTestPath));
  
  free((void*)testPath);
  free((void*)newTestPath);
  free((void*)origPath);
\stopCTest
\stopTestCase
\stopTestSuite

\startTestSuite[require lua modules]

\startCHeader
#define requireLuaModule(lstate, aLuaModule)  \
  lua_getglobal(lstate, "require");           \
  lua_pushstring(lstate, aLuaModule);         \
  if (lua_pcall(lstate, 1, 1, 0)) {           \
    /* there was an error...                  \
     * so return a copy of the error message  \
     */                                       \
    /*return*/ luaL_error(lstate,             \
      "Failed to load [%s]\n%s%s\n",          \
      aLuaModule,                             \
      "ERROR:\n",                             \
      lua_tostring(lstate, -1));              \
  }                                           \
  lua_pop(lstate, 1)
\stopCHeader

\startCHeader
#define requireLuaModuleInto(lstate, aLuaModule, result)  \
  Boolean result = TRUE;                                  \
  lua_getglobal(lstate, "require");                       \
  lua_pushstring(lstate, aLuaModule);                     \
  if (lua_pcall(lstate, 1, 1, 0)) {                       \
    /* there was an error... */                           \
    result = FALSE;                                       \
  }                                                       \
  lua_pop(lstate, 1)
\stopCHeader

\stopTestSuite

\setCHeaderStream{public}
\startCHeader
// modified from: Joe Rossi's initial question
// http://lua-users.org/lists/lua-l/2006-03/msg00335.html
#define stackDump(lstate, message)                \
  {                                               \
    int i = lua_gettop(lstate);                   \
    printf("%s ---- Stack Dump Started  ----\n",  \
      message);                                   \
    while( i ) {                                  \
      int aType = lua_type(lstate, i);            \
      switch (aType) {                            \
      case LUA_TSTRING:                           \
        printf("%d: [%s] (string)\n",             \
          i, lua_tostring(lstate, i));            \
        break;                                    \
      case LUA_TBOOLEAN:                          \
        printf("%d: %s (boolean)\n",              \
          i, (lua_toboolean(lstate, i) ?          \
            "true" : "false"));                   \
        break;                                    \
      case LUA_TNUMBER:                           \
        printf("%d: %g (number)\n",               \
          i, lua_tonumber(lstate, i));            \
        break;                                    \
      case LUA_TTABLE:                            \
        printf("%d: (%s)\n",                      \
          i, lua_typename(lstate, aType));        \
        lua_pushnil(lstate); /* first key */      \
        while (lua_next(lstate, i) != 0) {        \
          /* duplicate the key */                 \
          lua_pushvalue (lstate, -2);             \
          /* print:                           */  \
          /*  tostring of duplicate key at -1 */  \
          /*  type of value             at -2 */  \
          /*  type of key               at -3 */  \
          printf("   %s (%s) - %s\n",             \
              lua_tostring(lstate, -1),           \
              lua_typename(lstate,                \
                lua_type(lstate, -3)),            \
              lua_typename(lstate,                \
                lua_type(lstate, -2)));           \
          /* removes dupliate 'key' & 'value'; */ \
          /* keeps 'key' for next iteration    */ \
          lua_pop(lstate, 2);                     \
        }                                         \
        break;                                    \
      default:                                    \
        printf("%d: (%s)\n",                      \
          i, lua_typename(lstate, aType));        \
        break;                                    \
      }                                           \
      i--;                                        \
    }                                             \
    printf("%s ---- Stack Dump Finished ----\n",  \
      message);                                   \
  }
\stopCHeader

\startCCode
static void checkAllRequired(
  lua_State    *lstate,
  JoyLoLInterp *jInterp
) {
  // first check that core stdout and stderr are defined
  if (!(jInterp->writeStdOut) || !(jInterp->writeStdErr)) {
    /*return*/ luaL_error(lstate, "%s%s",
      "\nERROR:\n",
      "  Core output methods are missing\n");
  }
  
  // now check that all required CoAlg extensions 
  // have been registered
  //
  RequiredObjects *curCoAlg = requiredCoAlgs;
  for ( ; curCoAlg->name ; curCoAlg++) {
    size_t curTag = curCoAlg->tag;
    if (!getJClass(jInterp, curTag)) {
      /*return*/ luaL_error(lstate, "%s%s%s%s\n\n",
        "\nERROR:\n",
       "  Missing a required CoAlgebraic extension\n",
       "  need to require joylol.",
       curCoAlg->name);
    }
  }
  
  // finally check that all important interpreter 
  // objects have been assigned
  if (
    !(jInterp->dict) || 
    !(jInterp->loader) ||
    !(jInterp->rootCtx)
  ) {
      /*return*/ luaL_error(lstate, "%s%s",
        "\nERROR:\n",
        "  some required objects are missing\n\n");
  }
}

static int lua_jInterps_checkAllRequired(lua_State *lstate) {
  getJoyLoLInterpInto(lstate, jInterp);
  assert(jInterp);
  checkAllRequired(lstate, jInterp);
  return 0;
}

static int lua_jInterps_initializeAllLoaded(lua_State *lstate) {
  getJoyLoLInterpInto(lstate, jInterp);
  assert(jInterp);
  initializeAllLoadedImpl(lstate, jInterp);
  return 0;
}
\stopCCode

\startCCode
static const struct luaL_Reg lua_jInterps [] = {
  {"gitVersion",          lua_jInterps_getGitVersion},
  {"initializeAllLoaded", lua_jInterps_initializeAllLoaded},
  {"checkAllRequired",    lua_jInterps_checkAllRequired},
  {NULL, NULL}
};
\stopCCode

\startCCode
static void requireAllRequiredCoAlgs(
  lua_State    *lstate,
  JoyLoLInterp *jInterp
) {
  assert(lstate);
  assert(jInterp);
  assert(jInterp->coAlgs);
  
  // start by creating a master table
  lua_createtable(lstate, 0,
    NumRequiredCoAlgs + NumJoyLoLInterfaceFunctions + 5
  );                                                    // -3
  // add the joylol interface into the master (joylol) table
  luaL_setfuncs(lstate, lua_joylol_interface, 0);
  
  // now place jInterps into the master table
  lua_pushstring(lstate, "jInterps");                   // -2
  luaL_newlib(lstate, lua_jInterps);                    // -1
  lua_settable(lstate, -3);
  
  // now walk through the required coAlgs
  // requiring each in turn and 
  // placing it into the master table
  RequiredObjects *curCoAlg = requiredCoAlgs;
  for ( ; curCoAlg->name ; curCoAlg++) {
    size_t curTag = curCoAlg->tag;
    if (!(jInterp->coAlgs[curTag])) {
      if (jInterp->verbose) {
        char output[1000];
        memset(output, 0, 1000);
        snprintf(output, 999, 
          "    loading [joylol.%s]\n",
          curCoAlg->name);
        jInterp->writeStdErr(jInterp, output);
      }
      lua_pushstring(lstate, curCoAlg->name); // -2
      lua_getglobal(lstate, "require");
      luaL_Buffer modName;
      luaL_buffinit(lstate, &modName);
      luaL_addstring(&modName, "joylol.");
      luaL_addstring(&modName, curCoAlg->name);
      luaL_pushresult(&modName);
      if (lua_pcall(lstate, 1, 1, 0)) {       // -> -1
        /* return */ luaL_error(lstate,
          "Failed to load [%s]\n%s%s\n",
          curCoAlg->name,
          "ERROR:\n",
          lua_tostring(lstate, -1)
        );
      }
      lua_settable(lstate, -3);
      if (jInterp->verbose) {
        char output[1000];
        memset(output, 0, 1000);
        snprintf(output, 999,
          "    loaded [joylol.%s]\n",
          curCoAlg->name);
        jInterp->writeStdErr(jInterp, output);
      }
    }
  }  
}
\stopCCode

Now each shared library needs to implement the following code to provide 
an interface between the generic C-implementation and the Lua module for 
the shared library. 

\startCCode
int luaopen_joylol_jInterps (lua_State *lstate) {
  setJoyLoLInterpInto(lstate, jInterp);
  if (jInterp->verbose) {
    jInterp->writeStdErr(jInterp,
      "  loading [joylol.jInterps]\n"
    );
  }
  registerJInterps(jInterp);
  requireAllRequiredCoAlgs(lstate, jInterp);
  initializeAllLoaded(lstate, jInterp);
  checkAllRequired(lstate, jInterp);
  if (jInterp->verbose) {
    jInterp->writeStdErr(jInterp,
      "  loaded [joylol.jInterps]\n"
    );
  }
  return 1;
}
\stopCCode

In some instances, such as the typical \type{CTest} program 
\type{allCTests}, this Lua module (which can be \type{require}d as a 
shared library) is actually statically linked into the executable. In 
these cases we need the ability to mimic the standard Lua \type{require} 
process. The following \type{requireStaticallyLinkedJInterps} does just this. 

\setCHeaderStream{public}
\startCHeader
extern Boolean requireStaticallyLinkedJInterps(
  lua_State *lstate
);
\stopCHeader

\startCCode
Boolean requireStaticallyLinkedJInterps(
  lua_State *lstate
) {
  lua_getglobal(lstate, "package");
  lua_getfield(lstate, -1, "loaded");
  
  // fake a luaopen_joylol_jInterps 
  // BUT without the requireAllRequiredCoAlgs
  // AND without the checkAllRequired
  // AND without the initializeAllLoaded
  setJoyLoLInterpInto(lstate, jInterp);
  registerJInterps(jInterp);
  luaL_newlib(lstate, lua_jInterps);
  
  lua_setfield(lstate, -2, "joylol.jInterps");
  lua_setfield(lstate, -2, "loaded");
  lua_pop(lstate, 1);
  return TRUE;
}
\stopCCode
