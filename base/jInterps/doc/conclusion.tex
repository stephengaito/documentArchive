% A ConTeXt document [master document: coAlgs.tex]

\section[title=Conclusions]

\addDocumentDirectory{doc}

Before we actually we write out the \type{CoAlgs} C-header file, provide 
some standard definitions, the most important of which deal with 
memory allocation and alignment as well as a simple \type{DEBUG} system. 

\prependCHeader{public}
\startCHeader
#include <stdlib.h>
#include <string.h>
#include <inttypes.h>
#include <lua.h>
#include <lauxlib.h>
#include <lualib.h>

#ifndef JOYLOL_SYSTEM_CONFIG_PATH
#define JOYLOL_SYSTEM_CONFIG_PATH "/usr/local/etc/joyLoL"
#endif

#ifndef JOYLOL_SYSTEM_PLUGINS_PATH
#define JOYLOL_SYSTEM_PLUGINS_PATH "/usr/local/lib/joyLoL"
#endif

#ifndef JOYLOL_COALGS_INCREMENT
#define JOYLOL_COALGS_INCREMENT 10
#endif
 
#ifndef NULL
#define NULL (void*)0
#endif

#ifndef TRUE
#define TRUE 1
#endif
#ifndef FALSE
#define FALSE 0
#endif

#define MEM_ALIGNMENT ((size_t)0x7)
#define IS_MEM_ALIGNED(someMem) \
  (!( ((size_t)(someMem)) & (MEM_ALIGNMENT) ))

extern void* joyLoLCalloc0(size_t numItems, size_t sizeOfItem);
#define joyLoLCalloc(numItems, itemType) \
  (itemType*)joyLoLCalloc0((numItems), sizeof(itemType))

#ifndef NDEBUG
#include <stdio.h>
#define DEBUG(debugFlag, format, ... ) \
  if (debugFlag) { printf( format, __VA_ARGS__ ); fflush(stdout); }
#else
#define DEBUG(...)
#endif
\stopCHeader

\CHeaderIncludeGuard{public}{JOYLOL_COALG_COALGS}
\createCHeaderFile%
  {public}%
  {joylol/jInterps.h}%
  {}

\CHeaderIncludeGuard{private}{JOYLOL_COALG_COALGS_PRIVATE}
\createCHeaderFile%
  {private}%
  {joylol/jInterps-private.h}%
  {}

  
\setCCodeStream{handles}
\prependCCode{handles}
\startCCode
#include <stdlib.h>
#include <string.h>
#include <assert.h>

#include "joylol/jInterps.h"
#include "joylol/jInterps-private.h"
\stopCCode

\createCCodeFile%
  {handles}%
  {handles.c}%
  {}

\prependCCode{interpreter}
\startCCode
#include <stdlib.h>
#include <assert.h>

#include <joylol/jInterps.h>
\stopCCode

Now each shared library needs to implement the following code to provide 
an interface between the generic C-implementation and the Lua module for 
the shared library. 

\appendCCode{interpreter}
\startCCode
static int lua_jInterps_getGitVersion (lua_State *lstate) {
  const char* aKey   = lua_tostring(lstate, 1);
  const char* aValue = "no valid key provided";
  if (aKey) aValue = getGitVersion(gitVersionKeyValues, aKey);
  lua_pushstring(lstate, aValue);
  return 1;
}

static size_t joylol_register_jInterps(JoyLoLInterp *jInterp) {
  CoAlgebra* theCoAlg    = (CoAlgebra*) calloc(1, sizeof(CoAlgebra));
  theCoAlg->name         = strdup("JInterps");
  theCoAlg->registerFunc = joylol_register_jInterps;
  theCoAlg->equalityFunc = NULL;
  theCoAlg->printFunc    = NULL;
  theCoAlg->coAlgData    = NULL;
  registerCoAlgebra(jInterp, theCoAlg);
  
  // in future... register CoAlgs words with JoyLoL dictionary. 
  
  return TRUE;
}

static const struct luaL_Reg lua_jInterps [] = {
  {"gitVersion", lua_jInterps_getGitVersion},
  {NULL, NULL}
};

int luaopen_joylol_jInterps (lua_State *lstate) {
  JoyLoLInterp *jInterp = getJoyLoLInterp(lstate);
  joylol_register_jInterps(jInterp);
  luaL_newlib(lstate, lua_jInterps);
  return 1;
}
\stopCCode

\createCCodeFile%
  {interpreter}%
  {interpreter.c}%
  {}

\addCTestInclude{<joylol/jInterps.h>}
\addCTestInclude{<joylol/jInterps-private.h>}
\createCTestFile%
  {default}%
  {allCTests.c}%
  {}

\addMainDocument{jInterps.tex}
\addSubDocument{overview.tex} 
\addSubDocument{handles.tex}
\addSubDocument{interpreter.tex} 
\addSubDocument{conclusion.tex} 
\addConTeXtModuleDirectory{install}

\addCTestTargets{default}

\addJoyLoLTargets{default}

\compileLmsfile{default}

\createLmsfileFile%
  {default}%
  {lmsfile}%
  {-- A Lua Make System file}

