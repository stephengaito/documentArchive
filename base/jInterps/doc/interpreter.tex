% A ConTeXt document [master document: coAlgs.tex]

\section[title=JoyLoL interpreter]

\setCHeaderStream{public}
\setCCodeStream{interpreter}

In this section we concentrate on the C code required to implement the 
kernel JoyLoL CoAlgebra. The CoAlgebra, \type{CoAlg}, is the base of the 
JoyLoL interpreter. This CoAlgebra has two parts, a \quote{class} part, 
and an \quote{instance} part. This class part manages the loading, 
registration and listing of new \quote{external} CoAlgebras. The instance 
part, discussed in the next section, manages the creation and removal of 
CoAlgebra \quote{values} (more commonly know as \quote{object} instances). 


The ANSI-C implementation of a CoAlgebra is as a simple \type{struct} 
which contains the following items:

\startitemize

\item a \type{size_t} value which is unique for each registered CoAlgebra 
and hence acting as a test for identity, 

\item a \type{Symbol} value which provides a human readable \emph{name} 
for the CoAlgebra, 

\item a \type{C-function} which (recursively) tests for equality of two 
CoAlgebra instances of given type of CoAlgebra.

\item a \type{C-function} with (re)registers a given CoAlgebra with the 
JoyLoL interpreter. This registration function's primary purpose is two 
fold. It registers the CoAlgebra with the central collection of known 
CoAlgebras (to be discussed in the next subsection, below), as well as 
registers the individual JoyLoL functions provided by the CoAlgebra with 
the interpreter's dictionary of words. 

\stopitemize

\startCHeader
typedef size_t     Boolean;
typedef const char Symbol;


typedef size_t (CoAlgRegister)(JoyLoLInterp*);
typedef size_t (CoAlgEquality)(CoAlgebra*,
                               CoAlgHandle*,
                               CoAlgHandle*,
                               size_t);
typedef CoAlgHandle* (CoAlgPrint)(CoAlgHandle*);
typedef void *CoAlgData;

typedef struct coalgebra_struct {
  size_t         isA;
  Symbol*        name;
  CoAlgRegister* registerFunc;
  CoAlgEquality* equalityFunc;
  CoAlgPrint*    printFunc;
  CoAlgData      coAlgData;
} CoAlgebra;
\stopCHeader

\component gitVersion-c

\subsection[title=JoyLoLInterp structures]

The \type{JoyLoLInterp} structure provides access to all of the CoAlgebras 
known to a given JoyLoL interpreter. As such it will be accessed 
repeatedly, so should probably be accessible with as few memory accesses 
as possible. 

\startCHeader
typedef struct coalgebras_struct {
  size_t      num;
  size_t      maxNum;
  CoAlgebra** vec;
} CoAlgebras;

typedef struct joylol_interpreter_struct {
  HandleMemory hMem;
  CoAlgebras   coAlgs;
} JoyLoLInterp;
\stopCHeader

\startTestSuite[newJoyLoLInterp]

To be able to either test or use a JoyLoL interpreter, we must first be 
able to obtain a new one. The (privately defined) \type{newJoyLoLInterp} 
creates a new \type{JoyLoLInterp} structure and ensures it is properly 
initialized. Note that we \emph{could} have multiple instances of 
\type{JoyLoLInterp} in a running program. 

\setCHeaderStream{private}
\startCHeader
extern JoyLoLInterp* newJoyLoLInterp(void);
\stopCHeader
\setCHeaderStream{public}

\startCCode
JoyLoLInterp* newJoyLoLInterp(void) {
  // we need to create a new JoyLoLInterp structure...
  //
  JoyLoLInterp* jInterp = 
    (JoyLoLInterp*) calloc(1, sizeof(JoyLoLInterp));
  assert(jInterp);
    
  size_t maxNum =
    NumRequiredCoAlgs + JOYLOL_COALGS_INCREMENT;
  jInterp->coAlgs.num    = NumRequiredCoAlgs;
  jInterp->coAlgs.maxNum = maxNum;
  jInterp->coAlgs.vec    =
    (CoAlgebra**) calloc(maxNum, sizeof(CoAlgebra*));
  
  jInterp->hMem.freeHandles     = NULL;
  jInterp->hMem.rootHandleBlock = NULL; 
  return jInterp;
}
\stopCCode

\startTestCase[should create a valid JoyLoLInterp instance]

\startCTest
  JoyLoLInterp *jInterp = newJoyLoLInterp();
  AssertPtrNotNull(jInterp);
  
  AssertIntEquals(jInterp->coAlgs.num,
    NumRequiredCoAlgs);
  AssertIntEquals(jInterp->coAlgs.maxNum,
    NumRequiredCoAlgs + JOYLOL_COALGS_INCREMENT);
  AssertPtrNotNull(jInterp->coAlgs.vec);
  
  AssertPtrNull(jInterp->hMem.freeHandles);
  AssertPtrNull(jInterp->hMem.rootHandleBlock);
\stopCTest
\stopTestCase
\stopTestSuite

\startTestSuite[get Lua-state global JoyLoLInterp LightUserData]

We want the ability to have different JoyLoL interpreters for each Lua 
state. However, we also need fast access to the C-implementations of the 
loaded CoAlgebras registered with a given JoyLoL interpreter. To achieve 
this we use the Lua Registry with the \quote{well-known} key 
\type{joyLoLInterpKey}. We store a pointer to the JoyLoL interpreter 
\type{JoyLoLInterp}, as a LightUserData, in the Lua registry under the key 
\type{joyLoLInterpKey}, so that it is only accessible by the C 
implementation of any CoAlgebra. This ensures that the pointer to 
\type{JoyLoLInterp} is unique for any particular Lua state, but can be 
different for each distinct Lua state. 

This means that \type{joyLoLInterpKey} must be a symbol which is unique in 
the loaded executable, but has no meaningful value \emph{other than} its 
location in memory. It is declared external in \type{jInterp.h} and is 
declared \emph{only} in the \type{jInterp.so} itself. 

\startCHeader
extern void* joyLoLInterpKey;

extern JoyLoLInterp* getJoyLoLInterp(lua_State *lstate);
\stopCHeader

\startCCode
void* joyLoLInterpKey = NULL;

JoyLoLInterp* getJoyLoLInterp(lua_State *lstate) {
  lua_rawgetp(lstate, LUA_REGISTRYINDEX, (void *)&joyLoLInterpKey);
  JoyLoLInterp* jInterp = (JoyLoLInterp*)lua_touserdata (lstate, -1);
  lua_pop(lstate, 1);
  if (!jInterp) {
    // if jInterp has never been accessed before... 
    // create it now...
    jInterp = newJoyLoLInterp();
    lua_pushlightuserdata (lstate, (void *)jInterp);
    lua_rawsetp(lstate, LUA_REGISTRYINDEX, (void *)&joyLoLInterpKey);
  }
  return jInterp;
}
\stopCCode

\startTestCase[should get a Lua-State global JoyLoLInterp LightUserData] 

\startCTest
  JoyLoLInterp* jInterp = getJoyLoLInterp(lstate);
  AssertPtrNotNull(jInterp);
  AssertPtrNotNull(jInterp->coAlgs.vec);
  AssertIntEquals(jInterp->coAlgs.num, NumRequiredCoAlgs);
  AssertIntEquals(jInterp->coAlgs.maxNum,
    NumRequiredCoAlgs + JOYLOL_COALGS_INCREMENT);
  JoyLoLInterp * newJInterp = getJoyLoLInterp(lstate);
  AssertPtrNotNull(newJInterp);
  AssertPtrEquals(jInterp, newJInterp);
\stopCTest
\stopTestCase
\stopTestSuite

\startTestSuite[registerCoAlgebra]

\startCHeader
extern size_t registerCoAlgebra(
  JoyLoLInterp *jInterp,
  CoAlgebra *theCoAlgebra
);
\stopCHeader

\startCCode
size_t registerCoAlgebra(
  JoyLoLInterp *jInterp,
  CoAlgebra *theCoAlg
) {
  assert(jInterp);
  assert(theCoAlg);

  // we follow a policy of lazy management of the jInterp
  // if jInterp->coAlgs is too small we expand it
  for(size_t i = 0; i < jInterp->coAlgs.num; i++ ) {
    if (jInterp->coAlgs.vec[i]) {
      if (strcmp(theCoAlg->name,
        jInterp->coAlgs.vec[i]->name) == 0) {
    
        // a coAlgebra with this name has already been registered. 
        // so return its index...
        return i;
      }
    }
  }
  // NEED TO CHECK if this is a REQUIRED extension....
  
  if (jInterp->coAlgs.maxNum <= jInterp->coAlgs.num) {
    // we need to expand the existing coAlgs structure...
    //  
    CoAlgebras oldCoAlgs;
    oldCoAlgs.num    = jInterp->coAlgs.num;
    oldCoAlgs.maxNum = jInterp->coAlgs.maxNum;
    oldCoAlgs.vec    = jInterp->coAlgs.vec;
    
    size_t newMaxNumCoAlgs =
      oldCoAlgs.maxNum + JOYLOL_COALGS_INCREMENT;

    jInterp->coAlgs.vec =
      calloc(newMaxNumCoAlgs, sizeof(CoAlgebra*));
    assert(jInterp->coAlgs.vec);
    
    if (oldCoAlgs.vec) {
      memcpy(jInterp->coAlgs.vec,
        oldCoAlgs.vec,
        sizeof(CoAlgebra*)*oldCoAlgs.num);
      free(oldCoAlgs.vec);
    }
    
    jInterp->coAlgs.num    = oldCoAlgs.num;
    jInterp->coAlgs.maxNum = newMaxNumCoAlgs;
  }
  
  size_t newCoAlg = jInterp->coAlgs.num;
  
  jInterp->coAlgs.vec[newCoAlg]      = theCoAlg;
  jInterp->coAlgs.vec[newCoAlg]->isA = newCoAlg;
  
  jInterp->coAlgs.num++;
  
  return newCoAlg;
}
\stopCCode

\startTestCase[should add a lots of CoAlgs to a JoyLoLInterp]

We start by adding one single \type{CoAlgebra}. 

\startCTest
  JoyLoLInterp *jInterp = newJoyLoLInterp();

  char*          coAlgName        = strdup("newCoAlgA");
  char*          coAlgNameEnd     = coAlgName + strlen("newCoAlg");
  CoAlgRegister* fakeRegisterFunc = (CoAlgRegister*) 0x100;
  CoAlgEquality* fakeEqualityFunc = (CoAlgEquality*) 0x200;
  CoAlgData*     fakeCoAlgData    = (CoAlgData*)     0x300;
  
  CoAlgebra* aCoAlg    = (CoAlgebra*) calloc(1, sizeof(CoAlgebra));
  AssertPtrNotNull(aCoAlg);
  aCoAlg->name         = strdup("newCoAlgA");
  aCoAlg->registerFunc = fakeRegisterFunc;
  aCoAlg->equalityFunc = fakeEqualityFunc;
  aCoAlg->coAlgData    = fakeCoAlgData;
  
  size_t coAlgIdx = registerCoAlgebra(jInterp, aCoAlg);
  AssertIntEquals(coAlgIdx, NumRequiredCoAlgs);
  
  CoAlgebra** coAlgsVec = jInterp->coAlgs.vec;
  AssertPtrNotNull(coAlgsVec);
  AssertIntEquals(coAlgsVec[coAlgIdx]->isA, NumRequiredCoAlgs);
  AssertStrEquals(coAlgsVec[coAlgIdx]->name, "newCoAlgA");
  AssertPtrEquals(coAlgsVec[coAlgIdx]->registerFunc, fakeRegisterFunc);
  AssertPtrEquals(coAlgsVec[coAlgIdx]->equalityFunc, fakeEqualityFunc);
  AssertPtrEquals(coAlgsVec[coAlgIdx]->coAlgData,    fakeCoAlgData);
\stopCTest

Now we want to test the expansion of an existing CoAlgebras structure when 
the existing one runs out of \quote{space}. 

\startCTest
  AssertIntTrue(JOYLOL_COALGS_INCREMENT < 26);
  for(int i = 1; i < 26; i++) {
    *coAlgNameEnd        = 'A' + i;
    fakeRegisterFunc    += 1;
    fakeEqualityFunc    += 1;
    fakeCoAlgData       += 1;
    aCoAlg               = (CoAlgebra*) calloc(1, sizeof(CoAlgebra));
    aCoAlg->name         = strdup(coAlgName);
    aCoAlg->registerFunc = fakeRegisterFunc;
    aCoAlg->equalityFunc = fakeEqualityFunc;
    aCoAlg->coAlgData    = fakeCoAlgData;
    
    registerCoAlgebra(jInterp, aCoAlg);

    CoAlgebra** coAlgsVec = jInterp->coAlgs.vec;
    AssertPtrNotNull(coAlgsVec);
    size_t idx = NumRequiredCoAlgs + i;
    AssertIntEquals(coAlgsVec[idx]->isA, idx);
    AssertStrEquals(coAlgsVec[idx]->name, coAlgName);
    AssertPtrEquals(coAlgsVec[idx]->registerFunc, fakeRegisterFunc);
    AssertPtrEquals(coAlgsVec[idx]->equalityFunc, fakeEqualityFunc);
    AssertPtrEquals(coAlgsVec[idx]->coAlgData,    fakeCoAlgData);
  }
\stopCTest

Now we want to test the addition of CoAlgebra with an existing name does 
not change the number of registered CoAlgebras but simple returns the 
existing CoAlgebra index. 

\startCTest
  size_t oldNumCoAlgs  = jInterp->coAlgs.num;

  *coAlgNameEnd        = 'A' + 5;
  aCoAlg               = (CoAlgebra*) calloc(1, sizeof(CoAlgebra));
  aCoAlg->name         = strdup(coAlgName);
  aCoAlg->registerFunc = (CoAlgRegister*) 0x100 + 5;
  aCoAlg->equalityFunc = (CoAlgEquality*) 0x200 + 5;
  aCoAlg->coAlgData    = (CoAlgData*)     0x300 + 5;
  
  size_t anIndex = registerCoAlgebra(jInterp, aCoAlg);
  AssertIntEquals(anIndex, 5 + NumRequiredCoAlgs);
  AssertIntEquals(oldNumCoAlgs, jInterp->coAlgs.num);
\stopCTest
\stopTestCase
\stopTestSuite

\startTestSuite[getGitVersion]

We need a generic way of obtaining the git version of a given CoAlgebra. 
We do this in two parts. In the \type{bin} directory of any (collection 
of) CoAlgebra(s) there should be a pair of shell scripts. The 
\type{gitHook.sh} is a \type{bash} shell script which is run at various 
points in the git event cycle to capture the git version information into 
either a C-header file or a Lua file. These files can then be loaded as 
needed into either a C-implementation or required by Lua code, to provide 
information about the version of a given code artefact. The \type{bin} 
directory should also contain a \type{setupGitHooks} \type{bash} script 
which copies the \type{gitHook.sh} script into the correct places in the 
\type{.git} directory. 

For CoAlgebras implemented in ANSI-C, the following function will take the 
git version information in the \type{gitVersion.h} file which is 
statically linked in a given CoAlgabra's shared library and return the 
value associated with any valid key requested. 

\startCHeader
typedef struct keyValueStruct {
  const char *key;
  const char *value;
} KeyValues;

extern const char *getGitVersion(const KeyValues *gitVersion, 
                                 const char* gitVersionKey);
\stopCHeader

\startCCode
const char *getGitVersion(const KeyValues *gitVersion,
                          const char* gitVersionKey) {
  for(int i = 0 ; gitVersion[i].key ; i++) {
    if (strcmp(gitVersionKey, gitVersion[i].key) == 0) {
      return gitVersion[i].value;
    }
  }
  return "key not found";
}
\stopCCode

At the moment we can only (easily) test the \type{getGitVersion} function.

\startTestCase[should get value associated with keys]
\startCTest
  static const KeyValues testKVs[] = {
    { "key1", "value1" },
    { "key2", "value2" },
    { NULL,   NULL }
  };
  const char* aValue = getGitVersion(testKVs, "key1");
  AssertPtrNotNull(aValue);
  AssertStrEquals(aValue, "value1");
  aValue = getGitVersion(testKVs, "key2");
  AssertPtrNotNull(aValue);
  AssertStrEquals(aValue, "value2");
  aValue = getGitVersion(testKVs, "noKey");
  AssertPtrNotNull(aValue);
  AssertStrEquals(aValue, "key not found");
\stopCTest
\stopTestCase
\stopTestSuite

\subsection[title=Old tests]

\starttyping
  printf("\nStructure Sizes\n");
  printf("              void* = %zu bytes\n", sizeof(void*));
  printf("                int = %zu bytes\n", sizeof(int));
  printf("           long int = %zu bytes\n", sizeof(long int));
  printf("\n");

  PairAtom aPairAtom;
  printf("           PairAtom = %zu bytes\n", sizeof(PairAtom));
  printf("     PairAtom.coAlg = %zu bytes\n", sizeof(aPairAtom.coAlg));
  printf("       PairAtom.tag = %zu bytes\n", sizeof(aPairAtom.tag));
  printf("      PairAtom.pair = %zu bytes\n", sizeof(aPairAtom.pair));
  printf("  PairAtom.pair.car = %zu bytes\n", sizeof(aPairAtom.pair.car));
  printf("  PairAtom.pair.cdr = %zu bytes\n", sizeof(aPairAtom.pair.cdr));
  printf("   PairAtom.boolean = %zu bytes\n", sizeof(aPairAtom.boolean));
  printf("    PairAtom.symbol = %zu bytes\n", sizeof(aPairAtom.symbol));
  printf("      PairAtom.func = %zu bytes\n", sizeof(aPairAtom.func));
  printf("   PairAtom.natural = %zu bytes\n", sizeof(aPairAtom.natural));
  printf("\n");

  ListBlock aListBlock;
  printf("          ListBlock = %zu bytes\n", sizeof(ListBlock));
  printf("    ListBlock.block = %zu bytes\n", sizeof(aListBlock.block));
  printf("ListBlock.nextBlock = %zu bytes\n", sizeof(aListBlock.nextBlock));
  printf("\n");
\stoptyping

%\startsyntax

%\stopsyntax 

%\startinitialization

%\stopinitialization
