% A ConTeXt document [master document: jInterps.tex]

\section[title=JoyLoL interpreter]

\setCHeaderStream{public}
\setCCodeStream{interpreter}

In this section we concentrate on the C code required to implement the 
kernel JoyLoL CoAlgebra. The CoAlgebra, \type{CoAlg}, is the base of the 
JoyLoL interpreter. This CoAlgebra has two parts, a \quote{class} part, 
and an \quote{instance} part. This class part manages the loading, 
registration and listing of new \quote{external} CoAlgebras. The instance 
part, discussed in the next section, manages the creation and removal of 
CoAlgebra \quote{values} (more commonly know as \quote{object} instances). 

Any given implementation of the overall JoyLoL \emph{system} will consist 
of a number of CoAlgebraic extensions which are loaded into a JoyLoL 
\quote{core}. Any given core will provide some resources for the 
extensions to use. Before we can document the generic structure of an 
extension, we need to provide a method to allow the JoyLoL interpreter 
extension to obtain these resources from the core. 

\startTestSuite[get Lua-state global JoyLoL-Callback LightUserData]

Each JoyLoL core implementation will have a number of specific resources 
such as C-functions which it needs to register with the JoyLoL 
interpreter. In particular the core specific Input/Output will need to be 
available to the JoyLoL interpreter soon after it loads in order to allow 
it to correctly communicate with the core and hence the user. The 
\type{getJoyLoLCallbackInto} and \type{setJoyLoLCallbackFrom} macros 
provide this capability by registering a well known \quote{callback} 
C-function \emph{before} the JoyLoL interpreter is loaded. 

The callback C-function takes an unsigned integer as its sole argument and 
returns a \type{void*} to the returned resource. 

\startCHeader
typedef void *(JoyLoLCallback)(
  lua_State*,
  size_t
);
\stopCHeader

The currently well known resources are:

\setCHeaderStream{public}
\startCHeader
#define JoyLoLCallback_StdOutMethod   1
#define JoyLoLCallback_StdErrMethod   2
#define JoyLoLCallback_Verbose        3
#define JoyLoLCallback_Debug          4
#define JoyLoLCallback_Quiet          5
#define JoyLoLCallback_ConfigFile     6
#define JoyLoLCallback_UserPath       7
#define JoyLoLCallback_LocalPath      8
#define JoyLoLCallback_SystemPath     9
#define JoyLoLCallback_MinimalJoylol 10

typedef void (StdOutputMethod)(
  JoyLoLInterp *jInterp,
  Symbol       *aMessage
);
\stopCHeader

The core implementation can then use the \type{setJoyLoLCallbackFrom} 
macro to store the address of its specific callback C-function as a 
LightUserData in the Lua Registry at the \type{JoyLoLCallbackKey}. The 
JoyLoL interpreter, once it is loaded, can get the address of the callback 
C-function using the \type{getJoyLoLCallbackInto} macro. 

This means that \type{joyLoLCallbackKey} must be a unique value, but has no 
meaningful value \emph{other than} its uniqueness. To do this we 
transliterate the first 4 characters of the string \quote{JyLCallbackKey} 
into hexadecimal using the standard \type{ASCII} character codes. 

\startCHeader
//                           J y L C a l l b a c k K e y
#define joyLoLCallbackKey 0x4A794C43L

#define getJoyLoLCallbackInto(lstate, aCallback)  \
  lua_rawgetp(                                    \
    lstate,                                       \
    LUA_REGISTRYINDEX,                            \
    (void *)joyLoLCallbackKey                     \
  );                                              \
  JoyLoLCallback* aCallback =                     \
    (JoyLoLCallback*)lua_touserdata (lstate, -1); \
  lua_pop(lstate, 1);                             \
  if (!aCallback) {                               \
    /*return*/ luaL_error(lstate, "%s%s%s%s",     \
      "\nERROR:\n",                               \
      "  Could not get the Lua registered\n",     \
      "  JoyLoLCallback method!\n",               \
      "  Have you required joylol.core.xxx?\n");  \
  }
\stopCHeader

\startCHeader
#define setJoyLoLCallbackFrom(lstate, aCallback)    \
  lua_pushlightuserdata(lstate, (void *)aCallback); \
  lua_rawsetp(                                      \
    lstate,                                         \
    LUA_REGISTRYINDEX,                              \
    (void *)joyLoLCallbackKey                       \
  )
\stopCHeader

\startTestCase[should get a Lua-State global JoyLoL-Callback LightUserData] 

\startCTest
  // NOTE we MUST test using ctestsCallback 
  // OR we MUST reset the call back to ctestsCallback
  //
  JoyLoLCallback *origCallback = ctestsCallback;
  setJoyLoLCallbackFrom(lstate, origCallback);
  getJoyLoLCallbackInto(lstate, aCallback);
  AssertPtrNotNull(aCallback);
  AssertPtrEquals(origCallback, aCallback);
  getJoyLoLCallbackInto(lstate, newCallback);
  AssertPtrNotNull(newCallback);
  AssertPtrEquals(origCallback, newCallback);  
\stopCTest
\stopTestCase
\stopTestSuite

\section[title=CoAlgebra extensions]

The ANSI-C implementation of a CoAlgebra is as a simple \type{struct} 
which contains the following items:

\startitemize

\item a \type{size_t} value which is unique for each registered CoAlgebra 
and hence acting as a test for identity, 

\item a \type{Symbol} value which provides a human readable \emph{name} 
for the CoAlgebra, 

\item a \type{C-function} which (recursively) tests for equality of two 
CoAlgebra instances of given type of CoAlgebra.

\item a \type{C-function} with (re)registers a given CoAlgebra with the 
JoyLoL interpreter. This registration function's primary purpose is two 
fold. It registers the CoAlgebra with the central collection of known 
CoAlgebras (to be discussed in the next subsection, below), as well as 
registers the individual JoyLoL functions provided by the CoAlgebra with 
the interpreter's dictionary of words. 

\stopitemize

\startCHeader
typedef Boolean (JClassInitialize)(
  JoyLoLInterp*,
  JClass*
);

typedef Boolean (JObjEquality)(
  JoyLoLInterp *jInterp,
  JObj     *lolA,
  JObj     *lolB
);

typedef struct string_buffer_object_struct StringBufferObj;
typedef Boolean (JObjPrint)(
  JoyLoLInterp*,
  StringBufferObj*,
  JObj*
);

typedef JObj* (JObjCopy)(
  JoyLoLInterp*,
  JObj*
);

typedef Boolean (JObjRelease)(
  JoyLoLInterp*,
  JObj*
);

typedef void*   CoAlgData;

typedef struct joylol_class_struct {
  Symbol           *name;
  JoyLoLInterp     *jInterp;
  size_t            objectSize;
  size_t            numObjectsPerBlock;
  JObj             *freeObjects;
  ObjectBlock      *rootObjectBlock;
  JClassInitialize *initializeFunc;
  JObjEquality     *equalityFunc;
  JObjPrint        *printFunc;
  JObjCopy         *copyFunc;
  JObjRelease      *releaseFunc;
} JClass;

typedef JObj* (JInterpNewObject)(
  JoyLoLInterp *jInterp,
  TagType       aTag
);

typedef size_t (JInterpRegisterJClass)(
  JoyLoLInterp *jInterp,
  JClass   *theJClass
);

typedef struct crossCompiler_object_struct CrossCompilerObj;
typedef size_t (JInterpRegisterCrossCompiler)(
  JoyLoLInterp     *jInterp,
  CrossCompilerObj *aCompiler
);

typedef void (JInterpInitializeAllLoaded)(
  lua_State    *lstate,
  JoyLoLInterp *jInterp
);

typedef struct joylolinterp_class_struct {
  JClass                        super;
  JInterpNewObject             *newObjectFunc;
  JInterpRegisterJClass        *registerJClassFunc;
  JInterpRegisterCrossCompiler *registerCrossCompilerFunc;
  JInterpInitializeAllLoaded *initializeAllLoadedFunc;
} JoyLoLInterpClass;
\stopCHeader

\subsection[title=JoyLoLInterp structures]

The \type{JoyLoLInterp} structure provides access to all of the CoAlgebras 
known to a given JoyLoL interpreter. As such it will be accessed 
repeatedly, so should probably be accessible with as few memory accesses 
as possible. 

\startCHeader
typedef struct dictNode_object_struct DictNodeObj;

typedef struct loader_object_struct LoaderObj;
typedef struct context_object_struct ContextObj;
typedef struct dictionary_object_struct DictObj;
typedef struct crossCompiler_object_struct CrossCompilerObj;

typedef struct joylol_interpreter_struct {
  lua_State        *lstate;
  size_t            numCoAlgs;
  size_t            maxNumCoAlgs;
  JClass          **coAlgs;
  DictObj          *dict;
  LoaderObj        *loader;
  ContextObj       *rootCtx;
  size_t            numCompilers;
  size_t            maxNumCompilers;
  CrossCompilerObj **compilers;
  Boolean           verbose;
  Boolean           debug;
  Boolean           minimalJoylol;
  StdOutputMethod  *writeStdOut;
  StdOutputMethod  *writeStdErr;
} JoyLoLInterp;

#define getJClass(jInterp, aTag)     \
  (                                  \
    assert(jInterp),                 \
    assert(aTag),                    \
    assert((jInterp)->coAlgs),       \
    assert((jInterp)->coAlgs[aTag]), \
    ((jInterp)->coAlgs[aTag])        \
  )

#define getJInterpClass(jInterp)                            \
  ((JoyLoLInterpClass*)getJClass(jInterp, JInterpsTag))

typedef struct booleans_class_struct BooleansClass;
#define getBooleansClass(jInterp)                           \
  ((BooleansClass*)getJClass(jInterp, BooleansTag))

typedef struct cFunctions_class_struct CFunctionsClass;
#define getCFunctionsClass(jInterp)                         \
  ((CFunctionsClass*)getJClass(jInterp, CFunctionsTag))

typedef struct coAlgebras_class_struct CoAlgebrasClass;
#define getCoAlgebrasClass(jInterp)                         \
  ((CoAlgebrasClass*)getJClass(jInterp, CoAlgebrasTag))

typedef struct contexts_class_struct ContextsClass;
#define getContextsClass(jInterp)                           \
  ((ContextsClass*)getJClass(jInterp, ContextsTag))

typedef struct crossCompilers_class_struct CrossCompilersClass;
#define getCrossCompilersClass(jInterp)                     \
  ((CrossCompilersClass*)getJClass(jInterp, CrossCompilersTag))

typedef struct dictNodes_class_struct DictNodesClass;
#define getDictNodesClass(jInterp)                          \
  ((DictNodesClass*)getJClass(jInterp, DictNodesTag))

typedef struct dictionaries_class_struct DictionariesClass;
#define getDictionariesClass(jInterp)                       \
  ((DictionariesClass*)getJClass(jInterp, DictionariesTag))

typedef struct fragments_class_struct FragmentsClass;
#define getFragmentsClass(jInterp)                          \
  ((FragmentsClass*)getJClass(jInterp, FragmentsTag))

typedef struct implementations_class_struct ImplementationsClass;
#define getImplementationsClass(jInterp)                          \
  ((ImplementationsClass*)getJClass(jInterp, ImplementationsTag))

typedef struct loaders_class_struct LoadersClass;
#define getLoadersClass(jInterp)                            \
  ((LoadersClass*)getJClass(jInterp, LoadersTag))

typedef struct luaFunctions_class_struct LuaFunctionsClass;
#define getLuaFunctionsClass(jInterp)                       \
  ((LuaFunctionsClass*)getJClass(jInterp, LuaFunctionsTag))

typedef struct naturals_class_struct NaturalsClass;
#define getNaturalsClass(jInterp)                           \
  ((NaturalsClass*)getJClass(jInterp, NaturalsTag))

typedef struct pairs_class_struct PairsClass;
#define getPairsClass(jInterp)                              \
  ((PairsClass*)getJClass(jInterp, PairsTag))

typedef struct parsers_class_struct ParsersClass;
#define getParsersClass(jInterp)                            \
  ((ParsersClass*)getJClass(jInterp, ParsersTag))

typedef struct rules_class_struct RulesClass;
#define getRulesClass(jInterp)                            \
  ((RulesClass*)getJClass(jInterp, RulesTag))

typedef struct signals_class_struct SignalsClass;
#define getSignalsClass(jInterp)                      \
  ((SignalsClass*)getJClass(jInterp, SignalsTag))

typedef struct stringBuffers_class_struct StringBuffersClass;
#define getStringBuffersClass(jInterp)                      \
  ((StringBuffersClass*)getJClass(jInterp, StringBuffersTag))

typedef struct symbols_class_struct SymbolsClass;
#define getSymbolsClass(jInterp)                            \
  ((SymbolsClass*)getJClass(jInterp, SymbolsTag))
  
typedef struct templates_class_struct TemplatesClass;
#define getTemplatesClass(jInterp)                          \
  ((TemplatesClass*)getJClass(jInterp, TemplatesTag))

typedef struct texts_class_struct TextsClass;
#define getTextsClass(jInterp)                              \
  ((TextsClass*)getJClass(jInterp, TextsTag))
\stopCHeader

\startTestSuite[newJoyLoLInterp]

To be able to either test or use a JoyLoL interpreter, we must first be 
able to obtain a new one. The (privately defined) \type{newJoyLoLInterp} 
creates a new \type{JoyLoLInterp} structure and ensures it is properly 
initialized. Note that we \emph{could} have multiple instances of 
\type{JoyLoLInterp} in a running program. 

\setCHeaderStream{private}
\startCHeader
extern JoyLoLInterp* newJoyLoLInterp(lua_State *lstate);
\stopCHeader
\setCHeaderStream{public}

\startCCode
JoyLoLInterp* newJoyLoLInterp(lua_State *lstate) {
  // we need to create a new JoyLoLInterp structure...
  //
  JoyLoLInterp* jInterp = joyLoLCalloc(1, JoyLoLInterp);
  assert(jInterp);

  jInterp->lstate = lstate;
  
  // before we do anything else we need to install the
  // core's Output methods
  getJoyLoLCallbackInto(lstate, getCoreResources);
  assert(getCoreResources);
  jInterp->writeStdOut =
    getCoreResources(lstate, JoyLoLCallback_StdOutMethod);
  jInterp->writeStdErr =
    getCoreResources(lstate, JoyLoLCallback_StdErrMethod);
  if (!(jInterp->writeStdOut) || !(jInterp->writeStdErr)) {
    /*return*/ luaL_error(lstate, "%s%s",
      "\nERROR:\n",
      "  Could not get the core output methods\n");
  }
  jInterp->verbose =
    (Boolean)getCoreResources(lstate, JoyLoLCallback_Verbose);
    
  jInterp->debug = 
    (Boolean)getCoreResources(lstate, JoyLoLCallback_Debug);
    
  jInterp->minimalJoylol =
    (Boolean)getCoreResources(lstate, JoyLoLCallback_MinimalJoylol);

  DEBUG(jInterp, "Creating jInterp %p\n", jInterp);
    
  size_t maxNum =
    NumRequiredCoAlgs + JOYLOL_COALGS_INCREMENT;
  jInterp->numCoAlgs    = NumRequiredCoAlgs;
  jInterp->maxNumCoAlgs = maxNum;
  jInterp->coAlgs       = joyLoLCalloc(maxNum, JClass*);
  
  DEBUG(jInterp, "Created coAlgs vector %p %zu %zu\n",
    jInterp->coAlgs, jInterp->numCoAlgs, jInterp->maxNumCoAlgs);
  
  for (size_t i = 0; i < jInterp->maxNumCoAlgs; i++) {
    jInterp->coAlgs[i] = NULL;
  }

  DEBUG(jInterp, "Initialized coAlgs %p\n", jInterp);
  
  maxNum =
    NumRequiredCrossCompilers + JOYLOL_COMPILERS_INCREMENT;
  jInterp->numCompilers    = NumRequiredCrossCompilers;
  jInterp->maxNumCompilers = maxNum;
  jInterp->compilers       = joyLoLCalloc(maxNum, CrossCompilerObj*);

  DEBUG(jInterp, "Created crossCompilers vector %p %zu %zu\n",
    jInterp->compilers, 
    jInterp->numCompilers,
    jInterp->maxNumCompilers);

  for (size_t i = 0; i < jInterp->maxNumCompilers; i++) {
    jInterp->compilers[i] = NULL;
  }

  DEBUG(jInterp, "Initialized compilers %p\n", jInterp);

  jInterp->dict    = NULL;
  jInterp->loader  = NULL;
  jInterp->rootCtx = NULL;
    
  DEBUG(jInterp, "Initialized jInterp %p\n", jInterp);

  return jInterp;
}
\stopCCode

\startTestCase[should create a valid JoyLoLInterp instance]

\startCTest
  JoyLoLInterp *jInterp = newJoyLoLInterp(lstate);
  AssertPtrNotNull(jInterp);
  
  AssertIntEquals(jInterp->numCoAlgs,
    NumRequiredCoAlgs);
  AssertIntEquals(jInterp->maxNumCoAlgs,
    NumRequiredCoAlgs + JOYLOL_COALGS_INCREMENT);
  AssertPtrNotNull(jInterp->coAlgs);
  
  for (size_t i = 0; i < jInterp->maxNumCoAlgs; i++) {
    AssertPtrNull(jInterp->coAlgs[i]);
  }
  
  AssertPtrNull(jInterp->dict);
  AssertPtrNull(jInterp->loader);
\stopCTest
\stopTestCase
\stopTestSuite

\startTestSuite[get Lua-state global JoyLoLInterp LightUserData]

We want the ability to have different JoyLoL interpreters for each Lua 
state. However, we also need fast access to the C-implementations of the 
loaded CoAlgebras registered with a given JoyLoL interpreter. To achieve 
this we use the Lua Registry with the \quote{well-known} key 
\type{joyLoLInterpKey}. We store a pointer to the JoyLoL interpreter 
\type{JoyLoLInterp}, as a LightUserData, in the Lua registry under the key 
\type{joyLoLInterpKey}, so that it is only accessible by the C 
implementation of any CoAlgebra. This ensures that the pointer to 
\type{JoyLoLInterp} is unique for any particular Lua state, but can be 
different for each distinct Lua state. 

This means that \type{joyLoLInterpKey} must be a unique value, but has no 
meaningful value \emph{other than} its uniqueness. To do this we 
transliterate the first 4 characters of the string \quote{JoyLoLInterpKey} 
into hexadecimal using the standard \type{ASCII} character codes. 

\startCHeader
//                         J y L I n t e r p K e y
#define joyLoLInterpKey 0x4A794C49L

#define getJoyLoLInterpInto(lstate, jInterp)      \
  lua_rawgetp(                                    \
    lstate,                                       \
    LUA_REGISTRYINDEX,                            \
    (void *)joyLoLInterpKey                       \
  );                                              \
  JoyLoLInterp* jInterp =                         \
    (JoyLoLInterp*)lua_touserdata (lstate, -1);   \
  lua_pop(lstate, 1);                             \
  if (!jInterp) {                                 \
    /*return*/ luaL_error(lstate, "%s%s%s%s",     \
      "\nERROR:\n",                               \
      "  Could not get the Lua registered\n",     \
      "  JoyLoLInterp instance!\n",               \
      "  Have you required joylol.jInterps?\n");  \
  }
\stopCHeader

\setCHeaderStream{private}
\startCHeader
#define setJoyLoLInterpInto(lstate, jInterp)      \
  JoyLoLInterp *jInterp =                         \
    newJoyLoLInterp(lstate);                      \
  lua_pushlightuserdata(lstate, (void *)jInterp); \
  lua_rawsetp(                                    \
    lstate,                                       \
    LUA_REGISTRYINDEX,                            \
    (void *)joyLoLInterpKey                       \
  )
\stopCHeader
\setCHeaderStream{public}

\startTestCase[should get a Lua-State global JoyLoLInterp LightUserData] 

\startCTest
  getJoyLoLInterpInto(lstate, jInterp);
  AssertPtrNotNull(jInterp);
  AssertPtrEquals(jInterp->lstate, lstate);
  AssertPtrNotNull(jInterp->coAlgs);
  AssertIntEquals(jInterp->numCoAlgs, NumRequiredCoAlgs);
  AssertIntEquals(jInterp->maxNumCoAlgs,
    NumRequiredCoAlgs + JOYLOL_COALGS_INCREMENT);
  getJoyLoLInterpInto(lstate, newJInterp);
  AssertPtrNotNull(newJInterp);
  AssertPtrEquals(jInterp, newJInterp);
\stopCTest
\stopTestCase
\stopTestSuite

\startTestSuite[registerCoAlgebra]

\startCHeader
#define registerJClass(jInterp, aTag)                              \
  (                                                                    \
    assert(getJInterpClass(jInterp)->registerJClassFunc),          \
    (getJInterpClass(jInterp)->registerJClassFunc)(jInterp, aTag)  \
  )
\stopCHeader

\setCHeaderStream{private}
\startCHeader
size_t registerJClassImpl(
  JoyLoLInterp *jInterp,
  JClass *theCoAlg
);
\stopCHeader
\setCHeaderStream{public}

\startCCode
size_t registerJClassImpl(
  JoyLoLInterp *jInterp,
  JClass *theCoAlg
) {
  assert(jInterp);
  assert(jInterp->coAlgs);
  assert(theCoAlg);
  
  theCoAlg->jInterp = jInterp;
  
  for(size_t i = 0; i < jInterp->numCoAlgs; i++ ) {
    if (jInterp->coAlgs[i]) {
      if (strcmp(theCoAlg->name,
        jInterp->coAlgs[i]->name) == 0) {
    
        // a coAlgebra with this name has already been registered. 
        // so return its index...
        return i;
      }
    }
  }

  // now check to see if this is a required coAlg
  for (size_t i = 0; requiredCoAlgs[i].name; i++) {
    if (strcmp(theCoAlg->name, requiredCoAlgs[i].name) == 0) {
      size_t tag = requiredCoAlgs[i].tag;
      if (!(jInterp->coAlgs[tag])) {
        jInterp->coAlgs[tag] = theCoAlg;
      }
      return tag;
    }
  }

  // we follow a policy of lazy management of the jInterp
  // if jInterp->coAlgs is too small we expand it
  if (jInterp->maxNumCoAlgs <= jInterp->numCoAlgs) {
    // we need to expand the existing coAlgs structure...
    //  
    size_t oldnumCoAlgs     = jInterp->numCoAlgs;
    size_t oldmaxNumCoAlgs  = jInterp->maxNumCoAlgs;
    JClass **oldcoAlgs = jInterp->coAlgs;
    
    size_t newMaxNumCoAlgs =
      oldmaxNumCoAlgs + JOYLOL_COALGS_INCREMENT;

    jInterp->coAlgs =
      joyLoLCalloc(newMaxNumCoAlgs, JClass*);
    assert(jInterp->coAlgs);
    
    if (oldcoAlgs) {
      memcpy(jInterp->coAlgs,
        oldcoAlgs,
        sizeof(JClass*)*oldnumCoAlgs);
      free(oldcoAlgs);
    }
    
    jInterp->numCoAlgs    = oldnumCoAlgs;
    jInterp->maxNumCoAlgs = newMaxNumCoAlgs;
  }
  
  size_t newCoAlg = jInterp->numCoAlgs;
  
  jInterp->coAlgs[newCoAlg] = theCoAlg;
  
  jInterp->numCoAlgs++;
  
  return newCoAlg;
}
\stopCCode

\startTestCase[should add lots of CoAlgs to a JoyLoLInterp]

We start by adding one single \type{CoAlgebra}. 

\startCTest
  JoyLoLInterp *jInterp = newJoyLoLInterp(lstate);
  char*             coAlgName          = strdup("newCoAlgA");
  char*             coAlgNameEnd       = coAlgName + strlen("newCoAlg");
  JClassInitialize* fakeInitializeFunc = (JClassInitialize*) 0x100;
  JObjEquality*    fakeEqualityFunc    = (JObjEquality*)   0x200;
  JObjPrint*       fakePrintFunc       = (JObjPrint*)      0x300;
  
  JClass* aCoAlg   = joyLoLCalloc(1, JClass);
  AssertPtrNotNull(aCoAlg);
  aCoAlg->name           = strdup("newCoAlgA");
  aCoAlg->initializeFunc = fakeInitializeFunc;
  aCoAlg->equalityFunc   = fakeEqualityFunc;
  aCoAlg->printFunc      = fakePrintFunc;

  size_t coAlgIdx = registerJClassImpl(jInterp, aCoAlg);
  AssertIntEquals(coAlgIdx, NumRequiredCoAlgs);

  JClass** coAlgsVec = jInterp->coAlgs;
  AssertPtrNotNull(coAlgsVec);
  AssertStrEquals(coAlgsVec[coAlgIdx]->name, "newCoAlgA");
  AssertPtrEquals(coAlgsVec[coAlgIdx]->initializeFunc, fakeInitializeFunc);
  AssertPtrEquals(coAlgsVec[coAlgIdx]->equalityFunc, fakeEqualityFunc);
  AssertPtrEquals(coAlgsVec[coAlgIdx]->printFunc,    fakePrintFunc);
\stopCTest

Now we want to test the expansion of an existing CoAlgebras structure when 
the existing one runs out of \quote{space}. 

\startCTest
  AssertIntTrue(JOYLOL_COALGS_INCREMENT < 26);
  for(size_t i = 1; i < 26; i++) {
    *coAlgNameEnd          = 'A' + i;
    fakeInitializeFunc    += 1;
    fakeEqualityFunc      += 1;
    fakePrintFunc         += 1;
    aCoAlg                 = joyLoLCalloc(1, JClass);
    aCoAlg->name           = strdup(coAlgName);
    aCoAlg->initializeFunc = fakeInitializeFunc;
    aCoAlg->equalityFunc   = fakeEqualityFunc;
    aCoAlg->printFunc      = fakePrintFunc;

    registerJClassImpl(jInterp, aCoAlg);

    JClass** coAlgsVec = jInterp->coAlgs;
    AssertPtrNotNull(coAlgsVec);
    size_t idx = NumRequiredCoAlgs + i;
    AssertStrEquals(coAlgsVec[idx]->name, coAlgName);
    AssertPtrEquals(coAlgsVec[idx]->initializeFunc, fakeInitializeFunc);
    AssertPtrEquals(coAlgsVec[idx]->equalityFunc, fakeEqualityFunc);
    AssertPtrEquals(coAlgsVec[idx]->printFunc,    fakePrintFunc);
  }
\stopCTest

Now we want to test the addition of CoAlgebra with an existing name does 
not change the number of registered CoAlgebras but simple returns the 
existing CoAlgebra index. 

\startCTest
  size_t oldNumCoAlgs  = jInterp->numCoAlgs;

  *coAlgNameEnd          = 'A' + 5;
  aCoAlg                 = joyLoLCalloc(1, JClass);
  aCoAlg->name           = strdup(coAlgName);
  aCoAlg->initializeFunc = (JClassInitialize*) 0x100 + 5;
  aCoAlg->equalityFunc   = (JObjEquality*)     0x200 + 5;
  aCoAlg->printFunc      = (JObjPrint*)        0x300 + 5;
  
  size_t anIndex = registerJClassImpl(jInterp, aCoAlg);
  AssertIntEquals(anIndex, 5 + NumRequiredCoAlgs);
  AssertIntEquals(oldNumCoAlgs, jInterp->numCoAlgs);
\stopCTest
\stopTestCase
\stopTestSuite

\startTestSuite[registerCrossCompiler]

\startCHeader
#define registerCrossCompiler(jInterp, aTag)      \
  (                                               \
    assert(getJInterpClass(jInterp)               \
      ->registerCrossCompilerFunc),               \
    (getJInterpClass(jInterp)                     \
      ->registerCrossCompilerFunc)(jInterp, aTag) \
  )
\stopCHeader

\setCHeaderStream{private}
\startCHeader
size_t registerCrossCompilerImpl(
  JoyLoLInterp     *jInterp,
  CrossCompilerObj *aCompiler
);
\stopCHeader
\setCHeaderStream{public}

\startCCode
size_t registerCrossCompilerImpl(
  JoyLoLInterp     *jInterp,
  CrossCompilerObj *aCompiler
) {
  assert(jInterp);
  assert(jInterp->compilers);
  assert(aCompiler);
  
  for(size_t i = 0; i < jInterp->numCompilers; i++ ) {
    if (jInterp->compilers[i]) {
      if (strcmp(aCompiler->type,
        jInterp->compilers[i]->type) == 0) {
    
        // a compiler with this name has already been registered. 
        // so return its index...
        return i;
      }
    }
  }

  // now check to see if this is a required coAlg
  for (size_t i = 0; requiredCompilers[i].name; i++) {
    if (strcmp(aCompiler->type, requiredCompilers[i].name) == 0) {
      size_t tag = requiredCompilers[i].tag;
      if (!(jInterp->compilers[tag])) {
        jInterp->compilers[tag] = aCompiler;
      }
      return tag;
    }
  }

  // we follow a policy of lazy management of the jInterp
  // if jInterp->compilers is too small we expand it
  if (jInterp->maxNumCompilers <= jInterp->numCompilers) {
    // we need to expand the existing compilers structure...
    //  
    size_t oldNumCompilers          = jInterp->numCompilers;
    size_t oldMaxNumCompilers       = jInterp->maxNumCompilers;
    CrossCompilerObj** oldCompilers = jInterp->compilers;
    
    size_t newMaxNumCompilers =
      oldMaxNumCompilers + JOYLOL_COMPILERS_INCREMENT;

    jInterp->compilers =
      joyLoLCalloc(newMaxNumCompilers, CrossCompilerObj*);
    assert(jInterp->compilers);
    
    if (oldCompilers) {
      memcpy(jInterp->compilers,
        oldCompilers,
        sizeof(CrossCompilerObj*)*oldNumCompilers);
      free(oldCompilers);
    }
    
    jInterp->numCompilers    = oldNumCompilers;
    jInterp->maxNumCompilers = newMaxNumCompilers;
  }
  
  size_t newCompiler = jInterp->numCompilers;
  
  jInterp->compilers[newCompiler] = aCompiler;
  
  jInterp->numCompilers++;
  
  return newCompiler;
}
\stopCCode

\startTestCase[should add lots of Compilers to a JoyLoLInterp]

We start by adding one single \type{Compiler}. 

\startCTest
  JoyLoLInterp *jInterp = newJoyLoLInterp(lstate);
  char*             compilerType       = strdup("newCompilerA");
  char*             compilerTypeEnd    = compilerType + strlen("newCompiler");
  
  CrossCompilerObj* aCompiler = joyLoLCalloc(1, CrossCompilerObj);
  AssertPtrNotNull(aCompiler);
  aCompiler->type               = strdup("newCompilerA");

  size_t compilerIdx = registerCrossCompilerImpl(jInterp, aCompiler);
  AssertIntEquals(compilerIdx, NumRequiredCrossCompilers);

  CrossCompilerObj** compilers = jInterp->compilers;
  AssertPtrNotNull(compilers);
  AssertStrEquals(compilers[compilerIdx]->type, "newCompilerA");
\stopCTest

Now we want to test the expansion of an existing CoAlgebras structure when 
the existing one runs out of \quote{space}.

\startCTest
  AssertIntTrue(JOYLOL_COALGS_INCREMENT < 26);
  for(size_t i = 1; i < 26; i++) {
    *compilerTypeEnd = 'A' + i;
    aCompiler        = joyLoLCalloc(1, CrossCompilerObj);
    aCompiler->type  = strdup(compilerType);

    registerCrossCompilerImpl(jInterp, aCompiler);

    CrossCompilerObj** compilers = jInterp->compilers;
    AssertPtrNotNull(compilers);
    size_t idx = NumRequiredCrossCompilers + i;
    AssertStrEquals(compilers[idx]->type, compilerType);
  }
\stopCTest

Now we want to test the addition of CoAlgebra with an existing name does 
not change the number of registered CoAlgebras but simple returns the 
existing CoAlgebra index. 

\startCTest
  size_t oldNumCompilers = jInterp->numCompilers;

  *compilerTypeEnd       = 'A' + 5;
  aCompiler              = joyLoLCalloc(1, CrossCompilerObj);
  aCompiler->type        = strdup(compilerType);
  
  size_t anIndex = registerCrossCompilerImpl(jInterp, aCompiler);
  AssertIntEquals(anIndex, 5 + NumRequiredCrossCompilers);
  AssertIntEquals(oldNumCompilers, jInterp->numCompilers);
\stopCTest
\stopTestCase
\stopTestSuite

\startTestSuite[initializeAllLoaded]

\setCHeaderStream{public}
\startCHeader
#define initializeAllLoaded(lstate, jInterp)      \
  (                                               \
    assert(getJInterpClass(jInterp)               \
      ->initializeAllLoadedFunc),                 \
    (getJInterpClass(jInterp)                     \
      ->initializeAllLoadedFunc(lstate, jInterp)) \
  )
\stopCHeader

\setCHeaderStream{private}
\startCHeader
extern void initializeAllLoadedImpl(
  lua_State    *lstate,
  JoyLoLInterp *jInterp
);
\stopCHeader
\setCHeaderStream{public}

\startCCode
void initializeAllLoadedImpl(
  lua_State    *lstate,
  JoyLoLInterp *jInterp
) {
  assert(lstate);
  assert(jInterp);
  assert(jInterp->coAlgs);
  
  // now call the initialization function for each 
  // loaded CoAlgebraic extension
  for( size_t i = 1; i < NumRequiredCoAlgs; i++) {
    if (jInterp->coAlgs[i]) {
      JClass *aCoAlg = getJClass(jInterp, i);
      if (jInterp->verbose) {
        char output[1000];
        memset(output, 0, 1000);
        snprintf(output, 999, 
          "    initializing [joylol.%s]\n",
          aCoAlg->name);
        jInterp->writeStdOut(jInterp, output);
      }
      assert(aCoAlg->initializeFunc);
      aCoAlg->initializeFunc(jInterp, aCoAlg);
      if (jInterp->verbose) {
        char output[1000];
        memset(output, 0, 1000);
        snprintf(output, 999, 
          "    initialized [joylol.%s]\n",
          aCoAlg->name);
        jInterp->writeStdOut(jInterp, output);
      }
    }
  }
}
\stopCCode

\stopTestSuite

\startTestSuite[getGitVersion]

We need a generic way of obtaining the git version of a given CoAlgebra. 
We do this in two parts. In the \type{bin} directory of any (collection 
of) CoAlgebra(s) there should be a pair of shell scripts. The 
\type{gitHook.sh} is a \type{bash} shell script which is run at various 
points in the git event cycle to capture the git version information into 
either a C-header file or a Lua file. These files can then be loaded as 
needed into either a C-implementation or required by Lua code, to provide 
information about the version of a given code artefact. The \type{bin} 
directory should also contain a \type{setupGitHooks} \type{bash} script 
which copies the \type{gitHook.sh} script into the correct places in the 
\type{.git} directory. 

For CoAlgebras implemented in ANSI-C, the following function will take the 
git version information in the \type{gitVersion.h} file which is 
statically linked in a given CoAlgabra's shared library and return the 
value associated with any valid key requested. 

\startCHeader
typedef struct keyValueStruct {
  const char *key;
  const char *value;
} KeyValues;

#define getGitVersionInto(gitVersion, gitVersionKey, theValue)  \
  Symbol* theValue = "key not found";                           \
  for(size_t i = 0 ; gitVersion[i].key ; i++) {                 \
    if (strcmp(gitVersionKey, gitVersion[i].key) == 0) {        \
      theValue = gitVersion[i].value;                           \
      break;                                                    \
    }                                                           \
  }
\stopCHeader

At the moment we can only (easily) test the \type{getGitVersion} function.

\startTestCase[should get value associated with keys]
\startCTest
  static const KeyValues testKVs[] = {
    { "key1", "value1" },
    { "key2", "value2" },
    { NULL,   NULL }
  };
  getGitVersionInto(testKVs, "key1", aValue0);
  AssertPtrNotNull(aValue0);
  AssertStrEquals(aValue0, "value1");
  getGitVersionInto(testKVs, "key2", aValue1);
  AssertPtrNotNull(aValue1);
  AssertStrEquals(aValue1, "value2");
  getGitVersionInto(testKVs, "noKey", aValue2);
  AssertPtrNotNull(aValue2);
  AssertStrEquals(aValue2, "key not found");
\stopCTest
\stopTestCase
\stopTestSuite

\startCCode
static Boolean initializeJInterps(
  JoyLoLInterp *jInterp,
  JClass   *aJClass
) {
  assert(jInterp);
  assert(aJClass);
  registerJInterpWords(jInterp);
  return TRUE;
}
\stopCCode

\startTestSuite[regiserJInterps]

\setCHeaderStream{private}
\startCHeader
extern Boolean registerJInterps(JoyLoLInterp *jInterp);
\stopCHeader
\setCHeaderStream{public}

\startCCode
Boolean registerJInterps(JoyLoLInterp *jInterp) {
  assert(jInterp);
  
  JoyLoLInterpClass* theCoAlg =
    joyLoLCalloc(1, JoyLoLInterpClass);
  assert(theCoAlg);
  
  theCoAlg->super.name             = JInterpsName;
  theCoAlg->super.objectSize       = sizeof(JoyLoLInterp);
  theCoAlg->super.initializeFunc   = initializeJInterps;
  theCoAlg->super.equalityFunc     = NULL;
  theCoAlg->super.printFunc        = NULL;
  theCoAlg->newObjectFunc          = newObjectImpl;
  theCoAlg->registerJClassFunc     = registerJClassImpl;
  theCoAlg->registerCrossCompilerFunc =
    registerCrossCompilerImpl;
  theCoAlg->initializeAllLoadedFunc =
    initializeAllLoadedImpl;
    
  size_t tag =     
    registerJClassImpl(jInterp, (JClass*)theCoAlg);
  
  // do a sanity check...
  assert(tag == JInterpsTag);
  assert(tag < jInterp->numCoAlgs);
  assert(jInterp->coAlgs);
  assert(jInterp->coAlgs[tag]);
    
  return TRUE;
}
\stopCCode

\startTestCase[should register the JInterps coAlg]

\startCTest
  // CTestsSetup has already created a jInterp 
  // and run registerJInterps
  //
  AssertPtrNotNull(jInterp);
  AssertPtrNotNull(jInterp->coAlgs);
  AssertIntTrue(JInterpsTag < jInterp->numCoAlgs);
  AssertPtrNotNull(getJInterpClass(jInterp));
  
  JoyLoLInterpClass *jInterpClass = getJInterpClass(jInterp);
  size_t result = registerJInterps(jInterp);
  AssertIntTrue(result);
  AssertPtrNotNull(getJInterpClass(jInterp));
  AssertPtrEquals(getJInterpClass(jInterp), jInterpClass);
  AssertIntEquals(
    getJInterpClass(jInterp)->super.objectSize,
    sizeof(JoyLoLInterp)
  )
\stopCTest
\stopTestCase
\stopTestSuite

