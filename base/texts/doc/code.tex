% A ConTeXt document [master document: texts.tex]

\section[title=Code]

\dependsOn[jInterps]
%\dependsOn[context]

\component gitVersion-c

\starttyping
// texts.h
#ifndef JOYLOL_TEXT_H
#define JOYLOL_TEXT_H

typedef struct list_memory_struct ListMemory;
typedef struct pair_atom_struct PairAtom;
typedef struct avl_node_struct AVLNode;
typedef struct dictionary_struct Dictionary;

struct text_struct;

typedef void (NextLineFunc)(struct text_struct*);

typedef struct text_struct {
  //
  // fields used by all backing types
  //
  size_t        completed;
  PairAtom*     sym;
  const char*   curLine;
  const char*   curChar;
  const char*   lastChar;
  NextLineFunc* nextLine;
  CoAlgebras*   coAlgebras;
  //
  // array of strings specific fields
  //
  const char** textLines;
  size_t curLineNum;
  //
  // external file specific fields
  //
  FILE* inputFile;
  //
  // readline specific fields
  //
  const char* newPrompt;
  const char* continuePrompt;
  const char* curPrompt;
  AVLNode*    curNode;
  const char* curCompletionText;
  size_t      curCompletionLen;
} Text;


extern Text* createTextFromArrayOfStrings(CoAlgebras* coAlgs,
                                          const char* someTextLines[]);

extern Text* createTextFromString(CoAlgebras* coAlgs,
                                  const char* aString);

extern Text* createTextFromInputFile(CoAlgebras* coAlgs,
                                     FILE *anInputFile);

extern Text* createTextFromReadline(CoAlgebras* coAlgs);

extern void freeText(Text* aText);

extern void setReadlinePrompts(Text* aText,
  const char* newPrompt, const char* continuePrompt);
extern void clearReadlinePrompt(Text* aText);
extern void saveReadlineHistory(Text* aText);

extern void nextSymbol(Text* aText);
extern void reportError(Text* aText, const char* message);

#endif
\stoptyping

\starttyping
// texts-private.h
#ifndef JOYLOL_TEXT_PRIVATE_H
#define JOYLOL_TEXT_PRIVATE_H

extern void nextLineFromString(Text* aText);
extern void nextLineFromArray(Text* aText);
extern void nextLineFromFile(Text* aText);
extern void nextLineFromReadline(Text* aText);
extern void nextLineExit(Text* aText);

#endif
\stoptyping

\starttyping
// texts.c
#include <stdlib.h>
#include <string.h>
#include <assert.h>

#include "joyLoL/macros.h"
#include "joyLoL/coAlg/coAlgs.h"
#include "joyLoL/lists.h"
#include "joyLoL/text.h"
#include "joyLoL/text_private.h"
#include "joyLoL/dictionary.h"

// Texts are a collection of characters, which are used by the Parser
// to extract successive symbols.
//
// Texts are created on one of three backing suppliers of characters:
// 1. a fixed array of strings
// 2. an external file
// 3. a readline interaction with a user
//
// In all three cases, the Parser's nextSymbol method requests successive
// **lines** of characters (deliminated by new-line-characters).
//
// It is critical, for correct interaction with the user via readline,
// that the initial line is NOT obtained until actually requested by
// the parser's nextSymbol method.
//
// It is also critical that once completed, none of the sources, get
// asked for subsequent lines.
//
// When the text has been completed, the nextLine function ensures
// that aText->curLine is NULL.

void freeText(Text* aText) {
  if (!aText) return;

  free(aText);
}

void nextLineExit(Text* aText) {
  if (!aText) return; // there is nothing we can do!
  aText->sym       = NULL;
  aText->curLine   = NULL;
  aText->curChar   = NULL;
  aText->lastChar  = NULL;
  aText->completed = TRUE;
}

/////////////////////////////////////////////////////////////////////////////
// nextSymbol
//

void nextSymbol(Text* aText) {
  assert(aText);
  assert(aText->coAlgebras);
  assert(aText->coAlgebras->symbols);
  Dictionary* mainDic = aText->coAlgebras->symbols->dictionary;
  assert(mainDic);

  DEBUG(FALSE, "->nextSymbol [%s]{%s}\n", aText->curLine, aText->curChar);
  aText->sym = NULL;

  // ensure we have a non-empty line
  while ((!aText->completed) && (aText->curChar == aText->lastChar)) {
    aText->nextLine(aText);
  }
  if (!aText->curLine) {
    DEBUG(FALSE, "<-nextSymbol End of Text %s\n","");
    return; // we have exhausted this text
  }

  size_t      parsingNatural = TRUE;
  const char* symStart = aText->curChar;
  const char* symEnd   = aText->curChar;
  const char* lastChar = aText->lastChar;
  while (symStart == symEnd) {
    while (symEnd < lastChar) {
      if (*symEnd < 33) {
        //
        // white space
        //
        if (symStart != symEnd)  break; // we have found a symbol
        //
        // ignore whitespace (of any type)
        //
        symStart++;
        symEnd++;
        //
      } else if (
        *symEnd == '('  || *symEnd == ')'  ||
        *symEnd == '<'  || *symEnd == '>'  ||
        *symEnd == '['  || *symEnd == ']'  ||
        *symEnd == '{'  || *symEnd == '}' ) {
        //
        // '(', ')', '<', '>', '[', ']'
        //
        if (symStart != symEnd) break; // we have found a symbol
        //
        // '(' ')' are symbols in their own right
        //
        parsingNatural = FALSE;
        symEnd++;
        break;
        //
      } else if ( *symEnd == ';') {
        //
        if (symStart != symEnd) break; // we have found a symbol
        //
        const char* symEndP1 = symEnd + 1;
        if ( symEndP1 < lastChar && *symEndP1 == ';') {
          //
          // we have found the begining of a single line comment
          // ... so restart search for the next symbol on the next line
          //
          aText->nextLine(aText);
          if (!aText->curLine) return; // no more symbols
          symStart = aText->curChar;
          symEnd   = aText->curChar;
          lastChar = aText->lastChar;
        } else {
          //
          // ';' is a symbol in its own right
          //
          parsingNatural = FALSE;
          symEnd++;
          break;
          //
        }
        //
      } else {
        //
        // any other character is part of a symbol
        //
        // check to see if we are still parsing a natural
        //
        if ('0' <= *symEnd && *symEnd <= '9') {
          // do nothing
        } else {
          parsingNatural = FALSE;
        }
        //
        symEnd++;
        //
      }
    }

    if (symStart == symEnd) {
      aText->nextLine(aText);
      if (!aText->curLine) return; // no more symbols
      symStart = aText->curChar;
      symEnd   = aText->curChar;
      lastChar = aText->lastChar;
    }
  }

  aText->curChar = symEnd;
  char* aSymbol = strndup(symStart, symEnd - symStart); // TODO this thrashes memory ;-(
  DEBUG(FALSE, "--nextSymbol == [%s]\n", aSymbol);

  // check to see if this is the exit or quit symbol ...
  // ... and if so rest the nextLine function
  if (strcmp(aSymbol, "exit") == 0 || strcmp(aSymbol, "quit") == 0) {
    aText->nextLine = nextLineExit;
  }

  if (parsingNatural) {
    DEBUG(FALSE, "--nextSymbol == %s (natural)\n", aSymbol);
    // we have parsed a natural... so create a new natural
    //
    aText->sym = newNatural(aText->coAlgebras, aSymbol);
    //
    DEBUG(FALSE, "<-nextSymbol %s (natural)\n", aSymbol);
  } else {
    // we have parsed a symbol... so create a new symbol
    //
    AVLNode* aSym = createSymbol(mainDic, aSymbol);
    aText->sym = createAtom(aText->coAlgebras, aSym);
    //
    DEBUG(FALSE, "<-nextSymbol [%s] (symbol)\n", aText->sym->symbol);
  }
  free(aSymbol);
}

void reportError(Text *aText, const char* message) {
  fprintf(stderr, "\n\n%s\n", message);
  if (aText->curLine) {
    fprintf(stderr, "while parsing\n");
    fprintf(stderr, "\t[%s]\n", aText->curLine);
    fprintf(stderr, "at character %zu\n\n", aText->curChar - aText->curLine);
  } else {
    fprintf(stderr, "at end of text\n\n");
  }
}
\stoptyping

\starttyping
// arrayOfStrings.c
#include <stdlib.h>
#include <string.h>
#include <assert.h>

#include "joyLoL/macros.h"
#include "joyLoL/coAlg/coAlgs.h"
#include "joyLoL/lists.h"
#include "joyLoL/text.h"
#include "joyLoL/text_private.h"

// Texts are a collection of characters, which are used by the Parser
// to extract successive symbols.
//
// Texts are created on one of three backing suppliers of characters:
// 1. a single string
// 2. a NULL terminated array of strings
// 3. an external file
// 4. a readline interaction with a user
//
// In all four cases, the Parser's nextSymbol method requests successive
// **lines** of characters (deliminated by new-line-characters).
//
// It is critical, for correct interaction with the user via readline,
// that the initial line is NOT obtained until actually requested by
// the parser's nextSymbol method.
//
// It is also critical that once completed, none of the sources, get
// asked for subsequent lines.
//
// When the text has been completed, the nextLine function ensures
// that aText->curLine is NULL.

/////////////////////////////////////////////////////////////////////////////
// Text from array of strings
//

void nextLineFromArray(Text* aText) {
  DEBUG(FALSE, "->nextLineFromArray %s\n", "");
  if (!aText) return;  // there is nothing we can do!
  if (!aText->textLines) return; // there is nothing we can do!

  if (!aText->textLines[aText->curLineNum]) {
    // we have already reached the end of the text
    aText->completed  = TRUE;
    aText->curLine    = NULL;
    aText->curChar    = NULL;
    aText->lastChar   = NULL;
    return;
  }

  aText->curLine  = aText->textLines[aText->curLineNum];

  aText->curChar  = aText->curLine;

  if (aText->curLine) {
    aText->lastChar = aText->curChar + strlen(aText->curLine);
  } else aText->lastChar = aText->curChar;

  aText->curLineNum++;
}

Text* createTextFromArrayOfStrings(CoAlgebras* coAlgs,
                                   const char* someTextLines[]) {
  Text* aText = (Text*)calloc(1, sizeof(Text));
  assert(aText);
  //
  // array of strings specific initializations
  //
  aText->textLines  = someTextLines;
  aText->curLineNum = 0;
  aText->nextLine   = nextLineFromArray;
  //
  // general initializations
  //
  assert(coAlgs);
  aText->coAlgebras = coAlgs;
  aText->completed  = FALSE;
  aText->sym        = NULL;
  aText->curLine    = NULL;
  aText->curChar    = NULL;
  aText->lastChar   = NULL;
  //
  aText->inputFile  = NULL;
  //
  aText->newPrompt       = NULL;
  aText->continuePrompt  = NULL;
  aText->curPrompt       = NULL;

  return aText;
}
\stoptyping

\starttyping
// files.c
#include <stdlib.h>
#include <string.h>
#include <assert.h>

#include "joyLoL/macros.h"
#include "joyLoL/coAlg/coAlgs.h"
#include "joyLoL/lists.h"
#include "joyLoL/text.h"
#include "joyLoL/text_private.h"

// Texts are a collection of characters, which are used by the Parser
// to extract successive symbols.
//
// Texts are created on one of three backing suppliers of characters:
// 1. a single string
// 2. a NULL terminated array of strings
// 3. an external file
// 4. a readline interaction with a user
//
// In all four cases, the Parser's nextSymbol method requests successive
// **lines** of characters (deliminated by new-line-characters).
//
// It is critical, for correct interaction with the user via readline,
// that the initial line is NOT obtained until actually requested by
// the parser's nextSymbol method.
//
// It is also critical that once completed, none of the sources, get
// asked for subsequent lines.
//
// When the text has been completed, the nextLine function ensures
// that aText->curLine is NULL.

/////////////////////////////////////////////////////////////////////////////
// Text from file
//

void nextLineFromFile(Text* aText) {
  DEBUG(FALSE, "->nextLineFromFile %s\n", "");
  if (!aText) return; // there is nothing we can do!
  if (!aText->inputFile) return; // there is nothing we can do!

  // getline returns alloc'ed memory so we need to free it here.
  if (aText->curLine) free((void*)aText->curLine);

  aText->curLine    = NULL;
  aText->curChar    = NULL;
  aText->lastChar   = NULL;

  char *linePtr    = NULL;
  size_t n         = 0;
  if (!getline(&linePtr, &n, aText->inputFile) && n < 1) {
    DEBUG(FALSE, "<-nextLineFromFile %s\n", "getline returned 0");
    aText->completed = TRUE;
    return;
  }

  if (feof(aText->inputFile)) {
    // we have already reached the end of the text;
    DEBUG(FALSE, "<-nextLineFromFile %s\n", "feof returned file end");
    aText->completed = TRUE;
    aText->curLine   = NULL;
    aText->curChar   = NULL;
    aText->lastChar  = NULL;
    return;
  }

  aText->curLine    = linePtr;

  aText->curChar    = aText->curLine;
  aText->lastChar   = aText->curChar + strlen(aText->curLine);
  DEBUG(FALSE, "<-nextLineFromFile [%s]\n", aText->curLine);
}

Text* createTextFromInputFile(CoAlgebras* coAlgs,
                              FILE *anInputFile) {
  Text* aText = (Text*)calloc(1, sizeof(Text));
  assert(aText);
  //
  // external file specific initializations
  //
  aText->inputFile  = anInputFile;
  aText->nextLine   = nextLineFromFile;
  //
  // general initializations
  //
  assert(coAlgs);
  aText->coAlgebras = coAlgs;
  aText->completed  = FALSE;
  aText->sym        = NULL;
  aText->curLine    = NULL;
  aText->curChar    = NULL;
  aText->lastChar   = NULL;
  //
  aText->textLines  = NULL;
  aText->curLineNum = 0;
  //
  aText->newPrompt      = NULL;
  aText->continuePrompt = NULL;
  aText->curPrompt      = NULL;
  return aText;
}
\stoptyping

\starttyping
// readline.c
#include <stdlib.h>
#include <string.h>
#include <assert.h>
#include <readline/readline.h>
#include <readline/history.h>

#include "joyLoL/macros.h"
#include "joyLoL/coAlg/coAlgs.h"
#include "joyLoL/lists.h"
#include "joyLoL/text.h"
#include "joyLoL/text_private.h"
#include "joyLoL/dictionary.h"

// Texts are a collection of characters, which are used by the Parser
// to extract successive symbols.
//
// Texts are created on one of three backing suppliers of characters:
// 1. a single string
// 2. a NULL terminated array of strings
// 3. an external file
// 4. a readline interaction with a user
//
// In all four cases, the Parser's nextSymbol method requests successive
// **lines** of characters (deliminated by new-line-characters).
//
// It is critical, for correct interaction with the user via readline,
// that the initial line is NOT obtained until actually requested by
// the parser's nextSymbol method.
//
// It is also critical that once completed, none of the sources, get
// asked for subsequent lines.
//
// When the text has been completed, the nextLine function ensures
// that aText->curLine is NULL.

/////////////////////////////////////////////////////////////////////////////
// Text from readline
//

void clearReadlinePrompt(struct text_struct* aText) {
  aText->curPrompt = aText->newPrompt;
}

void setReadlinePrompts(Text* aText,
                        const char* newPrompt, const char* continuePrompt) {

  if (newPrompt) aText->newPrompt = newPrompt;
  else           aText->newPrompt = ">";

  if (continuePrompt) aText->continuePrompt = continuePrompt;
  else                aText->continuePrompt = ":";

  clearReadlinePrompt(aText);
}

void saveReadlineHistory(Text* aText) {
  write_history(".joyLoL-history");
}

void nextLineFromReadline(struct text_struct* aText) {
  DEBUG(FALSE, "->nextLineFromReadline %s\n", "");
  if (!aText) return; // there is nothing we can do!

  if (aText->completed) {
    // we have reached the end of the interaction with the user
    aText->completed = TRUE;
    aText->curLine   = NULL;
    aText->curChar   = NULL;
    aText->lastChar  = NULL;
    return;
  }

  // readline returns alloc'ed memory so we free it here.
  if (aText->curLine) free((void*)aText->curLine);

  aText->curLine  = NULL;
  aText->curChar  = NULL;
  aText->lastChar = NULL;

  aText->curLine = readline (aText->curPrompt);

  if (!aText->curLine) {
    aText->completed = TRUE;
    return;
  }

  if (*aText->curLine) {
    history_set_pos(0); // start at the begining
    if (0 <= history_search (aText->curLine, -1)) { // search backwards
      remove_history(where_history());
    } else if (0 <= history_search (aText->curLine, 1)) { // search forwards
      remove_history(where_history());
    }
    history_set_pos(0);
    add_history (aText->curLine);
  }

  aText->curChar = aText->curLine;
  aText->lastChar = aText->curChar + strlen(aText->curChar);

  aText->curPrompt = aText->continuePrompt;
}

Text* currentReadLineText = NULL;

char* dictionaryWalker(const char* text, int state) {
  if (!currentReadLineText) return NULL;
  if (!state) {

    assert(currentReadLineText->coAlgebras);
    assert(currentReadLineText->coAlgebras->symbols);
    Dictionary* mainDic =
      currentReadLineText->coAlgebras->symbols->dictionary;
    assert(mainDic);

    currentReadLineText->curNode = findLUBSymbol(mainDic, text);
    DEBUG(FALSE, "dictionaryWalker-start %p\n",
          currentReadLineText->curNode);
  }
  AVLNode* curNode = currentReadLineText->curNode;

  if (!curNode) return NULL;

  if (strncmp(curNode->symbol, text, strlen(text)) == 0) {
    DEBUG(FALSE, "dictionaryWalker %p {%s}[%s]\n",
          curNode, text, curNode->symbol);
    currentReadLineText->curNode = curNode->next;
    return strdup(curNode->symbol);
  }

  currentReadLineText->curNode = NULL;
  return NULL;
}

char** dictionaryCompletion(const char* text, int start, int end) {
 return rl_completion_matches(text, dictionaryWalker);
}

Text* createTextFromReadline(CoAlgebras* coAlgs) {
  Text* aText = (Text*)calloc(1, sizeof(Text));
  assert(aText);
  //
  // readline specific initializations
  //
  using_history();
  read_history(".joyLoL-history");
  rl_readline_name = "joyLoL";
  rl_attempted_completion_function = dictionaryCompletion;
  setReadlinePrompts(aText, NULL, NULL);
  aText->curNode = NULL;
  aText->curCompletionText = NULL;
  aText->curCompletionLen  = 0;
  aText->nextLine = nextLineFromReadline;
  //
  // general initialization
  //
  assert(coAlgs);
  aText->coAlgebras = coAlgs;
  aText->completed  = FALSE;
  aText->sym        = NULL;
  aText->curLine    = NULL;
  aText->curChar    = NULL;
  aText->lastChar   = NULL;
  //
  aText->inputFile = NULL;
  //
  aText->textLines  = NULL;
  aText->curLineNum = 0;

  assert(!currentReadLineText); // there should not be more than one
  currentReadLineText = aText;

  return aText;
}
\stoptyping

\starttyping
// string.c
#include <stdlib.h>
#include <string.h>
#include <assert.h>

#include "joyLoL/macros.h"
#include "joyLoL/coAlg/coAlgs.h"
#include "joyLoL/lists.h"
#include "joyLoL/text.h"
#include "joyLoL/text_private.h"

// Texts are a collection of characters, which are used by the Parser
// to extract successive symbols.
//
// Texts are created on one of three backing suppliers of characters:
// 1. a single string
// 2. a NULL terminated array of strings
// 3. an external file
// 4. a readline interaction with a user
//
// In all four cases, the Parser's nextSymbol method requests successive
// **lines** of characters (deliminated by new-line-characters).
//
// It is critical, for correct interaction with the user via readline,
// that the initial line is NOT obtained until actually requested by
// the parser's nextSymbol method.
//
// It is also critical that once completed, none of the sources, get
// asked for subsequent lines.
//
// When the text has been completed, the nextLine function ensures
// that aText->curLine is NULL.

/////////////////////////////////////////////////////////////////////////////
// Text from array of strings
//

void nextLineFromString(Text* aText) {
  DEBUG(FALSE, "->nextLineFromString [%s]{%p:%p}\n",
    aText->curLine, aText->curChar, aText->lastChar);
  if (!aText) return;  // there is nothing we can do!
  if (!aText->curLine) return; // there is nothing we can do!

  // there is no next line so... we have already reached the end of the text
  aText->completed  = TRUE;
  aText->curLine    = NULL;
  aText->curChar    = NULL;
  aText->lastChar   = NULL;
}

Text* createTextFromString(CoAlgebras* coAlgs,
                           const char* aString) {
  assert(coAlgs);
  assert(aString);

  Text* aText = (Text*)calloc(1, sizeof(Text));
  assert(aText);
  //
  // array of strings specific initializations
  //
  aText->curLine    = aString;
  aText->curChar    = aText->curLine;
  aText->lastChar   = aText->curChar + strlen(aText->curLine);
  aText->nextLine   = nextLineFromString;
  //
  // general initializations
  //
  aText->coAlgebras = coAlgs;
  aText->completed  = FALSE;
  aText->sym        = NULL;
  //
  aText->inputFile = NULL;
  //
  aText->newPrompt       = NULL;
  aText->continuePrompt  = NULL;
  aText->curPrompt       = NULL;
  //
  aText->textLines  = NULL;
  aText->curLineNum = 0;

  return aText;
}
\stoptyping
