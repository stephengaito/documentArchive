% A ConTeXt document [master document: templates.tex]

\section[title=Code]
\setCHeaderStream{public}

\dependsOn[jInterps]
%\dependsOn[context]

\startTestSuite[registerTemplates]

\startCHeader
typedef struct templates_class_struct {
  CoAlgClass super;
} TemplatesClass;
\stopCHeader

\setCHeaderStream{private}
\startCHeader
extern Boolean registerTemplates(JoyLoLInterp *jInterp);
\stopCHeader
\setCHeaderStream{public}

\startCCode
Boolean registerTemplates(JoyLoLInterp *jInterp) {
  assert(jInterp);
  
  TemplatesClass* theCoAlg =
    joyLoLCalloc(1, TemplatesClass);
  theCoAlg->super.name         = TemplatesName;
  theCoAlg->super.objectSize   = sizeof(CoAlgObj);
  theCoAlg->super.registerFunc = registerTemplates;
  theCoAlg->super.equalityFunc = NULL;
  theCoAlg->super.printFunc    = NULL;
  
  size_t tag =
    registerCoAlgClass(jInterp, (CoAlgClass*)theCoAlg);

  // do a sanity check...
  assert(tag == TemplatesTag);
  assert(jInterp->coAlgs[tag].sClass);
  
  registerTemplateWords(jInterp);
  
  return TRUE;
}
\stopCCode

\startTestCase[should register the Templates coAlg]

\startCTest
  // CTestsSetup has already created a jInterp
  // and run registerTemplates
  
  AssertPtrNotNull(jInterp);
  AssertPtrNotNull(jInterp->coAlgs);
  AssertPtrNotNull(getTemplatesClass(jInterp));
  TemplatesClass *coAlg =
    getTemplatesClass(jInterp);
  AssertIntTrue(registerTemplates(jInterp));
  AssertPtrNotNull(getTemplatesClass(jInterp));
  AssertPtrEquals(getTemplatesClass(jInterp), coAlg);
  AssertIntEquals(
    getTemplatesClass(jInterp)->super.objectSize,
    sizeof(CoAlgObj)
  )
\stopCTest
\stopTestCase
\stopTestSuite
