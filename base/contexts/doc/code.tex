% A ConTeXt document [master document: contexts.tex]

\section[title=Code]

\dependsOn[jInterps]

\component gitVersion-c

\starttyping
\startJoyLoLWord[eval]

\startJoyLoLCode

;; this is a comment
#| 
  This is a multi-line comment
  and another line
|#

(test () ())

\stopJoyLoLCode

\startCCode

void traceAction(const char* action, PairAtom* aLoL, PairAtom* command) {
  char* lolStr = printLoL(aLoL);
  if (command) {
    char* commandStr = printLoL(command);
    printf("%s: %s (%s)\\n", action, lolStr, commandStr);
    free(commandStr);
  } else {
    printf("%s: %s\\n", action, lolStr);
  }
  free(lolStr);
}

void evalCommandInContext(Context* aCtx, PairAtom* command) {
  assert(aCtx);
  assert(aCtx->coAlgebras);
  assert(aCtx->coAlgebras->symbols);
  Dictionary* mainDic = aCtx->coAlgebras->symbols->dictionary;
  assert(mainDic);
  //
  // if there is no command... there is nothing to do
  //
  if (!command) return;
  //
  DEBUG(FALSE, "evalCommandInContext > %p %p\\n", aCtx, command);
  //
  // push this command onto the top of the process stack
  //
  pushCtxProcess(aCtx, command);
  //
  while(aCtx->process) {
    if (aCtx->tracingOn) printf("\\n ctx: %s\\n", aCtx->name);
    //
    // pop the next command off the process stack
    //
    command = popCtxProcess(aCtx);
    aCtx->command = command;
    //
    if (isFunction(command)) {
      //
      // if the command is a function.. call the function
      //
      if (aCtx->tracingOn) traceAction("calling", command, NULL);
      (command->func)(aCtx);
      //
    } else if (isContext(command)) {
      //
      // if the command is a context... switch to this contex
      //
      DEBUG(FALSE, "evalCommandInContext -> switching origCtx: %s\\n",
            aCtx->name);
      if (aCtx->tracingOn) traceAction("switching", command, NULL);
      Context* switchCtx = command->context;
      assert(switchCtx);

      popCtxDataInto(aCtx, oldCtxTop);
      pushCtxData(switchCtx, oldCtxTop);
      switchCtx->tracingOn = aCtx->tracingOn;
      aCtx = switchCtx;
      DEBUG(FALSE, "evalCommandInContext <- switching newCtx: %s\\n",
            aCtx->name);
      //
    } else if (!isSymbol(command)) {
      //
      // if the command is not a Symbol ...
      //  ...  push it onto the top of the data stack
      //
      if (aCtx->tracingOn) traceAction("adding", command, NULL);
      pushCtxData(aCtx, command);
      //
    } else {
      //
      // if the command is a symbol ...
      //  ... look up the symbol's association in the dictionary
      //
      AVLNode* assoc = findSymbol(mainDic, command->symbol);
      assert(assoc);
      //
      if (!assoc->value) {
        //
        // if the association is empty.. push this symbol onto the top
        // of the data stack (re-evaluating this symbol would lead to an
        // infinite loop)
        //
        if (aCtx->tracingOn) traceAction("adding", command, NULL);
        pushCtxData(aCtx, command);
        //
      } else if (isPair(assoc->value)) {
        //
        // if the association is a LoL.. push this LoL onto the top of the
        // process stack
        //
        if (aCtx->tracingOn) traceAction("evaluating", assoc->value, command);
        assert(aCtx->coAlgebras);
        prependListCtxProcess(aCtx, copyLoL(aCtx->coAlgebras, assoc->value));
        //
      } else if (isFunction(assoc->value)) {
        //
        // if the association is a function.. call the function
        //
        if (aCtx->tracingOn) traceAction("calling", command, NULL);
        (assoc->value->func)(aCtx);
        //
      } else if (isContext(assoc->value)) {
        //
        // if the association is a context... switch to this contex
        //
        DEBUG(FALSE, "evalCommandInContext -> switching origCtx: %s\\n",
              aCtx->name);
        if (aCtx->tracingOn) traceAction("switching", command, NULL);
        Context* switchCtx = assoc->value->context;
        assert(switchCtx);

        popCtxDataInto(aCtx, oldCtxTop);
        pushCtxData(switchCtx, oldCtxTop);
        switchCtx->tracingOn = aCtx->tracingOn;
        aCtx = switchCtx;
        DEBUG(FALSE, "evalCommandInContext <- switching newCtx: %s\\n",
              aCtx->name);
        //
      } else {
        //
        // if the association is NOT a PairAtom or Function...
        // ... push this new ATOM onto the top of the process stack
        //
        if (aCtx->tracingOn) traceAction("evaluating", assoc->value, command);
        pushCtxProcess(aCtx, assoc->value);
        //
      }
    }
    if (aCtx->tracingOn) {
      DEBUG(FALSE, "evalCommandInContext -> tracing%s\\n", "");
      printf("d>>%s\\n", printLoL(aCtx->data));
      printf("p>>%s\\n", printLoL(aCtx->process));
      DEBUG(FALSE, "evalCommandInContext <- tracing%s\\n", "");
    }
  } // aCtx->process is empty
  DEBUG(FALSE, "evalCommandInContext < %p %p\\n", aCtx, command);
}

void evalContext(Context* aCtx) {
  popCtxProcessInto(aCtx, aCommand);
  evalCommandInContext(aCtx, aCommand);
}

void evalTextInContext(Context* aCtx, Text* aText) {
  DEBUG(FALSE, "evalTextInContext > %p %p\\n", aCtx, aText);
  while(TRUE) {
    PairAtom* aLoL = parseOneSymbol(aText);
//    if (aCtx->showStack) printf("<%s\\n", printLoL(aLoL));
    if (!aLoL) break;
    evalCommandInContext(aCtx, aLoL);
//    if (aCtx->showStack) printf(">%s\\n", printLoL(aCtx->data));
    if (aCtx->exceptionRaised) break;
  }
  DEBUG(FALSE, "evalTextInContext < %p %p\\n", aCtx, aText);
}

void evalArrayOfStringsInContext(Context* aCtx,
                                 const char* someStrings[]) {
  DEBUG(FALSE, "evalStringInContext > %p %p\\n", aCtx, someStrings);
  assert(aCtx);
  assert(aCtx->coAlgebras);
  Text* stringText = createTextFromArrayOfStrings(aCtx->coAlgebras,
                                                  someStrings);
  evalTextInContext(aCtx, stringText);
  freeText(stringText);
  DEBUG(FALSE, "evalStringInContext < %p %p\\n", aCtx, someStrings);
}

void evalStringInContext(Context* aCtx, const char* aString) {
  DEBUG(FALSE, "evalStringInContext > %p [%s]\\n", aCtx, aString);
  assert(aCtx);
  assert(aCtx->coAlgebras);
  Text* stringText = createTextFromString(aCtx->coAlgebras, aString);
  evalTextInContext(aCtx, stringText);
  freeText(stringText);
  DEBUG(FALSE, "evalStringInContext < %p [%s]\\n", aCtx, aString);
}

void runREPLInContext(Context* aCtx) {
  assert(aCtx);
  assert(aCtx->coAlgebras);
  Text* aText = createTextFromReadline(aCtx->coAlgebras);
  while(TRUE) {
    DEBUG(FALSE, "runREPLInContext %p reading\\n", aCtx);
    PairAtom* aLoL = parseOneSymbol(aText);
    DEBUG(FALSE, "runREPLInContext %p read %p\\n", aCtx, aLoL);
    //
    if (!aLoL) break;
    //
    evalCommandInContext(aCtx, aLoL);
    assert(!aCtx->process);
    //
    if (aCtx->showStack) {
      DEBUG(FALSE, "runREPLInContext %p printing data %p\\n", aCtx, aCtx->data);
      printf(">>%s\\n", printLoL(aCtx->data));
      DEBUG(FALSE, "runREPLInContext %p printed data %p\\n", aCtx, aCtx->data);
    }
    //
    reportException(aCtx); // report last exception if raised
  }
  saveReadlineHistory(aText);
  DEBUG(FALSE, "runREPLInContext %p COMPLETED\\n", aCtx);
}

\stopCCode

\stopJoyLoLWord

\stoptyping

\starttyping
// contexts.c
#include <stdlib.h>
#include <string.h>
#include <assert.h>

#include "joyLoL/macros.h"
#include "joyLoL/coAlg/coAlgs.h"
#include "joyLoL/lists.h"
#include "joyLoL/dictionary.h"
#include "joyLoL/printer.h"
#include "joyLoL/text.h"
#include "joyLoL/parser.h"

Context* createContext(CoAlgebras* coAlgs,
                       Symbol*   name,
                       Context*  parent,
                       PairAtom* dataLoL,
                       PairAtom* processLoL) {
  DEBUG(FALSE, "createContext %p [%s] %p %p\n", coAlgs, name, dataLoL, processLoL);
  assert(coAlgs);

  Context* context = (Context*) calloc(1, sizeof(Context));
  assert(context);

  context->coAlgebras         = coAlgs;
  context->name               = name;
  context->parent             = parent;
  context->data               = dataLoL;
  context->command            = NULL;
  context->process            = processLoL;
  context->messages           = NULL;
  context->listeners          = NULL;
  context->showSpecifications = TRUE;
  context->showStack          = TRUE;
  context->tracingOn          = FALSE;
  context->checkingOn         = FALSE;
  context->verbose            = FALSE;
  context->exceptionRaised    = FALSE;

  return context;
}

PairAtom* newContext(CoAlgebras* coAlgs, Context* aCtx) {
  assert(coAlgs);
  PairAtom* aNewCtx = newPairAtom(coAlgs);
  assert(aNewCtx);
  aNewCtx->coAlg    = (CoAlgebra*)coAlgs->contexts;
  aNewCtx->tag      = 0;
  aNewCtx->context  = aCtx;
  return aNewCtx;
}

void pushCtxData(Context* aCtx, PairAtom* lolToPush) {
  if (!aCtx) return;
  assert(aCtx->coAlgebras);
  aCtx->data = newPair(aCtx->coAlgebras, lolToPush, aCtx->data);
}

void pushNullCtxData(Context* aCtx) {
  if (!aCtx) return;
  assert(aCtx->coAlgebras);
  aCtx->data = newPair(aCtx->coAlgebras, NULL, aCtx->data);
}

void pushSymbolCtxData(Context* aCtx, Symbol* aSymbol) {
  if (!aCtx) return;
  if (!aSymbol) return;
  assert(aCtx->coAlgebras);
  CoAlgebras* coAlgs = aCtx->coAlgebras;

  assert(coAlgs->symbols);
  assert(coAlgs->symbols->dictionary);

  PairAtom* aSymbolPA =
    getSymbol(coAlgs, coAlgs->symbols->dictionary, aSymbol);
  if (!aSymbolPA) return;
  aCtx->data = newPair(coAlgs, aSymbolPA, aCtx->data);
}

void pushParsedArrayOfStringsCtxData(Context* aCtx, Symbol* someStrings[]) {
  if (!aCtx) return;
  if (!someStrings) return;
  Text* aText = createTextFromArrayOfStrings(aCtx->coAlgebras, someStrings);
  pushParsedTextCtxData(aCtx, aText);
  freeText(aText);
}

void pushParsedStringCtxData(Context* aCtx, Symbol* aString) {
  if (!aCtx) return;
  if (!aString) return;
  Text* aText = createTextFromString(aCtx->coAlgebras, aString);
  pushParsedTextCtxData(aCtx, aText);
  freeText(aText);
}

void pushParsedTextCtxData(Context* aCtx, Text* aText) {
  if (!aCtx) return;
  if (!aText) return;
  assert(aCtx->coAlgebras);
  PairAtom* aLoL = parseAllSymbols(aText);
  if (!aLoL) return;
  aCtx->data = newPair(aCtx->coAlgebras, aLoL, aCtx->data);
}

void prependListCtxData(Context* aCtx, PairAtom* lolToPrepend) {
  if (!aCtx) return;
  if (!lolToPrepend) return;

  // ensure lolToPrepend is a list
  if (!isPair(lolToPrepend)) {
    lolToPrepend = newPair(aCtx->coAlgebras, lolToPrepend, NULL);
    assert(lolToPrepend);
  }

  // find end of lolToPrepend
  PairAtom* lolList = lolToPrepend;
  while(lolList->pair.cdr && isPair(lolList->pair.cdr)) {
    lolList = lolList->pair.cdr;
  }

  // ensure that if lolToPrepend ends in a non-pair we make it a pair
  if (lolList->pair.cdr && !isPair(lolList->pair.cdr)) {
    lolList->pair.cdr = newPair(aCtx->coAlgebras, lolList->pair.cdr, NULL);
    assert(lolList->pair.cdr);
  }

  // place existing data stack at end of lolToPrepend
  lolList->pair.cdr = aCtx->data;
  aCtx->data   = lolToPrepend;
}

PairAtom* popCtxData(Context* aCtx) {
  if (!aCtx) return NULL;
  if (!aCtx->data) return NULL;

  PairAtom* poppedLoL = aCtx->data->pair.car;
  aCtx->data          = aCtx->data->pair.cdr;
  return poppedLoL;
}

void pushCtxProcess(Context* aCtx, PairAtom* lolToPush) {
  if (!aCtx) return;
  assert(aCtx->coAlgebras);
  aCtx->process = newPair(aCtx->coAlgebras, lolToPush, aCtx->process);
}

void pushSymbolCtxProcess(Context* aCtx, const char* aSymbol) {
  if (!aCtx) return;
  if (!aSymbol) return;
  assert(aCtx->coAlgebras);
  CoAlgebras* coAlgs = aCtx->coAlgebras;

  assert(coAlgs->symbols);
  assert(coAlgs->symbols->dictionary);

  PairAtom* aSymbolPA =
    getSymbol(coAlgs, coAlgs->symbols->dictionary, aSymbol);
  if (!aSymbolPA) return;
  aCtx->process = newPair(coAlgs, aSymbolPA, aCtx->process);
}

void pushParsedArrayOfStringsCtxProcess(Context* aCtx, Symbol* someStrings[]) {
  if (!aCtx) return;
  if (!someStrings) return;
  Text* aText = createTextFromArrayOfStrings(aCtx->coAlgebras, someStrings);
  pushParsedTextCtxProcess(aCtx, aText);
  freeText(aText);
}

void pushParsedStringCtxProcess(Context* aCtx, Symbol* aString) {
  if (!aCtx) return;
  if (!aString) return;
  Text* aText = createTextFromString(aCtx->coAlgebras, aString);
  pushParsedTextCtxProcess(aCtx, aText);
  freeText(aText);
}

void pushParsedTextCtxProcess(Context* aCtx, Text* aText) {
  if (!aCtx) return;
  if (!aText) return;
  assert(aCtx->coAlgebras);
  PairAtom* aLoL = parseAllSymbols(aText);
  if (!aLoL) return;
  aCtx->process = newPair(aCtx->coAlgebras, aLoL, aCtx->process);
}

void prependListCtxProcess(Context* aCtx, PairAtom* lolToPrepend) {
  if (!aCtx) return;
  if (!lolToPrepend) return;

  // ensure lolToPrepend is a list
  if (!isPair(lolToPrepend)) {
    lolToPrepend = newPair(aCtx->coAlgebras, lolToPrepend, NULL);
    assert(lolToPrepend);
  }

  // find end of lolToPrepend
  PairAtom* lolList = lolToPrepend;
  while(lolList->pair.cdr && isPair(lolList->pair.cdr)) {
    lolList = lolList->pair.cdr;
  }

  // ensure that if lolToPrepend ends in a non-pair we make it a pair
  if (lolList->pair.cdr && !isPair(lolList->pair.cdr)) {
    lolList->pair.cdr = newPair(aCtx->coAlgebras, lolList->pair.cdr, NULL);
    assert(lolList->pair.cdr);
  }

  // place existing process stack at end of lolToPrepend
  lolList->pair.cdr  = aCtx->process;
  aCtx->process = lolToPrepend;
}

PairAtom* popCtxProcess(Context* aCtx) {
  if (!aCtx->process) return NULL;

  PairAtom* poppedLoL = aCtx->process->pair.car;
  aCtx->process       = aCtx->process->pair.cdr;
  return poppedLoL;
}

size_t extendJoyLoL(CoAlgebras* coAlgs,
                    Symbol* definedName,
                    JoyLoLFunction* aFunc) {
  assert(coAlgs);
  assert(coAlgs->symbols);
  assert(coAlgs->symbols->dictionary);
  AVLNode* aSym =
    createSymbol(coAlgs->symbols->dictionary, definedName);
  aSym->value = newJoyLoLFunc(coAlgs, aFunc);

  return TRUE;
}

size_t defineJoyLoL(CoAlgebras* coAlgs,
                    Symbol* definedName,
                    PairAtom* aLoL) {
  assert(coAlgs);
  assert(coAlgs->symbols);
  assert(coAlgs->symbols->dictionary);
  AVLNode* aSym =
    createSymbol(coAlgs->symbols->dictionary, definedName);
  aSym->value = copyLoL(coAlgs, aLoL);

  return TRUE;
}

size_t defineContext(CoAlgebras* coAlgs,
                     Symbol* definedName,
                     Context* newCtx) {
  DEBUG(FALSE, "defineContext %p [%s] %p\n",
        coAlgs, definedName, newCtx);
  assert(coAlgs);
  assert(coAlgs->symbols);
  assert(coAlgs->symbols->dictionary);
  AVLNode* aSym =
    createSymbol(coAlgs->symbols->dictionary, definedName);
  aSym->value   = newContext(coAlgs, newCtx);

  return TRUE;
}
\stoptyping

\starttyping
// contestsAPControl.c
#include <stdlib.h>
#include <string.h>
#include <assert.h>


#include "joyLoL/macros.h"
#include "joyLoL/coAlg/coAlgs.h"
#include "joyLoL/lists.h"
#include "joyLoL/dictionary.h"
#include "joyLoL/loader.h"
#include "joyLoL/printer.h"

// Operators

static void failAP(Context* aCtx) {
  assert(aCtx);
  aCtx->data    = NULL;
  aCtx->process = NULL;
}

// Combinators

static void interpretAP(Context* aCtx) {
  popCtxDataInto(aCtx, top);
  prependListCtxProcess(aCtx, top);
}

static void forAP(Context* aCtx) {
  popCtxProcessInto(aCtx, nextCommand);
  if (symbolIs(nextCommand, "forDone")) {
    // this for loop is done
    DEBUG(FALSE, "forAP DONE%s\n", "");
  } else {
    DEBUG(FALSE, "forAP continue%s\n", "");
    pushSymbolCtxProcess(aCtx, "for");
    prependListCtxProcess(aCtx, nextCommand);
  }
}

static void forDoneAP(Context* aCtx) {
  // ignore
}

static void lispInterpretAP(Context* aCtx) {
  popCtxDataInto(aCtx, top);
  PairAtom* operationName = car(top);
  PairAtom* operationBody = cdr(top);
  pushCtxProcess(aCtx, operationName);
  pushCtxProcess(aCtx, operationBody);
}

static void lispForAP(Context* aCtx) {
  popCtxProcessInto(aCtx, nextCommand);
  if (symbolIs(nextCommand, "forDone")) {
    // this lisp for loop is done
    DEBUG(FALSE, "lispForAP DONE%s\n", "");
  } else {
    DEBUG(FALSE, "lispForAP continue%s\n", "");
    PairAtom* operationName = car(nextCommand);
    PairAtom* operationBody = cdr(nextCommand);
    pushSymbolCtxProcess(aCtx, "lispFor");
    pushCtxProcess(aCtx, operationName);
    pushCtxProcess(aCtx, operationBody);
  }
}

static void exitAP(Context* aCtx) {
  assert(aCtx);
  aCtx->process = NULL;
}

static void tryAP(Context* aCtx) {
  popCtxDataInto(aCtx, handlerExp);
  popCtxDataInto(aCtx, tryExp);
  pushCtxProcess(aCtx, handlerExp);
  pushSymbolCtxProcess(aCtx, "tryHandler");
  prependListCtxProcess(aCtx, tryExp);
}

static void raiseAP(Context* aCtx) {
  assert(aCtx);
  aCtx->exceptionRaised = TRUE;
  popCtxDataInto(aCtx, raiseExp);
  pushSymbolCtxProcess(aCtx, "findFirstTryHandler");
  prependListCtxProcess(aCtx, raiseExp);
}

static void raiseIfFalseAP(Context* aCtx) {
  assert(aCtx);
  popCtxDataInto(aCtx, condition);
  popCtxProcessInto(aCtx, raiseExp);
  if (isFalse(condition)) {
    pushSymbolCtxProcess(aCtx, "findFirstTryHandler");
    prependListCtxProcess(aCtx, raiseExp);
  }
}

static void findFirstTryHandlerAP(Context* aCtx) {
  assert(aCtx);
  if (aCtx->process) {
    popCtxProcessInto(aCtx, aCommand);
    if (isSymbol(aCommand) && (strcmp(aCommand->symbol, "tryHandler") == 0)) {
      assert(aCtx);
      aCtx->exceptionRaised = FALSE;
      popCtxProcessInto(aCtx, handlerExp);
      prependListCtxProcess(aCtx, handlerExp);
      return;
    }
    pushSymbolCtxProcess(aCtx, "findFirstTryHandler");
  } else {
    if (aCtx->parent) {
      PairAtom* parentCtx = newContext(aCtx->coAlgebras, aCtx->parent);
      pushSymbolCtxProcess(aCtx->parent, "raise");
      pushCtxProcess(aCtx, parentCtx);
    }
  }
}

static void tryHandlerAP(Context* aCtx) {
  popCtxProcessInto(aCtx, handlerExp);
}

void raiseException(Context* aCtx, Symbol* message) {
  assert(aCtx);
  pushSymbolCtxData(aCtx, message);
  pushSymbolCtxProcess(aCtx, "raise");
}

size_t reportException(Context* aCtx) {
  assert(aCtx);
  if (!aCtx->exceptionRaised) return FALSE;
  char* exceptionReport = printLoL(car(aCtx->data));
  printf("\nUNHANDLED EXCEPTION: %s\n", exceptionReport);
  if (exceptionReport) free(exceptionReport);
  aCtx->exceptionRaised = FALSE;
  return TRUE;
}

// Support

static void defineAP(Context* aCtx) {
  assert(aCtx);
  popCtxDataInto(aCtx, top1);
  popCtxDataInto(aCtx, top2);
  if (!isSymbol(top1)) return;
  defineJoyLoL(aCtx->coAlgebras, top1->symbol, top2);
}

static void defineContextAP(Context* aCtx) {
  popCtxDataInto(aCtx, contextName);
  popCtxDataInto(aCtx, contextData);
  popCtxDataInto(aCtx, contextProcess);
  if (!isSymbol(contextName)) return;
  Context* newCtx =
    createContext(aCtx->coAlgebras, contextName->symbol,
                  aCtx, contextData, contextProcess);
  defineContext(aCtx->coAlgebras, contextName->symbol, newCtx);
}

static void newContextAP(Context* aCtx) {
  popCtxDataInto(aCtx, contextData);
  popCtxDataInto(aCtx, contextProcess);
  Context* newCtx =
    createContext(aCtx->coAlgebras, "anonymous",
                  aCtx, contextData, contextProcess);
  assert(newCtx);
  PairAtom* newCtxLoL = newContext(aCtx->coAlgebras, newCtx);
  assert(newCtxLoL);
  pushCtxData(aCtx, newCtxLoL);
}

static void thisContextAP(Context* aCtx) {
  PairAtom* thisCtx = newContext(aCtx->coAlgebras, aCtx);
  assert(thisCtx);
  pushCtxData(aCtx, thisCtx);
}

void initContextsAPControl(CoAlgebras* coAlgs) {
  extendJoyLoL(coAlgs, "fail",                failAP);
  extendJoyLoL(coAlgs, "i",                   interpretAP);
  extendJoyLoL(coAlgs, "interpret",           interpretAP);
  extendJoyLoL(coAlgs, "for",                 forAP);
  extendJoyLoL(coAlgs, "forDone",             forDoneAP);
  extendJoyLoL(coAlgs, "lispInterpret",       lispInterpretAP);
  extendJoyLoL(coAlgs, "lispFor",             lispForAP);
  extendJoyLoL(coAlgs, "define",              defineAP);
  extendJoyLoL(coAlgs, "defineContext",       defineContextAP);
  extendJoyLoL(coAlgs, "thisContext",         thisContextAP);
  extendJoyLoL(coAlgs, "newContext",          newContextAP);
  extendJoyLoL(coAlgs, "exit",                exitAP);
  extendJoyLoL(coAlgs, "quit",                exitAP);
  extendJoyLoL(coAlgs, "try",                 tryAP);
  extendJoyLoL(coAlgs, "raise",               raiseAP);
  extendJoyLoL(coAlgs, "raiseIfFalse",        raiseIfFalseAP);
  extendJoyLoL(coAlgs, "findFirstTryHandler", findFirstTryHandlerAP);
  extendJoyLoL(coAlgs, "tryHandler",          tryHandlerAP);
}
\stoptyping

\starttyping
//contextsAPSupport.c
#include <stdlib.h>
#include <string.h>
#include <assert.h>


#include "joyLoL/macros.h"
#include "joyLoL/coAlg/coAlgs.h"
#include "joyLoL/lists.h"
#include "joyLoL/dictionary.h"
#include "joyLoL/loader.h"
#include "joyLoL/printer.h"

size_t isContext(PairAtom* aLoL) {
  if (!aLoL) return FALSE;
  if (aLoL->coAlg && (aLoL->coAlg->isA == CONTEXT_COALG)) return TRUE;
  return FALSE;
}

static void isContextAP(Context* aCtx) {
  assert(aCtx);
  assert(aCtx->coAlgebras);
  popCtxDataInto(aCtx, top);
  PairAtom* result = NULL;
  if (isContext(top)) result = newBoolean(aCtx->coAlgebras, TRUE);
  else                result = newBoolean(aCtx->coAlgebras, FALSE);
  pushCtxData(aCtx, result);
}

// Support
static void showStackOnAP(Context* aCtx) {
  assert(aCtx);
  aCtx->showStack = TRUE;
}

static void showStackOffAP(Context* aCtx) {
  assert(aCtx);
  aCtx->showStack = FALSE;
}

static void showStackAP(Context* aCtx) {
  assert(aCtx);
  printf("d>>%s\n", printLoL(aCtx->data));
  printf("p>>%s\n", printLoL(aCtx->process));
}

static void tracingOnAP(Context* aCtx) {
  assert(aCtx);
  aCtx->tracingOn = TRUE;
}

static void tracingOffAP(Context* aCtx) {
  assert(aCtx);
  aCtx->tracingOn = FALSE;
}

static void checkingOnAP(Context* aCtx) {
  assert(aCtx);
  aCtx->checkingOn = TRUE;
}

static void checkingOffAP(Context* aCtx) {
  assert(aCtx);
  aCtx->checkingOn = FALSE;
}

static void definitionsAP(Context* aCtx) {
  assert(aCtx);
  assert(aCtx->coAlgebras);
  assert(aCtx->coAlgebras->symbols);
  assert(aCtx->coAlgebras->symbols->dictionary);
  listDefinitions(aCtx->coAlgebras->symbols->dictionary, stdout);
}

static void showLoadExtensionsAP(Context* aCtx) {
  assert(aCtx);
  assert(aCtx->coAlgebras);
  assert(aCtx->coAlgebras->contexts);
  Loader* loader = aCtx->coAlgebras->contexts->loader;
  assert(loader);
  PairAtom* extensionList = loader->loadExtensions;
  while(extensionList) {
    PairAtom* anExtension = extensionList->pair.car;
    if (isSymbol(anExtension)) printf("%s\n", anExtension->symbol);
    extensionList = extensionList->pair.cdr;
  }
}

static void showLoadPathsAP(Context* aCtx) {
  assert(aCtx);
  assert(aCtx->coAlgebras);
  assert(aCtx->coAlgebras->contexts);
  Loader* loader = aCtx->coAlgebras->contexts->loader;
  assert(loader);
  PairAtom* pathList = loader->loadPaths;
  while(pathList) {
    PairAtom* aPath = pathList->pair.car;
    if (isSymbol(aPath)) printf("%s\n", aPath->symbol);
    pathList = pathList->pair.cdr;
  }
}

static void showLoadFilesAP(Context* aCtx) {
  assert(aCtx);
  assert(aCtx->coAlgebras);
  assert(aCtx->coAlgebras->contexts);
  Loader* loader = aCtx->coAlgebras->contexts->loader;
  assert(loader);
  PairAtom* fileList = loader->loadFiles;
  while(fileList) {
    PairAtom* aFile = fileList->pair.car;
    if (isSymbol(aFile)) printf("%s\n", aFile->symbol);
    fileList = fileList->pair.cdr;
  }
}
static void loadExtensionAP(Context* aCtx) {
  assert(aCtx);
  popCtxDataInto(aCtx, top);
  if (!isSymbol(top)) return;
  pushLoadExtension(aCtx->coAlgebras, top->symbol);
}

static void loadPathAP(Context* aCtx) {
  assert(aCtx);
  popCtxDataInto(aCtx, top);
  if (!isSymbol(top)) return;
  pushLoadPath(aCtx->coAlgebras, top->symbol);
}

static void loadFileAP(Context* aCtx) {
  DEBUG(FALSE, "loadFileAP > %p\n", aCtx);
  assert(aCtx);
  popCtxDataInto(aCtx, top);
  int oldVerboseFlag = aCtx->verbose;
  aCtx->verbose = aCtx->showStack;
  loadFiles(aCtx, top);
  aCtx->verbose = oldVerboseFlag;
  DEBUG(FALSE, "loadFileAP < %p\n", aCtx);
}

static void lispLoadFileAP(Context* aCtx) {
  DEBUG(FALSE, "listLoadFileAP > %p\n", aCtx);
  pushSymbolCtxProcess(aCtx, "lispInterpret");
  loadFileAP(aCtx);
  DEBUG(FALSE, "listLoadFileAP < %p\n", aCtx);
}

static void commentAP(Context* aCtx) {
  popCtxDataInto(aCtx, top);
}

void initContextsAPSupport(CoAlgebras* coAlgs) {
  extendJoyLoL(coAlgs, "isContext",           isContextAP);
  extendJoyLoL(coAlgs, "definitions",         definitionsAP);
  extendJoyLoL(coAlgs, "loadExtension",       loadExtensionAP);
  extendJoyLoL(coAlgs, "loadPath",            loadPathAP);
  extendJoyLoL(coAlgs, "load",                loadFileAP);
  extendJoyLoL(coAlgs, "lispLoad",            lispLoadFileAP);
  extendJoyLoL(coAlgs, "comment",             commentAP);
  extendJoyLoL(coAlgs, "---",                 commentAP);
  extendJoyLoL(coAlgs, "showLoadExtensions",  showLoadExtensionsAP);
  extendJoyLoL(coAlgs, "showLoadPaths",       showLoadPathsAP);
  extendJoyLoL(coAlgs, "showLoadFiles",       showLoadFilesAP);
  extendJoyLoL(coAlgs, "showStack",           showStackAP);
  extendJoyLoL(coAlgs, "showStackOn",         showStackOnAP);
  extendJoyLoL(coAlgs, "showStackOff",        showStackOffAP);
  extendJoyLoL(coAlgs, "tracingOn",           tracingOnAP);
  extendJoyLoL(coAlgs, "tracingOff",          tracingOffAP);
  extendJoyLoL(coAlgs, "checkingOn",          checkingOnAP);
  extendJoyLoL(coAlgs, "checkingOff",         checkingOffAP);
}
\stoptyping

\starttyping
//contextsCoAlg.c

#include <stdlib.h>
#include <string.h>
#include <assert.h>

#include "joyLoL/macros.h"
#include "joyLoL/coAlg/coAlgs.h"
#include "joyLoL/lists.h"
#include "joyLoL/loader.h"
#include "joyLoL/scheduler.h"

static size_t equalityContextsCoAlg(CoAlgebra* klass,
                                   PairAtom* lolA, PairAtom* lolB,
                                   size_t debugFlag) {
  DEBUG(debugFlag, "contextCoAlg->equal klass:%p a:%p b:%p\n",
        klass, lolA, lolB);
  if (!lolA && !lolB) return TRUE;
  if (!lolA && lolB)  return FALSE;
  if (lolA  && !lolB) return FALSE;
  if (lolA->coAlg != klass) return FALSE;
  if (lolB->coAlg != klass) return FALSE;
  size_t areEqual = TRUE;
  lolEqual(areEqual, lolA->context->data,      lolB->context->data);
  lolEqual(areEqual, lolA->context->command,   lolB->context->command);
  lolEqual(areEqual, lolA->context->process,   lolB->context->process);
  lolEqual(areEqual, lolA->context->messages,  lolB->context->messages);
  lolEqual(areEqual, lolA->context->listeners, lolB->context->listeners);
  return areEqual;
}

static size_t printSizeContextsCoAlg(PairAtom* aLoL, size_t debugFlag) {
  DEBUG(debugFlag, "contextCoAlg->printSize: > %p\n", aLoL);
  assert(aLoL);
  assert(aLoL->coAlg);
  assert(aLoL->coAlg->isA == CONTEXT_COALG);

  if ((aLoL->tag & PRINT_MARKER)) return 0;
  aLoL->tag |= PRINT_MARKER;

  DEBUG(debugFlag, "contextCoAlg-printSize: %p {%p:%zu} d:%p c:%p p:%p m:%p l:%p\n", aLoL,
        aLoL->coAlg, aLoL->tag,
        aLoL->context->data, aLoL->context->command, aLoL->context->process,
        aLoL->context->messages, aLoL->context->listeners);

  size_t lolSize = 12 + strlen(aLoL->context->name); // "\n[[ " and " ]]\n"

  lolPrintSize(lolSize, aLoL->context->data, 6,
               "context", "data",      debugFlag);
  lolPrintSize(lolSize, aLoL->context->command, 6,
               "context", "command",   debugFlag);
  lolPrintSize(lolSize, aLoL->context->process, 6,
               "context", "process",   debugFlag);
  lolPrintSize(lolSize, aLoL->context->messages, 6,
               "context", "messages",  debugFlag);
  lolPrintSize(lolSize, aLoL->context->listeners, 6,
               "context", "listeners", debugFlag);

  DEBUG(debugFlag, "contextCoAlg-printSize: < %zu %p\n", lolSize, aLoL);
  return lolSize;
}

static size_t printStrContextsCoAlg(PairAtom* aLoL,
                                   char* buffer, size_t bufferSize) {
  assert(aLoL);
  assert(aLoL->coAlg);
  assert(aLoL->coAlg->isA == CONTEXT_COALG);

  if (!(aLoL->tag & PRINT_MARKER)) return TRUE;
  aLoL->tag = aLoL->tag & (~ PRINT_MARKER);

  size_t printedOk = TRUE;
  strcat(buffer, "\n[");
  strcat(buffer, aLoL->context->name);
  strcat(buffer, "[ ");
  lolPrintStr(printedOk, aLoL->context->data,
              "d:( ", ") ", buffer, bufferSize);
  lolPrintStr(printedOk, aLoL->context->command,
              "c:( ", ") ", buffer, bufferSize);
  lolPrintStr(printedOk, aLoL->context->process,
              "p:( ", ") ", buffer, bufferSize);
  lolPrintStr(printedOk, aLoL->context->messages,
              "m:( ", ") ", buffer, bufferSize);
  lolPrintStr(printedOk, aLoL->context->listeners,
              "l:( ", ") ", buffer, bufferSize);
  strcat(buffer, " ]]\n");
  return printedOk;
}

Contexts* createContextsCoAlgebra(void) {
  Contexts* contexts  = (Contexts*) calloc(1, sizeof(Contexts));
  initACoAlgebra((CoAlgebra*)contexts);
  contexts->super.isA       = CONTEXT_COALG;
  contexts->super.name      = "context";
  contexts->super.equality  = equalityContextsCoAlg;
  contexts->super.printSize = printSizeContextsCoAlg;
  contexts->super.printStr  = printStrContextsCoAlg;

  // create the "singleton" objects associated to contexts
  contexts->loader          = createLoader();
  contexts->scheduler       = createScheduler();
  return contexts;
}

void initContextsCoAlgebra(CoAlgebras* coAlgs) {
  initStandardLoadPaths(coAlgs);

  initContextsAPControl(coAlgs);
  initContextsAPSupport(coAlgs);
}
\stoptyping