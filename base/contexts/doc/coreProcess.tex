% A ConTeXt document [master document: contexts.tex]

\section[title=Context core process code]
\setCHeaderStream{public}

\startCHeader
#define clearCtxProcess(aCtx)               \
  (                                         \
    assert(aCtx),                           \
    assert(getContextsClass(aCtx->jInterp)  \
      ->clearCtxProcessFunc),               \
    (getContextsClass(aCtx->jInterp)        \
      ->clearCtxProcessFunc(aCtx))          \
  )
\stopCHeader

\setCHeaderStream{private}
\startCHeader
extern void clearCtxProcessImpl(
  ContextObj *aCtx
);
\stopCHeader
\setCHeaderStream{public}

\startCCode
void clearCtxProcessImpl(
  ContextObj *aCtx
) {
  if (!aCtx) return;
  aCtx->process = NULL;
}
\stopCCode


\startCHeader
#define pushCtxProcess(aCtx, lolToPush)        \
  (                                            \
    assert(aCtx),                              \
    assert(getContextsClass(aCtx->jInterp)     \
      ->pushCtxProcessFunc),                   \
    (getContextsClass(aCtx->jInterp)           \
      ->pushCtxProcessFunc(aCtx, lolToPush))   \
  )
\stopCHeader

\setCHeaderStream{private}
\startCHeader
extern void pushCtxProcessImpl(
  ContextObj* aCtx,
  JObj* lolToPush
);
\stopCHeader
\setCHeaderStream{public}

\startCCode
void pushCtxProcessImpl(
  ContextObj* aCtx,
  JObj* lolToPush
) {
  if (!aCtx) return;
  assert(aCtx->jInterp);
  aCtx->process = newPair(aCtx->jInterp, lolToPush, aCtx->process);
}
\stopCCode

\startCHeader
#define pushNullCtxProcess(aCtx)            \
  (                                         \
    assert(aCtx),                           \
    assert(getContextsClass(aCtx->jInterp)  \
      ->pushNullCtxProcessFunc),            \
    (getContextsClass(aCtx->jInterp)        \
      ->pushNullCtxProcessFunc(aCtx))       \
  )
\stopCHeader

\setCHeaderStream{private}
\startCHeader
extern void pushNullCtxProcessImpl(
  ContextObj* aCtx
);
\stopCHeader
\setCHeaderStream{public}

\startCCode
void pushNullCtxProcessImpl(
  ContextObj* aCtx
) {
  if (!aCtx) return;
  assert(aCtx->jInterp);
  aCtx->process = newPair(aCtx->jInterp, NULL, aCtx->process);
}
\stopCCode

\startCHeader
#define pushBooleanCtxProcess(aCtx, aBool)      \
  (                                             \
    assert(aCtx),                               \
    assert(getContextsClass(aCtx->jInterp)      \
      ->pushBooleanCtxProcessFunc),             \
    (getContextsClass(aCtx->jInterp)            \
      ->pushBooleanCtxProcessFunc(aCtx, aBool)) \
  )
\stopCHeader

\setCHeaderStream{private}
\startCHeader
extern void pushBooleanCtxProcessImpl(
  ContextObj *aCtx,
  Boolean     aBool
);
\stopCHeader
\setCHeaderStream{public}

\startCCode
void pushBooleanCtxProcessImpl(
  ContextObj *aCtx,
  Boolean     aBool
) {
  if (!aCtx) return;
  assert(aCtx->jInterp);

  JObj* aBoolPA = newBoolean(aCtx->jInterp, aBool);
  if (!aBoolPA) return;
  aCtx->process = newPair(aCtx->jInterp, aBoolPA, aCtx->process);
}
\stopCCode

\startCHeader
#define pushNaturalCtxProcess(aCtx, aNaturalStr)      \
  (                                                   \
    assert(aCtx),                                     \
    assert(getContextsClass(aCtx->jInterp)            \
      ->pushNaturalCtxProcessFunc),                   \
    (getContextsClass(aCtx->jInterp)                  \
      ->pushNaturalCtxProcessFunc(aCtx, aNaturalStr)) \
  )
\stopCHeader

\setCHeaderStream{private}
\startCHeader
extern void pushNaturalCtxProcessImpl(
  ContextObj *aCtx,
  Symbol     *aNaturalStr
);
\stopCHeader
\setCHeaderStream{public}

\startCCode
void pushNaturalCtxProcessImpl(
  ContextObj *aCtx,
  Symbol     *aNaturalStr
) {
  if (!aCtx) return;
  assert(aCtx->jInterp);

  JObj* aNaturalPA = newNatural(aCtx->jInterp, aNaturalStr);
  if (!aNaturalPA) return;
  aCtx->process = newPair(aCtx->jInterp, aNaturalPA, aCtx->process);
}
\stopCCode

\startCHeader
#define pushSymbolCtxProcess(aCtx, aSymbol)      \
  (                                              \
    assert(aCtx),                                \
    assert(getContextsClass(aCtx->jInterp)       \
      ->pushSymbolCtxProcessFunc),               \
    (getContextsClass(aCtx->jInterp)             \
      ->pushSymbolCtxProcessFunc(aCtx, aSymbol)) \
  )
\stopCHeader

\setCHeaderStream{private}
\startCHeader
extern void pushSymbolCtxProcessImpl(
  ContextObj* aCtx,
  Symbol* aSymbol
);
\stopCHeader
\setCHeaderStream{public}

\startCCode
void pushSymbolCtxProcessImpl(
  ContextObj* aCtx,
  Symbol* aSymbol
) {
  if (!aCtx) return;
  if (!aSymbol) return;
  assert(aCtx->jInterp);

  JObj* aSymbolPA = getAsSymbol(aCtx->dict, aSymbol);
  if (!aSymbolPA) return;
  aCtx->process = newPair(aCtx->jInterp, aSymbolPA, aCtx->process);
}
\stopCCode

\startCHeader
#define pushParsedArrayOfStringsCtxProcess(      \
  aCtx, someStrings)                             \
  (                                              \
    assert(aCtx),                                \
    assert(getContextsClass(aCtx->jInterp)       \
      ->pushParsedArrayOfStringsCtxProcessFunc), \
    (getContextsClass(aCtx->jInterp)             \
      ->pushParsedArrayOfStringsCtxProcessFunc(  \
        aCtx, someStrings))                      \
  )
\stopCHeader

\setCHeaderStream{private}
\startCHeader
extern void pushParsedArrayOfStringsCtxProcessImpl(
  ContextObj* aCtx,
  Symbol* someStrings[]
);
\stopCHeader
\setCHeaderStream{public}

\startCCode
void pushParsedArrayOfStringsCtxProcessImpl(
  ContextObj* aCtx,
  Symbol* someStrings[]
) {
  if (!aCtx) return;
  if (!someStrings) return;
  TextObj* aText =
    createTextFromArrayOfStrings(aCtx->jInterp, someStrings);
  pushParsedTextCtxProcess(aCtx, aText);
  freeText(aText);
}
\stopCCode

\startCHeader
#define pushParsedStringCtxProcess(aCtx, aString)      \
  (                                                    \
    assert(aCtx),                                      \
    assert(getContextsClass(aCtx->jInterp)             \
      ->pushParsedStringCtxProcessFunc),               \
    (getContextsClass(aCtx->jInterp)                   \
      ->pushParsedStringCtxProcessFunc(aCtx, aString)) \
  )
\stopCHeader

\setCHeaderStream{private}
\startCHeader
extern void pushParsedStringCtxProcessImpl(
  ContextObj* aCtx,
  Symbol* aString
);
\stopCHeader
\setCHeaderStream{public}

\startCCode
void pushParsedStringCtxProcessImpl(
  ContextObj* aCtx,
  Symbol* aString
) {
  if (!aCtx) return;
  if (!aString) return;
  TextObj* aText =
    createTextFromString(aCtx->jInterp, aString);
  pushParsedTextCtxProcess(aCtx, aText);
  freeText(aText);
}
\stopCCode

\startCHeader
#define pushParsedTextCtxProcess(aCtx, aText)      \
  (                                                \
    assert(aCtx),                                  \
    assert(getContextsClass(aCtx->jInterp)         \
      ->pushParsedTextCtxProcessFunc),             \
    (getContextsClass(aCtx->jInterp)               \
      ->pushParsedTextCtxProcessFunc(aCtx, aText)) \
  )
\stopCHeader

\setCHeaderStream{private}
\startCHeader
extern void pushParsedTextCtxProcessImpl(
  ContextObj* aCtx,
  TextObj* aText
);
\stopCHeader
\setCHeaderStream{public}

\startCCode
void pushParsedTextCtxProcessImpl(
  ContextObj* aCtx,
  TextObj* aText
) {
  if (!aCtx) return;
  if (!aText) return;
  assert(aCtx->jInterp);
  JObj* aLoL = parseAllSymbols(aText);
  if (!aLoL) return;
  aCtx->process = newPair(aCtx->jInterp, aLoL, aCtx->process);
}
\stopCCode

\startCHeader
#define prependListCtxProcess(aCtx, lolToPrepend)      \
  (                                                    \
    assert(aCtx),                                      \
    assert(getContextsClass(aCtx->jInterp)             \
      ->prependListCtxProcessFunc),                    \
    (getContextsClass(aCtx->jInterp)                   \
      ->prependListCtxProcessFunc(aCtx, lolToPrepend)) \
  )
\stopCHeader

\setCHeaderStream{private}
\startCHeader
extern void prependListCtxProcessImpl(
  ContextObj* aCtx,
  JObj* lolToPrepend
);
\stopCHeader
\setCHeaderStream{public}

\startCCode
void prependListCtxProcessImpl(
  ContextObj* aCtx,
  JObj* lolToPrepend
) {
  if (!aCtx) return;
  if (!lolToPrepend) return;

  aCtx->process =
    concatLists(aCtx->jInterp, lolToPrepend, aCtx->process);
}
\stopCCode

\startCHeader
#define popCtxProcess(aCtx)                    \
  (                                            \
    assert(aCtx),                              \
    assert(getContextsClass(aCtx->jInterp)     \
      ->popCtxProcessFunc),                    \
    (getContextsClass(aCtx->jInterp)           \
      ->popCtxProcessFunc(aCtx))               \
  )

#define popCtxProcessInto(aCtx, aVar)                           \
assert(aCtx);                                                   \
JObj* aVar = popCtxProcess(aCtx);                               \
if (aCtx->tracingOn) {                                          \
  JoyLoLInterp *jInterp = aCtx->jInterp;                        \
  StringBufferObj *aStrBuf = newStringBuffer(jInterp);          \
  strBufPrintf(jInterp, aStrBuf, "%s = ", #aVar);               \
  printLoL(jInterp, aStrBuf, aVar);                             \
  strBufPrintf(jInterp, aStrBuf, "\n");                         \
  jInterp->writeStdOut(jInterp, getCString(jInterp, aStrBuf));  \
  strBufClose(jInterp, aStrBuf);                                \
}

#define popCtxProcessIntoImpl(aCtx, aVar)                       \
assert(aCtx);                                                   \
JObj* aVar = popCtxProcessImpl(aCtx);                           \
if (aCtx->tracingOn) {                                          \
  JoyLoLInterp *jInterp = aCtx->jInterp;                        \
  StringBufferObj *aStrBuf = newStringBuffer(jInterp);          \
  strBufPrintf(jInterp, aStrBuf, "%s = ", #aVar);               \
  printLoL(jInterp, aStrBuf, aVar);                             \
  strBufPrintf(jInterp, aStrBuf, "\n");                         \
  jInterp->writeStdOut(jInterp, getCString(jInterp, aStrBuf));  \
  strBufClose(jInterp, aStrBuf);                                \
}
\stopCHeader

\setCHeaderStream{private}
\startCHeader
extern JObj* popCtxProcessImpl(
  ContextObj* aCtx
);
\stopCHeader
\setCHeaderStream{public}

\startCCode
JObj* popCtxProcessImpl(
  ContextObj* aCtx
) {
  assert(aCtx);
  if (!aCtx->process) return NULL;

  JObj* poppedLoL = asCar(aCtx->process);
  aCtx->process   = asCdr(aCtx->process);
  return poppedLoL;
}
\stopCCode

\startCHeader
#define showCtxProcess(aCtx, aStrBuf)              \
  assert(aCtx);                                    \
  printLoL(aCtx->jInterp, aStrBuf, aCtx->process); \
\stopCHeader

