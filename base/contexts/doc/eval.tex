% A ConTeXt document [master document: contexts.tex]

\section[title=Evaluation]

\startCCode
void traceAction(const char* action, PairAtom* aLoL, PairAtom* command) {
  char* lolStr = printLoL(aLoL);
  if (command) {
    char* commandStr = printLoL(command);
    printf("%s: %s (%s)\\n", action, lolStr, commandStr);
    free(commandStr);
  } else {
    printf("%s: %s\\n", action, lolStr);
  }
  free(lolStr);
}

void evalCommandInContext(Context* aCtx, PairAtom* command) {
  assert(aCtx);
  assert(aCtx->coAlgebras);
  assert(aCtx->coAlgebras->symbols);
  Dictionary* mainDic = aCtx->coAlgebras->symbols->dictionary;
  assert(mainDic);
  //
  // if there is no command... there is nothing to do
  //
  if (!command) return;
  //
  DEBUG(FALSE, "evalCommandInContext > %p %p\\n", aCtx, command);
  //
  // push this command onto the top of the process stack
  //
  pushCtxProcess(aCtx, command);
  //
  while(aCtx->process) {
    if (aCtx->tracingOn) printf("\\n ctx: %s\\n", aCtx->name);
    //
    // pop the next command off the process stack
    //
    command = popCtxProcess(aCtx);
    aCtx->command = command;
    //
    if (isFunction(command)) {
      //
      // if the command is a function.. call the function
      //
      if (aCtx->tracingOn) traceAction("calling", command, NULL);
      (command->func)(aCtx);
      //
    } else if (isContext(command)) {
      //
      // if the command is a context... switch to this contex
      //
      DEBUG(FALSE, "evalCommandInContext -> switching origCtx: %s\\n",
            aCtx->name);
      if (aCtx->tracingOn) traceAction("switching", command, NULL);
      Context* switchCtx = command->context;
      assert(switchCtx);

      popCtxDataInto(aCtx, oldCtxTop);
      pushCtxData(switchCtx, oldCtxTop);
      switchCtx->tracingOn = aCtx->tracingOn;
      aCtx = switchCtx;
      DEBUG(FALSE, "evalCommandInContext <- switching newCtx: %s\\n",
            aCtx->name);
      //
    } else if (!isSymbol(command)) {
      //
      // if the command is not a Symbol ...
      //  ...  push it onto the top of the data stack
      //
      if (aCtx->tracingOn) traceAction("adding", command, NULL);
      pushCtxData(aCtx, command);
      //
    } else {
      //
      // if the command is a symbol ...
      //  ... look up the symbol's association in the dictionary
      //
      AVLNode* assoc = findSymbol(mainDic, command->symbol);
      assert(assoc);
      //
      if (!assoc->value) {
        //
        // if the association is empty.. push this symbol onto the top
        // of the data stack (re-evaluating this symbol would lead to an
        // infinite loop)
        //
        if (aCtx->tracingOn) traceAction("adding", command, NULL);
        pushCtxData(aCtx, command);
        //
      } else if (isPair(assoc->value)) {
        //
        // if the association is a LoL.. push this LoL onto the top of the
        // process stack
        //
        if (aCtx->tracingOn) traceAction("evaluating", assoc->value, command);
        assert(aCtx->coAlgebras);
        prependListCtxProcess(aCtx, copyLoL(aCtx->coAlgebras, assoc->value));
        //
      } else if (isFunction(assoc->value)) {
        //
        // if the association is a function.. call the function
        //
        if (aCtx->tracingOn) traceAction("calling", command, NULL);
        (assoc->value->func)(aCtx);
        //
      } else if (isContext(assoc->value)) {
        //
        // if the association is a context... switch to this contex
        //
        DEBUG(FALSE, "evalCommandInContext -> switching origCtx: %s\\n",
              aCtx->name);
        if (aCtx->tracingOn) traceAction("switching", command, NULL);
        Context* switchCtx = assoc->value->context;
        assert(switchCtx);

        popCtxDataInto(aCtx, oldCtxTop);
        pushCtxData(switchCtx, oldCtxTop);
        switchCtx->tracingOn = aCtx->tracingOn;
        aCtx = switchCtx;
        DEBUG(FALSE, "evalCommandInContext <- switching newCtx: %s\\n",
              aCtx->name);
        //
      } else {
        //
        // if the association is NOT a PairAtom or Function...
        // ... push this new ATOM onto the top of the process stack
        //
        if (aCtx->tracingOn) traceAction("evaluating", assoc->value, command);
        pushCtxProcess(aCtx, assoc->value);
        //
      }
    }
    if (aCtx->tracingOn) {
      DEBUG(FALSE, "evalCommandInContext -> tracing%s\\n", "");
      printf("d>>%s\\n", printLoL(aCtx->data));
      printf("p>>%s\\n", printLoL(aCtx->process));
      DEBUG(FALSE, "evalCommandInContext <- tracing%s\\n", "");
    }
  } // aCtx->process is empty
  DEBUG(FALSE, "evalCommandInContext < %p %p\\n", aCtx, command);
}

void evalContext(Context* aCtx) {
  popCtxProcessInto(aCtx, aCommand);
  evalCommandInContext(aCtx, aCommand);
}

void evalTextInContext(Context* aCtx, Text* aText) {
  DEBUG(FALSE, "evalTextInContext > %p %p\\n", aCtx, aText);
  while(TRUE) {
    PairAtom* aLoL = parseOneSymbol(aText);
//    if (aCtx->showStack) printf("<%s\\n", printLoL(aLoL));
    if (!aLoL) break;
    evalCommandInContext(aCtx, aLoL);
//    if (aCtx->showStack) printf(">%s\\n", printLoL(aCtx->data));
    if (aCtx->exceptionRaised) break;
  }
  DEBUG(FALSE, "evalTextInContext < %p %p\\n", aCtx, aText);
}

void evalArrayOfStringsInContext(Context* aCtx,
                                 const char* someStrings[]) {
  DEBUG(FALSE, "evalStringInContext > %p %p\\n", aCtx, someStrings);
  assert(aCtx);
  assert(aCtx->coAlgebras);
  Text* stringText = createTextFromArrayOfStrings(aCtx->coAlgebras,
                                                  someStrings);
  evalTextInContext(aCtx, stringText);
  freeText(stringText);
  DEBUG(FALSE, "evalStringInContext < %p %p\\n", aCtx, someStrings);
}

void evalStringInContext(Context* aCtx, const char* aString) {
  DEBUG(FALSE, "evalStringInContext > %p [%s]\\n", aCtx, aString);
  assert(aCtx);
  assert(aCtx->coAlgebras);
  Text* stringText = createTextFromString(aCtx->coAlgebras, aString);
  evalTextInContext(aCtx, stringText);
  freeText(stringText);
  DEBUG(FALSE, "evalStringInContext < %p [%s]\\n", aCtx, aString);
}

void runREPLInContext(Context* aCtx) {
  assert(aCtx);
  assert(aCtx->coAlgebras);
  Text* aText = createTextFromReadline(aCtx->coAlgebras);
  while(TRUE) {
    DEBUG(FALSE, "runREPLInContext %p reading\\n", aCtx);
    PairAtom* aLoL = parseOneSymbol(aText);
    DEBUG(FALSE, "runREPLInContext %p read %p\\n", aCtx, aLoL);
    //
    if (!aLoL) break;
    //
    evalCommandInContext(aCtx, aLoL);
    assert(!aCtx->process);
    //
    if (aCtx->showStack) {
      DEBUG(FALSE, "runREPLInContext %p printing data %p\\n", aCtx, aCtx->data);
      printf(">>%s\\n", printLoL(aCtx->data));
      DEBUG(FALSE, "runREPLInContext %p printed data %p\\n", aCtx, aCtx->data);
    }
    //
    reportException(aCtx); // report last exception if raised
  }
  saveReadlineHistory(aText);
  DEBUG(FALSE, "runREPLInContext %p COMPLETED\\n", aCtx);
}
\stopCCode