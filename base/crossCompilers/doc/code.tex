% A ConTeXt document [master document: crossCompiler.tex]

\section[title=Code]
\setCHeaderStream{public}

\dependsOn[jInterps]
%\dependsOn[context]

\component gitVersion-c

\startCHeader
typedef struct crossCompiler_object_struct {
  JObj     super;
  Symbol  *type;
  DictObj *dict;
} CrossCompilerObj;

#define asCFunc(aLoL) (((CFunctionObj*)(aLoL))->func)
\stopCHeader

\startTestSuite[newCrossCompiler]

\startCHeader
typedef CrossCompilerObj* (NewCrossCompiler)(
  JoyLoLInterp *jInterp,
  Symbol       *aType
);

#define newCrossCompiler(jInterp, aType)      \
  (                                           \
    assert(getCrossCompilersClass(jInterp)    \
      ->newCrossCompilerFunc),                \
    (getCrossCompilersClass(jInterp)          \
      ->newCrossCompilerFunc(jInterp, aType)) \
  )

//#define asCrossCompiler(aLoL) (((aLoL)->flags) & BOOLEAN_FLAG_MASK)
\stopCHeader

\setCHeaderStream{private}
\startCHeader
extern CrossCompilerObj* newCrossCompilerImpl(
  JoyLoLInterp *jInterp,
  Symbol       *aType
);
\stopCHeader
\setCHeaderStream{public}

\startCCode
CrossCompilerObj* newCrossCompilerImpl(
  JoyLoLInterp *jInterp,
  Symbol       *aType
) {
  assert(jInterp);
  assert(jInterp->coAlgs);
  
  CrossCompilerObj* result =
    (CrossCompilerObj*)newObject(jInterp, CrossCompilersTag);
  assert(result);
  
  result->type = strdup(aType);
  result->dict         = newDictionary(jInterp);
  
  result->super.type   = jInterp->coAlgs[CrossCompilersTag];
 
  return result;
}
\stopCCode

\startTestCase[should create a new crossCompiler]

\startCTest
  AssertPtrNotNull(jInterp);

  CrossCompilerObj* aNewCrossCompiler =
    newCrossCompiler(jInterp, "ansiC");
  AssertPtrNotNull(aNewCrossCompiler);
  AssertPtrNotNull(asType(aNewCrossCompiler));
  AssertIntEquals(asTag(aNewCrossCompiler), CrossCompilersTag);
  AssertIntTrue(isAtom(aNewCrossCompiler));
  AssertIntTrue(isCrossCompiler(aNewCrossCompiler));
  AssertIntFalse(isPair(aNewCrossCompiler));
  AssertPtrNotNull(aNewCrossCompiler->dict);
  AssertIntTrue(isDictionary(aNewCrossCompiler->dict));
\stopCTest
\stopTestCase

\startTestCase[print CrossCompiler]
\startCTest
  AssertPtrNotNull(jInterp);

  StringBufferObj *aStrBuf = newStringBuffer(jInterp);
  AssertPtrNotNull(aStrBuf);

  CrossCompilerObj* aLoL =
    newCrossCompiler(jInterp, "ansiC");
  AssertPtrNotNull(aLoL);
  printLoL(jInterp, aStrBuf, (JObj*)aLoL);
  AssertStrEquals(getCString(jInterp, aStrBuf), "crossCompiler ");
  strBufClose(jInterp, aStrBuf);
\stopCTest
\stopTestCase

\stopTestSuite

\startTestSuite[addImplementation]

\startCHeader
typedef void (AddImplementation)(
  JoyLoLInterp *jInterp,
  Symbol       *ccType,
  Symbol       *wordName,
  Symbol       *implBody
);

#define addImplementation(jInterp, ccType, wordName, implBody)      \
  (                                                                 \
    assert(getCrossCompilersClass(jInterp)                          \
      ->addImplementationFunc),                                     \
    (getCrossCompilersClass(jInterp)                                \
      ->addImplementationFunc(jInterp, ccType, wordName, implBody)) \
  )
\stopCHeader

\setCHeaderStream{private}
\startCHeader
extern void addImplementationImpl(
  JoyLoLInterp *jInterp,
  Symbol       *ccType,
  Symbol       *wordName,
  Symbol       *implBody
);
\stopCHeader
\setCHeaderStream{public}

\startCCode
void addImplementationImpl(
  JoyLoLInterp *jInterp,
  Symbol       *ccType,
  Symbol       *wordName,
  Symbol       *implBody
) {
  assert(jInterp);
  
//  if (jInterp->debug) {
  if (TRUE) {
    StringBufferObj *aStrBuf = newStringBuffer(jInterp);
    strBufPrintf(jInterp, aStrBuf, 
      "DEBUG addImplementation crossCompiler: [%s]\n", ccType);
    strBufPrintf(jInterp, aStrBuf, "Word: [%s]\n", wordName);
    strBufPrintf(jInterp, aStrBuf, "Body: [%s]\n", implBody);
    jInterp->writeStdErr(jInterp, getCString(jInterp, aStrBuf));
    strBufClose(jInterp, aStrBuf);
  }
  
  assert(jInterp->compilers);

  CrossCompilerObj *theCC = NULL;
  if (strcmp(ccType, AnsicName) == 0) {
    theCC = jInterp->compilers[AnsicCC];
  } else if (strcmp(ccType, AnsicLuaName) == 0) {
    theCC = jInterp->compilers[AnsicLuaCC];
  } else if (strcmp(ccType, PureLuaName) == 0) {
    theCC = jInterp->compilers[PureLuaCC];
  }
  
  if (theCC) {
    jInterp->writeStdErr(jInterp, "addImplementation found theCC\n");
    checkObj(jInterp, theCC, CrossCompilersTag);
    DictObj *ccDict = theCC->dict;
    checkObj(jInterp, ccDict, DictionariesTag);
    
    DictNodeObj *entry    = insertSymbol(ccDict, wordName);
    checkObj(jInterp, entry, DictNodesTag);
    jInterp->writeStdErr(jInterp, "addImplementation inserted symbol\n");
    ImplementationObj *implementation =
      newImplementation(jInterp, wordName, implBody);
    jInterp->writeStdErr(jInterp, "addImplementation created fragment\n");
    checkObj(jInterp, implementation, ImplementationsTag);
    entry->value = (JObj*)implementation;
  } else {
    // raise error -- NEEDS Context!
    
    StringBufferObj *aStrBuf = newStringBuffer(jInterp);
    strBufPrintf(jInterp, aStrBuf, 
      "ERROR(impl) could not find the [%s] cross compiler.\n", ccType);
    strBufPrintf(jInterp, aStrBuf, "Word: [%s]\n", wordName);
    strBufPrintf(jInterp, aStrBuf, "Body: [%s]\n", implBody);
    jInterp->writeStdErr(jInterp, getCString(jInterp, aStrBuf));
    strBufClose(jInterp, aStrBuf);
  }
  jInterp->writeStdErr(jInterp, "addImplementation DONE\n");
}

static int lua_crossCompilers_addImplementation(lua_State *lstate) {
  getJoyLoLInterpInto(lstate, jInterp);
  Symbol *ccType   = luaL_checkstring(lstate, 1);
  Symbol *wordName = luaL_checkstring(lstate, 2);
  Symbol *implBody = luaL_checkstring(lstate, 3);
  addImplementationImpl(jInterp, ccType, wordName, implBody);
  lua_pop(lstate, 3);
  return 0;
}
\stopCCode
\stopTestSuite

\startTestSuite[addFragment]

\startCCode
static void addFragmentAP(ContextObj* aCtx) {
  assert(aCtx);
  assert(aCtx->jInterp);
  JoyLoLInterp *jInterp = aCtx->jInterp;

  popCtxDataInto(aCtx, ccTypeLoL);
  if (!isSymbol(ccTypeLoL)) {
    raiseException(aCtx,
      "addFragment expected a symbol as ccType");
    return;
  }
  Symbol *ccType = asSymbol(ccTypeLoL);
  
  popCtxDataInto(aCtx, wordNameLoL);
  if (!isSymbol(wordNameLoL)) {
    raiseException(aCtx,
      "addFragment expected a symbol as wordName");
    return;
  }
  Symbol *wordName = asSymbol(wordNameLoL);
  
  popCtxDataInto(aCtx, fragmentBodyLoL);
  if (!isSymbol(fragmentBodyLoL)) {
    raiseException(aCtx,
      "addFragment expected a symbol as fragmentBody");
    return;
  }
  Symbol *fragmentBody = asSymbol(fragmentBodyLoL);
  
//  if (jInterp->debug) {
  if (TRUE) {
    StringBufferObj *aStrBuf = newStringBuffer(jInterp);
    strBufPrintf(jInterp, aStrBuf, 
      "DEBUG addFragment crossCompiler: [%s]\n", ccType);
    strBufPrintf(jInterp, aStrBuf, "Word: [%s]\n", wordName);
    strBufPrintf(jInterp, aStrBuf, "Body: [%s]\n", fragmentBody);
    jInterp->writeStdErr(jInterp, getCString(jInterp, aStrBuf));
    strBufClose(jInterp, aStrBuf);
  }
  assert(jInterp->compilers);

  CrossCompilerObj *theCC = NULL;
  if (strcmp(ccType, AnsicName) == 0) {
    theCC = jInterp->compilers[AnsicCC];
  } else if (strcmp(ccType, AnsicLuaName) == 0) {
    theCC = jInterp->compilers[AnsicLuaCC];
  } else if (strcmp(ccType, PureLuaName) == 0) {
    theCC = jInterp->compilers[PureLuaCC];
  }
  if (theCC) {
    checkObj(jInterp, theCC, CrossCompilersTag);
    DictObj *ccDict = theCC->dict;
    checkObj(jInterp, ccDict, DictionariesTag);
    
    DictNodeObj *entry    = insertSymbol(ccDict, wordName);
    FragmentObj *fragment =
      newFragment(jInterp, wordName, fragmentBody);
    entry->value = (JObj*)fragment;
  } else {
    // raise error -- NEEDS Context!
    
    StringBufferObj *aStrBuf = newStringBuffer(jInterp);
    strBufPrintf(jInterp, aStrBuf, 
      "ERROR could not find the [%s] cross compiler.\n", ccType);
    strBufPrintf(jInterp, aStrBuf, "Word: [%s]\n", wordName);
    strBufPrintf(jInterp, aStrBuf, "Body: [%s]\n", fragmentBody);
    jInterp->writeStdErr(jInterp, getCString(jInterp, aStrBuf));
    strBufClose(jInterp, aStrBuf);
  }
}
\stopCCode

\startTestCase[should add a fragment to the appropriate compiler - ansic]
\startCTest
  pushSymbolCtxProcess(jInterp->rootCtx, "addFragment");
  pushSymbolCtxProcess(jInterp->rootCtx, "ansic");
  pushSymbolCtxProcess(jInterp->rootCtx, "test-ansic");
  pushSymbolCtxProcess(jInterp->rootCtx, "this is a test");
  pushSymbolCtxProcess(jInterp->rootCtx, "tracingOn");
  pushSymbolCtxProcess(jInterp->rootCtx, "showStack");
  evalContext(jInterp->rootCtx);
  pushSymbolCtxProcess(jInterp->rootCtx, "showStack");
  pushSymbolCtxProcess(jInterp->rootCtx, "tracingOff");
  evalContext(jInterp->rootCtx);
\stopCTest
\stopTestCase

\startCCode
static int lua_crossCompilers_addFragment(lua_State *lstate) {
  getJoyLoLInterpInto(lstate, jInterp);
  assert(jInterp);
  
  pushSymbolCtxProcess(jInterp->rootCtx, "addFragment");

  Symbol *ccType       = luaL_checkstring(lstate, 1);
  pushSymbolCtxProcess(jInterp->rootCtx, ccType);
  
  Symbol *wordName     = luaL_checkstring(lstate, 2);
  pushSymbolCtxProcess(jInterp->rootCtx, wordName);
  
  Symbol *fragmentBody = luaL_checkstring(lstate, 3);
  pushSymbolCtxProcess(jInterp->rootCtx, fragmentBody);
    
  //evalContext(jInterp->rootCtx);
  
  lua_pop(lstate, 3);
  return 0;
}
\stopCCode

\startTestCase[should add a fragment to the appropriate compiler - lua]
\startLuaTest
  local joylol       = thirddata.joylol
  local ccType       = "ansic"
  local wordName     = "test"
  local fragmentBody = "this is a test"
  joylol.crossCompiler.addFragment(ccType, wordName, fragmentBody)
  print(">>>>>>>>>>>>>>>>>>>>>>>>>>>>>>")
  print(joylol.showData)
  print(joylol.showProcess)
  print("<<<<<<<<<<<<<<<<<<<<<<<<<<<<<<")
\stopLuaTest
\skipTestCase
\stopTestSuite

\startTestSuite[isCrossCompiler]

\startCHeader
#define isCrossCompiler(aLoL)             \
  (                                       \
    (                                     \
      (aLoL) &&                           \
      asType(aLoL) &&                     \
      (asTag(aLoL) == CrossCompilersTag)  \
    ) ?                                   \
      TRUE :                              \
      FALSE                               \
  )
\stopCHeader

\setCHeaderStream{private}
\startCHeader
extern Boolean equalityCrossCompilerCoAlg(
  JoyLoLInterp *jInterp,
  JObj     *lolA,
  JObj     *lolB
);
\stopCHeader
\setCHeaderStream{public}

\startCCode
Boolean equalityCrossCompilerCoAlg(
  JoyLoLInterp *jInterp,
  JObj     *lolA,
  JObj     *lolB
) {
  DEBUG(jInterp, "crossCompilerCoAlg-equal a:%p b:%p\n", lolA, lolB);
  if (!lolA && !lolB) return TRUE;
  if (!lolA && lolB)  return FALSE;
  if (lolA  && !lolB) return FALSE;
  if (asType(lolA) != asType(lolB)) return FALSE;
  if (!asType(lolA)) return FALSE;
  if (asTag(lolA)  != CrossCompilersTag) return FALSE;
  if (lolA != lolB) return FALSE;
  return TRUE;
}
\stopCCode

\startTestSuite[printing crossCompilers]

\setCHeaderStream{private}
\startCHeader
extern Boolean printCrossCompilerCoAlg(
  JoyLoLInterp    *jInterp,
  StringBufferObj *aStrBuf,
  JObj        *aLoL
);
\stopCHeader
\setCHeaderStream{public}

\startCCode
Boolean printCrossCompilerCoAlg(
  JoyLoLInterp    *jInterp,
  StringBufferObj *aStrBuf,
  JObj        *aLoL
) {
  assert(aLoL);
  assert(asTag(aLoL) == CrossCompilersTag);

  strBufPrintf(jInterp, aStrBuf, "crossCompiler ");
  return TRUE;
}
\stopCCode

\startTestCase[should print crossCompilers]

\startCTest
  AssertPtrNotNull(jInterp);
  AssertPtrNotNull(jInterp->coAlgs[CrossCompilersTag]);

  StringBufferObj *aStrBuf = newStringBuffer(jInterp);
  AssertPtrNotNull(aStrBuf);
  
  CrossCompilerObj* aNewCrossCompiler =
    newCrossCompiler(jInterp, "ansiC");
  AssertPtrNotNull(aNewCrossCompiler);
  printLoL(jInterp, aStrBuf, (JObj*)aNewCrossCompiler);
  AssertStrEquals(getCString(jInterp, aStrBuf), "crossCompiler ");
  strBufClose(jInterp, aStrBuf);
\stopCTest
\stopTestCase
\stopTestSuite

\startTestSuite[registerCrossCompilers]

\startCHeader
typedef struct crossCompilers_class_struct {
  JClass            super;
  NewCrossCompiler  *newCrossCompilerFunc;
  AddImplementation *addImplementationFunc;
} CrossCompilersClass;

\stopCHeader

\startCCode
static Boolean initializeCrossCompilers(
  JoyLoLInterp *jInterp,
  JClass   *aJClass
) {
  assert(jInterp);
  assert(aJClass);
  registerCrossCompilerWords(jInterp);

  CrossCompilerObj *ansic =
    newCrossCompilerImpl(jInterp, AnsicName);
  registerCrossCompiler(jInterp, ansic);
  CrossCompilerObj *ansicLua =
    newCrossCompilerImpl(jInterp, AnsicLuaName);
  registerCrossCompiler(jInterp, ansicLua);
  CrossCompilerObj *pureLua =
    newCrossCompilerImpl(jInterp, PureLuaName);
  registerCrossCompiler(jInterp, pureLua);

  return TRUE;
}
\stopCCode

\setCHeaderStream{private}
\startCHeader
extern Boolean registerCrossCompilers(JoyLoLInterp *jInterp);
\stopCHeader
\setCHeaderStream{public}

\startCCode
Boolean registerCrossCompilers(JoyLoLInterp *jInterp) {
  assert(jInterp);
  assert(jInterp->coAlgs);
  
  CrossCompilersClass* theCoAlg
    = joyLoLCalloc(1, CrossCompilersClass);
  assert(theCoAlg);
  
  theCoAlg->super.name            = CrossCompilersName;
  theCoAlg->super.objectSize      = sizeof(CrossCompilerObj);
  theCoAlg->super.initializeFunc  = initializeCrossCompilers;
  theCoAlg->super.equalityFunc    = equalityCrossCompilerCoAlg;
  theCoAlg->super.printFunc       = printCrossCompilerCoAlg;
  theCoAlg->newCrossCompilerFunc  = newCrossCompilerImpl;
  theCoAlg->addImplementationFunc = addImplementationImpl;

  size_t tag =
    registerJClass(jInterp, (JClass*)theCoAlg);
  
  // do a sanity check...
  assert(tag == CrossCompilersTag);
  assert(jInterp->coAlgs[tag]);
   
  return TRUE;
}
\stopCCode

\startTestCase[should register the CrossCompilers coAlg]

\startCTest
  // CTestsSetup has already created a jInterp
  // and run registerCrossCompilers
  AssertPtrNotNull(jInterp);
  AssertPtrNotNull(jInterp->coAlgs);
  AssertPtrNotNull(getCrossCompilersClass(jInterp));
  CrossCompilersClass *coAlg = getCrossCompilersClass(jInterp);
  registerCrossCompilers(jInterp);
  AssertPtrNotNull(getCrossCompilersClass(jInterp));
  AssertPtrEquals(getCrossCompilersClass(jInterp), coAlg);
  AssertIntEquals(
    getCrossCompilersClass(jInterp)->super.objectSize,
    sizeof(CrossCompilerObj)
  )
\stopCTest
\stopTestCase
\stopTestSuite
