% A ConTeXt document [master document: pairs.tex]

\section[title=Code]
\setCHeaderStream{public}

\dependsOn[jInterps]
\dependsOn[symbols]
\dependsOn[stringBuffers]
%\dependsOn[context]

\startCHeader
typedef struct pairs_object_struct {
  JObj  super;
  JObj* car;
  JObj* cdr;
} PairsObj;
\stopCHeader

\startTestSuite[newPair]

\startCHeader
typedef JObj* (NewPair)(
  JoyLoLInterp* jInterp,
  JObj* car,
  JObj* cdr
);

#define newPair(jInterp, aCar, aCdr)      \
  (                                       \
    assert(getPairsClass(jInterp)         \
      ->newPairFunc),                     \
    (getPairsClass(jInterp)               \
      ->newPairFunc(jInterp, aCar, aCdr)) \
  )

#define asCar(aLoL) (((PairsObj*)(aLoL))->car)
#define asCdr(aLoL) (((PairsObj*)(aLoL))->cdr)
\stopCHeader

\setCHeaderStream{private}
\startCHeader
extern JObj* newPairImpl(
  JoyLoLInterp* jInterp,
  JObj* car,
  JObj* cdr
);
\stopCHeader
\setCHeaderStream{public}

\startCCode
JObj* newPairImpl(
  JoyLoLInterp* jInterp,
  JObj* car,
  JObj* cdr
) {
  assert(jInterp);
  JObj* result = newObject(jInterp, PairsTag);
  assert(result);
  asCar(result) = car;
  asCdr(result) = cdr;
  return result;
}
\stopCCode

\startTestCase[should create a new pair]

\startCTest
  AssertPtrNotNull(jInterp);
  
  JObj *lolA = newPair(jInterp, NULL, NULL);
  JObj *lolB = newPair(jInterp, NULL, NULL);
  JObj *lol  = newPair(jInterp, lolA, lolB);
  AssertPtrEquals(asType(lol), jInterp->coAlgs[PairsTag]);
  AssertIntEquals(asTag(lol), PairsTag);
  AssertIntZero(asFlags(lol));
  AssertPtrEquals(asCar(lol), lolA);
  AssertPtrEquals(asCdr(lol), lolB);  
\stopCTest
\stopTestCase
\stopTestSuite

\startTestSuite[car and cdr]

\startCHeader
typedef JObj *(CarCdr)(JObj*);

#define getCar(jInterp, aLoL)               \
  (                                         \
    assert(getPairsClass(jInterp)->carFunc),\
    (getPairsClass(jInterp)->carFunc(aLoL)) \
  )

#define getCdr(jInterp, aLoL)               \
  (                                         \
    assert(getPairsClass(jInterp)->cdrFunc),\
    (getPairsClass(jInterp)->cdrFunc(aLoL)) \
  )
\stopCHeader

\setCHeaderStream{private}
\startCHeader
extern JObj* carImpl(JObj* aLoL);
extern JObj* cdrImpl(JObj* aLoL);
\stopCHeader
\setCHeaderStream{public}

\startCCode
JObj* carImpl(JObj* aLoL) {
  if (!aLoL) return NULL;
  if (asTag(aLoL) != PairsTag) return aLoL;
  return asCar(aLoL);
}

JObj* cdrImpl(JObj* aLoL) {
  if (!aLoL) return NULL;
  if (asTag(aLoL) != PairsTag) return NULL;
  return asCdr(aLoL);
}
\stopCCode

\startTestCase[should return the car and cdr]

\startCTest
  JObj *pairA = newPair(jInterp, NULL, NULL);
  JObj *pairB = newPair(jInterp, NULL, NULL);
  JObj *aPair = newPair(jInterp, pairA, pairB);
  AssertPtrEquals(getCar(jInterp, aPair), pairA);
  AssertPtrEquals(getCdr(jInterp, aPair), pairB);
\stopCTest
\stopTestCase
\stopTestSuite

\startTestSuite[concatLists]
\startCHeader
typedef JObj *(ConcatLists)(
  JoyLoLInterp *jInterp,
  JObj         *firstList,
  JObj         *secondList
);

#define concatLists(jInterp, firstList, secondList)       \
  (                                                       \
    assert(getPairsClass(jInterp)->concatListsFunc),      \
    (getPairsClass(jInterp)                               \
      ->concatListsFunc(jInterp, firstList, secondList))  \
  )
\stopCHeader

\setCHeaderStream{private}
\startCHeader
extern JObj* concatListsImpl(
  JoyLoLInterp *jInterp,
  JObj         *firstList,
  JObj         *secondList
);
\stopCHeader
\setCHeaderStream{public}

\startCCode
JObj* concatListsImpl(
  JoyLoLInterp *jInterp,
  JObj         *firstList,
  JObj         *secondList
) {
  assert(jInterp);
  
  JObj *result = NULL;
  
  if (firstList && isPair(firstList)) {
    result = firstList;
  } else {
    result = newPairImpl(jInterp, firstList, NULL);
  }
  
  if (!secondList) return result;
  // ensure that the second list is a LIST
  if (!isPair(secondList)) {
    secondList = newPair(jInterp, secondList, NULL);
  }

  // find end of firstList/result
  JObj* lolList = result;
  while(asCdr(lolList) && isPair(asCdr(lolList))) {
    lolList = asCdr(lolList);
  }

  // ensure that if firstList/result ends in a non-pair we make it a pair
  if (asCdr(lolList) && !isPair(asCdr(lolList))) {
    asCdr(lolList) = newPair(jInterp, asCdr(lolList), NULL);
    assert(asCdr(lolList));
    lolList = asCdr(lolList);
  }

  // place secondList at the end of firstList/result
  assert(!asCdr(lolList));
  asCdr(lolList) = secondList;
  
  return result;
}

\stopCCode

\startTestCase[should concatinate two LoLs]
\startCTest
  AssertPtrNotNull(jInterp);
  StringBufferObj *aStrBuf = newStringBuffer(jInterp->rootCtx);
  
  JObj* result = concatLists(jInterp, NULL, NULL);
  AssertPtrNotNull(result);
  AssertIntTrue(isPair(result));
  printLoL(aStrBuf, result);
  AssertStrEquals(getCString(aStrBuf), "( ) ");
  strBufClose(aStrBuf);
  
  JObj *trueBool  = newBoolean(jInterp, TRUE);
  result = concatLists(jInterp, trueBool, NULL);
  AssertPtrNotNull(result);
  AssertIntTrue(isPair(result));
  AssertPtrNotNull(asCar(result));
  AssertIntTrue(isTrue(asCar(result)));
  AssertPtrNull(asCdr(result));
  printLoL(aStrBuf, result);
  AssertStrEquals(getCString(aStrBuf), "true ");
  strBufClose(aStrBuf);
  
  JObj *falseBool = newBoolean(jInterp, FALSE);
  result = concatLists(jInterp, trueBool, falseBool);
  AssertPtrNotNull(result);
  AssertIntTrue(isPair(result));
  AssertPtrNotNull(asCar(result));
  AssertIntTrue(isTrue(asCar(result)));
  AssertPtrNotNull(asCdr(result));
  AssertIntTrue(isFalse(asCdr(result)));
  printLoL(aStrBuf, result);
  AssertStrEquals(getCString(aStrBuf), "true false ");
  strBufClose(aStrBuf);
  
  JObj* firstList = result;
  result = concatLists(jInterp, result, NULL);
  AssertPtrEquals(firstList, result);
  printLoL(aStrBuf, result);
  AssertStrEquals(getCString(aStrBuf), "true false ");
  strBufClose(aStrBuf);
  
  result = concatLists(jInterp, firstList, trueBool);
  AssertPtrNotNull(result);
  AssertPtrEquals(firstList, result);
  AssertPtrNotNull(asCar(result));
  AssertIntTrue(isTrue(asCar(result)));
  AssertPtrNotNull(asCdr(result));
  AssertIntTrue(isPair(asCdr(result)));
  AssertIntTrue(isFalse(asCar(asCdr(result))));
  printLoL(aStrBuf, result);
  AssertStrEquals(getCString(aStrBuf), "true false true ");
  strBufClose(aStrBuf);
\stopCTest
\stopTestCase
\stopTestSuite

\startTestSuite[copyLoL]

\startCHeader
typedef JObj* (CopyLoL)(
  JoyLoLInterp* jInterp,
  JObj* aLoL
);

#define copyLoL(jInterp, aLoL)      \
  (                                 \
    assert(getPairsClass(jInterp)   \
      ->copyLoLFunc),               \
    (getPairsClass(jInterp)         \
      ->copyLoLFunc(jInterp, aLoL)) \
  )
\stopCHeader

\setCHeaderStream{private}
\startCHeader
extern JObj* copyLoLImpl(
  JoyLoLInterp* jInterp,
  JObj* aLoL
);
\stopCHeader
\setCHeaderStream{public}

\startCCode
JObj* copyLoLImpl(
  JoyLoLInterp* jInterp,
  JObj* aLoL
) {
  assert(jInterp);
  if (!aLoL) return NULL;
  if (isAtom(aLoL)) return aLoL;

  return newPair(jInterp,
                 copyLoLImpl(jInterp, asCar(aLoL)),
                 copyLoLImpl(jInterp, asCdr(aLoL))
                 );
}
\stopCCode

\startTestCase[should make a correct copy]

\startCTest
  JObj *bool   = newBoolean(jInterp, TRUE);
  JObj *pairA  = newPair(jInterp, bool, NULL);
  JObj *pairB  = newPair(jInterp, pairA, bool);
  JObj *aPair0 = newPair(jInterp, pairA, pairB);
  JObj *aPair1 = copyLoL(jInterp, aPair0);
  AssertPtrNotNull(bool);
  AssertPtrNotNull(pairA);
  AssertPtrNotNull(pairB);
  AssertPtrNotNull(aPair0);
  AssertPtrNotNull(aPair1);
  AssertPtrNotNull(asType(aPair1));
  AssertPtrEquals(asType(aPair1),
    (JClass*)getPairsClass(jInterp));

  AssertPtrNotNull(asCar(aPair1));
  AssertPtrNotEquals(asCar(aPair1), pairA);

  AssertPtrNotNull(asCar(asCar(aPair1)));
  AssertPtrEquals(asCar(asCar(aPair1)), bool);
  AssertIntTrue(isBoolean(asCar(asCar(aPair1))));
  AssertIntTrue(isTrue(asCar(asCar(aPair1))));

  AssertPtrNull(asCdr(asCar(aPair1)));

  AssertPtrNotNull(asCdr(aPair1));
  AssertPtrNotEquals(asCdr(aPair1), pairB);

  AssertPtrNotNull(asCar(asCdr(aPair1)));
  AssertIntTrue(isPair(asCar(asCdr(aPair1))));

  AssertPtrNotNull(asCdr(asCdr(aPair1)));
  AssertPtrEquals(asCdr(asCdr(aPair1)), bool);
  AssertIntTrue(isBoolean(asCdr(asCdr(aPair1))));
  AssertIntTrue(isTrue(asCdr(asCdr(aPair1))));
\stopCTest
\stopTestCase
\stopTestSuite

\startTestSuite[equalLoL]

\startCHeader
typedef Boolean (EqualLoL)(
  JoyLoLInterp *jInterp,
  JObj         *lolA,
  JObj         *lolB,
  size_t        timeToLive
);

#define equalLoL(jInterp, lolA, lolB, ttl)      \
  (                                             \
    assert(getPairsClass(jInterp)               \
      ->equalLoLFunc),                          \
    (getPairsClass(jInterp)                     \
      ->equalLoLFunc(jInterp, lolA, lolB, ttl)) \
  )

#define lolEqual(jInterp, areEqual, lolA, lolB, ttl)  \
  if (areEqual && (lolA)) {			                      \
    assert(asType(lolA));  			                      \
    areEqual = (asType(lolA)->equalityFunc)	          \
      (jInterp, (lolA), (lolB), (ttl));               \
  }
\stopCHeader

\setCHeaderStream{private}
\startCHeader
Boolean equalLoLImpl(
  JoyLoLInterp *jInterp,
  JObj         *lolA,
  JObj         *lolB,
  size_t        timeToLive
);
\stopCHeader
\setCHeaderStream{public}

\startCCode
Boolean equalLoLImpl(
  JoyLoLInterp *jInterp,
  JObj         *lolA,
  JObj         *lolB,
  size_t        timeToLive
) {
  DEBUG(jInterp, "equalLoL %p %p %zu\n", lolA, lolB, timeToLive);
  if (!lolA && !lolB) return TRUE;
  if (!lolA || !lolB) return FALSE;
  if (asType(lolA) != asType(lolB)) return FALSE;
  if (asType(lolA) && 
     (asTag(lolA) != PairsTag)) {
    return (asType(lolA)->equalityFunc)
      (jInterp, lolA, lolB, timeToLive);
  }

  if (timeToLive < 1) return TRUE;

  if (!equalLoLImpl(
      jInterp, 
      asCar(lolA),
      asCar(lolB),
      (timeToLive-1)
    )) {
    return FALSE;
  }
  return equalLoLImpl(
    jInterp,
    asCdr(lolA),
    asCdr(lolB),
    (timeToLive -1)
  );
}
\stopCCode

\startTestCase[should return true if pairs are equal]
\startCTest
  JObj *pairA = newPair(jInterp, NULL, NULL);
  JObj *pairB = newPair(jInterp, NULL, NULL);
  AssertPtrNotNull(jInterp);
  AssertPtrNotNull(jInterp->coAlgs);
  AssertPtrNotNull(getPairsClass(jInterp));
  AssertPtrNotNull(getPairsClass(jInterp)->equalLoLFunc);
  AssertIntTrue(equalLoL(jInterp, NULL, NULL, 10));
  AssertIntTrue(equalLoL(jInterp, pairA, pairB, 10));
  // need to test with NON-Pairs
\stopCTest
\stopTestCase

\startTestCase[should return false if pairs are not equal]
\startCTest
  JObj *pairA = newPair(jInterp, NULL,  NULL);
  JObj *pairB = newPair(jInterp, pairA, NULL);
  AssertIntFalse(equalLoL(jInterp, pairA, pairB, 10));
  // need to test with NON-Pairs
\stopCTest
\stopTestCase
\stopTestSuite

\startTestSuite[printing pairs]

\startCHeader
#define lolPrintStr(                                \
  aStrBuf, printedOk, aLoL, opener, closer, ttl)    \
  if (aLoL) {								                        \
    size_t isList = asTag(aLoL) == PairsTag;        \
    if (isList) strBufPrintf((aStrBuf), (opener));  \
    assert(asType(aLoL));						                \
    (printedOk) = (printedOk) && 						        \
      (asType(aLoL)->printFunc)                     \
        ((aStrBuf), (aLoL), (ttl));                 \
    if (isList) strBufPrintf((aStrBuf), (closer));  \
  } else {								                          \
    strBufPrintf((aStrBuf), (opener));						  \
    strBufPrintf((aStrBuf), (closer));						  \
  }
\stopCHeader

\setCHeaderStream{private}
\startCHeader
extern Boolean printPairsCoAlg(
  StringBufferObj *aStrBuf,
  JObj            *aLoL,
  size_t           timeToLive
);
\stopCHeader
\setCHeaderStream{public}

\startCCode
Boolean printPairsCoAlg(
  StringBufferObj *aStrBuf,
  JObj            *aLoL,
  size_t           timeToLive
) {
  assert(aLoL);
  assert(asType(aLoL));
  assert(asTag(aLoL) == PairsTag);

  if (timeToLive < 1) {
    strBufPrintf(aStrBuf, "... ");
    return TRUE;
  }
  
  size_t printedOk = TRUE;
  lolPrintStr(
    aStrBuf,
    printedOk,
    asCar(aLoL),
    "( ", ") ",
    timeToLive-1
  );
  lolPrintStr(
    aStrBuf,
    printedOk,
    asCdr(aLoL),
    "",   "",
    timeToLive-1
  );
  return printedOk;
}
\stopCCode

\startTestCase[should print pairs]

\startCTest
  AssertPtrNotNull(jInterp);
  AssertPtrNotNull(getPairsClass(jInterp));

  JObj* aNewPair = newPair(jInterp,
                               newPair(jInterp, NULL, NULL),
                               newPair(jInterp, NULL, NULL));
  AssertPtrNotNull(aNewPair);
  StringBufferObj *aStrBuf = newStringBuffer(jInterp->rootCtx);
  printLoL(aStrBuf, aNewPair);
  AssertStrEquals(getCString(aStrBuf), "( ( ) ) ( ) ");
  strBufClose(aStrBuf);
\stopCTest
\stopTestCase
\stopTestSuite

\startTestSuite[registerPairs]

\startCHeader
typedef struct pairs_class_struct {
  JClass       super;
  NewPair     *newPairFunc;
  CarCdr      *carFunc;
  CarCdr      *cdrFunc;
  ConcatLists *concatListsFunc;
  EqualLoL    *equalLoLFunc;
  CopyLoL     *copyLoLFunc;
} PairsClass;
\stopCHeader

\startCCode
static Boolean initializePairs(
  JoyLoLInterp *jInterp,
  JClass   *aJClass
) {
  assert(jInterp);
  assert(aJClass);
  return TRUE;
}
\stopCCode

\setCHeaderStream{private}
\startCHeader
extern Boolean registerPairs(JoyLoLInterp *jInterp);
\stopCHeader
\setCHeaderStream{public}

\startCCode
Boolean registerPairs(JoyLoLInterp *jInterp) {
  assert(jInterp);
  
  PairsClass* theCoAlg           = joyLoLCalloc(1, PairsClass);
  theCoAlg->super.name           = PairsName;
  theCoAlg->super.objectSize     = sizeof(PairsObj);
  theCoAlg->super.initializeFunc = initializePairs;
  theCoAlg->super.registerFunc   = registerPairWords;
  theCoAlg->super.equalityFunc   = equalLoLImpl;
  theCoAlg->super.printFunc      = printPairsCoAlg;
  theCoAlg->newPairFunc          = newPairImpl;
  theCoAlg->carFunc              = carImpl;
  theCoAlg->cdrFunc              = cdrImpl;
  theCoAlg->concatListsFunc      = concatListsImpl;
  theCoAlg->equalLoLFunc         = equalLoLImpl;
  theCoAlg->copyLoLFunc          = copyLoLImpl;
  size_t tag =
    registerJClass(jInterp, (JClass*)theCoAlg);
  
  // sanity check...
  assert(tag == PairsTag);
  assert(jInterp->coAlgs[tag]);

  return TRUE;
}
\stopCCode

\startTestCase[should register the Pairs coAlg]

\startCTest
  // CTestSetup has already created a jInterp
  // and run registerPairs
  AssertPtrNotNull(jInterp);
  AssertPtrNotNull(getPairsClass(jInterp));
  PairsClass *coAlg = getPairsClass(jInterp);
  AssertIntTrue(registerPairs(jInterp));
  AssertPtrNotNull(getPairsClass(jInterp));
  AssertPtrEquals(getPairsClass(jInterp), coAlg);
  AssertIntEquals(
    getPairsClass(jInterp)->super.objectSize,
    sizeof(PairsObj)
  );
\stopCTest
\stopTestCase
\stopTestSuite

\setCHeaderStream{public}
