% LaTeX source for the joyOfLoL document
%

\documentclass[a4paper,openany]{amsart}
\usepackage[utf8]{inputenc}
\usepackage[english]{babel}
\usepackage{disitt}
\usepackage{disitt-symbols}
\usepackage{disitt-code}
\usepackage[backend=biber,style=alphabetic,citestyle=alphabetic]{biblatex}
\addbibresource{joyOfLoL.bib}
\usepackage{mdframed}
\newmdenv[linecolor=white,backgroundcolor=gray!10]{infobox}
\newenvironment{myQuote}{\begin{quotation}}{\end{quotation}}
\surroundwithmdframed[linecolor=white,backgroundcolor=gray!10]{myQuote}

\begin{document}
	
\sloppy
	
\title[Joy of LoL]{The Joy of Laughing out Loud: Lists of Lists and Joy
processes as a Computational Foundation for Mathematics}
 %% author: stg
\author{Stephen Gaito}
\address{PerceptiSys Ltd, 21 Gregory Ave, Coventry, CV3 6DJ, United Kingdom}%
\email{stephen@perceptisys.co.uk}%
\urladdr{http://www.perceptisys.co.uk}


%% version summary
\thanks{Created: 2016-10-12}
\thanks{Git commit \gitReferences{} (\gitAbbrevHash{}) commited on \gitAuthorDate{} by \gitAuthorName{}}
\thanks{AMS-\LaTeX{}'ed on \today{}.}

%% Copyrights
\thanks{\textbf{Copyright: \copyright{} Stephen Gaito, PerceptiSys Ltd \the\year{}; Some rights reserved}}
\thanks{\textbf{This work is licensed under a Creative Commons Attribution-ShareAlike 4.0 International License.}}

\subjclass[2010]{Primary unknown; Secondary unknown} %
\keywords{Keyword one, keyword two etc.}%
	
\begin{abstract}
We show that the collection of Lists of Lists and Joy processes form a
Computational Foundation of Mathematics.
\end{abstract} 
\maketitle 
\tableofcontents 
	
	
\section{Introduction} For historical reasons, all current foundations of
Mathematics are based upon first or higher order logic, or alternatively
(logically interpreted) Type Theory. \TODO{what about category theory?} In order
to provide a Computational Foundation of Mathematics we need to answer three
primary questions:

\begin{itemize}
\item What is Mathematics?
\item What constitutes a Foundation of Mathematics?
\item What is a Computational Process?
\end{itemize}

\subsection{What is Mathematics?}

Again, for historical reasons, Mathematics is currently identified with some
super-set of a formulation of the axioms of the well-founded sets, such as ZFC.
In particular it is assumed that Cantor's collection of Ordinals larger than
$\omega$ and their well-ordering are \emph{required} to provide a credible
formulation of real analysis as required for Mathematical Physics.

While we assert that what is normally considered computation, augmented with
something as powerful as the Axiom of Choice, can provide a Computational
Foundation for what is currently considered Mathematics, we will argue that
Computational \emph{processes} are more than sufficient to provide foundations
for real analysis. 

\TODO{fill in this gap: discuss Topos as a model of well-founded set theory.
Discuss Coalgebras as a model of processes. See fingerPiece01}

For the purposes of \emph{this paper} we will define Mathematics to be any fully
algebraic pair of Topos which contain an \emph{implementation} of the Reals.

\subsection{What constitutes a Foundation of Mathematics?}

\TODO{Discuss organizational/structural, formal/existential,
	ontological/material modes of foundation. See
	\cite{marquis1995categoryFoundationsMath} as well as \cite{awodey2009a}. Are
	these usefully seen as being in a linear adjoint chain?}

Following Hatcher, \cite[section 2.5]{hatcher1982logicalFoundationsMath}, we posit the
following list of criteria any Foundation of Mathematics must satisfy:

\begin{enumerate}
\item \textbf{A foundation of mathematics must be adequate for a reasonably large
portion of mathematics.}
	
\noindent If we ``assume'' something as powerful as the Axiom of Choice, we will
be able to recover the whole of current mathematical set theory (smaller than
the first strongly inaccessible cardinal).
	
\item \textbf{A foundation must derive from some intuitively natural principles.}
	
\noindent The foundational (co-)data language \emph{is} the List of Lists. The
only operations on this language are LISP's car, cdr, cons, together with
failure and repetition.
	
\item \textbf{The basic principles and primitive (undefined) notions should be
as economical as possible.}
	
\noindent See above.
	
\item \textbf{The foundation must be consistent.}
	
\noindent For a computational foundation, consistency is given since we are
\emph{computing} the (standard) model of the formal theory. What is more
important is that the formal theory is fully abstract, that is the denotation
and operational semantics are equivalent, in logical terms, the formal theory is
both sound and complete.
	
\item \textbf{The foundation should be expressed (or expressible) as a formal
system.}
	
\noindent In computational terms, each formal system is a (programming) language
complete with a syntax as well as denotational, operational and axiomatic
semantics. Every such formal system \emph{must} have (at least) a denotational
interpretation into the foundational (co-)data language.
	
\item \textbf{The construction of everyday mathematics in the system should be
``natural'' and ``orderly''.}
	
\noindent With the proposed computational foundations of mathematics, there is a
collection of formal theories each of which have (at least) an interpretation in
the (co-)data language. Each of these formal theories should be tailored to the
mathematical field in question, so by definition they should represent
``natural'' and ``orderly'' systems in which to conduct mathematics in a given
field.
	
\end{enumerate}

\subsection{What is a Computational Process?}

\subsection{Why Lists of Lists and Joy processes?}

\section{Lists of Lists}

\subsection{What is a List of Lists?} 

\TODO{see \cite{awodey2009a} ``From Sets to Types to Categories to Sets''. see \cite{marquis1995categoryFoundationsMath}}

In classical, logically defined, \textbf{set theory}, a set is a collection of
``objects'' which are all ``observed'' to have the same \emph{logical}
``property''. If we, like Frege, use any arbitrary logical predicate to
``select'' the objects of a given set, we can all to easily fall into various
logical paradoxes. To avoid these paradoxes we restrict the allowed predicates
to those that satisfy one or more ``standard'' collections of axioms. These
``standard'' collections of axioms are carefully crafted to avoid the known
paradoxes, as well as provide ``useful'' mathematical ``constructions''. Of
particular importance is that the axioms of set theory are constraining /
defining the ``element'' relation, $a \in A$.

\TODO{see \cite{awodeyCoquandVoevodsky2013homotopyTypeTheory} section 1.1 for a
good introduction}

In classical, \textbf{type theory}, a \emph{type} is a collection of ``objects''
which are (``logically'') judged to be of a specific type, $a:A$.

In classical, \textbf{category theory}, being structuralist, instead of sets or
types having elements or objects (respectively), a collection of objects (with
no internal structure) has a collection of relationships, ``arrows''. 

In each case, we have an unspecified collection of ``things'' for which we can
make observations.

Literally, the lists of lists is the ``collection'' of ``things'' to which we
can apply the basic joy operations to observe the ``thing's'' ``structure''. Being
a finite being, computational or organic, we can only ever apply a finite number
of operation/observations, but that does not mean that every list of lists is a
finite structure. It only means that there are potentially lists of lists whose
``full'' structure we can not explore in a finite number of
operation/observations.

For computer scientists, this is not a problematic concept. There are many
computational processes whose ``usefulness'' is not captured in a finite number
of steps. For example our ``operating systems'' (be they Windows, Mac, or Unix)
would not be much use if they were limited in the number of cycles they could
run. While we might turn them off, that is our external choice not an internal
limitation of the operating system itself. So the computational concept of
process is of a computational ``program'' with an potentially unlimited number
of cycles. It usefulness to us is in the external observations we might make of
the process's ongoing behaviour rather than any ``final'' answer it might
``compute''.

Similarly, a list of lists is a potentially infinite ``object'' to which any
finite computation can only build or observer a finite portion of it.

Given a collection of ``things'' we also need to understand when two potentially
different ``things'' are really the same. Set theory declares two sets to be
equal if they have the same collection of elements, that is if they are
extensionally equal. For the lists of lists, we declare that two potentially
distinct list of lists are equal if given the same collection of observations,
they always give the same collection of ``answers''. This is an extensional
concept of equality.

\subsubsection{What are the Joy operations}

\begin{itemize}

	\item push () \TODO{similar to car in Lisp; similar to write in Turing machine}

	\item pop  \TODO{same as cdr in Lisp; similar to erase in Turing machine}

	\item dup  \TODO{similar to car in Lisp; similar to move right in Turing
		machine; is this too powerful?}

	\item cons \TODO{same as cons in Lisp; similar to move left in Turing machine}
	
	\item ifte \TODO{similar to cond in Lisp; similar to a Turing machine's ability
		to choose its next state/action dependent upon its current state/tape-symbol;
		should consider Lisp's cond instead; there are really \emph{two} cond's, one is
		sequential, the other is parallel}

\end{itemize}

\section{Take two}
	
We will provide a triumvirate of a material, formal and structural theories,
corresponding to classical set, type and category theories respectively. While
the formal theory provides the syntaxtical foundational language, the material
and structural theories provide the semantical models of the formal theory. The
material theory provides ``coordinate-full'' / ``calculative'' models for
specific computations, the structural theory provides ``invariant'' models with
which to cleanly express the organizational aspects of the overall theory of
compuation. Each type of model has its own strengths and weaknesses. This
informal interpretation has been suggested by Awodey's comments, \cite[Section 5
``Conclusions'']{awodey2009a}.

We will follow the \emph{spirit} but not the letter of the formal presentation
of Type Theory provided by \cite[Appendix
A.2]{awodeyCoquandVoevodsky2013homotopyTypeTheory}. Classical type theory is
concerned with computing, via natural deduction rules, judgements of both the
``truth'' of existence of types and the ``truth'' of proofs that a given type is
``inhabited''. We \emph{instead} provide dual formal theories of construction
and observation of lists of lists, using computation via natural deduction
rules.

Our basic judgement is based upon one or more observations. Any observation is
built as a sequence of selectors applied to a given List of Lists. Our basic
selectors are $\text{id}$, $\text{car}$, and $\text{cdr}$. The result of a
selector applied to a list of lists such as, $\rho$, is another list of lists.

Similar to Type theory as presented by,
\cite{awodeyCoquandVoevodsky2013homotopyTypeTheory}, we have the following three
judgements:
%
\begin{center}
 $ \Gamma \: \text{ctx}$ \qquad
 $ \Gamma \vdash x : \sigma \rho $ \qquad 
 $ \Gamma \vdash x = y $
\end{center}
%
Informally, $\Gamma \: \text{ctx}$, is the judgement that the observations
contained in the context, $\Gamma$, are valid about one or more list of lists.
The judgement, $\Gamma \vdash x : \sigma \rho$ asserts that, in the context of
$\Gamma$, the observation, $\sigma \rho$, consisting of the selector, $\sigma$,
applied to the list of lists, $\rho$, is denoted by the meta-variable, $x$. The
judgement, $\Gamma \vdash x = y$, asserts that, in the context of, $\Gamma$,
that the two observations denoted by $x$ and $y$ respectively, are equal.

Again, similar to type theory as presented by,
\cite{awodeyCoquandVoevodsky2013homotopyTypeTheory}, we have inference rules of
the form:
%
\begin{center}\begin{prooftree}
\AxiomC{$\mathcal{J}_1$}
\AxiomC{$\cdots$}
\AxiomC{$\mathcal{J}_n$}
\RightLabel{Name}
\TrinaryInfC{$\mathcal{J}$}
\end{prooftree}\end{center}



We have the following nine judgements:
%
\vspace{1ex}\begin{center}\begin{tabular}{rccc}
selector & list of lists    &    null list       &   non-existing \\
id       & $x : * : id  : \rho$ & $x : ( ) : id  : \rho$ & $x : \emptyset : id  : \rho$ \\
car      & $x : * : car : \rho$ & $x : ( ) : car : \rho$ & $x : \emptyset : car : \rho$ \\
cdr      & $x : * : cdr : \rho$ & $x : ( ) : cdr : \rho$ & $x : \emptyset : cdr : \rho$
\end{tabular}\end{center}\vspace{1ex}
%
where $\rho$ is meta variable denoting a list of lists, $\sigma$ is a meta
variable denoting a list of lists selector sequence, and $x$ is a meta variable
denoting the selected portion of the list of lists obtained by applying the
selector, $\sigma$, to the list of lists, $\rho$. The judgement \emph{asserts}
that the selected portion of the list of lists is itself a valid list of lists.

\begin{prooftree}
\AxiomC{$ x : \sigma : \rho $,}
\AxiomC{$ y : \sigma' : x $}
\RightLabel{sequential composition}
\BinaryInfC{$ y : \sigma' \sigma : \rho $}
\end{prooftree}

Informally, the \emph{sequential composition} rule states that if we know that
the selector, $\sigma$, applied to the list of lists, $\rho$, results in a list
of lists, $x$, \emph{and} if the selector, $\sigma'$, applied to the list of
lists, $x$, results in the list of lists, $y$, then the sequential composition
of the selectors, $\sigma' \sigma$ (apply first $\sigma$ and then $\sigma'$),
applied to $\rho$ results in the list of lists, $y$.

\begin{prooftree}
\AxiomC{$x : \text{id} : \rho$}
\RightLabel{null list introduction}
\UnaryInfC{$ () : \text{car} : \rho'$, \quad $ x : \text{cdr} : \rho'$}
\end{prooftree}

\begin{racket}
(assert (judgement x id $\rho$))
pushnull
(assert (judgement () car $\rho'$) (judgement x cdr $\rho'$))
\end{racket}

\begin{prooftree}
\AxiomC{$ () : \text{car} : \rho$,}
\AxiomC{$ x : \text{cdr} : \rho$}
\RightLabel{null list elimination}
\BinaryInfC{$ x : \text{id} : \rho'$}
\end{prooftree}

\begin{racket}
(assert (judgement () car \rho) (judgement x cdr \rho))
popnull
(assert (judgement x id \rho'))
\end{racket}

\begin{prooftree}
\AxiomC{$ x : \sigma : \rho$,}
\AxiomC{$ IF ???? $}
\RightLabel{if exists}
\BinaryInfC{$ ???? $}
\end{prooftree}

\begin{prooftree}
\AxiomC{$ \emptyset : \sigma : \rho$,}
\AxiomC{$ IF ???? $}
\RightLabel{if does not exist}
\BinaryInfC{$ ???? $}
\end{prooftree}

\section{examples}

\begin{prooftree}
\AxiomC{$x : car : \rho$}
\AxiomC{$y : cdr : \rho$}
\RightLabel{dup}
\BinaryInfC{$x : car : \rho'$ \quad $x : cadr : \rho'$ \quad $y : cddr : \rho'$}
\end{prooftree}

\printbibliography
\end{document}


