% LaTeX source for the joyOfLoL document
%

\documentclass[a4paper,openany]{amsart}
\usepackage[utf8]{inputenc}
\usepackage[english]{babel}
\usepackage{disitt}
\usepackage{disitt-symbols}
%\usepackage{disitt-code}
\usepackage[backend=biber,style=alphabetic,citestyle=alphabetic]{biblatex}
\addbibresource{joyOfLoL.bib}
\usepackage{mdframed}
\newmdenv[linecolor=white,backgroundcolor=gray!10]{infobox}
\newenvironment{myQuote}{\begin{quotation}}{\end{quotation}}
\surroundwithmdframed[linecolor=white,backgroundcolor=gray!10]{myQuote}

\begin{document}
	
\sloppy
	
\title[Joy of LoL]{The Joy of Laughing out Loud: Lists of Lists and Joy
processes as a Computational Foundation for Mathematics}
 %% author: stg
\author{Stephen Gaito}
\address{PerceptiSys Ltd, 21 Gregory Ave, Coventry, CV3 6DJ, United Kingdom}%
\email{stephen@perceptisys.co.uk}%
\urladdr{http://www.perceptisys.co.uk}


%% version summary
\thanks{Created: 2016-10-12}
\thanks{Git commit \gitReferences{} (\gitAbbrevHash{}) commited on \gitAuthorDate{} by \gitAuthorName{}}
\thanks{AMS-\LaTeX{}'ed on \today{}.}

%% Copyrights
\thanks{\textbf{Copyright: \copyright{} Stephen Gaito, PerceptiSys Ltd \the\year{}; Some rights reserved}}
\thanks{\textbf{This work is licensed under a Creative Commons Attribution-ShareAlike 4.0 International License.}}

\subjclass[2010]{Primary unknown; Secondary unknown} %
\keywords{Keyword one, keyword two etc.}%
	
\begin{abstract}
We show that the collection of Lists of Lists and Joy processes form a
Computational Foundation of Mathematics.
\end{abstract} 
\maketitle 
\tableofcontents 
	
	
\section{Introduction} For historical reasons, all current foundations of
Mathematics are based upon first or higher order logic. In order to provide a Computational Foundation of Mathematics we need to answer three primary questions:

\begin{itemize}
\item What is Mathematics?
\item What constitues a Foundation of Mathematics?
\item What is a Computational Process?
\end{itemize}

\subsection{What is Mathematics?}

Again, for historical reasons, Mathematics is currently identified with some
super-set of a formulation of the axioms of the well-founded sets, such as ZFC.
In particular it is assumed that Cantor's collection of Ordinals larger than
$\omega$ and their well-ordering are \emph{required} to provide a credible
formulation of real analysis as required for Mathematical Physics.

While we assert that what is normally considered computation, augmented with
something as powerfull as the Well-Ordering Theorem, can provide a Computational
Foundation for what is currently considered Mathematics, we will argue that
Computational \emph{processes} are more than sufficient to provide foundations
for real analysis. 

\TODO{fill in this gap: discuss Topos as a model of well-founded set theory.
Discuss Coalgebras as a model of processes. See fingerPiece01}

For the purposes of \emph{this paper} we will define Mathematics to be any fully
algebraic pair of Topos which contain an \emph{implementation} of the Reals.

\subsection{What constitues a Foundation of Mathematics?}

Following Hatcher, \cite[section 2.5]{hatcher1982logicalFoundationsMath}, we posit the
following list of critera any Foundation of Mathematics must satisfy:

\begin{enumerate}
\item \textbf{A foundation of mathematics must be adequate for a reasonably large
portion of mathematics.}
	
\noindent If we ``assume'' something as powerful as the Axiom of Choice, we will
be able to recover the whole of current mathematical set theory (smaller than
the first strongly inaccessible cardinal).
	
\item \textbf{A foundation must derive from some intuitively natural principles.}
	
\noindent The foundational (co-)data language \emph{is} the List of Lists. The
only operations on this language are LISP's car, cdr, cons, together with
failure and repetition.
	
\item \textbf{The basic principles and primitive (undefined) notions should be
as economical as possible.}
	
\noindent See above.
	
\item \textbf{The foundation must be consistent.}
	
\noindent For a computational foundation, consistency is given since we are
\emph{computing} the (standard) model of the formal theory. What is more
important is that the formal theory is fully abstract, that is the denotation
and operational semantics are equivalent, in logical terms, the formal theory is
both sound and complete.
	
\item \textbf{The foundation should be expressed (or expressible) as a formal system.}
	
\noindent In computational terms, each formal system is a (programming) language
complete with a syntax as well as denotational, operational and axiomatic
semantics. Every such formal system \emph{must} have (at least) a denotational
interpretation into the foundational (co-)data language.
	
\item \textbf{The construction of everyday mathematics in the system should be
``natural'' and ``orderly''.}
	
\noindent With the proposed computational foundations of mathematics, there is a
collection of formal theories each of which have (at least) an interpretation in
the (co-)data language. Each of these formal theories should be tailored to the
mathematical field in question, so by definition they should represent
``natural'' and ``orderly'' systems in which to conduct mathematics in a given
field.
	
\end{enumerate}

\subsection{What is a Compuational Process?}

\subsection{Why Lists of Lists and Joy processes?}



\printbibliography
\end{document}


