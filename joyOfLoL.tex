% LaTeX source for the joyOfLoL document
%

\documentclass[a4paper,openany]{amsart}
\usepackage[utf8]{inputenc}
\usepackage[english]{babel}
\usepackage{disitt}
\usepackage{disitt-symbols}
\usepackage{disitt-code}
\usepackage[backend=biber,style=alphabetic,citestyle=alphabetic]{biblatex}
\addbibresource{joyOfLoL.bib}
\usepackage{mdframed}
\newmdenv[linecolor=white,backgroundcolor=gray!10]{infobox}
\newenvironment{myQuote}{\begin{quotation}}{\end{quotation}}
\surroundwithmdframed[linecolor=white,backgroundcolor=gray!10]{myQuote}

\begin{document}
	
\sloppy
	
\title[Joy of LoL]{The Joy of Laughing out Loud: Lists of Lists and Joy
processes as a Computational Foundation for Mathematics}
 %% author: stg
\author{Stephen Gaito}
\address{PerceptiSys Ltd, 21 Gregory Ave, Coventry, CV3 6DJ, United Kingdom}%
\email{stephen@perceptisys.co.uk}%
\urladdr{http://www.perceptisys.co.uk}


%% version summary
\thanks{Created: 2016-10-12}
\thanks{Git commit \gitReferences{} (\gitAbbrevHash{}) commited on \gitAuthorDate{} by \gitAuthorName{}}
\thanks{AMS-\LaTeX{}'ed on \today{}.}

%% Copyrights
\thanks{\textbf{Copyright: \copyright{} Stephen Gaito, PerceptiSys Ltd \the\year{}; Some rights reserved}}
\thanks{\textbf{This work is licensed under a Creative Commons Attribution-ShareAlike 4.0 International License.}}

\subjclass[2010]{Primary unknown; Secondary unknown} %
\keywords{Keyword one, keyword two etc.}%
	
\begin{abstract}
We show that the collection of Lists of Lists and Joy processes form a
Computational Foundation of Mathematics.
\end{abstract} 
\maketitle 
\tableofcontents 
	
	
\section{Introduction} For historical reasons, all current foundations of
Mathematics are based upon first or higher order logic, or alternatively
(logically interpreted) Type Theory. \TODO{what about category theory?} In order
to provide a Computational Foundation of Mathematics we need to answer three
primary questions:

\begin{itemize}
\item What is Mathematics?
\item What constitutes a Foundation of Mathematics?
\item What is a Computational Process?
\end{itemize}

\subsection{What is Mathematics?}

Again, for historical reasons, Mathematics is currently identified with some
super-set of a formulation of the axioms of the well-founded sets, such as ZFC.
In particular it is assumed that Cantor's collection of Ordinals larger than
$\omega$ and their well-ordering are \emph{required} to provide a credible
formulation of real analysis as required for Mathematical Physics.

While we assert that what is normally considered computation, augmented with
something as powerful as the Axiom of Choice, can provide a Computational
Foundation for what is currently considered Mathematics, we will argue that
Computational \emph{processes} are more than sufficient to provide foundations
for real analysis. 

\TODO{fill in this gap: discuss Topos as a model of well-founded set theory.
Discuss Coalgebras as a model of processes. See fingerPiece01}

For the purposes of \emph{this paper} we will define Mathematics to be any fully
algebraic pair of Topos which contain an \emph{implementation} of the Reals.

\subsection{What constitutes a Foundation of Mathematics?}

\TODO{Discuss organizational/structural, formal/existential,
	ontological/material modes of foundation. See
	\cite{marquis1995categoryFoundationsMath} as well as \cite{awodey2009a}. Are
	these usefully seen as being in a linear adjoint chain?}

Following Hatcher, \cite[section 2.5]{hatcher1982logicalFoundationsMath}, we posit the
following list of criteria any Foundation of Mathematics must satisfy:

\begin{enumerate}
\item \textbf{A foundation of mathematics must be adequate for a reasonably large
portion of mathematics.}
	
\noindent If we ``assume'' something as powerful as the Axiom of Choice, we will
be able to recover the whole of current mathematical set theory (smaller than
the first strongly inaccessible cardinal).
	
\item \textbf{A foundation must derive from some intuitively natural principles.}
	
\noindent The foundational (co-)data language \emph{is} the List of Lists. The
only operations on this language are LISP's car, cdr, cons, together with
failure and repetition.
	
\item \textbf{The basic principles and primitive (undefined) notions should be
as economical as possible.}
	
\noindent See above.
	
\item \textbf{The foundation must be consistent.}
	
\noindent For a computational foundation, consistency is given since we are
\emph{computing} the (standard) model of the formal theory. What is more
important is that the formal theory is fully abstract, that is the denotation
and operational semantics are equivalent, in logical terms, the formal theory is
both sound and complete.
	
\item \textbf{The foundation should be expressed (or expressible) as a formal
system.}
	
\noindent In computational terms, each formal system is a (programming) language
complete with a syntax as well as denotational, operational and axiomatic
semantics. Every such formal system \emph{must} have (at least) a denotational
interpretation into the foundational (co-)data language.
	
\item \textbf{The construction of everyday mathematics in the system should be
``natural'' and ``orderly''.}
	
\noindent With the proposed computational foundations of mathematics, there is a
collection of formal theories each of which have (at least) an interpretation in
the (co-)data language. Each of these formal theories should be tailored to the
mathematical field in question, so by definition they should represent
``natural'' and ``orderly'' systems in which to conduct mathematics in a given
field.
	
\end{enumerate}

\subsection{What is a Computational Process?}

\subsection{Why Lists of Lists and Joy processes?}

\section{Lists of Lists}

\subsection{What is a List of Lists?} 

\TODO{see \cite{awodey2009a} ``From Sets to Types to Categories to Sets''. see \cite{marquis1995categoryFoundationsMath}}

In classical, logically defined, \textbf{set theory}, a set is a collection of
``objects'' which are all ``observed'' to have the same \emph{logical}
``property''. If we, like Frege, use any arbitrary logical predicate to
``select'' the objects of a given set, we can all to easily fall into various
logical paradoxes. To avoid these paradoxes we restrict the allowed predicates
to those that satisfy one or more ``standard'' collections of axioms. These
``standard'' collections of axioms are carefully crafted to avoid the known
paradoxes, as well as provide ``useful'' mathematical ``constructions''. Of
particular importance is that the axioms of set theory are constraining /
defining the ``element'' relation, $a \in A$.

\TODO{see \cite{awodeyCoquandVoevodsky2013homotopyTypeTheory} section 1.1 for a
good introduction}

In classical, \textbf{type theory}, a \emph{type} is a collection of ``objects''
which are (``logically'') judged to be of a specific type, $a:A$.

In classical, \textbf{category theory}, being structuralist, instead of sets or
types having elements or objects (respectively), a collection of objects (with
no internal structure) has a collection of relationships, ``arrows''. 

In each case, we have an unspecified collection of ``things'' for which we can
make observations.

Literally, the lists of lists is the ``collection'' of ``things'' to which we
can apply the basic joy operations to observe the ``thing's'' ``structure''. Being
a finite being, computational or organic, we can only ever apply a finite number
of operation/observations, but that does not mean that every list of lists is a
finite structure. It only means that there are potentially lists of lists whose
``full'' structure we can not explore in a finite number of
operation/observations.

For computer scientists, this is not a problematic concept. There are many
computational processes whose ``usefulness'' is not captured in a finite number
of steps. For example our ``operating systems'' (be they Windows, Mac, or Unix)
would not be much use if they were limited in the number of cycles they could
run. While we might turn them off, that is our external choice not an internal
limitation of the operating system itself. So the computational concept of
process is of a computational ``program'' with an potentially unlimited number
of cycles. It usefulness to us is in the external observations we might make of
the process's ongoing behaviour rather than any ``final'' answer it might
``compute''.

Similarly, a list of lists is a potentially infinite ``object'' to which any
finite computation can only build or observer a finite portion of it.

Given a collection of ``things'' we also need to understand when two potentially
different ``things'' are really the same. Set theory declares two sets to be
equal if they have the same collection of elements, that is if they are
extensionally equal. For the lists of lists, we declare that two potentially
distinct list of lists are equal if given the same collection of observations,
they always give the same collection of ``answers''. This is an extensional
concept of equality.

\subsubsection{What are the Joy operations}

\begin{itemize}

	\item push () \TODO{similar to car in Lisp; similar to write in Turing machine}

	\item pop  \TODO{same as cdr in Lisp; similar to erase in Turing machine}

	\item dup  \TODO{similar to car in Lisp; similar to move right in Turing
		machine; is this too powerful?}

	\item cons \TODO{same as cons in Lisp; similar to move left in Turing machine}
	
	\item ifte \TODO{similar to cond in Lisp; similar to a Turing machine's ability
		to choose its next state/action dependent upon its current state/tape-symbol;
		should consider Lisp's cond instead; there are really \emph{two} cond's, one is
		sequential, the other is parallel}

\end{itemize}

\section{Take two}
	
We will provide a triumvirate of a material, formal and structural theories,
corresponding to classical set, type and category theories respectively. While
the formal theory provides the syntaxtical foundational language, the material
and structural theories provide the semantical models of the formal theory. The
material theory provides ``coordinate-full'' / ``calculative'' models for
specific computations, the structural theory provides ``invariant'' models with
which to cleanly express the organizational aspects of the overall theory of
compuation. Each type of model has its own strengths and weaknesses. This
informal interpretation has been suggested by Awodey's comments, \cite[Section 5
``Conclusions'']{awodey2009a}.

We will essentially follow the \emph{spirit} but not the letter of the formal
presentation of Type Theory provided by \cite[Appendix
A.2]{awodeyCoquandVoevodsky2013homotopyTypeTheory}. Classical type theory is
concerned with computing, via a form of Gentzen's Sequent Calculus, judgements
of both the ``truth'' of existence of types and the ``truth'' of proofs that a
given type is ``inhabited''. We \emph{instead} provide a Floyd-Hoare formal
theory for a dialect of the Joy language whose stacks are lists of lists, which
we denote JoyLoL. This Floyd-Hoare theory provides the rules required to
formally prove the correctness of individual JoyLoL programs using a process
similar to Gentzen's Natural Deduction. These formally proven programs provide
the computations required to dually construct and/or observe portions of any
given list of lists.

The duality here, between construction and observation, is integral to our
foundations of mathematics. It is essentially Cantor's distinction between
\emph{sets} and \emph{absolutely infinite multiplicities} or formal set theory's
\emph{proper classes}. This duality \emph{is} the distinction, in Computer
Science, between \emph{data} and \emph{processes}. In Category theory, it is the
distinction between Algebras and Co-Algebras. While Cantor and classical
Set-Theory, developed to make Cantor's naive set theory rigorous, has no
particularly good reasons for the distinction between set and class except to
avoid various paradoxes, Computer Science and Category theory both have good
computational reasons for this distinction. \TODO{provide refereneces such as
``The search for Certainty''}

The Floyd–Hoare formal theory for JoyLoL consists of a collection of rules which
link pairs of observations. There is a rule for each JoyLoL program construct.
There is also a collection of observation inference rules which allow a given
collection of observations to be appropriately weakened or strengthened as
required for a given correctness proof. A formal correctness proof of a JoyLoL
program, consists of a ordered sequence of inter-twined observations and JoyLoL
program constructs or observation inference rules. This sequence begins and ends
with observations, which are called the pre-conditions and post-conditions
respectively. In this sequence each JoyLoL program construct or observation
inference rule instance are bracketed by observations which correspond, up to
renaming of meta-variables, to that construct or inference rule's pre and post
conditions.

In terms of the material theory, the correctness proof asserts that the given
JoyLoL program will correctly transform any list of lists which satisfies the
pre-conditions into a list of lists which satisfies the post-condition. In terms
of the structural theory, the correctness proof specifies a particular directed
path of arrows in the category of Lists of Lists, between the objects denoted by
the pre-condition and post-condition.

Central to this formal theory are the structure and collection of the possible
observations. Any observation is built as a sequence of selectors applied to a
given List of Lists. Our basic selectors are $\text{id}$, $\text{car}$, and
$\text{cdr}$. The result of a selector applied to a list of lists such as,
$\rho$, is another list of lists. The denotations of any particular observation
consists of four symbols. The first symbol is the meta-variable which informally
denotes the list of lists which is the result of this observation. The second
symbol denotes the expected structure of the resulting list of lists. The third
symbol denotes the sequence of selectors. The last symbol is a meta-variable
which denotes the list of lists to which the selectors are applied. The
meta-variables are denoted by a collection of symbols for variables such as
$x_0$, $x_1$, etc. We will also use the symbol ``$\underline{\ }$'' to denote
any meta-variable of which we are not interested. The collection of symbols
which denote the expected structure consists of ``$\emptyset$'', ``$()$'', and
``$*$'', which denote the empty list of lists, the null list of
lists, and any unspecified list of lists.

Similar to Type theory as presented by,
\cite{awodeyCoquandVoevodsky2013homotopyTypeTheory}, we have the following three
judgements:
%
\begin{center}
 $ \Gamma \: \text{ctx}$ \qquad
 $ \Gamma \vdash x : \sigma \rho $ \qquad 
 $ \Gamma \vdash x = y $
\end{center}
%
Informally, $\Gamma \: \text{ctx}$, is the judgement that the observations
contained in the context, $\Gamma$, are valid about one or more list of lists.
The judgement, $\Gamma \vdash x : \sigma \rho$ asserts that, in the context of
$\Gamma$, the observation, $\sigma \rho$, consisting of the selector, $\sigma$,
applied to the list of lists, $\rho$, is denoted by the meta-variable, $x$. The
judgement, $\Gamma \vdash x = y$, asserts that, in the context of, $\Gamma$,
that the two observations denoted by $x$ and $y$ respectively, are equal.

Again, similar to type theory as presented by,
\cite{awodeyCoquandVoevodsky2013homotopyTypeTheory}, we have inference rules of
the form:
%
\begin{center}\begin{prooftree}
\AxiomC{$\mathcal{J}_1$}
\AxiomC{$\cdots$}
\AxiomC{$\mathcal{J}_n$}
\RightLabel{Name}
\TrinaryInfC{$\mathcal{J}$}
\end{prooftree}\end{center}

Given the collection of selectors and expected structures, we have the following
nine observations:
%
\vspace{1ex}\begin{center}\begin{tabular}{rccc}
selector & list of lists    &    null list       &   non-existing \\
id       & $x : * : id  : \rho$ & $x : ( ) : id  : \rho$ & $x : \emptyset : id  : \rho$ \\
car      & $x : * : car : \rho$ & $x : ( ) : car : \rho$ & $x : \emptyset : car : \rho$ \\
cdr      & $x : * : cdr : \rho$ & $x : ( ) : cdr : \rho$ & $x : \emptyset : cdr : \rho$
\end{tabular}\end{center}\vspace{1ex}
%
where $\rho$ is meta variable denoting a list of lists, $\sigma$ is a meta
variable denoting a list of lists selector sequence, and $x$ is a meta variable
denoting the selected portion of the list of lists obtained by applying the
selector, $\sigma$, to the list of lists, $\rho$. Each observations
\emph{asserts} that the selected portion of the list of lists is itself a valid
list of lists of the expected structure.

\begin{prooftree}
\AxiomC{$ x : * : \sigma : \rho $,}
\AxiomC{$ y : * : \sigma' : x $}
\RightLabel{sequential composition}
\BinaryInfC{$ y : * : \sigma' \sigma : \rho $}
\end{prooftree}

Informally, the \emph{sequential composition} rule states that if we know that
the selector, $\sigma$, applied to the list of lists, $\rho$, results in a list
of lists, $x$, \emph{and} if the selector, $\sigma'$, applied to the list of
lists, $x$, results in the list of lists, $y$, then the sequential composition
of the selectors, $\sigma' \sigma$ (apply first $\sigma$ and then $\sigma'$),
applied to $\rho$ results in the list of lists, $y$.

\begin{prooftree}
\AxiomC{$x : * : \text{id} : \rho$}
\RightLabel{null list introduction}
\UnaryInfC{$ \underline{\ } : () : \text{car} : \rho'$, \quad $ x : * : \text{cdr} : \rho'$}
\end{prooftree}

\begin{joyLoL}
((x * (id) $\rho$ observe) assert)
pushnull
(( _ () (car) $\rho'$ observe) (x * (cdr) $\rho'$ observe) assert)
\end{joyLoL}

\begin{prooftree}
\AxiomC{$ () : \text{car} : \rho$,}
\AxiomC{$ x : \text{cdr} : \rho$}
\RightLabel{null list elimination}
\BinaryInfC{$ x : \text{id} : \rho'$}
\end{prooftree}

\begin{joyLoL}
(assert (judgement () car \rho) (judgement x cdr \rho))
popnull
(assert (judgement x id \rho'))
\end{joyLoL}

\begin{prooftree}
\AxiomC{$ x : \sigma : \rho$,}
\AxiomC{$ IF ???? $}
\RightLabel{if exists}
\BinaryInfC{$ ???? $}
\end{prooftree}

\begin{prooftree}
\AxiomC{$ \emptyset : \sigma : \rho$,}
\AxiomC{$ IF ???? $}
\RightLabel{if does not exist}
\BinaryInfC{$ ???? $}
\end{prooftree}

\section{examples}

One of the key aspects of the JoyLoL language as opposed to standard programming
languages is that list of lists can be infinite (process) objects and in
particular, each ``element'' of any given list could itself be an infinite list.
This means that, for example, Joy's dup operator is potentially too powerful if
it is applied to a potential infinite process (rather than a finite data list).
This is essentially related to the concept of action at a distance in Physics,
what is the definition of ``locality'' and how powerful can a local action be? A
key aspect of Turing's (and Gandy's) definition of computation, as essentially
noted by G\"odel, is that computation is defined in as minimalistic a manner
possible.

Turing's machines were, as originally defined, special purpose computing agents,
used to show that universal computers could be built which used ``data'' to
define computation.

Thinking structurally for the moment, we can see the objects and morphisms of a
category as generalized ``data'' (though with a potentially infinite extent) and
generalized ``computation'' (though with a potentially finite description). This
is essentially the Harvard Architecture. That is the objects of the Category of
Lists of Lists, are Lists of Lists while the morphisms are JoyLoL programs
(which can ultimately be translated into a List of Lists).

\begin{prooftree}
\AxiomC{$x : * : car : \rho$}
\AxiomC{$y : * : cdr : \rho$}
\RightLabel{dup}
\BinaryInfC{$x : * : car : \rho'$ \quad $x : * : cadr : \rho'$ \quad $y : * : cddr : \rho'$}
\end{prooftree}

Translated into JoyLoL we have
%
\begin{joyLoL}
;; pre-condition
((x * (car) rho-pre observe)
 (y * (cdr) rho-pre observe) assert)
dup
;; post-condition
((x * (car) rho-post observe)
 (x * (car cdr) rho-post observe)
 (y * (cdr cdr) rho-post observe) assert)
\end{joyLoL}

A key feature of JoyLoL is that it can only change objects in limited ways. In
particular it can only add or remove one null list ``()'' per computational
step. To be able to ``dup(licate)'' a potentially infinite complex list of
lists, we have two related problems, firstly, where do we store the intermediate
structure if we can not do it all at once, and secondly, if we must inter-twine
temporary storage and final duplication of partial structures, how do we keep
track of ``where we are'' in the overall computation? One option is to use yet
another element on the stack as temporary storage, this would be a von Neumann
or Princeton architecture. Such a strategy will, when we come to investigate
space-time and computational traces, make ``tracing'' the identity of an
``object'' much more difficult. Another option is to allow JoyLoL to have an
explicitly Harvard architecture, where there is separate ``memory'' for both
data and process logic, and the JoyLoL operators can manipulate either. In
categorical terms, the ``data'' memory are the objects, while the ``process
logic'' memories are the categorical morphisms.

To this end we have the following JoyLoL operators:
%
\begin{joyLoL}
pushNull, popNull
selectCar, selectCdr
\end{joyLoL}

\printbibliography
\end{document}


