% LaTeX source for the joyOfLoL document
%

\documentclass[a4paper,openany]{amsart}
\usepackage[utf8]{inputenc}
\usepackage[english]{babel}
\usepackage{disitt}
\usepackage{disitt-symbols}
%\usepackage{disitt-code}
\usepackage[backend=biber,style=alphabetic,citestyle=alphabetic]{biblatex}
\addbibresource{joyOfLoL.bib}
\usepackage{mdframed}
\newmdenv[linecolor=white,backgroundcolor=gray!10]{infobox}
\newenvironment{myQuote}{\begin{quotation}}{\end{quotation}}
\surroundwithmdframed[linecolor=white,backgroundcolor=gray!10]{myQuote}

\begin{document}
	
\sloppy
	
\title[Joy of LoL]{The Joy of Laughing out Loud: Lists of Lists and Joy
processes as a Computational Foundation for Mathematics}
 %% author: stg
\author{Stephen Gaito}
\address{PerceptiSys Ltd, 21 Gregory Ave, Coventry, CV3 6DJ, United Kingdom}%
\email{stephen@perceptisys.co.uk}%
\urladdr{http://www.perceptisys.co.uk}


%% version summary
\thanks{Created: 2016-10-12}
\thanks{Git commit \gitReferences{} (\gitAbbrevHash{}) commited on \gitAuthorDate{} by \gitAuthorName{}}
\thanks{AMS-\LaTeX{}'ed on \today{}.}

%% Copyrights
\thanks{\textbf{Copyright: \copyright{} Stephen Gaito, PerceptiSys Ltd \the\year{}; Some rights reserved}}
\thanks{\textbf{This work is licensed under a Creative Commons Attribution-ShareAlike 4.0 International License.}}

\subjclass[2010]{Primary unknown; Secondary unknown} %
\keywords{Keyword one, keyword two etc.}%
	
\begin{abstract}
We show that the collection of Lists of Lists and Joy processes form a
Computational Foundation of Mathematics.
\end{abstract} 
\maketitle 
\tableofcontents 
	
	
\section{Introduction} For historical reasons, all current foundations of
Mathematics are based upon first or higher order logic. In order to provide a Computational Foundation of Mathematics we need to answer three primary questions:

\begin{itemize}
	\item What is Mathematics?
	\item What constitues a Foundation of Mathematics?
	\item What is a Computational Process?
\end{itemize}

Again, for historical reasons, Mathematics is currently identified with some
super-set of a formulation of the axioms of the well-founded sets, such as ZFC.
In particular it is assumed that Cantor's collection of Ordinals larger than
$\omega$ and their well-ordering are \emph{required} to provide a credible
formulation of real analysis as required for Mathematical Physics.

While we assert that what is normally considered computation, augmented with
something as powerfull as the Well-Ordering Theorem, can provide a Computational
Foundation for what is currently considered Mathematics, we will argue that
Computational \emph{processes} are more than sufficient to provide foundations
for real analysis. Unfortunately, for \emph{this paper} we will limit ourselves
to showing that the collection of Lists of Lists and Joy processes form a fully
algebraic pair of Topos which contain an \emph{implementation} of the Reals.

\printbibliography
\end{document}


