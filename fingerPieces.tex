% LaTeX source for the fingerPieces document
%

\documentclass[a4paper,openany]{amsart}
\usepackage[utf8]{inputenc}
\usepackage[english]{babel}
\usepackage{disitt}
\usepackage{disitt-symbols}
\usepackage[backend=biber,style=alphabetic,citestyle=alphabetic]{biblatex}
\addbibresource{fingerPieces.bib}
\usepackage{mdframed}
\newmdenv[linecolor=white,backgroundcolor=gray!10]{infobox}
\newenvironment{myQuote}{\begin{quotation}}{\end{quotation}}
\surroundwithmdframed[linecolor=white,backgroundcolor=gray!10]{myQuote}

\begin{document}

\sloppy

\title[Short Title]{Long Title}
%% author: stg
\author{Stephen Gaito}
\address{PerceptiSys Ltd, 21 Gregory Ave, Coventry, CV3 6DJ, United Kingdom}%
\email{stephen@perceptisys.co.uk}%
\urladdr{http://www.perceptisys.co.uk}


%% version summary
\thanks{Created: 2016-10-12}
\thanks{Git commit \gitReferences{} (\gitAbbrevHash{}) commited on \gitAuthorDate{} by \gitAuthorName{}}
\thanks{AMS-\LaTeX{}'ed on \today{}.}

%% Copyrights
\thanks{\textbf{Copyright: \copyright{} Stephen Gaito, PerceptiSys Ltd \the\year{}; Some rights reserved}}
\thanks{\textbf{This work is licensed under a Creative Commons Attribution-ShareAlike 4.0 International License.}}

\subjclass[2010]{Primary unknown; Secondary unknown} %
\keywords{Keyword one, keyword two etc.}%

\begin{abstract}
This mini-paper is a place to start ``finger pieces'' for re-use in other
papers.
\end{abstract} 
\maketitle 
\tableofcontents 


\section{Some philosophy}

\emph{What can an finite being do?} For our purposes it can \emph{do} two
important things:
\begin{enumerate}
\item It \emph{can} make a finite series of changes in reality (``markings in
the sand'') in a finite amount of time.
\item It can \emph{dream} of making trans-finite changes in reality in a
trans-finite amount of time.
\end{enumerate}

\emph{What can a finite being do?} Again for our purposes it can do two
important things:
\begin{enumerate}
\item It can \emph{construct} finite structures in a finite amount of time.
\item It can \emph{observe} a finite amount of a potentially non-finite
structure in a finite amount of time.
\end{enumerate}

Previous attempts at providing a foundation for Mathematics from the past 200
years have focused upon providing \emph{logical} arguments for the existence and
properties of mathematical entities. All existing definitions of set theory, such
as Zermelo-Fraenklel's set theory with the Axiom of Choice based upon first
order logic, or type theory, such as the recent Homotopy Type Theory,
\cite{awodeyCoquandVoevodsky2013homotopyTypeTheory}, continue this approach.

Existing (\emph{logical}) foundations of Mathematics focus upon using natural
deduction (in one form or another) to recursively \emph{compute} the \emph{truth
value} of a logical \emph{sentence} which is built out of component
sub-sentences. These logical sentences are semantically construed to be
\emph{about} the existence or properties of Mathematical structures\footnote{See
for example the discussion in \cite{hatcher1982logicalFoundationsMath}.}. It is
the essentially semantic nature of these logical sentences which provide the
\emph{bite} in the various logical paradoxes which the current foundations of
mathematics have been designed to avoid.

Unfortunately, the existing \emph{logical} foundations of mathematics avoid
these paradoxes by forbidding non-well-founded structures, which in turn
requires the use of what is essentially non-finite computational methods to
model the observation of physical reality. The primary thesis of the current
\emph{computational} foundations of mathematics, is that by analysing the
computational nature of the observational process of the sciences, and
\emph{explicitly} allowing non-well-founded structures, we obtain a simpler
mathematics which better reflects the way scientists actually work. Our
secondary thesis is that trans-finite mathematics is only required to provide an
understanding of how finitely constructive mathematics ``fits into'' existing
mathematical practice.

Being finite beings, we must limit ourselves to making a finite number of
``marks in the sand'' in a finite amount of time. Since we are assuming \emph{no
external} mathematical environment to this work, we need to define the whole of
the mathematics that we are using at once.

We make two primary assumptions:
\begin{enumerate}
\item We can make an arbitrary but finite number of \emph{different} marks or
symbols.
\item Given an existing collection of such symbols, we can add a finite
collection of additional symbols.
\end{enumerate}

We will provide what is essentially a type theoretic foundations for
mathematics, see, for example, the recent Homotopy Type Theory,
\cite{awodeyCoquandVoevodsky2013homotopyTypeTheory}. However, instead of
essentially \emph{logical} judgements that a given logical sentence is ``true'',
we use \emph{computational} judgements that a structure can be built or dually
has been observed.

\section{Defining the Universe using Natural Judgements}

Our primary aim is to begin building the Universe, \Universe{}{}.

When the dust settles we will have a \emph{functional} programming language
whose denotation will be a generalisation of a Topos, which we will call the
``Universal Topos''. The Universal Topos will be a generalisation of a Topos
\emph{because} it will \emph{explicitly} contain non-well founded co-algebraic
objects and not \emph{just} the well-founded algebraic ``sets''. Again, when the
dust settles, we will be able to work in the \emph{internal higher-order
intuitionistic logic} of the Topos. There will be \emph{no} external logic,
since this Topos \emph{will be a foundation for the \emph{whole} of
mathematics}. In fact the Universal Topos will be \emph{extensional} and will
satisfy the categorical equivalent of the Axiom of Choice, so the internal
logic, of the Universal Topos, will actually be higher-order classical logic.

\begin{prooftree}
\AxiomC{}
\RightLabel{empty-\Universe{}{}}
\UnaryInfC{\judgement{\emptyset}{\Universe{}{}}}
\end{prooftree}

\begin{prooftree}
\AxiomC{}
\RightLabel{\Universe{}{}-\Universe{}{}}
\UnaryInfC{\judgement{\Universe{}{}}{\Universe{}{}}}
\end{prooftree}

\begin{prooftree}
\AxiomC{\judgement{x}{y}}
\AxiomC{\judgement{y}{z}}
\RightLabel{transitive-\Universe{}{}}
\BinaryInfC{\judgement{x}{z}}
\end{prooftree}

\begin{prooftree}
\AxiomC{}
\RightLabel{\Universe{}{\emptyset}-\Universe{}{}}
\UnaryInfC{\judgement{\Universe{}{\emptyset}}{\Universe{}{}}}
\end{prooftree}

\begin{prooftree}
\AxiomC{\judgement{x}{\Universe{}{}}}
\AxiomC{\judgement{y}{\Universe{}{}}}
\RightLabel{$(\cdot)^{\emptyset}$-Intro}
\BinaryInfC{\judgement{\Delta_{\emptyset}(x)}{y^{\emptyset}}}
\end{prooftree}

\begin{prooftree}
\AxiomC{\judgement{y}{\Universe{}{}}}
\AxiomC{\judgement{x}{y^{\emptyset}}}
\RightLabel{\Universe{}{\emptyset}-Elim}
\BinaryInfC{\judgement{\textbf{Label}(x)}{\Universe{}{}}}
\end{prooftree}

\begin{prooftree}
\AxiomC{\judgement{x}{\Universe{}{}}}
\AxiomC{\judgement{y}{\Universe{}{}}}
\AxiomC{$ x \subSet y$}
\RightLabel{\Universe{}{\emptyset}-Elim}
\TrinaryInfC{\judgement{\textbf{Label}(x)}{\Universe{}{}}}
\end{prooftree}

\begin{prooftree}
\AxiomC{\judgement{x}{\Universe{}{\emptyset}}}
\RightLabel{\Universe{}{\emptyset}-Elim}
\UnaryInfC{\judgement{\textbf{Dim}(x)}{\Universe{}{}}}
\end{prooftree}

\begin{prooftree}
\AxiomC{\judgement{x}{\Universe{}{\emptyset}}}
\RightLabel{\Universe{}{\emptyset}-Elim}
\UnaryInfC{$\textbf{Dim}(x) =_{\Universe{}{}} \emptyset$}
\end{prooftree}

\hrule

\begin{prooftree}
\AxiomC{$\context{\Gamma} \vdash \judgement{a}{b}$}
\AxiomC{$\context{\Gamma} \vdash \judgement{b}{a}$}
\RightLabel{antisymmetry-partOf}
\BinaryInfC{$\context{\Gamma} \vdash a =_{\Universe{}{}} b$}
\end{prooftree}

\begin{prooftree}
\AxiomC{$c\mathcal{J}_0$}
\AxiomC{$\cdots$}
\AxiomC{$c\mathcal{J}_{\gamma}$}
\RightLabel{Name}
\TrinaryInfC{$c\mathcal{J}$}
\end{prooftree}

\section{Defining the Universe using Programming Language}

\begin{verbatim}
data Universe = DeltaE Universe
  | Delta0 Universe DeltaE
  | Delta1 Universe Delta0 Delta0

empty = ??

zero = DeltaE empty

one = DeltaE zero
  
label :: Universe -> Universe
  DeltaE x = x
  Delta0 x _ = x
  Delta1 x _ _ = x

pi0 :: Universe -> Universe
  Delta0 _ x = x
  Delta1 _ x _ = x
  
\end{verbatim}

\printbibliography
\end{document}


