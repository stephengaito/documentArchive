% A ConTeXt document [master document: joyLoL.tex]

\chapter[title=Overview]

\section[title=Implementation]

\subsection[title=Bridging the semantic gap]

JoyLoL \emph{is explicitly defined to be} a fixed point of the formal 
semantic functor, making JoyLoL its own formal semantic definition. 
However there is, currently, no existing computational device which 
\emph{implements} the JoyLoL language. That is, there is no computational 
device which \quote{runs} JoyLoL code natively. 

The objective of this document is to provide an implementation of JoyLoL 
in as transparently correct way as possible. As discussed in, 
\cite[gaito2017joyLoLDefinition], the formal definition of any 
computational language has two distinct components: one \emph{deductive} 
and the other \emph{inductive}. While we can rigorously check any 
deductive proofs of correctness, we can only ever hope to falsify any 
inductive tests of correctness. Any formally correct implementation of a 
computational language needs to be explicitly clear where the line between 
the deductively provable and the inductively testable is located.

The desired goal of any rigorous implementation is to keep as much as 
possible of the code deductively provable. Conversely any rigorous 
implementation needs to keep any code which is only inductively testable 
as clear and simple as possible. However how and were we draw the line 
between the deductively provable and the merely inductively testable 
implementation, will have profound impact upon the \emph{performance} of 
all resulting JoyLoL computations run using this implementation. Provable 
correctness \emph{and} performance are \emph{both} critically important. 

To obtain the correct balance of correctness, (potential) performance, and 
simplicity, JoyLoL has been designed as a \quote{trampolining} 
interpreter, written in ANSI-C, but meta-compiled from Literate sources 
written in \ConTeXt/\LuaTeX\ which are transcribed into ANSI-C source before 
being compiled to an executable on a given platform by an appropriately 
chosen ANSI-C compiler. 

Finally since JoyLoL is meant to form a foundation for Mathematics, and, 
as such, the basis of mathematical proof, we need to ensure the JoyLoL 
language is accessible within the most common tool, \TeX, used by 
mathematicians to communicate their proofs. To do this we wrap JoyLoL in a 
simple Lua interface. By wrapping the ANSI-C JoyLoL libraries in a Lua 
interface, we allow the JoyLoL libraries to be used, in particular, inside 
\LuaTeX\ and hence inside \LaTeX\ and \ConTeXt\ documents. At the moment, 
\LaTeX\ does not make integral use of \LuaTeX's Lua subsystem. Instead we 
make use of \ConTeXt\ for most of our documentation and mathematical 
writing, since \ConTeXt\ does make integral use of \LuaTeX's Lua subsystem. 

\subsection[title=Literate Sources]

The literate sources, provide human readable documentation and 
justifications for each JoyLoL Co-Algebraic extension, complete with 
formal semantic definitions of each axiomatic word in JoyLoL. 

\subsection[title=ANSI-C system code]

We build the lowest level system code for JoyLoL using ANSI-C with a few 
\quotation{standard} POSIX extension libraries. We have chosen ANSI-C for 
its: 

\startitemize

\item {\bf portability}: There are a large number of ANSI-C compatible C 
compilers which target almost \emph{all} computers currently in existence. 

\item {\bf inter-working}: There are a large number of code libraries 
implementing useful algorithms which can be \quote{loaded} into the 
runtime image of an ANSI-C compiled program.

\item {\bf performance}: \emph{If} desired, the overall JoyLoL interpreter 
can be compiled using any of the modern ANSI-C compilers' optimization 
modes. Since ANSI-C is so heavily used, the optimizing modes of most 
compilers are realatively well \quote{understood}, tested and stable. 

\item {\bf transparency}: The semantic gap between ANSI-C and the 
\quote{assembler} / \quote{machine-code} of almost any computer is small 
enough that a \emph{large number} of skilled programmers could, if needed, 
hand code any C code directly into a given machine-code. For our needs, 
this means that there is no obscure mapping between the short pieces of 
JoyLoL implementation code and a given CPU's machine-code. This ensures 
that what JoyLoL does when running is \quote{relatively} easy to 
understand for most programmers. 

\item {\bf familiarity}: While programmers are only a small part of our 
target audience, given we are explicitly dealing with the mathematics of 
computation, the programming community is an important part of the 
audience. More importantly \quote{most} programmers have a \quote{working} 
familiarity with the subset of ANSI-C used to implement the lowest levels 
of JoyLoL. 

\stopitemize

\subsection[title=Interpreter structure]

Since all JoyLoL words explicitly manage the context's data and process 
stacks, there is, in theory, no need for the ANSI-C call stack. The 
typical C-like language uses the call stack to hold both local data, any 
call parameters, as well as the process location to which to 
\quote{return}. Because data and process information are mixed on the call 
stack, to keep the call stack from growing without bounds, the explicit 
expectation of any C-like language is that calls \quote{return} in the 
\emph{strict} reverse order in which they are called, and that, more 
importantly, there is a finite limit on the number of calls a process 
might make. 

When using JoyLoL as a foundation of Mathematics, we will find that there 
are many processes which do not naturally follow this strict return in 
reverse order pattern. Keeping the data and process stacks separate 
ensures that JoyLoL does not need to be enforce this strict call pattern. 
Instead JoyLoL implements a \quote{continuation passing style} of 
programming, see, for example, \cite[stracheyWadsworth2000continuations], 
\cite[right={, section 5.1)}][gordon1979denotationalDescriptionProgLangs], 
\cite[right={, chapter 10)}][tennent1981programmingLanguages] 
\cite[right={, chapters 5 and 
6)}][friedmanWand2008essentialsProgrammingLanguages]. 

In typical programming languages, this continuation passing style is 
implemented using either explicit \quote{jumps}/\quote{gotos}, see 
\cite[right={, chapter 10)}][tennent1981programmingLanguages], or, 
alternatively, using \quote{tail calls}, see \cite[right={, chapter 
2)}][probst2001tailRecursionC], or \cite[right={, section 
6.2)}][friedmanWand2008essentialsProgrammingLanguages]. The use of 
explicit computed gotos, which are implemented as non-ANSI-C standard 
\emph{extensions}, requires the use of global variables to pass the data 
and process stacks. Unfortunately this use of global variables inhibits 
most standard C compiler optimizations\footnote{With considerably more 
effort, we \emph{could} arrange to keep the data and process stacks in 
\quote{local} variables in \quote{simulated} \quote{C-call stacks}. While 
this \emph{might} improve performance, the use of such simulated C-call 
stacks, being so non-standard, would seriously reduce the number of 
programmers who could \emph{easily} understand the resulting C-code.}. 

In the best of all worlds, we could implement JoyLoL's lowest levels using 
a \emph{systems programming language} with native \quote{tail calls}. 
Since the data and process stacks can now be passed as \quote{normal} 
procedure parameters, standard C compiler optimizations will not be 
inhibited. Unfortunately no widely used systems programming language 
currently implements tail calls. While the functional languages such as 
Haskell, and Lisp/Scheme have native tail calls, they do not map 
sufficiently cleanly onto the underlying machine-language of a given 
computer's CPU. All C-like languages, who do typically map reasonably 
cleanly onto a given CPU's architecture, do not have a tail call friendly 
call structure. 

To solve this problem, following \cite[right={, Section 
5.2)}][friedmanWand2008essentialsProgrammingLanguages], we use a 
\quote{trampolining} interpreter as the main \quote{eval loop} for JoyLoL. 
The use of trampolining, ensures that the C-call stack never grows very 
large. JoyLoL words are implemented as simple C procedures keeping the 
structure of the resulting C-code simple. Trampolining also allows the use 
of external libraries which expect C-call stacks. For each cycle around 
the JoyLoL eval loop, the top of the process stack is used to determine 
which C procedure (JoyLoL word) to call next. 

Unfortunately, while providing simply structured C-code, trampolining of 
small C procedures, is not as performant as a system which makes use of 
native tail calls. Instead by using the \ConTeXt/\LuaTeX\ based 
meta-compiler we can pre-compile any complex JoyLoL word definitions as 
explicit C procedures which \emph{can allow} a given C compiler's 
optimization mode to produce performant code. This means that the JoyLoL 
system itself can be written in JoyLoL, allowing it to be proven 
deductively correct, yet still be performant. 

\subsection[title=Call structure]

JoyLoL is a Forth-like language which manipulates \quote{stacks}. Almost 
all existing general purpose computational devices are \quote{register 
based}. Cleanly and performantly implementing stack based languages on a 
register based computational device has been previously explored in Ertl's 
thesis, \cite[ertl1996implementationStackLanguages]. 

\subsection[title=Bootstrapping JoyLoL]

Since we want your tool set to be as rigorous as possible, we ultimately 
need to use JoyLoL to deductively prove its own correctness. 
Unfortunately, at least initially, most users do not have a running 
version of JoyLoL. In order to obtain the \emph{first} running version, we 
need to \quote{bootstrap} the tool set by building an initial version of 
JoyLoL which is not rigorously checked. 

Since we assume that any serious user of JoyLoL will be using JoyLoL to 
develop mathematical arguments, and hence will be using \ConTeXt, we will 
provide this \quote{bootstrapped} JoyLoL using Lua. To do this each JoyLoL 
CoAlgebra will contain a highly simplified version of itself as pure Lua 
using the \type{MinJoyLoL} environment. 

\startMkIVCode
\defineLitProgs[MinJoyLoL]
\stopMkIVCode

\section[title=Organization]

While we assert that all of the CoAlgebraic extensions provided in this 
document are conservative extensions over JoyLoL provided with only Lists 
of Lists, it is useful, for \quote{bears of \emph{very} little brains} 
such as myself, to work, at least initially, with the extra structure 
provided by these CoAlgebraic extensions. We will show in a subsequent 
paper that all of the CoAlgebraic extensions provided in this document, 
are conservative extensions over JoyLoL provided with only lists of lists. 

