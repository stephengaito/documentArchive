% A ConTeXt document [master document: hilbertsProgram.tex]

\section[title=Some Software Engineering]

\subsection[title=\joylol's Software Architecture]

Software Engineering looks at the practical aspects of making 
computational software \emph{work} for a specific purpose.

An important practical aspect of a computation is its ultimate speed of 
performance. A computation which takes an apreaciable part of the lifetime 
of an activity to complete is not much use for managing that activity. It 
is certainly important that the software writen in \joylol\ be performant. 

However since software is (re)-engineered by many individuals over a 
considerable length of time, the \quote{maintainability} of the textual 
structure of the software can be equally or even more important than raw 
performance in many cases. 

For our purposes, this maintainability, this ability to understand the 
software's (textual) structure is very important. In terms of software 
engineering, this is called the \quote{systems architecture} of the 
overall problem solution. 

For our needs there are two distinct ways to understand \joylol's software 
architecture. Firstly as a layering where each additional layer adds 
complexity. For our purposes, these layers of complexity are: 
\quote{purest \joylolZero}, \quote{pure \joylol} and \quote{performant 
\joylol}.

Secondly, we can understand \joylol's software architecture as collections 
of interacting \quote{components}, where each \quote{component} 
performs a specific conceptual task. This collection of \quote{components} 
can be organized either by what depends upon what to perform a 
computation, \emph{what \quote{calls} what}, or what depends upon what 
through the time-line of a computation, \emph{what happens first, second, 
third, etc}.

\subsection[title=\joylol's Complexity Layers]

In the following three diagrams, we attempt to depict the inter-related 
complexity layers as well as hint at the \quote{call} dependency of 
important components. At the \quote{highest level} of \quote{call} 
dependency, we have a number of \quote{\joylol\ coalgebras}. These in turn 
depend upon, for example, the \joylol\ interpreter, the \quote{CONS-pairs} 
and ultimately \quote{memory management}. 

\placefigure[][fig:pureJoylolZero]{Purest \joylolZero}
\bgroup\startMPcode
input hatching.mp;
picture dots; dots := dashpattern(on 0.1mm off 0.5mm);

% JoyLol0
path joylolZero;
joylolZero := (0.1cm,0cm) -- (6cm, 0cm) --
  (6cm, -4cm) -- (0.1cm, -4cm) -- cycle;

draw image (
  hatchfill joylolZero
    withcolor (45,2mm,-.5bp);
) dashed dots withcolor darkgray;

draw joylolZero
  withpen pencircle scaled 1mm
  withcolor lightgray;

% CoAlgs

draw (1.1cm,3cm) -- (1.9cm,3cm) -- (1.9cm,-2.5cm) -- (1.1cm,-2.5cm) -- cycle;
draw (2.1cm,3cm) -- (2.9cm,3cm) -- (2.9cm,-2.5cm) -- (2.1cm,-2.5cm) -- cycle;
draw (3.1cm,3cm) -- (3.9cm,3cm) -- (3.9cm,-2.5cm) -- (3.1cm,-2.5cm) -- cycle;
draw (4.1cm,3cm) -- (4.9cm,3cm) -- (4.9cm,-2.5cm) -- (4.1cm,-2.5cm) -- cycle;

draw (1cm,-1.7cm) -- (5cm,-1.7cm) -- (5cm,-5cm) -- (1cm,-5cm) -- cycle;
draw (0.8cm,-2.1cm) -- (11.6cm,-2.1cm) -- (11.6cm,-3.1cm) -- (0.8cm,-3.1cm) -- cycle;

% labels

label("Joylol interpreter", (3cm,-2.8cm));
label("CONS-Pairs", (3cm,-3.5cm));

label("$\bold{JoyLoL}_0$", origin) rotated 90 shifted (0.5cm,-2cm);
label("\bold{Characters}", origin) rotated 90 shifted (1.5cm, -1.4cm);
label("\bold{Symbols}", origin) rotated 90 shifted (2.5cm, -1.4cm);
label("\bold{Numbers}", origin) rotated 90 shifted (3.5cm, -1.4cm);
label("\bold{...}", origin) rotated 90 shifted (4.5cm, -1.4cm);

path nonJoylolZero;
nonJoylolZero := (0,-3.9cm) -- (5.9cm,-3.9cm) --
  (5.9cm,-0.1cm) -- (0cm, -0.1cm) -- (0cm, 4cm) -- (12cm, 4cm) --
  (12cm,-10cm) -- (0cm, -10cm) -- cycle;

fill nonJoylolZero withcolor (1, 1, 1);

draw (0.1cm, -3.9cm) -- (5.9cm,-3.9cm) -- (5.9cm, -0.1cm) -- (0.1cm, -0.1cm) -- cycle
  withpen pencircle scaled 1mm
  withcolor lightgray;

path justJoylolZero;
justJoylolZero := (0cm,-4cm) -- (6cm,-4cm) -- (6cm,0cm) -- (0cm,0cm) -- cycle;

setbounds currentpicture to justJoylolZero;
clip currentpicture to justJoylolZero ;
\stopMPcode\egroup

\placefigure[][fig:pureJoylol]{Pure \joylol}
\bgroup\startMPcode
input hatching.mp;
picture dots; dots := dashpattern(on 0.1mm off 0.5mm);

% JoyLoL

draw (0.1cm,0cm) -- (11.9cm, 0cm) -- (11.9cm, 3.9cm) -- (0.1cm, 3.9cm) -- cycle
  withpen pencircle scaled 1mm
  withcolor lightgray;

% JoyLol0
path joylolZero;
joylolZero := (0.1cm,0cm) -- (6cm, 0cm) --
  (6cm, -4cm) -- (0.1cm, -4cm) -- cycle;

draw image (
  hatchfill joylolZero
    withcolor (45,2mm,-.5bp);
) dashed dots withcolor darkgray;

draw joylolZero
  withpen pencircle scaled 1mm
  withcolor lightgray;

% CoAlgs

draw (1.1cm,3cm) -- (1.9cm,3cm) -- (1.9cm,-2.5cm) -- (1.1cm,-2.5cm) -- cycle;
draw (2.1cm,3cm) -- (2.9cm,3cm) -- (2.9cm,-2.5cm) -- (2.1cm,-2.5cm) -- cycle;
draw (3.1cm,3cm) -- (3.9cm,3cm) -- (3.9cm,-2.5cm) -- (3.1cm,-2.5cm) -- cycle;
draw (4.1cm,3cm) -- (4.9cm,3cm) -- (4.9cm,-2.5cm) -- (4.1cm,-2.5cm) -- cycle;

draw (1cm,-1.7cm) -- (5cm,-1.7cm) -- (5cm,-5cm) -- (1cm,-5cm) -- cycle;
draw (0.8cm,-2.1cm) -- (11.6cm,-2.1cm) -- (11.6cm,-3.1cm) -- (0.8cm,-3.1cm) -- cycle;

% labels

label("\bold{Pure JoyLoL}", (3cm,3.5cm));
label("Joylol interpreter", (3cm,-2.8cm));
label("CONS-Pairs", (3cm,-3.5cm));

label("$\bold{JoyLoL}_0$", origin) rotated 90 shifted (0.5cm,-2cm);
label("\bold{Characters}", origin) rotated 90 shifted (1.5cm, 0cm);
label("\bold{Symbols}", origin) rotated 90 shifted (2.5cm, 0cm);
label("\bold{Natural numbers}", origin) rotated 90 shifted (3.5cm, 0cm);
label("\bold{...}", origin) rotated 90 shifted (4.5cm, 0cm);

path impureJoylol;
impureJoylol := (0,-3.9cm) -- (5.9cm,-3.9cm) -- (5.9cm,4cm) -- (12cm, 4cm) --
  (12cm,-10cm) -- (0cm, -10cm) -- cycle;

fill impureJoylol withcolor (1, 1, 1);

draw (0.1cm, -3.9cm) -- (5.9cm,-3.9cm) -- (5.9cm, 3.9cm) -- (0.1cm, 3.9cm) -- cycle
  withpen pencircle scaled 1mm
  withcolor lightgray;
  
path pureJoylol;
pureJoylol := (0cm,-4cm) -- (6cm,-4cm) -- (6cm,4cm) -- (0cm,4cm) -- cycle;

setbounds currentpicture to pureJoylol;
clip currentpicture to pureJoylol;

\stopMPcode\egroup

\placefigure[][fig:performantJoylol]{A Software Architectural description 
of a preformant \joylol\ implementation} 
\bgroup\startMPcode
input hatching.mp;
picture dots; dots := dashpattern(on 0.1mm off 0.5mm);

% JoyLoL

draw (0.1cm,0cm) -- (11.9cm, 0cm) -- (11.9cm, 3.9cm) -- (0.1cm, 3.9cm) -- cycle
  withpen pencircle scaled 1mm
  withcolor lightgray;

% JoyLol0
path joylolZero;
joylolZero := (0.1cm,0cm) -- (6cm, 0cm) --
  (6cm, -4cm) -- (0.1cm, -4cm) -- cycle;

draw image (
  hatchfill joylolZero
    withcolor (45,2mm,-.5bp);
) dashed dots withcolor darkgray;

draw joylolZero
  withpen pencircle scaled 1mm
  withcolor lightgray;

% ANSI-C implementation
path ansic;
ansic := (0.1cm,-4cm) -- (6cm, -4cm) --
     (6cm,0cm) -- (11.9cm, 0cm) -- (11.9cm, -7.9cm) -- (0.1cm, -7.9cm) -- cycle;

draw image (
  hatchfill ansic
    %withcolor (45,2mm,-.5bp)
    withcolor (-45,2mm,-.5bp);
) dashed dots withcolor darkgray;

draw ansic
  withpen pencircle scaled 1mm
  withcolor lightgray;

path operatingSystem;
operatingSystem := (0.1cm,-7.9cm) -- (11.9cm,-7.9cm) --
  (11.9cm,-9.9cm) -- (0.1cm,-9.9cm) -- cycle;
  
draw image (
  hatchfill operatingSystem
    withcolor (0,2mm,-.5bp);
) dashed dots withcolor darkgray;

draw operatingSystem
  withpen pencircle scaled 1mm
  withcolor lightgray;

% CoAlgs

draw (1.1cm,3cm) -- (1.9cm,3cm) -- (1.9cm,-2.5cm) -- (1.1cm,-2.5cm) -- cycle;
draw (2.1cm,3cm) -- (2.9cm,3cm) -- (2.9cm,-2.5cm) -- (2.1cm,-2.5cm) -- cycle;
draw (3.1cm,3cm) -- (3.9cm,3cm) -- (3.9cm,-2.5cm) -- (3.1cm,-2.5cm) -- cycle;
draw (4.1cm,3cm) -- (4.9cm,3cm) -- (4.9cm,-2.5cm) -- (4.1cm,-2.5cm) -- cycle;

draw (6.6cm,3cm) -- (7.4cm,3cm) -- (7.4cm,-8.5cm) -- (6.6cm,-8.5cm) -- cycle;
draw (7.6cm,3cm) -- (8.4cm,3cm) -- (8.4cm,-5.5cm) -- (7.6cm,-5.5cm) -- cycle;
draw (8.6cm,3cm) -- (9.4cm,3cm) -- (9.4cm,-8.5cm) -- (8.6cm,-8.5cm) -- cycle;
draw (9.6cm,3cm) -- (10.4cm,3cm) -- (10.4cm,-5.5cm) -- (9.6cm,-5.5cm) -- cycle;
draw (10.6cm,3cm) -- (11.4cm,3cm) -- (11.4cm,-8.5cm) -- (10.6cm,-8.5cm) -- cycle;

draw (1cm,-1.7cm)   -- (5cm,-1.7cm)    -- (5cm,-5cm)      -- (1cm,-5cm)     -- cycle;
draw (0.8cm,-2.1cm) -- (11.6cm,-2.1cm) -- (11.6cm,-3.1cm) -- (0.8cm,-3.1cm) -- cycle;
draw (0.8cm,-4.5cm) -- (11.6cm,-4.5cm) -- (11.6cm,-6cm)   -- (0.8cm,-6cm)   -- cycle;

% labels

label("\bold{JoyLoL}", (6cm,3.5cm));
label("Joylol interpreter", (6cm,-2.8cm));
label("CONS-Pairs", (3cm,-3.5cm));
label("Memory Management", (4cm,-5.5cm));
label("ANSI-C Implementation", (4cm,-6.7cm));
label("/ Lua wrapper", (4cm,-7.2cm));
label("Operating System / \LuaTeX", (6cm, -9cm));

label("$\bold{JoyLoL}_0$", origin) rotated 90 shifted (0.5cm,-2cm);
label("\bold{Characters}", origin) rotated 90 shifted (1.5cm, 0cm);
label("\bold{Symbols}", origin) rotated 90 shifted (2.5cm, 0cm);
label("\bold{Natural numbers}", origin) rotated 90 shifted (3.5cm, 0cm);
label("\bold{...}", origin) rotated 90 shifted (4.5cm, 0cm);

label("\bold{Characters} Alt", origin) rotated 90 shifted (7cm, 0cm);
label("(UTF-8 implementation)", origin) rotated 90 shifted (7cm, -6.5cm);
label("\bold{Symbols} Alt", origin) rotated 90 shifted (8cm, 0cm);
label("\bold{Natural numbers} Alt", origin) rotated 90 shifted (9cm, 0cm);
label("(GMP libraries)", origin) rotated 90 shifted (9cm, -6.5cm);
label("\bold{...}", origin) rotated 90 shifted (10cm, 0cm);
label("\bold{OS interfaces}", origin) rotated 90 shifted (11cm, 0cm);
label("(OS / \LuaTeX\ libraries)", origin) rotated 90 shifted (11cm, -6.5cm);

draw (0cm,-10cm) -- (12cm,-10cm) -- (12cm,4cm) -- (0cm,4cm) -- cycle;

\stopMPcode\egroup

In terms of conceptual complexity, (Logical) set theory has two distinct 
variants, pure set theory without \quote{ur-elements}, and naive set 
theory which allows \quote{ur-elements}. \quote{Ur-elements} are 
\quote{atomic} elements which might be elements of some set, but which 
contain no (sub)elements of their own. 

Similarly there are three distinct types of \joylol: 

\startitemize[1]

\item \bold{Purest \joylolZero} which allows only the characters \quote{(} 
and \quote{)} and lists of lists. It is this version of \joylol\ which is 
the most similar to pure set theory. (See figure \in[fig:pureJoylolZero]) 

\item \bold{Pure \joylol} which allows explicit textual \quote{symbols} 
and \quote{natural numbers}, as well as lists of lists. This is the verion 
of \joylol\ which is most similar to naive set theory as it is used in 
typical mathematical practice. (See figure \in[fig:pureJoylol]) 

\item \bold{Performant \joylol} implementations which allow arbitrary 
more performant implementations of coalgebras using non-\joylol\ computing 
languages. This is the version of \joylol\ which is most similar to 
various computational proof assistants such as Agda, Coq, ACL2, NuPRL, 
HOL, and Isabelle. (See figure \in[fig:performantJoylol]) 

\stopitemize 

The dialect of \joylol\ without \quote{ur-elements}, \joylolZero, is the 
purest form of \joylol. In \joylolZero, the \emph{only} allowed characters 
are \quote{(} and \quote{)}. In \joylolZero\ \emph{everything} is a list 
of lists. Unfortunately, for \quote{bears of little brain}, such as 
myself, this is a far too austere language in which to think about 
mathematics. Hence, to help \quote{bears of little brain} understand what 
they are doing, we allow \joylol\ to have atomic symbols. 

Since \joylol\ \emph{is} computational, to be useful, we need to embed 
\joylol\ \emph{implementations} into one or more \emph{modern} computation 
systems. Commonly used \joylol\ coalgebras, such as for example, the 
natural numbers could be implemented as \joylolZero\ list of list 
structures complete with associated (recursive) addition and 
multiplication. However, since the addition and multiplication of natural 
numbers is so pervasive in mathematics, a pure (computational) 
implementation as \joylolZero\ lists of lists is unlikely to be performant 
enough for general use. This means we will generally need to allow more 
performant alternative implementations of many important coalgebras such 
as the natural numbers. 

This brings the question of how do we tell if two \quote{implementations} 
of a concept behave in the \quote{same} way?. The behaviour of a 
collection of \quote{things} are traditionally captured as a coalgebra. 
The most important thing about the definition of coalgebras is that 
\quote{identity} is captured via (behavioural) \quote{bisimulation}. Two 
coalebras behave the same if they are bisimular. Hence, two 
implementations behave the same if they are bisimular. While an 
implementation of the natural numbers in \joylolZero\ as lists of lists 
will be transparently \joylolZero\ code whose behaviour can be rigorously 
verified, typical implementations of the natural numbers in, for example 
ANSI-C, contain behaviours which are not as transparent. For example, both 
the \joylolZero as well as the GNU Multiple Precision Arithmetic (GMP) 
Library's implementation of the natural numbers on any particular computer 
system, \emph{will} have a concrete bound on the size of the largest 
natural number it can represent, if only because the computer system has 
run out of \emph{all} of the memory resources it may use. However, GMP's 
implementation will also have internal structures whose memory use will 
potentially impose additional behaviours which might not be explicitly 
documented in GMP's formal description. Given the complexity of GMP's 
implementation, GMP does not have a rigorous proof of correctness. Indeed 
the ANSI-C compilers used to compile \joylolZero\ and \joylol\ typically 
do not have rigorous proofs of correctness either. 

The GNU Multiple Precision Arithmetic Library (GMP) is a free library for 
arbitrary-precision arithmetic, operating on signed integers, rational 
numbers, and floating point numbers (Wikipedia). We only use the unsigned 
integers. 

UTF-8 is a variable width character encoding capable of encoding all 
1,112,064 valid code points in Unicode using one to four 8-bit bytes. The 
encoding is defined by the Unicode standard, and was originally designed 
by Ken Thompson and Rob Pike (Wikipedia). We (explicitly) only use the 
ASCII subset. 

\subsection[title=\joylol's processing components]

In diagram \in[fig:joylolProcess] we depict the major components 
required for a given computation. Focusing upon the processing steps, a 
given computation typically unfolds as follows: 

\startitemize[n]

\item A \bold{parser} takes some \bold{code as text} and transforms it 
using the \emph{syntactical} rules of the language \joylol, into an 
(potentially) \bold{untyped abstract syntax tree (AST)}. 

\item A \bold{type inferencer} takes an \bold{untyped AST} and computes 
types for all implied pre and post stack frames for all \joylol\ 
statements. Our type inferencer uses the Hindley-Milner type inference 
algorithm essentially altered for stack based languages as suggested in 
\cite{diggins2008simpleTypeInference}.

\item A \bold{verifier} takes the \bold{typed AST} and verifies that the 
pre and post conditions for each \joylol\ statement is correct. Our 
verifier is based the $\mu$-modal logic satisfication checking algorithms 
in \cite{demriGorankoLange2016temporalLogics} and 
\cite{harelKozenTiuryn2000dynamicLogic}. 

\item An \bold{interpreter} takes the \bold{verified AST} and interprets 
it given any supplied input data. The interpreter might create either 
output data or a new \bold{code as text} (or both). Our interpreter is 
based upon ideas taken from \cite{vonThun1996interpreter} and 
continuations taken from \cite{gordon1979denotationalDescriptionProgLangs} 
(as well as the \quote{standard} ideas of \quote{trampolining tail 
recursion in ANSI-C}). 

\stopitemize

%define \diagramBox == \db for use in diagrams
\def\db#1#2{\vbox{\hsize=#1 \startalign[center,nothyphenated]#2\stopalign}}

\placefigure[][fig:joylolProcess]{Compiling \joylol}
\bgroup\startMPcode
path dataBox;
dataBox := 
  (-0.75cm,0.75cm) -- (0.75cm,0.75cm) -- (0.75cm,-0.75cm) --
  (-0.5cm, -0.75cm) -- (-0.75cm, -0.5cm) -- cycle;

path processBox;
processBox := 
  (-1cm,0.75cm) -- (1cm,0.75cm) --
  (1cm,-0.75cm) -- (-1cm,-0.75cm) -- cycle;

draw image(
  draw dataBox;
  label("\db{1cm}{Code as Text}", origin);
);

draw image(
  draw dataBox;
  label("\db{1.5cm}{Code as untyped AST}", origin);
) shifted (0cm, -3cm);

drawarrow (0cm,-0.75cm){dir 270} .. (1.5cm,-1.5cm) .. {dir 270}(0cm, -2.25cm)
  withpen pencircle scaled .25mm ;

draw image(
  draw processBox;
  label("\db{1.5cm}{Parser}", origin);
) shifted (2.5cm, -1.5cm);

draw image(
  draw dataBox;
  label("\db{1.5cm}{Code as typed AST}", origin);
) shifted (0cm, -6cm);

drawarrow (0cm, -3.75cm){dir 270} .. (1.5cm,-4.5cm) .. {dir 270}(0cm, -5.25cm)
  withpen pencircle scaled .25mm ;

draw image(
  draw processBox;
  label("\db{1.5cm}{Type Inferencer}", origin);
) shifted (2.5cm, -4.5cm);

draw image(
  draw dataBox;
  label("\db{1.5cm}{Code as verified AST}", origin);
) shifted (0cm, -9cm);

drawarrow (0cm, -6.75cm){dir 270} .. (1.5cm,-7.5cm) .. {dir 270}(0cm, -8.25cm)
  withpen pencircle scaled .25mm ;

draw image(
  draw processBox;
  label("\db{1.5cm}{Verifier}", origin);
) shifted (2.5cm, -7.5cm);

draw image(
  draw dataBox;
  label("\db{1.5cm}{Input data}", origin);
) shifted (-4cm, -12cm);

draw image(
  draw dataBox;
  label("\db{1.5cm}{Output data}", origin);
) shifted (4cm, -12cm);

draw image(
  draw processBox;
  label("\db{1.5cm}{Interpreter}", origin);
) shifted (0cm, -11.5cm);

drawarrow (0cm, -9.75cm) -- (0cm, -10.75cm)
  withpen pencircle scaled .25mm ;

drawarrow (-3.25cm, -12cm) -- (-1cm, -12cm)
  dashed evenly
  withpen pencircle scaled .25mm;

drawarrow (1cm, -12cm) -- (3.25cm, -12cm)
  dashed evenly
  withpen pencircle scaled .25mm ;
  
drawarrow (-1cm, -11cm){dir 180} .. (-4cm, -5.5cm) .. {dir 0}(-0.75cm, 0cm)
  dashed evenly
  withpen pencircle scaled .25mm;
\stopMPcode\egroup


\TODO{Discuss how to deal with lack of rigor in implementations (See 
JoyLoL implementation paper(s)). Note that this is a problem even in 
Science. How do we know that a given model, models \quote{Reality}.} 

\TODO{Discuss the importance of being able to add new Coalgebra 
implementations using, for example, \joylol, ANSI-C or Lua.} 

