% A ConTeXt document [master document: ../hilbertsProgram.tex]

\section[title=Strategy]

This document will provide a rigorous \emph{computational} foundation of 
Mathematics, by... 

\TODO{Need to talk about lists of lists}

We will do this in a number of distinct steps. Firstly, by defining a 
computational langauge, JoyLoL (\quote{The Joy of Lists of 
Lists}\footnote{Or is it \quote{The Joy of Laughing out Loud}?}). JoyLoL 
is a functional \emph{concatenative} language based upon Manfred von 
Thun's language Joy, \cite{vonThun1994overview}. The critically important 
aspect of JoyLoL is that it is constructed to be a fixed point of the 
semantics functor. This means that JoyLoL provides its own denotation, 
operational and axiomatic semantics. JoyLoL does not rely upon any other 
\quote{pre-existing} structures or set theory to define its meaning. An 
other important aspect of JoyLoL is that it is a \emph{concatenative} 
function language. Almost all other functional programming languages are 
based upon Church's $\lambda$-calculus, importantly, this means that most 
such langauges are focused upon function evaluation and substitution. From 
a categorical point of view, this means that the collection of 
computational traces forms a Topos. Being \emph{concatenative}, the 
collection of JoyLoL computational traces forms a Category, which also 
happens to be a Topos. The distinction here is important. The requirements 
of being a Category are much simpler and valid of many more distinct 
sub-collections of computational traces. 

, we can construct the structure of all JoyLoL computational traces. We 
can define the collection of \emph{finite} substructures as those 
substructures for which a simple \emph{short} JoyLoL program \emph{halts}. 
These finite substructures 

Since there \emph{are} JoyLoL computations which do not halt, this 
structure 
