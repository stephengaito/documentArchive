% A ConTeXt document [master document: hilbertsProgram.tex]

\chapter[title=Introduction]

\section[title=Goals]

Every long term researcher should have a silly question, if only to keep 
them focused upon whole problems amidst all of the unending detail. Your 
ultimate destination, whether or not you get there, influences how you 
plan to get there. The problem for any researcher, is that the research 
\quote{space} is infinite dimensional. From the dazzle of choices, you 
must make \quote{one} sequence of choices which, one hopes, ultimately 
leads to your answer(s). The vantage point provided by your ultimate goal, 
influences the questions asked and hence the answers found. The choice of 
where you want to get to, does have a profound influence on how you choose 
to get there. 

My silly question is:

\startalignment[center] Why do babies babble?\stopalignment

\blank[big]

There are (at least) two aspects to this question:

\startitemize[n]

\item How do organic naive \quote{brains} build models of reality. Too 
much of classical artificial intelligence focuses on the incremental 
learning based upon the capabilities of \quote{adult} learners. However 
this misses the point of learning models of Reality \emph{ab initio}. 

\item Equally important, is the question of what constitutes an efficient 
model. Animal brains must keenly balance energy use with the 
comprehensiveness of a given model. Any animal that gets this wrong too 
much of the time, becomes someone else's lunch. 

\stopitemize 

So put simply, my objective, as a \emph{mathematician}, is to build 
efficient and mathematically rigorous models Reality. However, if you are 
going to build a mathematically rigorous theory of Reality, you must first 
provide a mathematically rigorous theory of Mathematics itself. This 
document is devoted to providing just such a rigorous foundation for 
Mathematics. 

Since the ancient Greeks, western influenced Philosophical, Scientific, 
Mathematical practice has been to equate rigour with proofs of 
\quote{Truth}. Euclid's \quote{Elements} has, for just over two millennia, 
provided the pre-eminent example of this paradigm of rigorous proof. 
However, G\"odel's two Incompleteness theorems, show that classical proofs 
through logic are unable to provide the rigorous foundations we require. 
This is the failure of Hilbert's \emph{logical} Program. 

Essentially, G\"odel's work around the early 1930's represent the first 
contributions to the theory of computation. G\"odel's theorems from this 
period concern themselves with our ability to compute \quote{Truth}, and 
the fact that such computation of \quote{Truth}, is only partially 
recursive, rather than totally recursive.

Instead of computing \quote{Truth}, is there a more useful computation 
which might provide a foundation of Mathematics? By exploring Hilbert's 
Program within a \emph{computational} framework, the objective of this 
document is to show that the answer to this question is yes. 
