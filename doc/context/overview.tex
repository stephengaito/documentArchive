% A ConTeXt document [master document: joyLoL.tex ]

\startcomponent overview

\startchapter[title=Overview]

\startsection[title=Introduction]

We will provide full formal descriptions of six distinctly different 
programming languages, JoyLoL, WhileLoL, WhileRecLoL, EagerLambdaLoL, 
LazyLambdaLoL and LogicLoL. The languages WhileLoL and WhileRecLoL will be 
in the same overall class of languages as most standard imperative 
programming languages such as, Lua, C, Pascal, Algol, Java, Python, PHP, 
Ruby, etc. Most programmers will recognize WhileRecLoL as a close, but 
simplified, cousin to the languages they are currently using. The 
languages EagerLambaLoL and LazyLambdaLoL represent the class of eager and 
lazy functional languages such as ML and Haskell respectively. The 
LogicLoL language represents the class of logic programming lanaguages 
such as Prolog. 

The JoyLoL language, is in a new class of \quote{concatenative} languages 
originally explored by Manfred von Thun\footnote{For example, see 
\cite[vonThun1994overview], \cite[vonThun1994mathematicalFoundations] or 
\cite[vonThun1994rational] all of which can be found in 
\cite[vonThun2011archive] or \cite[vonThun2005website].}. The primary 
importance of JoyLoL is that it \emph{is} a fixed point of the formal 
semantics operator. As a fixed point of the semantics operator, JoyLoL 
provides a foundation for both computation and more importantly 
Mathematics\footnote{While we assert that JoyLoL provides a 
\emph{computational} foundation for Mathematics, proving this assertion 
will be the work of many (future) papers. We will not even attempt a proof 
in this document.}. 

Across all of these languages the constant similarity is the \quote{LoL} 
or List of Lists. In each language the \emph{only} expressions are Lists 
of Lists. Depending upon the language, these lists of lists are 
potentially infinite expressions, which will, however, always have a 
finite description at any particular point in a computation. 

As developed over the past 50 years, the formal semantics operator has 
three parts: 

\startitemize[n, packed]

\item Denotational Semantics (roughly equivalent to Tarski's model 
semantics) 

\item Operational Semantics (roughly equivalent to Gentzen's natural 
deduction) 

\item Axiomatic Semantics (roughly equivalent to Type theory) 

\stopitemize

Good introductions to these three types of formal semantics 
can be found in \cite[winskel1993formalSemanticsProgrammingLanguages] and 
\cite[gunter1992semainticProgrammingLanguages]. (TODO see 
\cite[aptVanEmden1980logicProgramming] and 
\cite[billaud1990semanticsOfPrologWithCut] for early reports of Prolog's 
semantics) 

The collection of Lists of Lists \emph{is} an \quotation{infinitely} 
\quotation{complex} \quotation{structure}. In its complete incarnation, it 
is strictly \emph{more} complex than the whole of any formal \emph{set} 
theory such as ZFC\footnote{In this paper we will only consider the 
component which corresponds to classical, $\omega$-computation.}. This is, 
for finite beings, such as mere mortal mathematicians, a large and complex 
\quote{world} to explore. It is a world in which it is very easy to get 
lost. While we assert that JoyLoL provides a computational foundation for 
Mathematics, to help us \quote{mere mortal mathematicians} orient 
ourselves, we will often make reference to classical mathematical 
concepts. It is important to realize that these classical mathematical 
concepts are simply aids to our mathematical intuition, and not formal 
statements. 

The most important classical intuition is that of Algebra and CoAlgebra, 
or equivalently, for a Computer Scientist, that of Data and Process. Both 
classical Mathematics and Computational theory have, by and large, 
explicitly limited themselves to the well-founded and terminating, largely 
to avoid Poincar\'{e}'s \quote{Vicious Circles}. We will see that the 
non-well-founded, non-terminating processes, Cantor's \quote{Absolute 
Infinite}, have a surprisingly easily understood structure, essentially 
dual to classical set theory. However, the \emph{computational} theory of 
these non-terminating processes, has a profound impact on Mathematics. By 
ignoring this computational theory, we, as mathematicians, make simple 
problems, hard. 

\emph{Intuitively}, \emph{a} List of Lists, is a potentially infinite 
structure which records a potentially infinite \emph{structured} 
collection of observations of a potentially non-terminating process of 
processes. As \emph{finite} mathematicians, we can only ever manipulate 
finite structures, \emph{finite records of observations} of a potentially 
infinte process. One such record of observations might be denoted by, for 
example: 

\startcenteraligned
\starttyping
(() ( ( )))
\stoptyping
\stopcenteraligned

\noindentation This is of course the denotation of Lists in John 
McCarthy's Lisp, see 
\cite[mcCarthyAbrahamsEdwardsHartLevin1965lispManual]. 

Where classically, formal semantics concentrated on \emph{one} denotation, 
to provide a \emph{formal} description of these potentially 
non-terminating process, it is critical that we carefully distinguish 
\emph{two} distinct denotations: the classical \emph{algebraic} denotation 
(corresponding to a least fixed point of the semantics operator) and the 
non-classical \emph{coAlgebraic} denotation (corresponding to a greatest 
fixed point of the semantics operator). While for data, the data itself is 
its own denotation, for a potentially non-terminating process, \emph{an 
answer} (a data object) is insufficient to \emph{denote} that process. 
Instead the appropriate denotation of a non-terminating process is its 
trace of observations (or any finite record of this trace which is 
\quote{sufficient} for current purposes). Similarly, while the 
\quote{big-step} operational semantics might suffice for a data object, 
the \quote{one-step} operational semantics is the only definition of 
operational semantics appropriate for a potentially non-terminating 
process. Finally, to provide an Axiomatic semantics, with out recourse to 
classical first order set theory, we will make essential use of finite 
descriptions of the traces of potentially non-terminating processes. 

Philosophically, it is important to know when two \quote{things} are the 
\quote{same}. For an algebraic list of lists, two lists are equal if there 
is a \emph{finite}, structurally inductive, comparison of the two objects. 
This is the familiar concept of recursive equality. For sets this is 
\emph{extensive} equality. For a coAlgebraic list of lists, two 
potentially non-terminating processes are equal if they respond in the 
same way to any collection of \quote{observations}. This is the concept, 
from Theoretical Computer Science and CoAlgebraic Category theory, of 
\quote{bisimulation}. 

\stopsection 

\startsection[title=How certain can certainty be?]

We intend to show that JoyLoL provides a foundation for Mathematics. Any 
foundation for Mathematics, must be \quote{certain}, but what exactly 
does \quote{certainty} mean and how \quote{certain} can a finite 
computational artefact be? 

see \cite[mackenzie2001mechanizingProof] and \cite[lakatos1976proofsRefutations]

\cite[giaquinto2002searchCertainty]

\stopsection

\stopchapter

\stopcomponent