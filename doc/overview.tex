% A ConTeXt document [master document: protoJoyLoL.tex ]

\chapter[title=Introduction]

\section[title=Overview]

The \joylol\ compiler is self-hosted, this means that it compiles itself. 
Building a self-hosted compiler is a multi-step process. While all 
\joylol\ compilers are, technically, transpilers to ANSI-C, the critically 
important feature of any fully self-hosting \joylol\ compiler is that it 
produces fully verified code. Since ANSI-C compilers are generally not 
fully verified, we need to go through a number of steps to go from 
\quote{hand written} and unverified code to a fully self-hosted and 
verified \joylol\ compiler. 

In the rest of this section we outline the steps we will use to produce 
the first self-hosting \joylol\ compiler.

We will create a fully verifying self-hosted \joylol\ compiler in the 
following five steps: 

\startitemize[n]

\item \bold{Step 1:} Use ANSI-C to produce a command line, hand-coded, 
unverified, \joylol\ interpreter complete with an interpreted \jPeg\ 
parser. We use this \jPeg\ parser to implement a simple \joylol\ 
s-expression parser (\joylolSExp) so that the interpreter is capable of 
loading \joylol\ code in \joylolSExp\ form. 

\item \bold{Step 2:} Use the interpreted \jPeg\ parser to produce \joylol\ 
configuration file (\joylolCF) and \joylol\ register machine (\joylolRM) 
parsers written in \joylolSExp\ code. 

\item \bold{Step 3:} Use the \joylolCF\ and \joylolRM\ parsers to produce 
a \emph{self-hosted} (unverified) \joylol\ compiler. 

\item \bold{Step 4:} Use the unverified \joylol\ compiler to produce a 
self-hosted, \emph{verified}, \joylol\ compiler. 

\item \bold{Step 5:} Use the verified \joylol\ compiler to produce a 
self-hosted, verified, \emph{optimized}, \joylol\ compiler. 

\stopitemize

\subsection[title=Step 0: The simplest interpreter]

Early compilers (or probably more correctly assemblers) were, in modern 
terms, a parser / template system coupled together in, usually, one-pass. 
The parser would take a string of human understandable characters 
conforming to a, preferably context free grammar (CFG), and parse the 
sting into an abstract syntax tree (AST). This AST would be used to drive 
the machine code based templates to produce an object which could be 
executed on a particular computer\footnote{In fact early one-pass 
compiler/assemblers probably did not bother to produce an AST. They would 
simply emit some machine code dependent upon the particular grammar 
element recognized at a particular point in the parse. However, while only 
notional, they did effectively produce an intermediary AST. However, this 
AST will help structure the complexity of \joylol's self-hosting 
compiler.}. 

The syntax of any modern computer language has a very complex context free 
grammar (CFG). Ideally this CFG grammar should be either LL(1) or LR(1). 
That is the parser proceeds from \bold{L}eft to right along the string, 
and constructs either a \bold{L}eft or \bold{R}ight derivation (of the 
AST) while only looking ahead by \bold{one} symbol. We will use a 
composition of a number of independent Parsing Expression Grammars (PEGs) 
together with a global LR(1) grammar for \emph{(infix) expressions}. 

The complexity of the full \joylol\ grammar is such that we need a tool 
which can take the description of the full grammar and automatically 
produce a parser. The objective of step 0 is to provide a \joylol\ 
interpreter which has enough functionality to build this parser. 

This means that we need a \joylol\ interpreter which can:

\startitemize[n]

\item read a file into a sequence of characters

\item parse a sequence of characters into an AST

\item interpret this AST as joylol code

\item produce a sequence of characters using a template system

\item write a file from the produced sequence of characters

\stopitemize

The appeal of a PEG for our purposes, comes from the work of the LPeg 
parser designers, \cite{ierusalimschy2008lpegArticle}, 
\cite{medeirosIerusalimschy2008parsingMachinePEGs}, 
\cite{mascarenhasMedeirosIerusalimschy2014CFGandPEG}, 
\cite{maidlMascarenhasMedeirosIerusalimschy2016errorReportingPEGs}, and 
\cite{medeirosMascarenhas2018syntaxErrorRecoveryPEGs}. Critically for us 
is the fact that the LPeg's underlying \quote{parsing machine}, 
\cite{ierusalimschy2008lpegArticle}, 
\cite{medeirosIerusalimschy2008parsingMachinePEGs}, can be easily 
implemented in \joylol\ code. 

Our objective in step 0 is to hand-code, in ANSI-C, a \joylol\ interpreter 
rich enough to interpret a jPeg based loosely upon the LPegLabel, 
\cite{medeiros2019lpegLabel}. 

\subsection[title=Step 1: ????]

