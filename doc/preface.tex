% A ConTeXt document [master document: hilbertsProgram.tex]

\chapter[title=Preface] 

Despite the wealth of fruitful work inspired by Hilbert's Program, it is 
generally assumed that G\"odel's two Incompleteness Theorems show that 
Hilbert's Program has failed. The purpose of this document is to show that 
while Hilbert's \emph{logical} Program is doomed to failure, a 
\emph{computational} version of Hilbert's Program is very definitely 
capable of providing a \emph{neo-platonist} foundation of mathematics.

The important distinction here is that G\"odel's theorems show that the 
set of \quote{True} statements of a given \emph{consistent} formal theory 
which is at least as powerful as Peano Arithmetic, is a recursively 
enumerable set which is not recursive. This is what makes Hilbert's 
\emph{logical} Program fail. In this analysis, one of the two key 
requirements is that the formal theory is \emph{consistent}. That is you 
can never \quote{prove} both a statement and its negation. It is this 
assumption of \emph{consistency} which forces the set of \quote{True} 
statements to be recursively enumerable but not recursive. 

By working \emph{computationally}, we can, with an assumption equivalent 
to the Axiom of Choice, build \emph{recursive} sets which can provide 
interpretations of standard Zermelo–Fraenkel set theory (with the Axiom 
of Choice). For each ordinal, $\lambda$, we can define 
$\lambda$-computation. Standard computation is then $\omega$-computation. 
Using a computationally natural form of Vop\v{e}nka's 
Principle\footnote{See Ad\'amek and Rosik\'y's \emph{Locally Presentable 
and Accessible Categories}, we can build recursive sets of the size of a 
Vop\v{e}nka cardinal, 
\cite{adamekRosicky1994locallyPresentableAccessibleCategories}.} into 
which we can then build an interpretation of Zermelo–Fraenkel set theory 
(ZFC), hence showing that ZFC is (\emph{computationally}) consistent. 

By taking Cantor's Absolutely infinite multiplicities (AIMs)\footnote{See 
Cantor's letter to Dedekind dated 1899, 
\cite{vanHeijenoort1967fregeToGodel}.} seriously, we can do a significant 
amount of classical analysis even with $\omega$-computational resources. 
For a given amount of computational resources, we build a dual pair of 
structures which correspond to well-founded sets and 
non-well-founded\footnote{See Aczel's \emph{Non-Well-Founded Sets}, 
\cite{aczel1988nonWellFoundedSets}, or Barwise and Moss' \emph{Vicious 
Circles}, \cite{barwiseMoss1996viciousCircles}.} proper classes, or in 
terms of the terminology of Computer Science, data and processes. 
Categorically these dual structures are a dual Topos / Co-Topos pair. 
While the classical Reals, are located in the well-founded sets (Topos) of 
ZFC and hence obey classical logic, the $\omega$-computational Reals are 
processes (Co-Topos) and obey a weaker process logic. The important 
realization here is that the \quote{strangeness} of Quantum Mechanics 
comes from trying to understand the essentially process nature of Quantum 
Mechanics within the logic of classical sets. We assert that the natural 
outcome of developing the \quote{logic} of finite process structures 
\emph{is} Quantum Relativity. Unfortunately, this is a topic for a 
subsequent document. 

In this volume, we will concentrate on the base case of 
$\omega$-computation. Subsequent volumes will look at how to define the 
transfinite ordinals and the corresponding $\lambda$-computation for 
$\omega \leq \lambda$. With $\lambda$-computation, we can then show that 
ZFC has a model and hence is \emph{computationally} consistent. 

\TODO{make the recursively-enumerable/partially-recursive versus 
recursive/totally-recursive terminology more uniform.} 

