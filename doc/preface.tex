% A ConTeXt document [master document: shaggyDog.tex]

\chapter[title=Preface]

This work is the result of 45 years of thought. The original problem was 
essentially posed to me by my father. As an Experimental Psychologist, my 
father was looking for a chemical basis of memory based upon the, then 
recently discovered, structures of DNA and RNA. I realized that the tools 
he had to solve this problem in the late 1960s and early 1970s were too 
crude to provide the answers he needed. As a teenage \emph{Mathematician} 
I wanted to provide stronger \emph{mathematical} tools with which to 
assist his understanding of how the brain learns. Not surprisingly, he 
finished his life's work long before I had found enough mathematical 
skills to be of much assistance. 

As a late teen and young adult I spent time trying to understand the 
foundations of Mathematics. What does mathematics \emph{mean}? How do we 
know that it is \emph{correct}?\footnote{Note that I am carefully 
\emph{not} using the word \quote{True} though at the time I did not 
realize the importance of the distinction between \quote{correct} and 
\quote{True} or even \quote{true}.} While my mathematical maturity was, at 
that time, not broad enough to fully understand the philosophical 
discussion surrounding the foundations of mathematics, I was left 
dissatisfied by most of what I read. As a graduate student studying 
Dynamical Systems, fixed points were natural objects, yet Russell’s 
paradox is a fixed point of a logical semantics, which the mathematical 
community went to great lengths to avoid. 

