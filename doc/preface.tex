% A ConTeXt document [master document: programmingInJoyLoL.tex]

\chapter[title=Preface] 

The programming language, \type{JoyLoL} (the Joy of Lists of 
Lists\footnote{Or alternatively, the Joy of Laughing out Loud}), is a 
concatenative programming language based loosely upon Manfred von Thun's 
programming language \type{Joy}, 
\cite{vonThun1994mathematicalFoundations}, \cite{vonThun1994overview}. 
Most programming languages, whether imperative, like \type{C}, \type{C++}, 
\type{C#} or Java, or functional, like \type{Lisp}, \type{ML}, or 
\type{Haskell}, depend upon the \emph{application} of functions to 
arguments. In a concatenative programming language, all expressions behave 
as functions, and the juxtaposition of expressions denotes function 
composition. The \quote{space} of all computations of a concatenative 
language has a natural a categorical structure which we will explicitly 
exploit. 

The JoyLoL language's primary goal is, in its most austere form, to be a 
fixed point of the semantics functor. Being a fixed point of the semantic 
functor, it is its own denotational, operational and axiomatic semantics. 
This means that, in particular, it provides a sufficient foundation for 
both computational theory, as well as, if granted transfinite 
computational power equivalent to Vopěnka's principle, all of 
conventional Mathematical discourse\footnote{Transfinite computational 
power is equivalent to the Axiom of choice, \emph{so} any one who accepts 
the Axiom of Choice must accept transfinite computational power, 
conversely, any one who rejects transfinite computational power, must 
reject the Axiom of Choice.}. 

Unfortunately for any mere human, JoyLoL's most austere form, while very 
easy for computers to parse, is very difficult for humans to understand. 
This document will instead discuss programming in an augmented form of 
JoyLoL (probably best called \type{JoyLoL++}). Eventually we will show 
that this augmented form of JoyLoL is in fact a \quote{conservative 
extension} of the austere form of JoyLoL. However, for this document, we 
will simply assume that the two variants of JoyLoL are equivalent. 
