% A ConTeXt document [master document: sortingJoyLoL.tex]

\chapter[title=Theory]

How do we verify that a particular code fragment, such as an 
implementation of the merge sort algorithm, actually computes what it 
purports to compute? This is the most critical question if we are to 
provide a rigorous \emph{computational} foundation for mathematics. 

Fortunately, most of the hard work required to answer this question 
already exists in one form or another. In the 1960's and 1970's, numerous 
computer scientists attempted to answer this question. Their answer was in 
the form of: 

\startitemize

\item {\bf Denotational semantics}

\item {\bf Operational semantics}

\item {\bf Axiomatic semantics}

\stopitemize

See \cite{winskel1993formalSemanticsProgrammingLanguages}, and 
\cite{gunter1992semainticProgrammingLanguages}.

Of course, denotational semantics and axiomatic semantics are effectively 
based upon Tarski's work on formal concepts of Truth, 
\cite{tarski1983truthFormalizedLanguages}, and Gentzen's work on both 
Natural Deduction and Sequential Calculus, 
\cite{gentzen1969collectedPapers}, all of which were conducted in the 
1930's. 

The syntax, semantics and computation of satisfaction will be taken from a 
mixture of the more recent Dynamic Logic, 
\cite{harelKozenTiuryn2000dynamicLogic}, and Modal $\mu$-Logic, 
\cite{demriGorankoLange2016temporalLogics}. 

\section[title=Syntax]

Both Dynamic Logic and Modal $\mu$-Logic have been developed assuming an 
external \emph{classical} logic, which includes the use of the \quote{law 
of the excluded middle}, and the axiom of choice. While ultimately both of 
these principles will be valid in various sub-sections of computational 
theory, we will not assume their validity until we have proven them valid. 
Hence we will develop an intuitionistic and constructive logic, based upon 
the type theory of Martin-L\"of, \cite{martinLofSambin1984typeTh}. 

The effect of this upon our syntax is that we can not depend upon De 
Morgan's laws to supply the duals of any of the operators we define. So 
our syntax is more explicit. 

In fact we need to consider the syntax of two languages, \joylolZero\ and 
\joylol. The syntax of \joylolZero\ is particularly simple, it consists of 
lists of the characters \quote{(} and \quote{)}. Formally the syntax of 
\joylolZero\ is an LL(1) context free language with the following 
Backus-Naur Form (BNF): 

\def\bnfProd{:\,:\,=}

\startformula\startmathalignment
\NC \text{<listOfLists>}
  \NC \text{ ::= \quote{(} <listOfLists> \quote{)} <listOfLists> 
    \| \quote{} }
  \NR
\stopmathalignment\stopformula

\noindent where \quote{} denotes the empty string.

The syntax of \joylol\ is only slightly more complex. It is designed to be 
slightly easier for a human to work with, as it adds quoted strings, 
single line comments, numbers, and symbols. It is based upon the ASCII 
character set, and is described with the following BNF extended with IEEE 
POSIX Extended Regular Expressions (ERE):

\startformula\startmathalignment
\NC \text{<whiteSpace>}
  \NC \text{ ::= [\textbackslash x00-\textbackslash x20]+} \NR
\NC \text{<singleLineComment>}
  \NC \text{ ::= \quote{;} [^\textbackslash nl\textbackslash cr]* } \NR
\NC \text{<naturalNumber>}
  \NC \text{ ::= [1-9][0-9]*} \NR
\NC \text{<sQuotedString>}
  \NC \text{ ::= \quote{'} [^']* \quote{'}} \NR
\NC \text{<dQuotedString>}
  \NC \text{ ::= \quote{"} [^"]* \quote{"}} \NR
\NC \text{<symbol>}
  \NC \text{ ::= [^\textbackslash x00-\textbackslash x20;'"]+} \NR
\NC \text{<abLoL>}
  \NC \text{ ::= \quote{<} <listOfLists> \quote{>}} \NR
\NC \text{<cbLoL>}
  \NC \text{ ::= \quote{\{} <listOfLists> \quote{\}}} \NR
\NC \text{<rbLoL>}
  \NC \text{ ::= \quote{(} <listOfLists> \quote{)}} \NR
\NC \text{<sbLoL>}
  \NC \text{ ::= \quote{[} <listOfLists> \quote{]}} \NR
\NC \text{<listOfLists>}
  \NC \text{ ::= <abLoL> <listOfLists> } \NR
\NC \NC \text{ \quad \| <cbLoL> <listofLists> } \NR
\NC \NC \text{ \quad \| <rbLoL> <listofLists> } \NR
\NC \NC \text{ \quad \| <sbLoL> <listofLists> } \NR
\NC \NC \text{ \quad \| <naturalNumber> } \NR
\NC \NC \text{ \quad \| <sQuotedString> } \NR
\NC \NC \text{ \quad \| <dQuotedString> } \NR
\NC \NC \text{ \quad \| <symbol> } \NR
\NC \NC \text{ \quad \| \quote{} } \NR
\stopmathalignment\stopformula

\noindent where <whiteSpace> and <singleLineComments> (as defined above) 
are ignored and may occur anywhere in a <listOfLists>.

\section[title=Semantics]

\section[title=Satisfaction]

