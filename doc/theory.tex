% A ConTeXt document [master document: sortingJoyLoL.tex]

\chapter[title=Theory]

How do we verify that a particular code fragment, such as an 
implementation of the merge sort algorithm, actually computes what it 
purports to compute? This is the most critical question if we are to 
provide a rigorous \emph{computational} foundation for mathematics. 

Fortunately, most of the hard work required to answer this question 
already exists in one form or another. In the 1960's and 1970's, numerous 
computer scientists attempted to answer this question. Their answer was in 
the form of: 

\startitemize

\item {\bf Denotational semantics}

\item {\bf Operational semantics}

\item {\bf Axiomatic semantics}

\stopitemize

See \cite{winskel1993formalSemanticsProgrammingLanguages}, and 
\cite{gunter1992semainticProgrammingLanguages}.

Of course, denotational semantics and axiomatic semantics are effectively 
based upon Tarski's work on formal concepts of Truth, 
\cite{tarski1983truthFormalizedLanguages}, and Gentzen's work on both 
Natural Deduction and Sequential Calculus, 
\cite{gentzen1969collectedPapers}, all of which were conducted in the 
1930's. 

The syntax, semantics and computation of satisfaction will be taken from a 
mixture of the more recent Dynamic Logic, 
\cite{harelKozenTiuryn2000dynamicLogic}, and Modal $\mu$-Logic, 
\cite{demriGorankoLange2016temporalLogics}. 

\section[title=Syntax]

Both Dynamic Logic and Modal $\mu$-Logic have been developed assuming an 
external classical logic, which includes the use of the \quote{law of the 
excluded middle}, and the axiom of choice. While ultimately both of these 
principles will be valid in various sub-sections of computational theory, 
we will not assume their validity until we have proven them valid. Hence 
we will develop an intuitionistic and constructive logic, based upon the 
type theory of Martin-L\"of, \cite{martinLofSambin1984typeTh}. 

The effect of this upon our syntax is that we can not depend upon De 
Morgan's laws to supply the duals of any of the operators we define. So 
our syntax is more explicit. 



\section[title=Semantics]

\section[title=Satisfaction]

