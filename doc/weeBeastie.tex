% A ConTeXt document [master document: hilbertsProgram.tex]

\chapter[title=Of Wee Beasties and Idealized Mathematicians]

In this chapter, over the next four definitions, \in[beastieBeing], 
\in[beastieActions], \in[beastieTests], \in[beastieIdentity], below, we 
will explore what a wee beastie \emph{is}, can \emph{do}, can \emph{know}, 
and can \emph{identify}. We do this by formally and \emph{computationally} 
defining a wee beastie's \quote{Reality} using the rigorous computational 
language \joylol. In fact, since we identify any wee beastie with a 
\joylol\ computation, our definitions in this chapter are actually 
defining the \quote{core} of the \joylol\ computer language. These four 
definitions are, of necessity, inter-referential. They are essentially 
different facets of the same combined definition of what the \joylol\ 
computer langauge is, as well as what a wee beastie's reality is. 

Intuitionistic Mathematics, as initiated by Brouwer, has identified the 
concept of the \quote{Idealized Mathematician}. Since this document is 
focused upon the \emph{Mathematics} of the \quote{Reality} which a wee 
beastie can know, for this document, we will identify any wee beastie with 
this Idealized Mathematician, and conversely, any Idealized Mathematician, 
with this wee beastie. 

Unlike Kant, this idealization makes \emph{no} assumptions about the 
structure of time or space. One consequence of our analysis of what a wee 
beastie can know, is space-time itself. 

Following Hilbert, we will assume a number of undefined terms which will 
be defined by their use in our theory. In the following definitions, using 
well known practice from Computer Science, we \emph{could} use the word 
\quote{widget} to emphasize, with Hilbert, that our undefined terms can 
represent anything which behaves in the prescribed way. Having made this 
point, we will actually use the undefined but slightly more suggestive 
terminology of \quote{Lists of Lists} or \lols. 

\section[title={A wee beastie's reality, the \joylol\ interpreter}]

\startDefinition[beastieBeing]

Any wee beastie \emph{is} a \quote{triple} of \lols, called the 
\quote{data} \quote{stack}, the \quote{process} \quote{stack} and the 
\quote{definition} \lol, respectively. 

At each \quote{computational instant}, the wee beastie takes the 
\quote{top} \lol\ off of the \quote{process} \quote{stack} and 
\quote{interprets} it by either: 

\startitemize[n] 

\item \bold{Reserved word}: If the \quote{top} \lol\ is a \quote{reserved} 
\quote{word}, as defined in any of the four definitions, 
\in[beastieBeing], \in[beastieActions], \in[beastieTests], or 
\in[beastieIdentity], then the beastie makes the described changes to the 
\quote{data} and \quote{process} \quote{stacks}. 

\item \bold{Known word}: If the \quote{top} \lol\ is a \quote{known} 
\quote{word}, then the wee beastie \quote{\type{push}}es the 
\quote{definition} of the known word onto the \quote{top} of the 
\quote{data} \quote{stack} and the \quote{\type{interpret}} \quote{word} 
onto the \quote{top} of the \quote{process} \quote{stack}. 

\item \bold{Otherwise}: The wee beastie \quote{\type{push}}es the \lol\ 
onto the \quote{top} of the \quote{data} \quote{stack}. 

\stopitemize 

\stopDefinition 

These \lols\ are important. For \emph{our} wee beasties, every thing is a 
\lol. Since we are considering the \emph{computational} basis of 
\quote{Reality}, we will assume that any wee beastie is a computational 
entity. At any one instant of \quote{time}, this wee beastie (or its 
associated computation) will be in a particular \quote{state}. For our 
purposes, these states \emph{will be} \lols.

At the same time these \lols\ will also capture the \quote{textual} 
description of any given program. While we could use any programming 
language, we will base our \emph{computational} foundations on the 
language \joylol\ which is in turn loosely based upon Manfred von 
Thun's \emph{concatenative} stack based language, \type{joy}\footnote{To 
better understand Manfred von Thun's programming language, \type{joy}, see 
\cite{vonThun1994overview} for an overview of the language and 
\cite{vonThun1994mathematicalFoundations} for an overview of the 
mathematical foundations of the language. The \quote{original} 
\emph{concatenative} stack based programming language was \type{Forth}. 
See \cite{brodie1987StartingForth} and \cite{brodie2004thinkingForth} for 
introductions to \type{Forth}.}. While most programming languages are 
applicative and make extensive use of variables, we will use a 
semantically simpler concatenative stack based language. 

\startMMundi In classical terms the category of computations of an 
applicative functional language, such as \type{Haskell}, corresponds to a 
Topos, and that of a concatenative stack based functional language, such 
as \joylol\ or \type{joy}, assumes no more structure than that of a 
Category. Not surprisingly, we will, eventually, show that the 
computational power of \joylol\ is equivalent to that of any other 
programming language. In classical terms this means that the Category of 
\joylol's computations actually has the structure of a Topos (or 
more correctly a Topos-CoTopos pair). \stopMMundi 

In fact to make \joylol\ semantically even simpler, we explicitly expose 
the stack of process \quote{continuations}\footnote{For a good, early, 
introduction to the use of continuations to capture the semantics of 
programming languages, see chapter 5 of 
\cite{gordon1979denotationalDescriptionProgLangs}.}. This means that the 
\quote{instantaneous} \quote{state} of any \joylol\ computation, is 
captured by a pair of \lols\ interpreted as the \quote{data stack} and 
\quote{process stack} respectively. 

The \joylol\ interpreter then in each computational \quote{instance}, 
takes one \lol\ off the top of the \quote{process stack} 

\section[title=What \emph{can} a \quote{wee beastie} do?]

We begin our trilogy of definitions by considering what a wee beastie can 
\emph{do} to its environment. 

\startDefinition[beastieActions]

We assume any wee beastie exists in a \quote{collection} of \quote{List of 
Lists} (\lols) represented itself as a \lol. With these \lols, a wee 
beastie can: 

\startitemize[n]

\item \bold{Atomic actions}

\startitemize[n]

\item \bold{Constructive actions}

\startitemize[n]

\item \bold{\quote{\type{nil}} action}: create a new \quote{top} 
\quote{\type{nil}} \lol. 

\item \bold{\quote{\type{cons}} action}: create a new \quote{top} \lol\ by 
removing the \quote{top} two \lols\ and \quote{\type{cons}}ing them 
together. 

\stopitemize

\item \bold{Destructive actions}

\startitemize[n]

\item \bold{\quote{\type{pop}} action}: delete the \quote{top} \lol\ by 
\quote{\type{pop}}ing it off the stack. 

\item \bold{\quote{\type{carCdr}} action}: create two new \quote{top} 
\lols\ by removing the \quote{top} \lol\ and replacing it with both the 
\quote{\type{car}} and \quote{\type{cdr}} of the original \lol. 

\stopitemize

\item \bold{Data Stack manipulation}

\startitemize[n]

\item \bold{\quote{\type{dup}} action}: \quote{\type{dup}}licate the 
\quote{top} \lol. 

\item \bold{\quote{\type{swap}} action}: \quote{\type{swap}} the 
\quote{order} of the \quote{top} two \lols. 

\stopitemize

\item \bold{Process Stack manipulation}

\startitemize[n]

\item \bold{\quote{\type{interpret}} action}:

\item \bold{\quote{\type{choice}} action}:

\item \bold{\quote{\type{repeat}} action}: 

\item \bold{\quote{\type{test}} action}:

\stopitemize

\item \bold{Definition list manipulation}

\startitemize[n]

\item \bold{\quote{\type{define}} action}: 

\stopitemize

\stopitemize

\item \bold{Action Combinators}

\startitemize[n]

\item \bold{composition}: the action which \quote{performs} a pair of 
actions one followed by the other. If $a$ and $b$ are two actions, then 
the \quote{composition} action is explicitly denoted as 
\quote{\explicitCompose{a}{b}} or more simply as \quote{\compose{a}{b}}. 

\item \bold{non-deterministic choice}: The action which \quote{performs} 
one or other of a pair of actions. There is, however, \emph{no 
pre-determined} reason for the choice. If $a$ and $b$ are two actions, 
then the \quote{(non-determinitsic) choice} action is denoted 
\quote{\ndChoice{a}{b}}. 

\item \bold{non-deterministic repeat}: The action which performs an action 
multiple times before \quote{completing}. However, the number of 
repetitions is \emph{not pre-determined}. If $a$ is an action, then we 
denote the \quote{(non-deterministic) repetition} action as 
\quote{\repets{a}}. 

\item \bold{test}: The action which \quote{completes} if a given 
test\footnote{Tests are defined in \in{Definition}[beastieTests].} 
succeeds. If $t$ is a test then we denote the \quote{test} action by 
\quote{\test{t}}. 

\stopitemize

\stopitemize

\noindent This is \emph{all} that any wee beastie, or (Idealized) 
Mathematician can \emph{do} to \lols\ in its environment\footnote{In 
Volume II, we will add one more thing that a \emph{transfinite} wee 
beastie can do.}. 

\stopDefinition

\startMMundi

The atomic actions provide the \emph{only} actions that a wee beastie can 
perform \emph{\quote{directly}} on its environment. All other actions are 
formed via various action combinators from these four atomic actions and, 
via the \quote{test} action, any of the tests defined below. 

The pair of definitions, \in[beastieActions] and \in[beastieTests], are 
essentially Dynamic Logic with the addition of the least, $\mu$, and 
greatest, $\nu$, fixed point operators taken from $\mu$-modal logic. See 
\cite{harelKozenTiuryn2000dynamicLogic} and 
\cite{demriGorankoLange2016temporalLogics} respectively for a more 
in-depth discussion. 

Notice our \emph{explicit} use of John McCarthy's notation for the 
programming language LISP,
\cite{mcCarthyAbrahamsEdwardsHartLevin1965lispManual}. 

Since the definitions \in[beastieActions], \in[beastieTests], and 
\in[beastieIdentity] are co-dependent, we postpone any further commentary 
until Section \in[beastieReality]. 

\stopMMundi

\section[title=What \emph{can} a \quote{wee beastie} know?]

\startDefinition[beastieTests]

Again, we assume that any wee beastie exists in a 
\quote{collection} of \lols. With these \lols, a wee beastie 
can preform (\emph{compute}) the following tests:

\startitemize[n]

\item Atomic Tests

\startitemize[n]

\item The test which always succeeds, denoted \quote{\true} or 
\quote{\top}. 

\item The test which always fails, denoted \quote{\false} or 
\quote{\bottom}. 

\item The test which succeeds if the \quote{top} \lol\ is 
\quote{\type{nil}}, denoted \quote{\isNil}.

\stopitemize

\item Static Test Combinators

\startitemize[n]

\item The test which succeeds if given pair of other tests both succeed. 
If $s$ and $t$ are two tests, then the \quote{conjunction} test is denoted 
\quote{$s \land t$}.

\item The test which succeeds if one or more of a given pair of other 
tests succeeds. If $s$ and $t$ are two tests, the \quote{disjunction} test 
is denoted \quote{$s \lor t$}.

\item The test which succeeds if a given other test fails. If $t$ is a 
test, then the \quote{negation} test is denoted \quote{$\lnot\; t$}. 

\stopitemize

\item Dynamic Test Combinators

\startitemize[n]

\item The test which succeeds if a given other test succeeds after all 
possible computations of a given action. If $t$ is a test and $a$ is an 
action, then the \quote{necessity} test is denoted \quote{$[a]t$}. 

\item The test which succeeds if a given other test succeeds after at 
least one possible computation of a given action. If $t$ is a test and $a$ 
is an action, then the \quote{possibility} test is denoted 
\quote{$\langle a\rangle t$}. 

\item The test which succeeds if \TODO{complete the description of the 
$\mu$ operator}.

\item The test which succeeds if \TODO{complete the description of the 
$\nu$ operator}. 

\stopitemize

\stopitemize

If any of the above tests does \emph{not} succeed, then that test fails. 
This is \emph{all} that any wee beastie (or Idealized Mathematician) can 
know by testing \lols\ in its environment. 

\stopDefinition

\startMMundi

The atomic tests provide the \emph{only} tests that a wee beastie can 
perform \emph{\quote{directly}} on its environment. All other tests are 
formed via various test combinators from these three atomic tests, and, 
via the \quote{necessity} and \quote{posibility} tests, any actions 
defined above. Of these three atomic tests, only the third atomic test 
increases the wee beastie's knowledge of its environment. Neither of the 
first two atomic tests reference the \quote{state} of the wee beastie's 
environment. 

The three static test combinators correspond to the core of classical 
(static) propositional logic. Note that we do \emph{not} assume the 
classical \quote{principle of the excluded middle}, hence we must define 
both the \quote{logical} \quote{conjunction} (\quote{and}) as well as the 
\quote{logical} \quote{disjunction} (\quote{or}) operators. It will turn 
out that while the principle of the excluded middle \emph{is} valid for 
any finitely defined collection of (non-)well-founded \lols, it need 
\emph{not} be valid for \quote{asymptotically} defined collection 
(non-)well-founded \lols. 

Classical logic, including propositional, first or higher order, concerns 
itself with \quote{properties} of a single \emph{fixed} \quote{unchanging} 
\quote{world}. Since there is no change, there is no corresponding sense 
of \quote{time}. Wee beasties, such as ourselves, experience a 
\quote{world} of constant change. While we will not (yet) explicitly 
define time, as wee beasties, we are naturally interested in what will 
happen to our world if we change it. The dynamic test combinators capture 
how given tests will change if a wee beastie changes its environment by 
first performing one of its possible actions on its environment. 

As mentioned above the pair of definitions, \in[beastieActions] and 
\in[beastieTests], are essentially Dynamic Logic, 
\cite{harelKozenTiuryn2000dynamicLogic}, augmented with the $\mu$ and 
$\nu$ operators taken from $\mu$-modal logic, 
\cite{demriGorankoLange2016temporalLogics}. 

\stopMMundi 

\section[title=Identity and Bisimulation]

When can a wee beastie know that two \lols, in its environment, are the 
\emph{same} \lol? Given the capabilities listed above, a wee beastie can 
never know if two \lols\ are \emph{identical}, the \emph{same} or 
\emph{equal}. The best a wee beastie can do is to compare a pair of \lol's 
using the same collection of \emph{tests}. 

\TODO{\emph{If} we have a proof of the well-foundedness of a given 
collection of tests, \emph{then} we can talk about the identity of a given 
well-founded object. The critical thing here is that since everything is 
well-founded, we know that the computation of identity will terminate with 
either an affermation or a denial of the claim of identity.} 

\startDefinition[beastieIdentity]

Two \lols\ are \emph{Bisimilar} if they pass the same collection of tests.

\stopDefinition

\startMMundi This concept of Bisimulation has been identified by a number 
of Mathematicians and Computer Scientists. From the theory of 
non-well-founded sets, see \cite{aczel1988nonWellFoundedSets}. From the 
theory of co-algebras, see Chapter 3 of \cite{jacobs2017coalgebras}. 
\stopMMundi 

Finally, linking the above three definitions together we make the 
following assertion about any wee beastie and its environment of \lols: 

\begingroup\startAssertion

\startitemize[n]

\item \quote{\type{car}} of the \quote{\type{nil}} \lol\ is bisimular to 
\quote{\type{nil}}. \TODO{Is this correct?}

\item \quote{\type{cdr}} of the \quote{\type{nil}} \lol\ is bisimular to 
\quote{\type{nil}}. \TODO{Is this correct?}

\item \quote{\type{car}} of the \quote{\type{cons}} of two \lols\ is 
bisimular to the first \lol. 

\item \quote{\type{cdr}} of the \quote{\type{cons}} of two \lols\ is 
bisimular to the second \lol. 

\item \quote{\type{cons}} of the pair of \lols\ obtained by 
\quote{\type{car}} and \quote{\type{cdr}} of the same \lol, is bisimular 
to the original \lol. 

\stopAssertion\endgroup

As a result of these assertions, the only \quote{meaningful} tests are 
those which consist of collections of \quote{\type{car}}s and 
\quote{\type{cdr}}s. Assertions 3 and 4 imply that any 
\quote{\type{cons}}s can be undone by appropriate use of 
\quote{\type{car}}s and \quote{\type{cdr}}s. \TODO{Is this a meaningful 
observation?} 

\startChapterAppendices

\chapterAppendix[title={Texts: transcribing Lists of Lists}]

To be useful, any \joylol\ program needs to be written down in a format 
which can be stored over time and transmitted through space. The 
traditional way to do this is to transcribe a \joylol\ program to and from 
a \quote{string of characters}. This chapter appendix specifies how this 
is done. 

While it might seem strange to worry about texts and transcription of 
textual artefacts in a \emph{mathematical} document, any \emph{formal} 
foundation of mathematics does start by listing the collection of allowed 
symbols, see for example \cite{kleene2009introMetaMathematics}. In 
particular, we \emph{could} develop the whole of mathematics using just 
two symbols: \quote{(}, and \quote{)}. However, to be kind to \quote{bears 
of very little brain} (like myself), we will use, at least initially, a 
richer collection of symbols. In line with the typical practice of 
Computer Science, we parse tree structures of symbols from texts as 
sequences of characters. This enables the collection of meaningful symbols 
to expand as needed for a particular discussion. 

\subChapterAppendix[title=Characters, reference=characterSpec]

A character is a fundamental representation of a glyph of one or other 
human language. \quote{Printed} characters are currently used by humans, 
to communicate ideas between themselves. For our purposes, a character is 
an indivisible entity. The collection of sets have an associated order 
relationship. So we know when a given character is ordered before, 
equal-to, or ordered after another given character. 

In current computational systems, characters are usually implemented using 
the Unicode consortium's \quote{UTF-8} specifications. 

Characters are often sub-divided into specific \quote{useful} 
sub-collections. For our purposes, we will subdivide our character set 
into \quote{white space}, \quote{structural} and \quote{all other} 
characters. 

\subChapterAppendix[title=Symbols, reference=symbolSpec]

A symbol is a sequence of characters which does not include any 
white-space or structural characters. Symbols can be lexicographically 
ordered using the order of the characters. 

\subChapterAppendix[title=Texts, reference=textSpec]

A text is a list or sequence of characters.

\subChapterAppendix[title=Parsing, reference=parsingSpec]

Parsing is the process of extracting a \quote{parse tree} of symbols from 
a text. 

\chapterAppendix[title=Containers]

In our specification of Wee Beasties we assumed a number of specifications 
of the containers, Lists, Stacks and Dictionaries. This chapter appendix 
provides the full specifications for each of these concepts. 

\subChapterAppendix[title=Lists, reference=listSpec]

A list is a container which contains a sequence of \lols.

\subChapterAppendix[title=Stacks, reference=stackSpec]

A stack is a container which only allows access via the push and pop 
methods. A typical implementation would be using a list. If a dictionary 
is known to be well-founded, then it is easy to specify that the 
dictionary has exactly one instance of any key, value pair. If a 
dictionary is potentially non-well-founded, it is impossible to implement 
this uniqueness constraint. This means that the concept of 
\quote{dictionary} has both well-founded, dictionary, as well as a 
non-well-founded, multi-dictioanry, variants. 

\subChapterAppendix[title=Dictionaries, reference=dictionarySpec]

A dictionary is a container which implements a key to value mapping. It 
allows you to find a value given its key, or to add a key, value pair. 
Similar to a dictionary, if a set is known to be well-founded, then it is 
easy to specify that the set has exactly one instance of any element. 
However, if a set is potentially non-well-founded, it is impossible to 
implement this uniqueness constraint. This means that the concept of 
\quote{set} has both well-founded, set, as well as a non-well-founded, 
multi-set, variants. Similarly, we have well-founded powersets as well as 
potentially non-well-founded multi-powersets respectively. 

\subChapterAppendix[title={Sets and Power Sets}, reference=setSpec]

\stopChapterAppendices
