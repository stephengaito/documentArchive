% A ConTeXt document [master document: hilbertsProgram.tex]

\chapter[title=Introduction]

\section[title=Goals]

Every long term researcher should have a silly question, if only to keep 
them focused upon whole problems amidst all of the unending detail. Your 
ultimate destination, whether or not you get there, influences how you 
plan to get there. The problem for any researcher, is that the research 
\quote{space} is infinite dimensional. From the dazzle of choices, you 
must make \quote{one} sequence of choices which, one hopes, ultimately 
leads to your answer(s). The vantage point provided by your ultimate goal, 
influences the questions asked and hence the answers found. The choice of 
where you want to get to, does have a profound influence on how you choose 
to get there. 

My silly question is:

\startalignment[center] Why do babies babble?\stopalignment

\blank[big]

There are (at least) two aspects to this question:

\startitemize[n]

\item How do organic naive \quote{brains} build models of reality. Too 
much of classical artificial intelligence focuses on the incremental 
learning based upon the capabilities of \quote{adult} learners. However 
this misses the point of learning models of Reality \emph{ab initio}. 

\item Equally important, is the question of what constitutes an efficient 
model. Animal brains must keenly balance energy use with the 
comprehensiveness of a given model. Any animal that gets this wrong too 
much of the time, becomes someone else's lunch. 

\stopitemize 

So put simply, my objective, as a \emph{mathematician}, is to build 
efficient and mathematically rigorous models Reality. However, if you are 
going to build a mathematically rigorous theory of Reality, you must first 
provide a mathematically rigorous theory of Mathematics itself. This 
document is devoted to providing just such a rigorous foundation for 
Mathematics. 

Since the ancient Greeks, western influenced Philosophical, Scientific, 
Mathematical practice has been to equate rigour with proofs of 
\quote{Truth}. Euclid's \quote{Elements} has, for just over two millennia, 
provided the pre-eminent example of this paradigm of rigorous proof. 
However, G\"odel's two Incompleteness theorems, show that classical proofs 
through logic are unable to provide the rigorous foundations we require. 
This is the failure of Hilbert's \emph{logical} Program. 

Essentially, G\"odel's work around the early 1930's represent the first 
contributions to the theory of computation. G\"odel's theorems from this 
period concern themselves with our ability to compute \quote{Truth}, and 
the fact that such computation of \quote{Truth}, is only partially 
recursive, rather than totally recursive.

Instead of computing \quote{Truth}, is there a more useful computation 
which might provide a foundation of Mathematics? By exploring Hilbert's 
Program within a \emph{computational} framework, the objective of this 
document is to show that the answer to this question is yes. 

\component introPhilosophy

\component introMathematics

\section[title=Strategy]

This document will provide a rigorous \emph{computational} foundation of 
Mathematics, by... 

\TODO{Need to talk about lists of lists}

We will do this in a number of distinct steps. Firstly, by defining a 
computational langauge, JoyLoL (\quote{The Joy of Lists of 
Lists}\footnote{Or is it \quote{The Joy of Laughing out Loud}?}). JoyLoL 
is a functional \emph{concatenative} language based upon Manfred von 
Thun's language Joy, \cite{vonThun1994overview}. The critically important 
aspect of JoyLoL is that it is constructed to be a fixed point of the 
semantics functor. This means that JoyLoL provides its own denotation, 
operational and axiomatic semantics. JoyLoL does not rely upon any other 
\quote{pre-existing} structures or set theory to define its meaning. An 
other important aspect of JoyLoL is that it is a \emph{concatenative} 
function language. Almost all other functional programming languages are 
based upon Church's $\lambda$-calculus, importantly, this means that most 
such langauges are focused upon function evaluation and substitution. From 
a categorical point of view, this means that the collection of 
computational traces forms a Topos. Being \emph{concatenative}, the 
collection of JoyLoL computational traces forms a Category, which also 
happens to be a Topos. The distinction here is important. The requirements 
of being a Category are much simpler and valid of many more distinct 
sub-collections of computational traces. 

, we can construct the structure of all JoyLoL computational traces. We 
can define the collection of \emph{finite} substructures as those 
substructures for which a simple \emph{short} JoyLoL program \emph{halts}. 
These finite substructures 

Since there \emph{are} JoyLoL computations which do not halt, this 
structure 

