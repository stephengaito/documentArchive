% A ConTeXt document [master document: hilbertsProgram.tex]

\chapter[title=Introduction]

\section[title=Goals]

Every long term researcher should have a silly question, if only to keep 
them focused upon whole problems amidst all of the unending detail. My 
silly question is:

\startalignment[center] Why do babies babble?\stopalignment

\blank[big]

There are (at least) two aspects to this question:

\startitemize[n]

\item How do organic naive \quote{brains} build models of reality. Too 
much of classical artificial intelligence focuses on the incremental 
learning based upon the capabilities of \quote{adult} learners. However 
this misses the point of learning models of Reality \emph{ab initio}. 

\item Equally important, is the question of what constitutes an efficient 
model. Animal brains must keenly balance energy use with the 
comprehensiveness of a given model. Any animal that gets this wrong too 
much of the time, becomes someone else's lunch. 

\stopitemize 

So put simply, my objective, as a \emph{mathematician}, is to build 
efficient and mathematically rigorous models or theories of Reality. 
However, if you are going to build a mathematically rigorous theory of 
Reality, you must first provide a mathematically rigorous theory of 
Mathematics itself. This document is devoted to providing just such a 
rigorous foundation for Mathematics. 

Since the ancient Greeks, western influenced Philosophical, Scientific, 
Mathematical practice has been to equate rigour with proofs of 
\quote{Truth}. Euclid's \quote{Elements} has, for just over two millennia, 
provided the pre-eminent example of this paradigm of rigorous proof. 
However, G\"odel's two Incompleteness theorems, show that classical logic 
is unable to provide the rigorous foundations we require. This is the 
failure of Hilbert's \emph{logical} Program. 

Essentially, G\"odel's work around the early 1930's represent the first 
contributions to the theory of computation. G\"odel's theorems from this 
period concern themselves with our ability to compute \quote{Truth}, and 
the fact that such computation of \quote{Truth}, is only partially 
recursive, rather than totally recursive.

Instead of computing \quote{Truth}, is there a more useful computation 
which might provide a foundation of Mathematics? The objective of this 
document is to show that the answer to this question is yes by exploring 
Hilbert's Program within a \emph{computational} framework. 

\section[title=Some philosophy]

In Western philosophy, since at least the time of the ancient Greeks, 
there have been a wide range of Philosophical theories of the 
\quote{Reality} of \quote{Reality}. Our objective in building a rigorous 
Mathematical theory of Reality is not to prove any of these Philosophical 
theories (in)correct. Instead our objective, and really the only one 
available \emph{Mathematically}, is to explore what a finite computational 
device can learn about Reality. 

Ignoring Hume's \quote{Problem of Induction} for the moment, as a 
Scientist and Engineer, like any young child, I live in the belief that I 
can both learn about and, more importantly, \emph{interact} with Reality. 
To bastardize Descartes, \emph{from moment to moment, I can make marks in 
the sand, therefore I am}.

It is naive to assert that finite beings, such as ourselves, can not learn 
to predict at least some of the future. Russell's farmyard birds, given 
their limited cognitive abilities, \emph{are rational} to expect to be 
feed daily\footnote{\TODO{See \cite{Russell's The Problem of 
Philosophy}}}. However, for any \emph{finite} being, there will always be 
events, some highly critical events, which are outside of that being's 
ability to know about and hence predict. Understanding these limits of 
being finite, is the true import of Hume's Problem. 

We \emph{can} build (finite) computational devices. I am writing this 
document using one such device, you are no doubt reading this document 
using at least one other. So it is reasonable to expect \emph{finitist} 
Mathematics to be founded computationally. \TODO{Talk about 
meta-mathematics} Hilbert's primary objective was to provide a rigorous 
foundation for Analysis whose metamathematics is finitist. 

The classical Reals, which are required for classical Analysis, are 
uncountable. Once we have defined the Ordinals, we will find that there is 
a reasonable definition of $\lambda$-computation for each ordinal, 
$\lambda$. \emph{If} we assume a computational equivalent to the 
\quote{standard} Axiom of Choice, then we can define the transfinite 
ordinals and hence computational structures which can interpret the 
uncountable collection of the Reals (as \emph{data}). Using this 
structure, we will then be able to develop the classical theory of 
Analysis, which was Hilbert's ultimate goal. 

Alternatively, we can define the Reals as (measurement) \emph{processes}, 
with out requiring any \emph{transfinite} ordinals. The resulting theory 
of Analysis will not be quite classical, since the \emph{internal} logic 
of the collection of processes, is not classical. However, I conjecture 
that this process logic provides an explanation of the \quote{strangeness} 
of, and hence the correct foundations for, Quantum Mechanics. 

\section[title=Strategy]

This document will provide a rigorous \emph{computational} foundation of 
Mathematics, by... 

We will do this in a number of distinct steps. Firstly, by defining a 
computational langauge, JoyLoL (\quote{The Joy of Lists of 
Lists}\footnote{Or is it \quote{The Joy of Laughing out Loud}?}). JoyLoL 
is a functional \emph{concatenative} language based upon Manfred von 
Thun's language Joy, \cite{vonThun1994overview}. The critically important 
aspect of JoyLoL is that it is constructed to be a fixed point of the 
semantics functor. This means that JoyLoL provides its own denotation, 
operational and axiomatic semantics. JoyLoL does not rely upon any other 
\quote{pre-existing} structures or set theory to define its meaning. An 
other important aspect of JoyLoL is that it is a \emph{concatenative} 
function language. Almost all other functional programming languages are 
based upon Church's $\lambda$-calculus, importantly, this means that most 
such langauges are focused upon function evaluation and substitution. From 
a categorical point of view, this means that the collection of 
computational traces forms a Topos. Being \emph{concatenative}, the 
collection of JoyLoL computational traces forms a Category, which also 
happens to be a Topos. The distinction here is important. The requirements 
of being a Category are much simpler and valid of many more distinct 
sub-collections of computational traces. 

, we can construct the structure of all JoyLoL computational traces. We 
can define the collection of \emph{finite} substructures as those 
substructures for which a simple \emph{short} JoyLoL program \emph{halts}. 
These finite substructures 

Since there \emph{are} JoyLoL computations which do not halt, this 
structure 

