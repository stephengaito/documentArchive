% A ConTeXt document [master document: hilbertsProgram.tex]

\chapter[title=Introduction]

Every long term researcher should have a silly question, if only to keep 
them focused upon whole problems amidst all of the unending detail. My 
silly question is:

\startalignment[center] Why do babies babble?\stopalignment

\vspace[eX]

There are (at least) two aspects to this question:

\startitemize[n]

\item How do organic naive \quote{brains} build models of reality. Too 
much of classical artificial intelligence focuses on the incremental 
learning based upon the capabilities of \quote{adult} learners. However 
this misses the point of learning models of Reality \emph{ab initio}. 

\item Equally important, is the question of what constitutes an efficient 
model. Animal brains must keenly balance energy use with the 
comprehensiveness of a given model. Any animal that gets this wrong too 
much of the time, becomes someone else's lunch. 

\stopitemize 

So put simply, my objective, as a \emph{mathematician}, is to build 
efficient and mathematically rigorous models or theories of Reality. 
However, if you are going to build a mathematically rigorous theory of 
Reality, you must first provide a mathematically rigorous theory of 
Mathematics itself. This document is devoted to providing just such a 
rigorous foundation for Mathematics. 

Since the Greeks, western influenced Philosophical, Scientific, 
Mathematical practice has been to equate rigour with proofs of 
\quote{Truth}. Euclid's \quote{Elements} has, for just over two millennia, 
provided the pre-eminent example of this paradigm of rigorous proof. 
However, G\"odel's two Incompleteness theorems, show that classical logic 
is unable to provide the rigorous foundations we require. This is the 
failure of Hilbert's \emph{logical} Program. 



