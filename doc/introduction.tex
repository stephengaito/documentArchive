% A ConTeXt document [master document: hilbertsProgram.tex]

\chapter[title=Introduction]

\section[title=Goals]

Every long term researcher should have a silly question, if only to keep 
them focused upon whole problems amidst all of the unending detail. Your 
ultimate destination, whether or not you get there, influences how you 
plan to get there. The problem for any researcher, is that the research 
\quote{space} is infinite dimensional. From the dazzle of choices, you 
must make \quote{one} sequence of choices which, one hopes, ultimately 
leads to your answer(s). The vantage point provided by your ultimate goal, 
influences the questions asked and hence the answers found. The choice of 
where you want to get to, does have a profound influence on how you choose 
to get there. 

My silly question is:

\startalignment[center] Why do babies babble?\stopalignment

\blank[big]

There are (at least) two aspects to this question:

\startitemize[n]

\item How do organic naive \quote{brains} build models of reality. Too 
much of classical artificial intelligence focuses on the incremental 
learning based upon the capabilities of \quote{adult} learners. However 
this misses the point of learning models of Reality \emph{ab initio}. 

\item Equally important, is the question of what constitutes an efficient 
model. Animal brains must keenly balance energy use with the 
comprehensiveness of a given model. Any animal that gets this wrong too 
much of the time, becomes someone else's lunch. 

\stopitemize 

So put simply, my objective, as a \emph{mathematician}, is to build 
efficient and mathematically rigorous models Reality. However, if you are 
going to build a mathematically rigorous theory of Reality, you must first 
provide a mathematically rigorous theory of Mathematics itself. This 
document is devoted to providing just such a rigorous foundation for 
Mathematics. 

Since the ancient Greeks, western influenced Philosophical, Scientific, 
Mathematical practice has been to equate rigour with proofs of 
\quote{Truth}. Euclid's \quote{Elements} has, for just over two millennia, 
provided the pre-eminent example of this paradigm of rigorous proof. 
However, G\"odel's two Incompleteness theorems, show that classical proofs 
through logic are unable to provide the rigorous foundations we require. 
This is the failure of Hilbert's \emph{logical} Program. 

Essentially, G\"odel's work around the early 1930's represent the first 
contributions to the theory of computation. G\"odel's theorems from this 
period concern themselves with our ability to compute \quote{Truth}, and 
the fact that such computation of \quote{Truth}, is only partially 
recursive, rather than totally recursive.

Instead of computing \quote{Truth}, is there a more useful computation 
which might provide a foundation of Mathematics? By exploring Hilbert's 
Program within a \emph{computational} framework, the objective of this 
document is to show that the answer to this question is yes. 

\section[title=Some philosophy]

In Western philosophy, since at least the time of the ancient Greeks, 
there have been a wide range of Philosophical theories of the 
\quote{Reality} of \quote{Reality}. Our objective in building a rigorous 
Mathematical theory of Reality is not to prove any of these Philosophical 
theories (in)correct. Instead our objective, and really the only one 
available \emph{Mathematically}, is to explore what a finite computational 
device can learn about Reality. 

\subsection[title=Hume's problem]

Ignoring Hume's \quote{Problem of Induction} for the moment, as a 
Scientist and Engineer, like any young child, I live in the belief that I 
can both learn about and, more importantly, \emph{interact} with Reality. 
To bastardize Descartes, \emph{from moment to moment, I can see that I 
have made marks in the sand, therefore I am}. 

It is naive to assert that finite beings, such as ourselves, can not learn 
to predict at least some of the future. Russell's farmyard birds, given 
their limited cognitive abilities, \emph{are rational} to expect to be 
feed daily\footnote{See chapter VI, \quote{On Induction}, in Russell's 
\quote{The Problems of Philosophy}, \cite{russell1912problemsOfPhilosophy} 
near page 98}. However, for any \emph{finite} being, there will always be 
events, some highly critical events, which are outside of that being's 
ability to know about and hence predict. Understanding these limits of 
being finite, is the true import of Hume's Problem. 

It is equally naive to assert that a finite being can not interact with 
their environment. I \emph{can} communicate with you over both distance 
and time. We \emph{can} build (finite) computational devices. I am writing 
this document using one such device, you are no doubt reading this 
document using at least one other. So it is at least potentially 
reasonable to expect that \emph{finitist} Mathematics could be founded 
computationally. 

\TODO{What does it mean to \emph{exist} mathematically?}

\subsection[title=What is a thing?]

\blank[big]\startblockquote

From the range of the basic questions of metaphysics we shall here ask 
this one question: “What is a thing?” The question is quite old. What 
remains ever new about it is merely that it must be asked again and 
again.\footnote{Martin Heidegger, page 1, first paragraph, in 
\cite{heidegger1967whatIsAThing}, as quoted by 
\cite{doeringIsham2008thingTheoryFoundationsPhysics}.} 

\stopblockquote\blank[big]

A Zen Master would respond that there is \emph{no-thing}, there is only 
\emph{is-isness}, \emph{existence in its entirety}\footnote{Indeed a Zen 
Master would refuse to use mere words. To slightly mix philosophies, 
\quote{The Tao which can be named is not the Tao}, or again in Jewish 
tradition, God is not to be directly \emph{named}. Words differentiate 
\quote{things}, and Zen's \quote{existence-in-entirety}, the Tao and God 
are beyond all human limits to differentiate, identify or understand. We, 
as limited beings, can only experience. This is similar to Cantor's 
expressed understanding of his Absolute Infinite Magnitudes, again, see 
Cantor's letter to Dedekind dated 1899, 
\cite{vanHeijenoort1967fregeToGodel}.}. That quarrelsome \quote{thing}, 
\emph{I}, is only an illusion. The hardest thing any \quote{one} can do is 
to ignore the \emph{I} in order to \quote{see} the \emph{is}. The 
dissolution of this \emph{I} is un-important in the context of the 
\emph{is}. In our work, this point of view will be indispensable. 

However, for most of \quote{us}, such a view point is very hard to hold. 
We all play a \quote{me}-\quote{environment} game with existence. This 
view point is equally important for our work.

A \quote{quark} plays the absolute simplest of games, a 
\quote{quark}-\quote{everything-else} game. A \quote{quark}, re-acts to 
its environment. Any model of a \quote{quark} is a (fairly) simple 
S-Matrix. 

A frog's game is only slightly less simple, there is the frog, there are 
\quote{things} that are small enough to be potential food, there are 
\quote{things} that are so large they might be predators, and finally 
there are \quote{things} which might be potential mates. A frog's 
\quote{environment} has some substructure, \quote{prey}, 
\quote{predators}, and \quote{mates}. We assume that a frog's brain 
models, to a sufficiently complex level of detail, these three 
\quote{things}, however, by and large, frogs do not need to expend much 
more energy on playing any more complex games so they don't. By modelling 
only the most important categories, frogs can save energy by not building 
and maintaining complex and energetically expensive nervous 
systems\footnote{See, for example, Ewert's \emph{Motion Perception Shapes 
the Visual World of Amphibians}, 
\cite{ewert2004motionPerceptionAmphibians}. While this reference focuses 
primarily on the visual system of amphibians, it indicates an overall lack 
of need for complex models in an amphibian's nervous system. It is 
estimated, \cite{raichleGusnard2002brainEnergyBudget}, that the adult 
human brain consumes around 20\% of all calories consumed each day, yet 
the human brain only represents around 2\% of our body weight. Nervous 
systems \emph{are} relatively expensive to keep running. For most of a 
frog's needs, this additional complexity is not needed, this is a frog's 
evolutionary niche.} 

A human's game is \emph{much} more complex. We regularly, split 
\quote{our} environment into many many \quote{things}. \quote{Objects} for 
which we build wide classes of models of their behaviour and even their 
potential internal, \quote{intentional}, state of \quote{mind}. Chairs and 
mugs have different uses. Metals and glasses, have different abilities to 
be re-fashioned into useful tools. Animals have widely different 
behaviours providing useful companions or dangerous enemies. Even more 
complex, though, are our \quote{models} of other humans. Each person in 
our environment, has widely differing objectives of their \quote{own}. All 
of which we must, and, by and large, do, keep track of. For each of these 
objects we take an \emph{intentional stance}\footnote{See Daniel Dennett's 
\quote{The Intentional Stance}, \cite{dennett1987a}.}, they each have 
various, though widely, differing abilities to re-act, or intend with 
\quote{me}. These different intentional abilities are reflected in the 
overall complexity in the various models we build to represent any 
particular \quote{thing}. 

So how do we, most efficiently, build these models? Naively, we all 
\quote{know} what an object \quote{is} when we \quote{see} it. Physics 
suggests that all material things are made up of sub-atomic particles 
which are in turn made up of \quote{quarks}. It is \quote{obvious} at one 
level that \quote{I} and \quote{you} are \quote{different} things. 
However, when I shake your hand, where do \quote{my} \quote{quarks} end 
and \quote{your} \quote{quarks} begin? At the \quote{most basic} level, we 
can not separate one \quote{thing} from another \quote{thing}. A Zen 
master's view is actually deeply entwined with any complete mathematical 
model of \quote{things}. How do we reconcile these multiple levels of 
\quote{being} into one comprehensive and complete model of 
\quote{Reality}? 

Andreas D\"oring and Chris Isham in their paper, \emph{What is a thing?}, 
\cite{doeringIsham2008thingTheoryFoundationsPhysics}, suggest that Physics 
can best be captured using the variable sets point of view provided by 
Categorical Topos\footnote{See Lawvere's concept of variable sets, 
\cite{lawvere1975continuouslyVariableSets} or 
\cite{lawvereRosebrugh2003setsForMathematics}.}. For us, a Topos over a 
collection of descriptive \quote{levels}, provides the natural tool with 
which to capture the coherent variation of \quote{thing-ness} as our level 
of description varies. This suggests that the use of the structuralist 
Categorical point of view, in general, and the associated variable 
descriptive Topos point of view, in particular, will be very important to 
our work. 

\subsection[title=A Neuron's eye view]

So lets reflect for a moment on a neuron's point of view. When a neuron 
fires, it \quote{means} something, but what does it mean? Deeply embedded 
inside a complex collection of other neurons, each collecting the spike 
trains from countless other neurons. In some sense each neuron integrates 
the \quote{information} conveyed by each of these spike trains from 
up-stream neurons. What sort of information should these spike trains be 
communicating? At the very least they should be communicating some 
important value. This is what artificial neural networks model. 

\TODO{Bayesian brain, \cite{doyaIshiiPougetRao2007bayesianBrain}, Spikes, 
\cite{riekeWarlandDeRuyterVanSteveninck1999spikesNeuralCode} Pouget, 
\cite{beckPouget2007inferencesImplementationMarkov} and 
\cite{knillPouget2004bayesianUncertaintyComputation}, markov models, 
Shalizi spatial models, \cite{shalizi2001thesis}, section 10.2.1 \emph{Why 
Global States Are not Enough}, required. Markov has restrictive scope of 
description... but by recoding and using levels of description, we can 
recover the effect of history on the present and future.} 

\TODO{talk about Wally's oversight... \cite{walley1991impreciseProb}. 
There is a deep distinction between the ideal asymptotic reals and the 
finite subsequences of processes} 

\TODO{rework the following paragraphs}

So this paper generalizes the collected work of Spitters, Coquand, (see, 
for example, \cite{coquandSpitters2009integralsAndValuations})\footnote{It 
is also important to see the related work of Heunen, Landsman and 
Spitters, \cite{heunenLandsmanEtAl2009toposForAlgebraicQuantumTheory}}, 
together with Walley's work on imprecise probabilities, to produce a 
computable measure theory which a beastie could use. \TODO{Add the work of 
\cite{jackson2006phdThMeasureThSheaves} and its generalization in Isham 
\cite{doeringIsham2008thingTheoryFoundationsPhysics}, Section 8.2} 

As has become traditional in Imprecise Probability theory, in his book, 
\emph{Statistical Reasoning with Imprecise Probabilities}, 
\cite{walley1991impreciseProb}, Walley makes the upper and lower 
previsions (expectations) the primary objects of study with upper and 
lower probabilities as derived concepts. Since we will be concerned with 
probability based Markov structures we will reverse this orientation. One 
of the reasons Walley choose to work with previsions (expectations) 
instead of probabilities is because of his belief that Lower Probablities 
did not determine Lower Previsions (see section 2.7.3 page 82, 
\cite{walley1991impreciseProb}). In fact we will show below that with the 
correct definition of upper and lower measures and upper and lower 
integrals, lower probabilities do determine lower previsions. This will be 
the substance of the (Imprecise) Dedekind-Riesz Representation Theorms 
proven below. 

\TODO{paragraphs above}

\subsection[title=Hilbert's program]

\TODO{Talk about meta-mathematics} 

Hilbert's primary objective was to provide a rigorous 
foundation for Analysis whose metamathematics is finitist. 

\TODO{discuss the concept of data versus processes. classical computation 
theory has been mostly focused upon data not proceses.} 

The classical Reals, which are required for classical Analysis, are 
uncountable. Once we have defined the Ordinals, we will find that there is 
a reasonable definition of $\lambda$-computation for each ordinal, 
$\lambda$. \emph{If} we assume a computational equivalent to the 
\quote{standard} Axiom of Choice, then we can define the transfinite 
ordinals and hence computational structures which can interpret the 
uncountable collection of the Reals (as \emph{data}). Using this 
structure, we will then be able to develop the classical theory of 
Analysis, which was Hilbert's ultimate goal. 

Alternatively, we can define the Reals as (measurement) \emph{processes}, 
with out requiring any \emph{transfinite} ordinals. The resulting theory 
of Analysis will not be quite classical, since the \emph{internal} logic 
of the collection of processes, is not classical. However, I conjecture 
that this process logic provides an explanation of the \quote{strangeness} 
of, and hence the correct foundations for, Quantum Mechanics. 

\TODO{Need to introduce collection of processes as a (co)algebraic 
collection/structure.... we distinguish different processes by observing 
them...} 
\TODO{For this work the focus upon the well-founded/data/algebraic versus 
the non-well-founded/process/co-algebraic is all pervasive. } 

\TODO{Categorical thought == structuralist point of view. Quote 
\cite{awodey2009a}.} 
\TODO{discuss reals as data versus reals as processes == imprecise Reals.} 



\section[title=Strategy]

This document will provide a rigorous \emph{computational} foundation of 
Mathematics, by... 

\TODO{Need to talk about lists of lists}

We will do this in a number of distinct steps. Firstly, by defining a 
computational langauge, JoyLoL (\quote{The Joy of Lists of 
Lists}\footnote{Or is it \quote{The Joy of Laughing out Loud}?}). JoyLoL 
is a functional \emph{concatenative} language based upon Manfred von 
Thun's language Joy, \cite{vonThun1994overview}. The critically important 
aspect of JoyLoL is that it is constructed to be a fixed point of the 
semantics functor. This means that JoyLoL provides its own denotation, 
operational and axiomatic semantics. JoyLoL does not rely upon any other 
\quote{pre-existing} structures or set theory to define its meaning. An 
other important aspect of JoyLoL is that it is a \emph{concatenative} 
function language. Almost all other functional programming languages are 
based upon Church's $\lambda$-calculus, importantly, this means that most 
such langauges are focused upon function evaluation and substitution. From 
a categorical point of view, this means that the collection of 
computational traces forms a Topos. Being \emph{concatenative}, the 
collection of JoyLoL computational traces forms a Category, which also 
happens to be a Topos. The distinction here is important. The requirements 
of being a Category are much simpler and valid of many more distinct 
sub-collections of computational traces. 

, we can construct the structure of all JoyLoL computational traces. We 
can define the collection of \emph{finite} substructures as those 
substructures for which a simple \emph{short} JoyLoL program \emph{halts}. 
These finite substructures 

Since there \emph{are} JoyLoL computations which do not halt, this 
structure 

