% A ConTeXt document [master document: hilbertsProgram.tex]

\section[title=Some Computer Science and Engineering]

\subsection[title=JoyLoL]

Just as with (logical) set theory with and without \quote{ur-elements}, 
there are two distinct types of \joylol, with and without the \joylol\ 
equivalent of \quote{ur-elements}. In set theory \quote{ur-elements} are 
\quote{atomic} elements which might be elements of some set, but which 
contain no (sub)elements of their own. In the \joylol\ case, there are two 
types of \quote{ur-elements}, explicit textual \quote{symbols} as well as 
more performant implementations of various coalgebras.

The dialect of \joylol\ without \quote{ur-elements}, \joylolZero, is the 
purest form of \joylol. In \joylolZero, the \emph{only} allowed characters 
are \quote{(} and \quote{)}. In \joylolZero\ \emph{everything} is a list 
of lists. Unfortunately, for \quote{bears of little brain}, such as 
myself, this is a far too austere language in which to think about 
mathematics. Hence, to help \quote{bears of little brain} understand what 
they are doing, we allow \joylol\ to have atomic symbols. 

Since \joylol\ \emph{is} computational, to be useful, we need to embed 
\joylol\ \emph{implementations} into one or more \emph{modern} computation 
systems. Commonly used \joylol\ coalgebras, such as for example, the 
natural numbers could be implemented as \joylolZero\ list of list 
structures complete with associated (recursive) addition and 
multiplication. However, since the addition and multiplication of natural 
numbers is so pervasive in mathematics, a pure (compuational) 
implementation as \joylolZero\ lists of lists is unlikely to be performant 
enough for general use. This means we will generally need to allow more 
performant alternative implementations of many important coalgebras such 
as the natural numbers. 

This brings the question of how do we tell if two \quote{implementations} 
of a concept behave in the \quote{same} way?. The behaviour of a 
collection of \quote{things} are traditionally captured as a coalgebra. 
The most important thing about the definition of coalgebras is that 
\quote{identity} is captured via (behavioural) \quote{bisimulation}. Two 
coalebras behave the same if they are bisimular. Hence, two 
implementations behave the same if they are bisimular. While an 
implementation of the natural numbers in \joylolZero\ as lists of lists 
will be transparently \joylolZero\ code whose behaviour can be rigorously 
verified, typical implementations of the natural numbers in, for example 
ANSI-C, contain behaviours which are not as transparent. For example, both 
the \joylolZero as well as the GNU Multiple Precision Arithmetic (GMP) 
Library's implementation of the natural numbers on any particular computer 
system, \emph{will} have a concrete bound on the size of the largest 
natural number it can represent, if only because the computer system has 
run out of \emph{all} of the memory resources it may use. However, GMP's 
implementation will also have internal structures whose memory use will 
potentially impose additional behaviours which might not be explicitly 
documented in GMP's formal description. Given the complexity of GMP's 
implementation, GMP does not have a rigorous proof of correctness. Indeed 
the ANSI-C compilers used to compile \joylolZero\ and \joylol\ typically 
do not have rigorous proofs of correctness either. 

\TODO{Discuss how to deal with lack of rigor in implementations (See 
JoyLoL implementation paper(s)). Note that this is a problem even in 
Science. How do we know that a given model, models \quote{Reality}.} 

\TODO{Discuss the importance of being able to add new Coalgebra 
implementations using, for example, \joylol, ANSI-C or Lua.} 

\placefigure{A Software Architectural desciption of \joylol}
\bgroup\startMPcode
input hatching.mp;
picture dots; dots := dashpattern(on 0.1mm off 0.5mm);

draw (0cm,-10cm) -- (12cm,-10cm) -- (12cm,4cm) -- (0cm,4cm) -- cycle;

% JoyLoL

draw (0.1cm,0cm) -- (11.9cm, 0cm) -- (11.9cm, 3.9cm) -- (0.1cm, 3.9cm) -- cycle
  withpen pencircle scaled 1mm
  withcolor lightgray;

% JoyLol0
path joylolZero;
joylolZero := (0.1cm,0cm) -- (6cm, 0cm) --
  (6cm, -4cm) -- (0.1cm, -4cm) -- cycle;

draw image (
  hatchfill joylolZero
    withcolor (45,2mm,-.5bp);
) dashed dots withcolor darkgray;

draw joylolZero
  withpen pencircle scaled 1mm
  withcolor lightgray;

% ANSI-C implementation
path ansic;
ansic := (0.1cm,-4cm) -- (6cm, -4cm) --
     (6cm,0cm) -- (11.9cm, 0cm) -- (11.9cm, -7.9cm) -- (0.1cm, -7.9cm) -- cycle;

draw image (
  hatchfill ansic
    %withcolor (45,2mm,-.5bp)
    withcolor (-45,2mm,-.5bp);
) dashed dots withcolor darkgray;

draw ansic
  withpen pencircle scaled 1mm
  withcolor lightgray;

path operatingSystem;
operatingSystem := (0.1cm,-7.9cm) -- (11.9cm,-7.9cm) --
  (11.9cm,-9.9cm) -- (0.1cm,-9.9cm) -- cycle;
  
draw image (
  hatchfill operatingSystem
    withcolor (0,2mm,-.5bp);
) dashed dots withcolor darkgray;

draw operatingSystem
  withpen pencircle scaled 1mm
  withcolor lightgray;

% CoAlgs

draw (1.1cm,3cm) -- (1.9cm,3cm) -- (1.9cm,-2.5cm) -- (1.1cm,-2.5cm) -- cycle;
draw (2.1cm,3cm) -- (2.9cm,3cm) -- (2.9cm,-2.5cm) -- (2.1cm,-2.5cm) -- cycle;
draw (3.1cm,3cm) -- (3.9cm,3cm) -- (3.9cm,-2.5cm) -- (3.1cm,-2.5cm) -- cycle;
draw (4.1cm,3cm) -- (4.9cm,3cm) -- (4.9cm,-2.5cm) -- (4.1cm,-2.5cm) -- cycle;

draw (6.6cm,3cm) -- (7.4cm,3cm) -- (7.4cm,-8.5cm) -- (6.6cm,-8.5cm) -- cycle;
draw (7.6cm,3cm) -- (8.4cm,3cm) -- (8.4cm,-5.5cm) -- (7.6cm,-5.5cm) -- cycle;
draw (8.6cm,3cm) -- (9.4cm,3cm) -- (9.4cm,-8.5cm) -- (8.6cm,-8.5cm) -- cycle;
draw (9.6cm,3cm) -- (10.4cm,3cm) -- (10.4cm,-5.5cm) -- (9.6cm,-5.5cm) -- cycle;
draw (10.6cm,3cm) -- (11.4cm,3cm) -- (11.4cm,-8.5cm) -- (10.6cm,-8.5cm) -- cycle;

draw (1cm,-1.7cm) -- (5cm,-1.7cm) -- (5cm,-5cm) -- (1cm,-5cm) -- cycle;
draw (0.8cm,-2.1cm) -- (11.6cm,-2.1cm) -- (11.6cm,-3.1cm) -- (0.8cm,-3.1cm) -- cycle;
draw (0.5cm,-4.5cm) -- (11.6cm,-4.5cm) -- (11.6cm,-6cm) -- (0.5cm,-6cm) -- cycle;

% labels

label("\bold{JoyLoL}", (6cm,3.5cm));
label("Joylol interpreter", (6cm,-2.8cm));
label("CONS-Pairs", (3cm,-3.5cm));
label("Memory Management", (4cm,-5.5cm));
label("ANSI-C Implementation", (4cm,-6.7cm));
label("/ Lua wrapper", (4cm,-7.2cm));
label("Operating System / \LuaTeX", (6cm, -9cm));

label("$\bold{JoyLoL}_0$", origin) rotated 90 shifted (0.5cm,-2cm);
label("\bold{Characters}", origin) rotated 90 shifted (1.5cm, 0cm);
label("\bold{Symbols}", origin) rotated 90 shifted (2.5cm, 0cm);
label("\bold{Natural numbers}", origin) rotated 90 shifted (3.5cm, 0cm);
label("\bold{...}", origin) rotated 90 shifted (4.5cm, 0cm);

label("\bold{Characters} Alt", origin) rotated 90 shifted (7cm, 0cm);
label("(UTF-8 implementation)", origin) rotated 90 shifted (7cm, -6.5cm);
label("\bold{Symbols} Alt", origin) rotated 90 shifted (8cm, 0cm);
label("\bold{Natural numbers Alt}", origin) rotated 90 shifted (9cm, 0cm);
label("(GMP libraries)", origin) rotated 90 shifted (9cm, -6.5cm);
label("\bold{...}", origin) rotated 90 shifted (10cm, 0cm);
label("\bold{OS interfaces}", origin) rotated 90 shifted (11cm, 0cm);
label("(OS / \LuaTeX\ libraries)", origin) rotated 90 shifted (11cm, -6.5cm);
\stopMPcode\egroup

The GNU Multiple Precision Arithmetic Library (GMP) is a free library for 
arbitrary-precision arithmetic, operating on signed integers, rational 
numbers, and floating point numbers (Wikipedia). We only use the unsigned 
integers. 

UTF-8 is a variable width character encoding capable of encoding all 
1,112,064 valid code points in Unicode using one to four 8-bit bytes. The 
encoding is defined by the Unicode standard, and was originally designed 
by Ken Thompson and Rob Pike (Wikipedia). We (explicitly) only use the 
ASCII subset. 

