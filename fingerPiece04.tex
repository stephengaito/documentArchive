% Peano Arithmetic

In this section we count. We use a List of Lists form of Cantor's normal form,
\cite[Theorem 2.26]{jech2003setTheory}, to represent the Ordinals, \Ordinal{}.

\begin{bnf*}
 \bnfProd{Ordinal}{ ( \bnfSP \bnfKS{\bnfPN{OrdinalSumand}}{*} \bnfSP )}
 \bnfProd{OrdinalSumand}{ 
   ( \bnfSP 
     ( \bnfSP \bnfPN{Ordinal} \bnfSP ) \bnfSP
     ( \bnfSP \bnfPN{Natural} \bnfSP ) \bnfSP
   ) }
 \bnfProd{Natural}{ ( \bnfSP ) }
 \bnfAlt{ ( \bnfSP \bnfPN{Natural} \bnfSP ( \bnfSP \bnfPN{Natural} \bnfSP ) \bnfSP ) }
\end{bnf*}

\begin{racket}
;; Peano Arithmetic

(println "Hello from Peano Arithmetic!")

(define zero (lambda () ( list ) ) )

;;(define omega (lambda () 0))

(define isZero? (lambda (n) (null? n)))

;;(define isOmega? (lambda (n) (zero? n)))

(define suc (lambda (n) ( list* '() n) ) )

(define pred (lambda (n) (cdr n) ) )
\end{racket}