% A ConTeXt document [master document: joylolCoAlg.tex] 

\chapter[title=Rules]

\startTestSuite[rule environment]

% depends upon the 
%   \buff_pickup{name}{startsequence}{stopsequence}{before}{after}{undent}
% macro defined in buff-ini.mkiv
% where:
%  name          is the name of the buffer to be extracted
%                  from the following text
%  startsequence is the *macro* which starts the text
%  stopsequence  is the *macro* (used as a literal) which
%                  stops the text
%  before        is the *macro* to be executed *before* text
%                  is extracted
%  after         is the *macro* to be executed *after* the 
%                  text has been extracted
%  undent        is .... ??

\startMkIVCode
\let\stopRule\relax

\def\stopRuleDone{
  \directlua{thirddata.joyLoLCoAlgs.stopRule()}
}

\def\startRule[#1]{
  \directlua{thirddata.joyLoLCoAlgs.startRule('#1')}
  \buff_pickup{_rules_buffer_}%
    {startRule}{stopRule}%
    {\relax}{\stopRuleDone}\plusone%
}
\stopMkIVCode

\startLuaCode
local function startRule(ruleName)
  texio.write_nl("starting rule: ["..ruleName.."]")
end

coAlgs.startRule = startRule

local sectionHeaders = tConcat({
  'arguments',
  'returns',
  'preDataStack',
  'preProcessStack',
  'preConditions',
  'postDataStack',
  'postProcessStack',
  'postConditions'
}, '|'):lower()

local function stopRule()
  local pp = require 'pl.pretty'
  local rulesBody  = buffers.getcontent('_rules_buffer_'):gsub("\13", "\n")
  local rules      = { }
  local lines      = { }
  local curSection = 'ignore'
  for aLine in rulesBody:gmatch("[^\r\n]+") do
    local aMatch = aLine:match("^%s*\\(%a+)%s*$")
    if aMatch and
      sectionHeaders:find(aMatch:lower(), 1, true)
    then
      rules[curSection] = lines
      lines             = { }
      curSection        = aMatch
    else
      tInsert(lines, aLine)
    end
  end
  rules[curSection] = lines
  texio.write_nl('---------rules-buffer-------------')
  texio.write_nl(pp.write(rules))
  texio.write_nl('---------rules-buffer-------------')
end


coAlgs.stopRule = stopRule
\stopLuaCode

\startTestCase[should manipulate buffers]
\startConTest
  \startRule[testRule]
    \arguments
      some argument content
    \returns
      some returns content
    \preDataStack
      some preDataStack content
    \preProcessStack
      some preProcessStack content
    \preConditions
      some preConditions content
    \postDataStack
      some postDataStack content
    \postProcessStack
      some postProcessStack content
    \postConditions  
      some postConditions content
  \stopRule
\stopConTest
\stopTestCase
\stopTestSuite

\startTestSuite[implementation environment]

\startMkIVCode
\let\stopImplementation\relax

\def\stopImplementationDone{
  \directlua{thirddata.joyLoLCoAlgs.stopImplementation()}
}

\def\startImplementation[#1]{
  \directlua{thirddata.joyLoLCoAlgs.startImplementation('#1')}
  \buff_pickup{_implementation_buffer_}%
    {startImplementation}{stopImplementation}%
    {\relax}{\stopImplementationDone}\plusone%
}
\stopMkIVCode

\startLuaCode
local function startImplementation(implementationType)
end

coAlgs.startImplementation = startImplementation

local function stopImplementation()
  local implBody  = buffers.getcontent('_implementation_buffer_'):gsub("\13", "\n")
  tex.sprint("\\starttyping")
  tex.print(implBody)
  tex.sprint("\\stoptyping")
  texio.write_nl('---------impl-buffer-------------')
  texio.write_nl(implBody)
  texio.write_nl('---------impl-buffer-------------')
end

coAlgs.stopImplementation = stopImplementation
\stopLuaCode

\stopTestSuite

\startTestSuite[fragment environment]

\startMkIVCode
\let\stopFragment\relax

\def\stopFragmentDone{
  \directlua{thirddata.joyLoLCoAlgs.stopFragment()}
}

\def\startFragment[#1]{
  \directlua{thirddata.joyLoLCoAlgs.startFragment('#1')}
  \buff_pickup{_fragment_buffer_}%
    {startFragment}{stopFragment}%
    {\relax}{\stopFragmentDone}\plusone%
}
\stopMkIVCode

\startLuaCode
local function startFragment(fragmentType)
  theCoAlg.curFragment      = { }
  theCoAlg.curFragment.type = fragmentType
end

coAlgs.startFragment = startFragment

local function stopFragment()
  local curFragment   = setDefs(theCoAlg, 'curFragment')
  local fragementType = setDefs(curFragment, 'type', 'ansic')
  local fragmentBody  = buffers.getcontent('_fragment_buffer_'):gsub("\13", "\n")

  joylol.crossCompilers.addFragment(fragmentType, fragmentBody)

  tex.sprint("\\starttyping")
  tex.print(fragmentBody)
  tex.sprint("\\stoptyping")
end

coAlgs.stopFragment = stopFragment
\stopLuaCode

\stopTestSuite
