% A ConTeXt document [master document: joylolCoAlg.tex]

\chapter[title=JoyLoL Tests] 

QUESTION: How do we load a *.joy file? Where do we put this command?

\startMkIVCode
\def\addCTestJoyLoLCallbacks#1{%
  \directlua{
    thirddata.joylolCoAlgs.addCTestJoyLoLCallbacks('#1')
  }
}
\stopMkIVCode

\startLuaCode
local function addCTestJoyLoLCallbacks(aCodeStream)
  local contests      = setDefs(thirddata, 'contests')
  local tests         = setDefs(contests, 'tests')
  local methods       = setDefs(tests, 'methods')
  local setup         = setDefs(methods, 'setup')
  local cTests        = setDefs(setup, 'cTests')
  aCodeStream         = aCodeStream         or 'default'
  cTests[aCodeStream] = cTests[aCodeStream] or { }
  tInsert(cTests[aCodeStream], [=[
void ctestsWriteStdOut(
  JoyLoLInterp *jInterp,
  Symbol       *aMessage
) {
  fprintf(stdout, "%s", aMessage);
}

void ctestsWriteStdErr(
  JoyLoLInterp *jInterp,
  Symbol       *aMessage
) {
  fprintf(stderr, "%s", aMessage);
}
void *ctestsCallback(
  lua_State *lstate,
  size_t resourceId
) {
  if (resourceId == JoyLoLCallback_StdOutMethod) {
    return (void*)ctestsWriteStdOut;
  } else if (resourceId == JoyLoLCallback_StdErrMethod) {
    return (void*)ctestsWriteStdErr;
  } else if (resourceId == JoyLoLCallback_Verbose) {
    return (void*)FALSE;
  } else if (resourceId == JoyLoLCallback_Debug) {
    return (void*)FALSE;
  }
  return NULL;
} 
]=])
  setup               = setDefs(tests, 'setup')
  cTests              = setDefs(setup, 'cTests')
  cTests[aCodeStream] = cTests[aCodeStream] or { }
  tInsert(cTests[aCodeStream], [=[
setJoyLoLCallbackFrom(lstate, ctestsCallback);
]=])
end

coAlgs.addCTestJoyLoLCallbacks = addCTestJoyLoLCallbacks
\stopLuaCode

\subsection{JoylolTests}

see ConTests LuaTests.tex file

To integrate into ConTests inside ConTeXt runner we need to create 
something like: 

local function runCurLuaTestCase(suite, case)
  runALuaTest(case.lua, suite, case)
end

contests.testRunners.runCurLuaTestCase = runCurLuaTestCase 

Anything in the testRunners table must be a function taking two arguments 
as above. 

\startMkIVCode
\def\setJoylolVerboseOn{%
  \directlua{thirddata.joylol.setVerbose(true)}
}

\def\setJoylolVerboseOff{%
  \directlua{thirddata.joylol.setVerbose(false)}
}

\def\setJoylolDebuggingOn{%
  \directlua{thirddata.joylol.setDebugging(true)}
}

\def\setJoylolDebuggingOff{%
  \directlua{thirddata.joylol.setVDebugging(false)}
}

\def\setJoylolTracingOn{%
  \directlua{thirddata.joylol.setTracing(true)}
}

\def\setJoylolTracingOff{%
  \directlua{thirddata.joylol.setTracing(false)}
}

\def\setJoylolShowStackOn{%
  \directlua{thirddata.joylol.setShowStack(true)}
}

\def\setJoylolShowStackOff{%
  \directlua{thirddata.joylol.setShowStack(false)}
}

\def\setJoylolShowSpecificationsOn{%
  \directlua{thirddata.joylol.setShowSpecifications(true)}
}

\def\setJoylolShowSpecificationsOff{%
  \directlua{thirddata.joylol.setShowSpecifications(false)}
}

\def\setJoylolCheckingOn{%
  \directlua{thirddata.joylol.setChecking(true)}
}

\def\setJoylolCheckingOff{%
  \directlua{thirddata.joylol.setChecking(false)}
}
\stopMkIVCode

\startLuaCode
function showStack(aMessage) 
  texio.write_nl('-----------------------------------------------')
  if aMessage and type(aMessage) == 'string' and 0 < #aMessage then
    texio.write_nl(aMessage)
  end
  dataStack    = joylol.showData()
  processStack = joylol.showProcess()
  texio.write_nl("Data:")
  texio.write_nl(dataStack)
  texio.write_nl("Process:")
  texio.write_nl(processStack)
  texio.write_nl('AT: '..status.filename..'::'..status.linenumber)
  texio.write_nl('-----------------------------------------------')

end

contests.showStack = showStack
\stopLuaCode

\startMkIVCode
\definetyping[JoylolTest]
\setuptyping[JoylolTest][option=lisp]

\let\oldStopJoylolTest=\stopJoylolTest
\def\stopJoylolTest{%
  \oldStopJoylolTest%
  \directlua{thirddata.contests.addJoylolTest('_typing_')}
}

\def\showJoylolTest{%
  \directlua{thirddata.contests.showJoylolTest()}
}
\stopMkIVCode

\startLuaCode
local function addJoylolTest(bufferName)
  local bufferContents = buffers.getcontent(bufferName):gsub("\13", "\n")
  local suite = tests.curSuite
  local case  = suite.curCase
  case.joylol    = case.joylol or {}
  tInsert(case.joylol, bufferContents)
end

contests.addJoylolTest = addJoylolTest

local function buildJoylolChunk(joylolChunk, curSuite, curCase)
  if type(joylolChunk) == 'table' then
    joylolChunk = tConcat(joylolChunk, '\n')
  end

  if type(joylolChunk) ~= 'string' then
    return nil
  end

  if joylolChunk:match('^%s*$') then
    return nil
  end

  return [=[
(
]=]..joylolChunk..[=[

)
(
 "]=]..curCase.desc..[=["
  ]=]..curCase.fileName..[=[

  ]=]..curCase.startLine..[=[

  ]=]..status.linenumber..[=[

)
runTestCase
showStack
true
]=]
end

contests.buildJoylolChunk = buildJoylolChunk

local function showJoylolTest()
  local curSuite = setDefs(tests, 'curSuite')
  local curCase  = setDefs(curSuite, 'curCase')
  texio.write_nl('===============================================')
  local joylolChunk =
    buildJoylolChunk(curCase.joylol, curSuite, curCase)
  if joylolChunk then
    texio.write_nl('Joylol Test: ')
    texio.write_nl('-----------------------------------------------')
    texio.write_nl(joylolChunk)
    texio.write_nl('-----------------------------------------------')
  else
    texio.write_nl('NO Joylol Test could be built')
  end
  texio.write_nl('AT: '..status.filename..'::'..status.linenumber)
  texio.write_nl('===============================================')
end

contests.showJoylolTest = showJoylolTest
\stopLuaCode

\startLuaCode
local function runAJoylolTest(joylolTest, suite, case)
  case.passed = case.passed or true
  local joylolChunk = buildJoylolChunk(joylolTest, suite, case)
  if not joylolChunk then
    -- nothing to test
    return true
  end

  local caseStats = tests.stats.joylol.cases
  caseStats.attempted = caseStats.attempted + 1
  tex.print("\\starttyping")
  joylol.evalString(joylolChunk)
  tex.print("\\stoptyping")
  local testResult = joylol.popData()
  if not testResult then
    local errObj = joylol.popData()
    local failure = logFailure(
      "LuaTest FAILED",
      suite.desc,
      case.desc,
      errObj.message,
      toStr(errObj[1]),
      sFmt("in file: %s between lines %s and %s",
        case.fileName, toStr(case.startLine), toStr(case.lastLine))
      )
    reportFailure(failure, false)
    tInsert(tests.failures, failure)
    return false
  end

  -- all tests passed
  caseStats.passed = caseStats.passed + 1
  tex.print("\\noindent{\\green PASSED}")
  return true
end

contests.runAJoylolTest = runAJoylolTest

local function runCurJoylolTestCase(suite, case)
  runAJoylolTest(case.joylol, suite, case)
end

contests.testRunners.runCurJoylolTestCase = runCurJoylolTestCase
\stopLuaCode

\startMkIVCode
\def\createJoylolTestFile#1#2#3{%
  \directlua{
    thirddata.contests.createJoylolTestFile('#1', '#2', '#3')
  }
}
\stopMkIVCode

\startLuaCode
local function createJoylolTestFile(
  aCodeStream, aFilePath, aFileHeader
)
  texio.write("\n-------------------------------\n")
  texio.write("aCodeStream = ".. aCodeStream.."\n")
  texio.write("aFilePath   = ".. aFilePath.."\n")
  texio.write("\n-------------------------------\n")

  if not build.buildDir then
    texio.write('\nERROR: document directory NOT yet defined\n')
    texio.write('       NOT creating code file ['..aFilePath..']\n\n')
    return
  end

  if type(aFilePath) ~= 'string'
    or #aFilePath < 1 then
    texio.write('\nERROR: no file name provided for joylolTests\n\n')
    return
  end

  build.joylolTestTargets = build.joylolTestTargets or { }
  local aTestExec = aFilePath:gsub('%..+$','')
  tInsert(build.joylolTestTargets, aTestExec)

  aFilePath = build.buildDir .. '/buildDir/' .. aFilePath
  local outFile = io.open(aFilePath, 'w')
  if not outFile then
    return
  end
  
  texio.write('creating JoylolTest file: ['..aFilePath..']\n')

  outFile:write('#!/usr/bin/env joylol\n')
  outFile:write('\n\n')
  
  if type(aFileHeader) == 'string'
    and 0 < #aFileHeader then
    outFile:write(aFileHeader)
    outFile:write('\n\n')
  end

  tests.suites = tests.suites or { }

  if type(aCodeStream) ~= 'string'
    or #aCodeStream < 1 then
    aCodeStream = 'default'
  end

  outFile:close()
end

contests.createJoylolTestFile = createJoylolTestFile
\stopLuaCode

\section[title=Lua Make System files] 

In this section we add the code required to produce Lua Make System files 
which know how to compile JoyLoL Tests. 

\startMkIVCode
\def\addJoyLoLTestTargets#1{%
  \directlua{
    thirddata.joylolCoAlgs.addJoylolTestTargets('#1')
  }
}
\stopMkIVCode

\startLuaCode
local function addJoylolTestTargets(aCodeStream)
  litProgs.setCodeStream('Lmsfile', aCodeStream)
  litProgs.markCodeOrigin('Lmsfile')
  local lmsfile = {}
  tInsert(lmsfile, "require 'lms.joylolTests'\n")
  tInsert(lmsfile, "joylolTests.targets(lpTargets, {")
  tInsert(lmsfile, "  testExecs = {")
  for i, aTestExec in ipairs(build.joylolTestTargets) do
    tInsert(lmsfile, "    '"..aTestExec.."',")
  end
  tInsert(lmsfile, "  },")
  
  build.srcTargets = build.srcTargets or { }
  local srcTargets = build.srcTargets
  
  srcTargets.cHeader = srcTargets.cHeader or { }
  local cHeader      = srcTargets.cHeader
  tInsert(lmsfile, "  cHeaderFiles = {")
  for i, aSrcFile in ipairs(cHeader) do
    tInsert(lmsfile, "    '"..aSrcFile.."',")
  end
  tInsert(lmsfile, "  },")
  
  srcTargets.cCode = srcTargets.cCode or { }
  local cCode      = srcTargets.cCode
  tInsert(lmsfile, "  cCodeFiles = {")
  for i, aSrcFile in ipairs(cCode) do
    tInsert(lmsfile, "    '"..aSrcFile.."',")
  end
  tInsert(lmsfile, "  },")

  if build.cCodeLibDirs then 
    tInsert(lmsfile, "  cCodeLibDirs = {")
    for i, aLibDir in ipairs(build.cCodeLibDirs) do
      tInsert(lmsfile, "    '"..aLibDir.."',")
    end
    tInsert(lmsfile, "  },")
  end
  if build.cCodeLibs then 
    tInsert(lmsfile, "  cCodeLibs = {")
    for i, aLib in ipairs(build.cCodeLibs) do
      tInsert(lmsfile, "    '"..aLib.."',")
    end
    tInsert(lmsfile, "  },")
  end

  tInsert(lmsfile, "  coAlgLibs = {")
  for i, aCoAlgDependency in ipairs(build.coAlgDependencies) do
    tInsert(lmsfile, "    '"..aCoAlgDependency.."',")
  end
  tInsert(lmsfile, "  },")
  tInsert(lmsfile, "})")
  litProgs.setPrepend('Lmsfile', aCodeStream, true)
  litProgs.addCode.default('Lmsfile', tConcat(lmsfile, '\n'))
end

coAlgs.addJoylolTestTargets = addJoylolTestTargets
\stopLuaCode
