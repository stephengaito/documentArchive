% A ConTeXt document [master document: joylolCoAlg.tex]

\chapter[title=Code Manipulation] 

In this chapter we define the \ConTeXt\ tools we will use to define the 
JoyLoL language. 

The JoyLoL CoAlgebra \ConTeXt\ module provides the tools required to fully 
describe the formal semantics of a particular JoyLoL CoAlgebraic extension 
including any defined JoyLoL words. It consists of literate documentation 
of the actual source code produced to implement the JoyLoL CoAlgebraic 
extension. 

\subsection{Implementation} 

In this section we load the Syntax Highlighter modules used by the code 
display commands (below). We also load the ConTests module used to test 
the JoyLoL CoAlgebra module itself. We then load the lua code associated 
with the \type{t-joylol} module. 

\prependMkIVCode{default}
\startMkIVCode
\writestatus{loading}{ConTeXt User Module / JoyLoL CoAlgebra Extensions} 

\usemodule[t-literate-progs]
\usemodule[t-high-lisp]
\usemodule[t-contests]
\usemodule[t-joylol]

\ctxloadluafile{t-joylol-coalg}
\stopMkIVCode

\startTestSuite[JoyLoLCoAlg environment]

The JoyLoLCoAlg environment provides a highly structured environment in 
which to describe the formal semantics and implementation of a particular 
JoyLoL CoAlgebraic extension. 

A typical JoyLoLCoAlg environment consists of a collection of JoyLoL 
words. This includes a \quote{global} word which defines any global code 
required by the CoAlgegraic extension as a whole. 

\subsection{Examples}

\starttyping
\startJoyLoLCoAlg[title=List of Lists][lists]
\stoptyping

The first argument provides the arguments to an embedded 
\type{\startchapter} command. 

The second argument provides the arguments to an embedded 
\type{\startcomponent} command. It also provides the base file name of 
all of the automatically generated code fragments. 

The second argument also determines the name of any JoyLoL, ANSI-C, or 
Lua source file artefacts produced by this literate code documentation. 

\startMkIVCode
\def\declareJoyLoLCoAlg[#1][#2]{
  \startcomponent[#2]
  \startchapter[#1]
  \directlua{thirddata.joyLoLCoAlgs.newCoAlg('#2')}
}

\let\startJoyLoLCoAlg=\declareJoyLoLCoAlg

\def\stopJoyLoLCoAlg{\relax}
\stopMkIVCode

\startLuaCode
--local function newCoAlg(coAlgName)
--  local lCoAlg        = setDefs(theCoAlg, coAlgName)
--  lCoAlg.name         = coAlgName
--  lCoAlg.words        = lCoAlg.words or {}
--  lCoAlg.words.global = {}
--end

local function newCoAlg(coAlgName)
  texio.write_nl('newCoAlg: ['..coAlgName..']')
  theCoAlg               = {}
  theCoAlg.name          = coAlgName
  theCoAlg.ctx           = nil --joylol.newContext()
  theCoAlg.hasJoyLoLCode = false;
  theCoAlg.hasLuaCode    = false;
  theCoAlg.hasCHeader    = false;
  theCoAlg.hasCCode      = false;
  build.coAlgsToBuild = build.coAlgsToBuild or {}
  tInsert(build.coAlgsToBuild, coAlgName)
  build.coAlgDependencies = build.coAlgDependencies or {}
  -- 
  --local aCtx = theCoAlg.ctx
  --pushProcess(aCtx, 'addToDict')
  --pushProcessQuoted(aCtx, coAlgName)
  --pushProcessQuoted(aCtx, 'coAlgName')
  --addNewList(aCtx, 'dependsOn')
  --addNewList(aCtx, 'wordOrder')
  --addNewDict(aCtx, 'words')
  --newDictionary(aCtx)
  --jEval(aCtx)
  --
  -- add the new word: "global"
  --
  --coAlgs.newWord('global')
end

coAlgs.newCoAlg = newCoAlg
\stopLuaCode

\subsubsection{Implementation: Start: Tests}

\startTestCase[should do something]
\startConTest
  \mockContextMacro{startcomponent}{1}
  \mockContextMacro{startchapter}{1}
  \startJoyLoLCoAlg[title=List of Lists][lists]
  \assertMacroNthArgumentOnMthExpansionMatches%
    {startcomponent}{1}{1}{lists}{}
  \assertMacroNthArgumentOnMthExpansionMatches%
    {startchapter}{1}{1}{title=List of Lists}{}
  \startLuaConTest
    local theCoAlg = thirddata.joylol.theCoAlg
    assert.isTable(theCoAlg)
    assert.hasKey(theCoAlg, 'lists')
    local lists = theCoAlg.lists
    assert.isTable(lists)
    assert.hasKey(lists, 'name')
    assert.matches(lists.name, 'lists')
    assert.hasKey(lists, 'words')
    local words = lists.words
    assert.hasKey(words, 'global')
  \stopLuaConTest
\stopConTest
\stopTestCase
\stopTestSuite

\subsection{Implementation: Stop}

\startMkIVCode
\def\stopJoyLoLCoAlg{
  \directlua{thirddata.joylol.createCoAlg()}
  \stopchapter
  \stopcomponent
}
\stopMkIVCode

\startLuaCode
local function createCoAlg()
end

joylol.createCoAlg = createCoAlg
\stopLuaCode

\section{Source licenses}

\subsection{Examples}

\subsection{Implementation}

\startMkIVCode
\unexpanded\def\srcCopyrightCCBYSA{}
\stopMkIVCode

% <a rel="license" 
% href="http://creativecommons.org/licenses/by-sa/4.0/"><img alt="Creative 
% Commons License" style="border-width:0" 
% src="https://i.creativecommons.org/l/by-sa/4.0/88x31.png" /></a><br 
% /><span xmlns:dct="http://purl.org/dc/terms/" 
% property="dct:title">JoyLoL</span> by <a 
% xmlns:cc="http://creativecommons.org/ns#" 
% href="https://www.perceptisys.co.uk" property="cc:attributionName" 
% rel="cc:attributionURL">Perceptisys Ltd (Stephen Gaito)</a> is licensed 
% under a <a rel="license" 
% href="http://creativecommons.org/licenses/by-sa/4.0/">Creative Commons 
% Attribution-ShareAlike 4.0 International License</a>.<br />Based on a 
% work at <a xmlns:dct="http://purl.org/dc/terms/" href="gitHub" 
% rel="dct:source">gitHub</a>. 

\section{Target licenses}

\subsection{Examples}

\subsection{Implementation}

\startMkIVCode
\unexpanded\def\targetCopyrightMIT{}
\stopMkIVCode

\section{Describing CoAlgebraic dependencies}

\subsection{Examples}

\subsection{Implementation}

\startMkIVCode
\def\dependsOn[#1]{
  \directlua{thirddata.joyLoLCoAlgs.addDependency('#1')}
}
\stopMkIVCode

\startLuaCode
local function addDependency(dependencyName)
  build.coAlgDependencies = build.coAlgDependencies or {}
  tInsert(build.coAlgDependencies, dependencyName)
  --texio.write_nl('addDependency: ['..dependencyName..']')
  --local aCtx = theCoAlg.ctx
  --pushProcess(aCtx, 'popData')
  --pushProcess(aCtx, 'appendToEndList')
  --pushProcessQuoted(aCtx, dependencyName)
  --pushProcess(aCtx, 'lookupInDict')
  --pushProcess(aCtx, 'dependsOn')
  --jEval(aCtx)
end

coAlgs.addDependency = addDependency
\stopLuaCode

\section{JoyLoL stack action: In }

A JoyLoL stack action (either in or out) contains one or more sections of 
\emph{implementation} code, either ANSI-C or Lua, together with a 
collection of descriptors of the JoyLoL \{pre, post\} \{data, process\} 
stacks. These stack actions provide the only allowed interface between an 
implementation language's \quote{local} variables and the JoyLoL stack 
context. \quote{In} actions take a data structure in a local variable and 
place it on either the data or process stacks. 

\subsection{Examples}

\subsection{Implementation: start}

\startMkIVCode
\def\startJoyLoLStackActionIn[#1]{
  \directlua{thirddata.joylol.newStackActionIn('#1')}
}
\stopMkIVCode

\startLuaCode
local function newStackActionIn(aWord)
end

joylol.newStackActionIn = newStackActionIn
\stopLuaCode

\subsection{Implementation: stop}

\startMkIVCode
\def\stopJoyLoLStackActionIn{
  \directlua{thirddata.joylol.endStackActionIn()}
}
\stopMkIVCode

\startLuaCode
local function endStackActionIn()
end

joylol.endStackActionIn = endStackActionIn
\stopLuaCode

\section{JoyLoL stack action: Out }

A JoyLoL stack action (either in or out) contains one or more sections of 
\emph{implementation} code, either ANSI-C or Lua, together with a 
collection of descriptors of the JoyLoL \{pre, post\} \{data, process\} 
stacks. These stack actions provide the only allowed interface between an 
implementation language's \quote{local} variables and the JoyLoL stack 
context. \quote{Out} actions take an item on either the data or process 
stack and place it into a data structure in a local variable and 
\emph{possibly} \quote{removing} it from the appropriate stack. 

\subsection{Examples}

\subsection{Implementation: start}

\startMkIVCode
\def\startJoyLoLStackActionOut[#1]{
  \directlua{thirddata.joylol.newStackActionOut('#1')}
}
\stopMkIVCode

\startLuaCode
local function newStackActionOut(aWord)
end

joylol.newStackActionOut = newStackActionOut
\stopLuaCode

\subsection{Implementation: stop}

\startMkIVCode
\def\stopJoyLoLStackActionOut{
  \directlua{thirddata.joylol.endStackActionOut()}
}
\stopMkIVCode

\startLuaCode
local function endStackActionOut()
end

joylol.endStackActionOut = endStackActionOut
\stopLuaCode

\section{Describing the data stack}

\subsection{Examples}

\subsection{Implementation}

\startMkIVCode
\def\preDataStack[#1][#2]{
  \directlua{thirddata.joylol.addPreDataStackDescription('#1', '#2')}
}

\def\postDataStack[#1]{
  \directlua{thirddata.joylol.addPostDataStackDescription('#1')}
}
\stopMkIVCode

\startLuaCode
local function addPreDataStackDescription(arg1, arg2)
end

joylol.addPreDataStackDescription = addPreDataStackDescription

local function addPostDataStackDescription(arg1, arg2)
end

joylol.addPostDataStackDescription = addPostDataStackDescription
\stopLuaCode

\section{Describing the process stack}

\subsection{Examples}

\subsection{Implementation}

\startMkIVCode
\def\preProcessStack[#1][#2]{
  \directlua{thirddata.joylol.addPreProcessStackDescription('#1', '#2')}
}

\def\postProcessStack[#1]{
  \directlua{thirddata.joylol.addPostProcessStackDescription('#1')}
}
\stopMkIVCode

\startLuaCode
local function addPreProcessStackDescription(arg1, arg2)
end

joylol.addPreProcessStackDescription = addPreProcessStackDescription

local function addPostProcessStackDescription(arg1, arg2)
end

joylol.addPostProcessStackDescription = addPostProcessStackDescription
\stopLuaCode

\section{JoyLoL code environment}

\subsection{Examples}

\subsection{Implementation}

\startMkIVCode
\defineLitProgs
  [JoyLoLCode]
  [ option=lisp, numbering=line,
    before={\noindent\startLitProgFrame}, after=\stopLitProgFrame
  ]

\defineLitProgs
  [MinJoyLoLCode]
  [ option=lisp, numbering=line,
    before={\noindent\startLitProgFrame}, after=\stopLitProgFrame
  ]
\stopMkIVCode

\startMkIVCode
\def\addCTestJoyLoLCallbacks#1{%
  \directlua{
    thirddata.joylol.addCTestJoyLoLCallbacks('#1')
  }
}
\stopMkIVCode

\startLuaCode
local function addCTestJoyLoLCallbacks(aCodeStream)
  local contests      = setDefs(thirddata, 'contests')
  local tests         = setDefs(contests, 'tests')
  local methods       = setDefs(tests, 'methods')
  local setup         = setDefs(methods, 'setup')
  local cTests        = setDefs(setup, 'cTests')
  aCodeStream         = aCodeStream         or 'default'
  cTests[aCodeStream] = cTests[aCodeStream] or { }
  tInsert(cTests[aCodeStream], [=[
void ctestsWriteStdOut(
  JoyLoLInterp *jInterp,
  Symbol       *aMessage
) {
  fprintf(stdout, "%s", aMessage);
}

void ctestsWriteStdErr(
  JoyLoLInterp *jInterp,
  Symbol       *aMessage
) {
  fprintf(stderr, "%s", aMessage);
}
void *ctestsCallback(
  lua_State *lstate,
  size_t resourceId
) {
  if (resourceId == JoyLoLCallback_StdOutMethod) {
    return (void*)ctestsWriteStdOut;
  } else if (resourceId == JoyLoLCallback_StdErrMethod) {
    return (void*)ctestsWriteStdErr;
  } else if (resourceId == JoyLoLCallback_Verbose) {
    return (void*)FALSE;
  } else if (resourceId == JoyLoLCallback_Debug) {
    return (void*)FALSE;
  }
  return NULL;
} 
]=])
  setup               = setDefs(tests, 'setup')
  cTests              = setDefs(setup, 'cTests')
  cTests[aCodeStream] = cTests[aCodeStream] or { }
  tInsert(cTests[aCodeStream], [=[
setJoyLoLCallbackFrom(lstate, ctestsCallback);
]=])
end

joylol.addCTestJoyLoLCallbacks = addCTestJoyLoLCallbacks
\stopLuaCode