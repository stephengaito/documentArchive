% A ConTeXt document [master document: protoJoyLoL.tex ]

\section[title=JoyLoL bootstrap code] 

QUESTION: How do we load a *.joy file? Where do we put this command?

\subsection{JoyLoL code environment}

\subsubsection{Examples}

\subsubsection{Implementation}

We begin by registering the \type{JoylolCode} code type with the build 
srcTypes system. This will ensure the \type{\createJoylolCodeFile} macro 
(and corresponding lua code) knows how to deal with files of 
\type{JoylolCode}. 

\startLuaCode
build.srcTypes = build.srcTypes or { }
build.srcTypes['JoylolCode'] = 'joylolCode'
\stopLuaCode

\startMkIVCode
\defineLitProgs
  [Joylol]
  [ option=lisp, numbering=line,
    before={\noindent\startLitProgFrame}, after=\stopLitProgFrame
  ]
  
\setLitProgsOriginMarker[JoylolCode][markJoylolCodeOrigin]
\stopMkIVCode

\startLuaCode
local function markJoylolOrigin()
  local codeType       = setDefs(code, 'Joylol')
  local codeStream     = setDefs(codeType, 'curCodeStream', 'default')
  codeStream           = setDefs(codeType, codeStream)
  return sFmt('// from file: %s after line: %s',
    codeStream.fileName,
    toStr(
      mFloor(
        codeStream.startLine/code.lineModulus
      )*code.lineModulus
    )
  )
end

litProgs.markJoylolOrigin = markJoylolOrigin
\stopLuaCode
