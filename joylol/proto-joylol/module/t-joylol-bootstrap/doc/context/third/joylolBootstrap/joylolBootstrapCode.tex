% A ConTeXt document [master document: protoJoyLoL.tex ]

\section[title=JoyLoL bootstrap code] \goto{website}[url(https://disimplex.org/papers/workingDraft/2019/04/protoJoyLoL/protoJoyLoL.html#pf103)]

QUESTION: How do we load a *.joy file? Where do we put this command?

\subsection{JoyLoL code environment}

\subsubsection{Examples}

\subsubsection{Implementation}

We begin by registering the \type{JoylolCode} code type with the build 
srcTypes system. This will ensure the \type{\createJoylolCodeFile} macro 
(and corresponding lua code) knows how to deal with files of 
\type{JoylolCode}. 

\startLuaCode
build.srcTypes = build.srcTypes or { }
build.srcTypes['Joylol'] = 'joylol'
\stopLuaCode

\startMkIVCode
\defineLitProgs
  [JoylolCode]
  [ option=lisp, numbering=line,
    before={\noindent\startLitProgFrame}, after=\stopLitProgFrame
  ]
  
\setLitProgsOriginMarker[JoylolCode][markJoylolCodeOrigin]
\stopMkIVCode

\startLuaCode
local function markJoylolCodeOrigin()
  local codeType       = setDefs(code, 'JoylolCode')
  local codeStream     = setDefs(codeType, 'curCodeStream', 'default')
  codeStream           = setDefs(codeType, codeStream)
  return sFmt('// from file: %s after line: %s',
    codeStream.fileName,
    toStr(
      mFloor(
        codeStream.startLine/code.lineModulus
      )*code.lineModulus
    )
  )
end

litProgs.markJoylolCodeOrigin = markJoylolCodeOrigin
\stopLuaCode

\subsection[title=Automatic generation of parsers]

In this subsection we write the Lua/LPeg code required to extract the 
concrete parsers from the \joylol\ CoAlgebra step 3 code.

\startMkIVCode
\def\loadCoAlgebraCollection#1{
  \directlua{
    thirddata.literateProgs.loadCoAlgebraCollection('#1')
  }
}

\def\extractConcreteParserFromTo#1#2{
  \directlua{
    thirddata.literateProgs.extractConcreteParserFromTo('#1', '#2')
  }
}
\stopMkIVCode

\startLuaCode

local function loadCoAlgebraCollection(aCollectionPath)

  texio.write_nl(">>>>>>>>>>LOAD>>>>>>>>>>")
  texio.write_nl("from CoAlgebra collection: ["..aCollectionPath.."]")
  local newPath = replaceEnvironmentVarsInPath(aCollectionPath)
  texio.write_nl("from CoAlgebra collection: ["..newPath.."]")
  recursivelyFindFiles(newPath, '%.joy$', function(aPath)
    texio.write_nl(aPath)
  end)
--  texio.write_nl(lpPP(os.env))
--  texio.write_nl(lpPP(fromCodeStream))
--  texio.write_nl(">>>>>>>>>>LOAD<<<<<<<<<<")
--  texio.write_nl("to code stream: [coAlgTable]")
--  texio.write_nl(lpPP(coAlgTable))
--  texio.write_nl(">>>>>>>>>>LOAD<<<<<<<<<<")
--  texio.write_nl("to code stream: ["..toCodeStreamName.."]")
--  texio.write_nl(lpPP(toCodeStream))  
--  texio.write_nl("<<<<<<<<<<LOAD<<<<<<<<<<\n")
--  texio.write_nl(lpPP(codeType))
  texio.write_nl("<<<<<<<<<<LOAD<<<<<<<<<<\n")
end

litProgs.loadCoAlgebraCollection = loadCoAlgebraCollection
\stopLuaCode

\startLuaCode
local P, R, S, V, C, Cc, Cg, Ct = 
  lpeg.P, lpeg.R, lpeg.S, lpeg.V, lpeg.C, lpeg.Cc, lpeg.Cg, lpeg.Ct
local lMatch = lpeg.match

local currentCoAlg = "unknown"
local coAlgTable   = {}

local function extractCoAlg(aMatch)
  currentCoAlg       = aMatch
  coAlgTable[aMatch] = {}
end

local function extractInherit(aMatch)
  coAlgTable[currentCoAlg] = coAlgTable[currentCoAlg] or {}
  
  coAlgTable[currentCoAlg]['parents'] = 
    coAlgTable[currentCoAlg]['parents'] or {}
  tInsert(coAlgTable[currentCoAlg]['parents'], aMatch)  
  
  coAlgTable[aMatch] = coAlgTable[aMatch] or { }
  
  coAlgTable[aMatch]['children'] = 
    coAlgTable[aMatch]['children'] or {}
  tInsert(coAlgTable[aMatch]['children'], currentCoAlg)
end

local function extractParser(anId, aParser)
  coAlgTable[anId] = coAlgTable[anId] or {}
  
  coAlgTable[anId]['parser'] = aParser
end

local function buildParserStream(parserStream)
  local coAlgNames = { }
  for aCoAlg, aBody in pairs(coAlgTable) do
    tInsert(coAlgNames, aCoAlg)
    if aBody['children'] then
      local choiceBody = {}
      for i, aChild in ipairs(aBody['children']) do
        tInsert(choiceBody, "parse"..aChild)
      end
      aBody['parser'] = "( "..tConcat(choiceBody, '\n').." ) choice"
    end
  end
  
  tSort(coAlgNames)
  
  for i, aCoAlg in ipairs(coAlgNames) do
    local aBody   = coAlgTable[aCoAlg]
    local aParser = "( "..aBody['parser'].." ) \"parse"..aCoAlg.."\" define"
    tInsert(parserStream, aParser)
  end
end

local function extractConcreteParserFromTo(fromCodeStreamName, toCodeStreamName)
  local codeType       = setDefs(code, 'Joylol')
  fromCodeStreamName   = fromCodeStreamName or 'default'
  toCodeStreamName     = toCodeStreamName   or 'parser'
  local fromCodeStream = setDefs(codeType, fromCodeStreamName)
  local toCodeStream   = setDefs(codeType, toCodeStreamName)

  local ws       = S('\r\n\f\t ')^1
  local id       = R('AZ', 'az') * (R('AZ', 'az', '09') + S('_-'))^0
\stopLuaCode

This is the most important section of the extract code. At this point we 
define a crude LPeg parser which scans for three patterns.

\startitemize[n]

\item \bold{CoAlgebra} We look for the pattern: \type{CoAlgebra <id>}. 

\startLuaCode
  local coAlg =
    ( P('CoAlgebra') * ws * C(id) ) / extractCoAlg
\stopLuaCode

\item \bold{Inheritance} We look for the pattern: \type{inherit <id> ;}. 
Note that at the moment we can \emph{only} deal with single inheritance. 

\startLuaCode
  local inherit  =
    ( P('inherit') * ws * C(id) * ws * P(';') ) / extractInherit
\stopLuaCode

\item \bold{(Concrete) Parser} In order to identify the concrete parser 
code, we scan for a much more complex pattern. This pattern starts by 
looking for \type{feature -- Concrete Syntax}. It then looks for the 
pattern \type{parse} followed by an identifier with \emph{no} intervening 
white space. Finally it captures all of the text between a \emph{single} 
pair of \type{beginSExp} and \type{endSExp} words. 

\startLuaCode
  local concreteParser   = (
    P('feature') * ws * P('--') * ws * 
    P('Concrete') * ws * P('Syntax') * ws *
    P('parse') * C(id) * 
    ( 1 - P('beginSExp') )^1 *  P('beginSExp') *
    C( ( 1 - P('endSExp') )^1 ) * P('endSExp')
    ) / extractParser
\stopLuaCode

\stopitemize

Note that each of the above sub-parsers, takes any captures and stores 
them in our \type{coAlgTable} using the three functions 
\type{extractCoAlg}, \type{extractInherit}, and \type{extractParser} 
respectively. Each of these functions were defined above.

We now put these three sub-parsers together in an ordered choice and wrap 
this choice in the LPeg code required to scan a whole text multiple times. 
The final line in the following code simply runs the LPeg parser over the 
existing code in the \type{fromCodeStream}. All captured information is 
stored in the \type{coAlgTable} Lua table. 

\startLuaCode
  local parts    = coAlg + inherit + concreteParser
  local matchPat = Ct( P{ parts + 1 * lpeg.V(1) }^0 )
  lMatch(matchPat, tConcat(fromCodeStream, '\n\n'))
\stopLuaCode

Having run the LPeg parser, we now need to add the extracted concrete 
parsers after any existing \joylol\ code in the \type{toCodeStream} using 
the lua function \type{buildParserStream} defined above.

\startLuaCode
  tInsert(codeType[toCodeStreamName], 
    "// START of automatically generated jPEG parser")
  tInsert(codeType[toCodeStreamName],
    "// listed alphabetically by definition")
  buildParserStream(codeType[toCodeStreamName])
  tInsert(codeType[toCodeStreamName], 
    "// END of automatically generated jPEG parser")
  
--  texio.write_nl(">>>>>>>>>>EXTRACT>>>>>>>>>>")
--  texio.write_nl("from code stream: ["..fromCodeStreamName.."]")
--  texio.write_nl(lpPP(fromCodeStream))
--  texio.write_nl(">>>>>>>>>>EXTRACT<<<<<<<<<<")
--  texio.write_nl("to code stream: [coAlgTable]")
--  texio.write_nl(lpPP(coAlgTable))
--  texio.write_nl(">>>>>>>>>>EXTRACT<<<<<<<<<<")
--  texio.write_nl("to code stream: ["..toCodeStreamName.."]")
--  texio.write_nl(lpPP(toCodeStream))  
--  texio.write_nl("<<<<<<<<<<EXTRACT<<<<<<<<<<\n")
--  texio.write_nl(lpPP(codeType))
--  texio.write_nl("<<<<<<<<<<EXTRACT<<<<<<<<<<\n")
end

litProgs.extractConcreteParserFromTo = extractConcreteParserFromTo
\stopLuaCode


