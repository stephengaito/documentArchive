% A ConTeXt document [master document: joyLoLMinus.tex ]


\section[title=Overview]

\joylolMinus\ is the most minimal \joylol\ compiler/interpreter hand coded 
in ANSI-C. 

Its only purpose is to run jPeg \quote{programs}, hence we specialize the 
overall \joylol\ structure, principally \quote{\type{context}s}, to focus 
upon making jPeg programs run reasonably performantly. Any \joylol\ 
program which \quote{keeps} a given finite number of structures on top of 
the data stack, can be \quote{assisted} by adding the same structures as 
\quote{registers} of a modified \quote{\type{context}}. We do exactly this 
for this most minimal \joylol. 

Why \joylol\ for the bootstrap steps? Why not just hand code something 
that will run some sort of PEG? Ultimately our argument is that the 
semantics of any computational program and mathematical argument, rests 
upon the fact that \joylol\ is a fixed point of a semantic operator. As a 
fixed point it provides its own semantic meaning. Hence to give meaning to 
any computation or mathematical argument we \quote{simply} need to show 
how that computation or mathematical argument can be translated into 
\quote{raw} \joylol. Essentially, our bootstrapping jPeg programs will do 
just that. 

