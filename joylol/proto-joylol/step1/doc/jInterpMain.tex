% A ConTeXt document [master document: joyLoLMinus.tex ]

\section[title=ANSI-C Main]

\setCCodeStream{jInterpMain}
\startCCode

/* option parser configuration */
static struct option longOpts[] = {
  {"trace", no_argument, 0, 't'},
  {"quite", no_argument, 0, 'q'},
  {"debug", no_argument, 0, 'd'},
  {"help",  no_argument, 0, 'h'},
  {0, 0, 0, 0}
};
 /* getopt_long stores the option index here. */
static int optId = 0;

int main(int argc, char *argv[]){

  printf("joylolMinus\n");
  ContextObj *rootCtx = newContext(NULL, NULL, "rootCtx");
  assert(isContextObj(rootCtx));
  int opt = 0;
  
  int tracingOn = false;
  
  while ((opt = getopt_long(argc, argv,
                            "dtqh",
                            longOpts, &optId)) != -1) {
    switch (opt) {
      case 'd': rootCtx->debug = true; break;
      case 't': tracingOn  = true; break;
      case 'q':
        tracingOn          = false;
        rootCtx->tracingOn = false;
        rootCtx->debug     = false;
        break;
      case 'h':
      default:
        fprintf(stderr, "Usage: %s [options] [file...]\n\n", argv[0]);
        fprintf(stderr, "options:\n");
        fprintf(stderr, "  -d --debug\n");
        fprintf(stderr, "  -t --trace\n");
        fprintf(stderr, "  -q --quiet\n\n");
        fprintf(stderr, "files to load:\n");
        fprintf(stderr, "  Any remaining options are treated as files to be loaded.\n");

        exit(-1);
    }
  }

  registerBasicWords(rootCtx);
  registerControlWords(rootCtx);
  registerJPegWords(rootCtx);
  registerSExpressionParser(rootCtx);
  
  for(; optind < argc; optind++) {
    rootCtx->tracingOn = false;
    printf("interpreting: [%s]\n", argv[optind]);
    printf("------------------------------------------\n");
    if (tracingOn) printf("==============\ncalling parseFile\n\n");
    parseFile(rootCtx, argv[optind]);
    if (tracingOn) {
      printf("input file: [%s]\n", rootCtx->parserTextName);
      printf("contents:   [%s]\n", rootCtx->parserText);
      printf("==============\ncalling parseSExpression\n\n");
    }
    parseSExpressionWord(rootCtx);
    if (tracingOn) {
      showCtxData(rootCtx);
      showCtxProcess(rootCtx);
      printf("==============\ncalling evalCommandInContext (parseSExpressionWord)\n\n");
    }
    evalCommandInContext(rootCtx, NULL);
    if (tracingOn) {
      showCtxData(rootCtx);
      showCtxProcess(rootCtx);
      printf("==============\ncalling interpretWord\n\n");
    }
    interpretWord(rootCtx);
    if (tracingOn) {
      showCtxData(rootCtx);
      showCtxProcess(rootCtx);
      printf("==============\ncalling evalCommandInContext (interpretWord)\n\n");
      rootCtx->tracingOn = tracingOn;
    }
    evalCommandInContext(rootCtx, NULL);
    printf("\n------------------------------------------\n");
    printf("finished interpreting [%s]\n\n", argv[optind]);
  }
}
\stopCCode

\prependCCode{jInterpMain}
\startCCode
#include <stdio.h>
#include <stdlib.h>
#include <string.h>
#include <assert.h>
#include <unistd.h>
#include <getopt.h>

#include "joylol/jInterp.h"
#include "joylol/jInterp-private.h"
\stopCCode

\prependCCode{jInterpMain}
\placeCCodeCopyright{PerceptiSys Ltd (Stephen Gaito)}

\createCCodeFile%
  {jInterpMain}%
  {jInterp.c}%
  {}

\setCCodeStream{default}