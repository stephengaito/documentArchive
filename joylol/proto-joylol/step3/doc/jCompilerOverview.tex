% A ConTeXt document [master document: jPeg.tex ]

\section[title=Overview]

For the initial version of the syntax for both the JoyLoL Configuration 
Language, and the JoyLoL Language itself, we make essential use of 
Bertrand Meyer's definition of the early version of Eiffel contained in 
\cite{meyer1992eiffelTheLanguage}. We will however make extensive changes 
to suit a new purpose. Eiffel's essential use of \quote{programming by 
contract} is however very close to one of our twin purposes for JoyLoL. 

We structure the various parsers using ideas from Meyer's 
\cite{meyer1990theoryProgrammingLanguages}. The specific parser languages 
are (originally) loosely based upon the \quote{Language for assembling 
classes in Eiffel} (Lace) (\cite{meyer1992eiffelTheLanguage} Appendix D) 
as well as the Eiffel language itself (\cite{meyer1992eiffelTheLanguage} 
summarized in Appendix H). However we structure the \joylol\ language as a 
\quote{enclosing} \joylol\ CoAlgebra Specification language 
(\joylolCoAlg), enclosing the (initial) pair of \joylol\ implementation 
languages, \joylol\ S-Expression language (\joylolSExp) together with the 
\joylol\ Register Machine language (\joylolRM). While the \joylol\ 
language we develop here is a \quote{conservative extension} of the fixed 
point of the \joylol\ semantic functor, \joylolZero, the \joylol\ 
S-Expression language (\joylolSExp) is closer in structure to the fixed 
point language, \joylolZero. 

\setJoylolCodeStream{concreteParser}
\startJoylolCode
( pattern ) "P" define
( 1 repeatAtMost) "?" define
( 0 repeatAtLeast ) "*" define
( 1 repeatAtLeast ) "+" define
( 1 any ) "." define
//( 1 charSet ) "??" define
//( 0 span ) "???" define
//( 1 span ) "???" define
//( 1 coSpan ) "????" define
\stopJoylolCode
\setJoylolCodeStream{default}


