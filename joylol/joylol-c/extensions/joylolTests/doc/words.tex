% A ConTeXt document [master document: joylolTests.tex]

\section[title=Words]
\setJoylolTracingOff\setJoylolVerboseOff

\TODO{ Consider changing the defineContext and defineNaming syntax to be 
similar to new define syntax; Fix dynamic definitions: defineTestsSetup, 
defineTestsTeardown, defineTestSuiteSetup, defineTestSuiteTeardown, } 

\startTestSuite[tests and assert naming scopes]

We start by defining a naming scope to contain all of the Joylol test 
associated words. This naming space is denoted \quote{tests} in the 
\quote{globals} naming scope. The \quote{tests} naming scope's parent 
naming scope is also \quote{globals}. 

\startJoylolCode
;; create a tests naming scope
globals      ;; parent naming scope
globals      ;; defining naming scope
tests        ;; defined name of this new naming scope
defineNaming ;; make the definition
\stopJoylolCode

We also create a naming scope to contain all of the assertion related 
words. 

\startJoylolCode
;; create an assert naming scope
tests        ;; parent naming scope
globals      ;; defining naming scope
assert       ;; defined name of this new naming scope
defineNaming ;; make the definition
\stopJoylolCode
\stopTestSuite

\startTestSuite[running all tests]

The following figure is a summary of the word/switch pattern of 
\type{runAllTests} word: 

\externalfigure[runJoylolTestAll.pdf][width=10cm]

\startJoylolCode
(
  runAllTests           ;; name of word
  { true }              ;; pre condition
  (                     ;; start of the runAllTests word
    "running ALL TESTS" ;;
    writeOutNL          ;; announce our presence
\stopJoylolCode

We start by localizing the running context's naming scope and then 
defining the testsRunnerNS and testsMonitorNS naming scopes in the 
localized naming scope. 

\startJoylolCode
    ;; localize the naming scope of the current context
    allTestsLocals  ;; name of the localized naming scope
    localizeNaming  ;; localize the naming scope
\stopJoylolCode

\startJoylolCode
    ;; undefine any existing testsRunnerNS
    ( testsRunnerNS )
    thisNaming
    undefine
  
    ;; create the testsRunnerNS
    thisNaming    ;; define the parent's naming scope
    thisNaming    ;; define in the localized naming scope
    testsRunnerNS ;; name of the new naming scope
    defineNaming  ;; create the new naming scope
                  ;; to be used as the testsRunner's naming scope
\stopJoylolCode

\startJoylolCode
    ;; undefine any existing testsMonitorNS
    ( testsMonitorNS )
    thisNaming
    undefine
  
    ;; create the testsMonitorNS
    thisNaming    ;; define the parent's naming scope
    thisNaming    ;; define in the localized naming scope
    testsMonitorNS;; name of the new naming scope
    defineNaming  ;; create the new naming scope
                  ;; to be used as the testsMonitor's naming scope
\stopJoylolCode

We now want to define default tests setup and teardown words in the 
testsRunnerNS. 

\startJoylolCode
    (
      testsSetup  ;; name of the new word
      { true }    ;; pre condition
      ()          ;; empty default setup actions
      { true }    ;; post condition
    ) 
    testsRunnerNS ;; define in the testsRunnerNS
    define        ;; define the testsSetup word
\stopJoylolCode

\startJoylolCode
    (
      testsTeardown ;; name of the new word
      { true }    ;; pre condition
      ()          ;; empty default teardown actions
      { true }    ;; post condition
    ) 
    testsRunnerNS ;; define in the testsRunnerNS
    define        ;; define the testsTeardown word
\stopJoylolCode

We now create the testsRunner context.

\startJoylolCode
    ;; undefine any existing testsRunner context
    ( testsRunner )
    thisNaming
    undefine

    ;; define the testsRunner context
                  ;; testsRunner process (uses provided list of words)
    ( tests.testsFinished )
    append        ;; ensure we get the tests results
                  ;; when all suites have been run
    ()            ;; testsRunner data
    testsRunnerNS ;; naming scope of the new context
    thisNaming    ;; defined in the localized naming scope
    testsRunner   ;; name of the new context
    defineContext ;; define the new context
\stopJoylolCode

We now create the testsMonitor context.

\startJoylolCode
    ;; undefine any existing testsMonitor context
    ( testsMonitor )
    thisNaming
    undefine

    ;; define the testsMonitor context
    ( interpret )   ;; begin the testsMonitor process stack
    thisContext     ;; interpret thisContext onto data stack
    append          ;; append the result
    ( switchCtx )   ;;
    append          ;; append the word switchCtx to the
                    ;; testsMonitor's initial process stack
  
    (
      0             ;; number of attempted suites
      0             ;; number of failed suites
      ()            ;; list of failure reports
    )               ;; testsMonitor data
  
    testsMonitorNS  ;; naming scope of the new context
    thisNaming      ;; defining naming scope
    testsMonitor    ;; defined name of the new context
    defineContext   ;; define the context
\stopJoylolCode

Finally we want to switch to the testsRunner context, however, we also 
want to specify what actions to take when the testsMonitor context 
switches context back to this context. We do this by placing our context 
switch to testsRunner and any subsequent actions into a list and then we 
interpret this list. 

\startJoylolCode
    ;; switch to the testsRunner context and run all of the tests
    (
      ;; switch to the testsRunner context
      testsRunner ;; place the testsRunner context on the stack
      switchCtx   ;; switch to the testsRunner context
    
      ;; when we are re-activated
      "finished ALL TESTS"  ;;
      writeOutNL            ;; say we are done
    )
    interpret     ;; place this list of actions onto
                  ;; our process stack and do them
  )
  { true }        ;; post condition
)
tests
define
\stopJoylolCode

We now define the \type{defineTestsSetup} and \type{defineTestsTeardown} 
words. While defined in the tests naming scope, they will be run in the 
testsRunner and its associated testsRunnerNS. 

\startJoylolCode
( 
  defineTestsSetup ;; name of word
  { true }         ;; pre condition
  (
    "defined tests setup"
    writeOutNL     ;; tell everyone we have been run
  
                   ;; we use the provided definition
    (
      setupTests   ;; name of the new word
      { true }     ;; pre condition
      swap12D
      { true }     ;; post condition
    )
    thisNaming     ;; defining naming scope
    define
  )
  { true }         ;; post condition
)
tests
define
\stopJoylolCode

\startJoylolCode
(
  defineTestsTeardown ;; name of word
  { true }            ;; pre condition
  (
    "defined tests teardown"
    writeOutNL        ;; tell everyone we have been run

                      ;; we use the provided definition
    ( 
      teardownTests   ;; name of the new word
      { true }        ;; pre condition
      swap12D
      { true }        ;; post condition
    )
    thisNaming        ;; defining naming scope
    define
  )
  { true }            ;; post condition
)
tests
define
\stopJoylolCode

Once a particular test suite has finished, we need summarize the results 
and pass them to the tests runner. We do this with a pair of words one, 
\type{suiteFinished}, run in the suiteRunner and the other, 
\type{reportSuiteResults}, run in the suiteMonitor. 

\startJoylolCode
(
  testsFinished  ;; name of word
  { true }       ;; pre condition
  (
    ( tests.reportTestsResult )
    testsMonitor
    switchCtx
  )
  { true }       ;; post condition
)
tests
define
\stopJoylolCode

\startJoylolCode
(
  reportTestsResult ;; name of word
  { true }          ;; pre condition
  (
                    ;; there is nothing more to do
                    ;; the testsMonitor already has
                    ;; the context switch 
  )
  { true }          ;; post condition
)
tests
define
\stopJoylolCode

\stopTestSuite

\startTestSuite[running a test suite]

The following figure is a summary of the word/switch pattern of 
\type{runTestSuite} word: 

\externalfigure[runJoylolTestSuite.pdf][width=10cm]

\startJoylolCode
(
  runTestSuite
  { true }                ;; pre condition
  (                       ;; start of the runTestSuite word
    "running test suite"  ;;
    writeOutNL            ;; announce our presence
\stopJoylolCode

We are running in the testsRunner context so we already have a 
\quote{localized} naming scope. So for the \type{runTestSuite} word we 
only need to create the suiteRunnerNS and suiteMonitorNS naming scopes. 

\startJoylolCode
    ;; undefine any existing suiteRunnerNS
    ( suiteRunnerNS )
    thisNaming
    undefine

    ;; create the suiteRunnerNS
    thisNaming    ;; define the parent's naming scope
    thisNaming    ;; define in the localized naming scope
    suiteRunnerNS ;; name of the new naming scope
    defineNaming  ;; create the new naming scope
                  ;; to be used as the suiteRunner's naming scope
\stopJoylolCode

\startJoylolCode
    ;; undefine any existing suiteMonitorNS
    ( suiteMonitorNS )
    thisNaming
    undefine

    ;; create the suiteMonitorNS
    thisNaming      ;; define the parent's naming scope
    thisNaming      ;; define in the localized naming scope
    suiteMonitorNS  ;; name of the new naming scope
    defineNaming    ;; create the new naming scope
                    ;; to be used as the suiteMonitor's naming scope
\stopJoylolCode

We now want to define the default suite setup and teardown words in the 
suiteRunnerNS. 

\startJoylolCode
    (
      suiteSetup  ;; name of the new word
      { true }    ;; pre condition
      ()          ;; empty default setup actions
      { true }    ;; post condition
    )
    suiteRunnerNS ;; define in the suiteRunnerNS
    define        ;; define the suiteSetup word
\stopJoylolCode

\startJoylolCode
    (
      suiteTeardown ;; name of the new word
      { true } ;; pre condition
      ()            ;; empty default teardown actions
      { true } ;; post condition
    )
    suiteRunnerNS ;; define in the suiteRunnerNS
    define        ;; define the suiteTeardown word
\stopJoylolCode

We now create the suiteRunner context.

\startJoylolCode
    ;; undefine any existing suiteRunner context
    ( suiteRunner )
    thisNaming
    undefine

    ;; define the suiteRunner context
                ;; suiteRunner process (uses provided list of words)
    ( tests.suiteFinished )
    append        ;; ensure we get the suite results
                  ;; at the end of the suite
    ()            ;; suiteRunner data
    suiteRunnerNS ;; naming scope of the new context
    thisNaming    ;; defined in the localized naming scope
    suiteRunner   ;; name of the new context
    defineContext ;; define the new context
\stopJoylolCode

We now create the suiteMonitor context.

\startJoylolCode
    ;; undefine any existing suiteMonitor context
    ( suiteMonitor )
    thisNaming
    undefine

    ;; define the suiteMonitor context
    ( interpret )   ;; begin the suiteMonitor process stack
    thisContext     ;; interpret thisContext onto data stack
    append          ;; append the result
    ( switchCtx )   ;;
    append          ;; append the word switchCtx to the 
                    ;; suiteMonitor's initial process stack
  
    (
      0             ;; number of attempted cases
      0             ;; number of failed cases
      ()            ;; list of failure reports
    )               ;; suiteMonitor data
  
    suiteMonitorNS  ;; naming scope of the new context
    thisNaming      ;; defining naming scope
    suiteMonitor    ;; name of the new context
    defineContext   ;; define the context
\stopJoylolCode

Finally we want to switch to the suiteRunner context, however, again, we 
also want to specify what actions to take when the suiteMonitor context 
switches back to this context. We do this by placing our context switch to 
suiteRunner and any subsequent actions into a list and then we interpret 
this list. 

\startJoylolCode
    ;; switch to the suiteRunner context and run the suite
    (
      ;; switch to the suiteRunner context
      suiteRunner ;; place the suiteRunner context on the stack
      switchCtx   ;; switch to the suiteRunner context
    
      ;; when we are re-activated
      ;; ... HOW DO WE SWITCH BACK???!!
    )
    interpret
  )
  { true }        ;; post condition
)
tests
define
\stopJoylolCode

We now define the \type{recordTestSutieDetails}, 
\type{defineTestSuiteSetup}, and \type{defineTestSuiteTeardown} words. 

\startJoylolCode
(
  recordTestSuiteDetails
  { true } ;; pre condition
  (
    (
      newLine
      "-------------------------------------"
      newLine
      "jTS:"
    )           ;; format a sign that we are running
                ;; the test suite
    prepend     ;; prepend this sign to the name of the suite
    100         ;; limit the printLoL
    swap12D     ;; put the lol and the depth limit 
                ;; in the correct order
    printLoL    ;; print it to a string
    writeOutNL  ;; write out the string
  )
  { true }      ;; post condition
)
tests
define
\stopJoylolCode

\startJoylolCode
(
  defineTestSuiteSetup
  { true }        ;; pre condition
  (
    "defined test suite setup"
    writeOutNL    ;; tell everyone we have been run

                  ;; we use the provided definition
    (
      setupSuite  ;; name of the new word
      { true }    ;; pre condition
      swap12D
      { true }    ;; post condition
    )
    thisNaming  ;; defining naming scope
    define
  )
  { true } ;; post condition
)
tests
define
\stopJoylolCode

\startJoylolCode
(
  defineTestSuiteTeardown
  { true } ;; pre condition
  (
    "defined test suite teardown"
    writeOutNL    ;; tell everyone we have been run

    (
      teardownSuite ;; name of the new word
                    ;; we use the provided definition
      { true }      ;; pre condition
      swap12D
      { true }      ;; post condition
    )
    thisNaming      ;; defining naming scope
    define
  )
  { true }        ;; post condition
)
tests
define
\stopJoylolCode

Once a particular test suite has finished, we need summarize the results 
and pass them to the tests runner. We do this with a pair of words one, 
\type{suiteFinished}, run in the suiteRunner and the other, 
\type{reportSuiteResults}, run in the suiteMonitor. 

\startJoylolCode
(
  suiteFinished
  { true }   ;; pre condition
  (
    ( tests.reportSuiteResults )
    suiteMonitor
    switchCtx
  )
  { true } ;; post condition
)
tests
define
\stopJoylolCode

\startJoylolCode
( 
  reportSuiteResults
  { true }  ;; pre condition
  (
            ;; there is nothing more to do
            ;; the suiteMonitor already has
            ;; the context switch 
  )
  { true }  ;; post condition
)
tests
define
\stopJoylolCode

\stopTestSuite

\startTestSuite[runTestCase]

Our next task, over the next following code fragments, is to define the 
runTestCase word. This word takes a test case, as a list of the words to 
be tested intermingled with assertions, and creates the caseMonitor and 
caseRunner contexts which together run the test case provided. To be 
completely re-entrant, the caseMonitor and caseRunner contexts are stored 
in the running context's localized naming scope. 

The following figure is a summary of the word/switch pattern of 
\type{runTestCase} word: 

\externalfigure[runJoylolTestCase.pdf][width=10cm]

\startJoylolCode
( 
  runTestCase   ;; the name of the new word
  { true }      ;; pre condition
  (             ;; start of the runTestCase word
\stopJoylolCode

We are running in the suiteRunner context so we already have a 
\quote{localized} naming scope. So for the \type{runTestCase} word we only 
need to create the caseRunnerNS and caseMonitorNS naming scopes. 

\startJoylolCode
    ;; undefine any existing caseRunnerNS
    ( caseRunnerNS )
    thisNaming
    undefine

    ;; create the caseRunnerNS
    thisNaming   ;; parent naming scope
    thisNaming   ;; define in localized naming scope
    caseRunnerNS ;; name of the new naming scope
    defineNaming ;; create a new naming scope
                 ;; to be used as the caseRunner's naming scope
\stopJoylolCode

\startJoylolCode
    ;; undefine any existing caseMonitorNS
    ( caseMonitorNS )
    thisNaming
    undefine

    ;; create the caseMonitorNS
    thisNaming    ;; parent naming scope
    thisNaming    ;; define in the localized naming scope
    caseMonitorNS ;; name of the new naming scope
    defineNaming  ;; create a new naming scope
                  ;; to be used as the caseMonitor's naming scope
\stopJoylolCode

At this point the test case's list of words and assertions are back on the 
top of the stack. This list will be used as the caseRunner's initial 
process stack. We can now add the initial data stack for the caseRunner 
context, and then setup the new context's naming scope, the defining 
naming scope and finally the name of the new context.

\startJoylolCode
    ;; undefine any existing caseRunner context
    ( caseRunner )
    thisNaming
    undefine

    ;; define the caseRunner context
                  ;; caseRunner process (uses provided list of words)
    ( tests.caseFinished )
    append        ;; ensure we get the case results
                  ;; at the end of the case
    ()            ;; caseRunner data
    caseRunnerNS  ;; the naming scope of the new context
    thisNaming    ;; define in the localized naming scope
    caseRunner    ;; name of the new context
    defineContext ;; create and define the context
\stopJoylolCode

Now we create a caseMonitor context which will keep track of the test 
results. The caseMonitor's initial process stack should contain the three 
words, \type{interpret}, the current context, and \type{switchCtx} in this 
order. When an assertion switches from the caseRunner's context to the 
caseMonitor context the top of the caseRunner's stack is place onto the 
top of the caseMonitor's data stack. The \type{interpret} word will then 
place the words on the top of the caseMonitor's data stack onto the 
caseMonitor's process stack to be interpreted. The assertion being 
interpreted \emph{should} ensure it re-places the \type{interpret} word 
back on the caseMonitor's process stack for use by the next assertion. 

The last two words, the current context and \type{switchCtx}, as used by the 
textMonitor interpreting the \type{results} word to know which context to 
switch to with the summary results. 

Since the \type{thisContext} word has to be interpreted in the calling 
context, we have to \type{append} the result of interpreting the 
\type{thisContext} word rather than just placing the \type{thisContext} 
word directly on the caseMonitor's process stack. Once the 

\startJoylolCode
    ;; undefine any existing caseMonitor context
    ( caseMonitor )
    thisNaming
    undefine

    ;; define the caseMonitor context
    ( interpret ) ;; begin the caseMonitor process stack
    thisContext   ;; interpret thisContext 
    append        ;; append the result of interpreting thisContext
    ( switchCtx ) ;; 
    append        ;; append the word switch to the
                  ;; caseMonitor's initial process stack
  
    ( 
      false       ;; should fail indicator
      0           ;; number of attempted assertions
      0           ;; number of failed assertions
      ()          ;; list of failure reports
    )             ;; caseMonitor data

    caseMonitorNS ;; the naming scope of the new context
    thisNaming    ;; defining naming scope
    caseMonitor   ;; defined name of this new context
    defineContext ;; create and define the context
\stopJoylolCode

Finally we switch to the caseRunner context.

\startJoylolCode
    ;; switch to the caseRunner context and run the assertions
    caseRunner    ;; place the caseRunner context on the stack
    switchCtx     ;; switch to the caseRunner context
  
  )               ;; end of the runTestCase definition
  { true }        ;; post condition
)
tests             ;; define the new word in this naming scope
define            ;; define the new runTestCase word
\stopJoylolCode

We now define the \type{recordTestCaseDetails} word.

\startJoylolCode
(
  recordTestCaseDetails
  { true }      ;; pre condition
  (
    "  jTC:"    ;; format a sign that we are running
                ;; the test case
    prepend     ;; prepend this sign to the name of the case
    100         ;; limit the printLoL
    swap12D     ;; put the lol and the depth limit
                ;; in the correct order
    printLoL    ;; print it to string
    writeOutNL  ;; write out the string
  )
  { true }      ;; post condition
)
tests
define
\stopJoylolCode
\stopTestSuite

\startTestSuite[communicate case results]

Once a particular test case has finished, we need summarize the results 
and pass them to the suite runner. We do this with a pair of words one, 
\type{caseFinished}, run in the caseRunner and the other, 
\type{reportCaseResults}, run in the caseMonitor. 

\startJoylolCode
(
  caseFinished
  { true } ;; pre condition
  (
    ( tests.reportCaseResults )
    caseMonitor
    switchCtx
  )
  { true } ;; post condition
)
tests
define
\stopJoylolCode

\startJoylolCode
( 
  reportCaseResults
  { true } ;; pre condition
  (
    pop1D   ;; ignore the should fail boolean
            ;; there is nothing more to do
            ;; the caseMonitor already has
            ;; the context swtich 
  )
  { true }  ;; post condition
)
tests
define
\stopJoylolCode

\stopTestSuite

\startTestSuite[assert.reportAssertion]

In this section we develop the \type{reportAssertion} word in the 
\type{assert} naming scope. This word takes the result of an individual 
assertion, switches to the \type{caseMonitor} context and records it 
before switching back to the \type{caseRunner} context. 

The \type{reportAssertion} word assumes that any previous assertion has 
left an assertion result list on the top of the caseRunner's data stack. 
This assertion result list consists of true/false followed by a failure 
report list (only if the assertion failed). 

\startJoylolCode
(
  reportAssertion ;; the name of the new word
  { true }        ;; pre condition
  (               ;; start of the reportAssertion word
                  ;; the assertion report is assumed to be on top
    (
      assert.recordAssertion  ;; the task to be done
      caseRunner  ;; return to caseRunner
      switchCtx   ;; perform the switch/return
      interpret   ;; leave an initial interpret for the next switch/call
    )
    append
    caseMonitor   ;; place the caseMonitor context on top of stack
    switchCtx     ;; switch to the caseMonitor context
                  ;; (the caseMonitor context will interpret to
                  ;;  current top of the stack)
  )               ;; end of the reportAssertion word
  { true }        ;; post condition
)
assert          ;;define the new word in the assert naming scope
define          ;; define the reportAssertion word
\stopJoylolCode
\stopTestSuite

\startTestSuite[assert.recordAssertion]

The \type{recordAssertion} word is run on the caseMonitor context. It 
assumes that the top of the data stack is a true/false depending upon the 
success/failure of the previous assertion. If the top of the stack is 
false, then the \type{recordAssertion} word also expects a list containing 
the assertion failure reasons as the second item on the stack. The 
subsequent item on the stack is then the context to which to return so 
that any further assertions can be run. Finally the last item on the stack 
is the caseMonitor's record of assertion results. 

\startJoylolCode
(
  recordAssertion
  { true }    ;; pre condition
  (
    showStack ;; need to deal with shouldFail....
    (
      isTrue
    )  ;; test
    (
      ;; the top of the stack is the caseMonitor's
      ;; record of assertion results
      extract
      ;; the top of the stack is the number of attempts value
      1 + ;; add one to the top of the stack
      prepend
      prepend
      ;; the top of the stack is the
      ;; updated record of assertion results
    )  ;; then action ;; just update the number of attempted tests 
    (
      ;; the top of the stack is an assertion failure report list
      ;; then comes the caseMonitor's record of assertion results
      append
      extract
      ;; top    comes the number of attempts value
      ;; second comes the number of failures value
      ;; third  comes the collection of failure reports
      ;; fourth comes the assertion failure report list
      1 + ;; add one to the number of attempted
      swap12D
      1 + ;; add one to the number of failed
      append
      rollupD ;; put current record third
      append  ;; append the assertion failure report to list or failures
      append  ;; complete the record of assertion results
      ;; top comes the updated record of assertion results
    )  ;; else action ;; update everything
    ifte
  )
  { true }   ;; post condition
)
assert
define
\stopJoylolCode
\stopTestSuite

\startTestSuite[assertShouldFail]

The \type{assert.shouldFail} word wraps the interpretation of the body of 
the should fail assertion in a pair of \type{tests.startShouldFail} and 
\type{tests.stopShouldFail} words. The \type{startShouldFail} word ensures 
the top of the caseMonitor data stack contains \type{true} so that any 
failures are interpreted as successes. The \type{stopShouldFail} word 
ensures that the top of the caseMonitor data stack contains a \type{false} 
to ensure any failures are reported as failures. 

We begin by defining the \type{tests.startShouldFail} word. It builds the 
actions which will pop the top of the data stack and replaces it with a 
\type{true} and then switch back to the current context. Having setup the 
appropriate list of actions on the top of the current context's data 
stack, the \type{startShouldFail} word then switches to the caseMonitor 
context. The \type{switchCtx} word pops the top of the current context's 
data stack and pushes it on the top of the caseMonitor's data stack. This 
all assumes that the top of the caseMonitor's process stack is the 
\type{interpret} word which will have the effect of interpreting the list 
of actions copied over from the current context to the caseMonitor's data 
stack. 

\startJoylolCode
(
  startShouldFail
  { true }        ;; pre condition
  (
    (
      pop1D
      true
      dup
    ) 
    thisContext
    append
    ( switchCtx )
    append
    caseMonitor
    switchCtx
    pop1D
  )
  { true } ;; post condition
)
tests
define 
\stopJoylolCode 

\startJoylolCode
(
  stopShouldFail
  { true } ;; pre condition
  (
    (
      pop1D
      false
      dup
    )
    thisContext
    append
    ( switchCtx )
    append
    caseMonitor
    switchCtx
    pop1D
  )
  { true } ;; post condition
)
tests
define
\stopJoylolCode

The \type{assert.shouldFail} word now uses the 
\type{tests.startShouldFail} and \type{tests.stopShouldFail} words to 
bracket the interpretation of the actions found quoted on the current 
context's data stack. 

\startJoylolCode
(
  shouldFail
  { true } ;; pre condition
  (
    tests.startShouldFail
    interpret
    tests.stopShouldFail
  )
  { true } ;; post condition
)
assert
define
\stopJoylolCode

\startTestCase[assert.shouldFail]
\startJoylolTest
( assert.fail ) assert.shouldFail
\stopJoylolTest
\stopTestCase
\stopTestSuite

\startTestSuite[assert.fail]

\startJoylolCode
(
  fail
  { true } ;; pre condition
  (
    false
    assert.reportAssertion
  )
  { true } ;; post condition
)
assert
define
\stopJoylolCode

\startTestCase[assert.fail should fail]
\startJoylolTest
( assert.fail ) assert.shouldFail
\stopJoylolTest
\stopTestCase
\stopTestSuite

\startTestSuite[assert.succeed]
\startJoylolCode
( 
  succeed
  { true } ;; pre condition
  (
    true
    assert.reportAssertion
  )
  { true } ;; post condition
)
assert
define
\stopJoylolCode

\startTestCase[assert.succeed should succeed]
\startJoylolTest
assert.succeed
\stopJoylolTest
\stopTestCase
\stopTestSuite

\startTestSuite[assert.isBoolean]
\startJoylolCode
(
  isBoolean
  { true } ;; pre condition
  (
    isBoolean
    assert.reportAssertion
  )
  { true } ;; post condition
)
assert
define
\stopJoylolCode

\startTestCase[should succeed if an object is a boolean ]
\startJoylolTest
true
assert.isBoolean
\stopJoylolTest
\stopTestCase

\startTestCase[should fail if an object is not a boolean]
\startJoylolTest
(
  notABoolean
  assert.isBoolean
) assert.shouldFail
\stopJoylolTest
\stopTestCase
\stopTestSuite

\startTestSuite[assert.isTrue]

\startJoylolCode
(
  isTrue
  { true } ;; pre condition
  (
    isTrue
    assert.reportAssertion
  )
  { true } ;; post condition
)
assert
define
\stopJoylolCode

\startTestCase[should succeed if an object is true]
\startJoylolTest
true
assert.isTrue

notABooleanButTrue
assert.isTrue
\stopJoylolTest
\stopTestCase

\startTestCase[should fail if an object is not true]
\startJoylolTest
(
  false
  assert.isTrue 
) assert.shouldFail
(
  ()
  assert.isTrue
) assert.shouldFail
\stopJoylolTest
\stopTestCase
\stopTestSuite

\startTestSuite[assert.isFalse]

\startJoylolCode
(
  isFalse
  { true } ;; pre condition
  (
    isFalse
    assert.reportAssertion
  )
  { true } ;; post condition
)
assert
define
\stopJoylolCode

\startTestCase[should succeed if an object is false]
\startJoylolTest
false
assert.isFalse

()
assert.isFalse
\stopJoylolTest
\stopTestCase

\startTestCase[should fail if an object is not false]
\startJoylolTest
(
  true
  assert.isFalse
) assert.shouldFail
(
  notABooleanButTrue
  assert.isFalse
) assert.shouldFail
\stopJoylolTest
\stopTestCase
\stopTestSuite

\startTestSuite[assert.isNil]

\startJoylolCode
(
  isNil
  { true } ;; pre condition
  (
    isNil
    assert.reportAssertion
  )
  { true } ;; post condition
)
assert
define
\stopJoylolCode

\startTestCase[should succeed if an object is nil]
\startJoylolTest
()
assert.isNil
\stopJoylolTest
\stopTestCase

\startTestCase[should fail if an object is not nil]
\startJoylolTest
(
  ( someThingNotNil )
  assert.isNil
) assert.shouldFail
\stopJoylolTest
\stopTestCase

\stopTestSuite

\startTestSuite[assert.isNotNil]

\startJoylolCode
(
  isNotNil
  { true } ;; pre condition
  (
    isNil
    not
    assert.reportAssertion
  )
  { true } ;; post condition
)
assert
define
\stopJoylolCode

\startTestCase[should succeed if an object is not nil]
\startJoylolTest
( this is a list )
assert.isNotNil
\stopJoylolTest
\stopTestCase

\startTestCase[should fail if an object is nil]
\startJoylolTest
(
  ( )
  assert.isNotNil
) assert.shouldFail
\stopJoylolTest
\stopTestCase
\stopTestSuite

\startTestSuite[assert.isAnAtom]

\startJoylolCode
(
  isAnAtom
  { true } ;; pre condition
  (
    isAnAtom
    assert.reportAssertion
  )
  { true } ;; post condition
)
assert
define
\stopJoylolCode

\startTestCase[should succeed if an object is an atom]
\startJoylolTest
thisIsAnAtom
assert.isAnAtom
thisContext
assert.isAnAtom
123
assert.isAnAtom
\stopJoylolTest
\stopTestCase

\startTestCase[should fail if an object is not an atom]
\startJoylolTest
(
  ( this is a list )
  assert.isAnAtom
) assert.shouldFail
\stopJoylolTest
\stopTestCase
\stopTestSuite

\startTestSuite[assert.isAPair]

\startJoylolCode
(
  isAPair
  { true } ;; pre condition
  (
    isAPair
    assert.reportAssertion
  )
  { true } ;; post condition
)
assert
define
\stopJoylolCode

\startTestCase[should succeed if an object is a pair]
\startJoylolTest
( this is a list)
assert.isAPair
\stopJoylolTest
\stopTestCase

\startTestCase[should succeed if an object is not a pair]
\startJoylolTest
(
  true
  assert.isAPair
) assert.shouldFail
\stopJoylolTest
\stopTestCase
\stopTestSuite

\startTestSuite[assert.isANatural]

\startJoylolCode
(
  isANatural
  { true } ;; pre condition
  (
    isANatural
    assert.reportAssertion
  )
  { true } ;; post condition
)
assert
define
\stopJoylolCode

\startTestCase[should succeed if an object is a natural]
\startJoylolTest
1234
assert.isANatural
\stopJoylolTest
\stopTestCase

\startTestCase[should succeed if an object is not a natural]
\startJoylolTest
(
  aSymbol
  assert.isANatural
) assert.shouldFail
\stopJoylolTest
\stopTestCase
\stopTestSuite

\startTestSuite[assert.isASymbol]

\startJoylolCode
(
  isASymbol
  { true } ;; pre condition
  (
    isASymbol
    assert.reportAssertion
  )
  { true } ;; post condition
)
assert
define
\stopJoylolCode

\startTestCase[should succeed if an object is a symbol]
\startJoylolTest
aSymbol
assert.isASymbol
\stopJoylolTest
\stopTestCase

\startTestCase[should succeed if an object is not a symbol]
\startJoylolTest
(
  12345
  assert.isASymbol
) assert.shouldFail
\stopJoylolTest
\stopTestCase
\stopTestSuite

\startTestSuite[assert.isAContext]

\startJoylolCode
(
  isAContext
  { true } ;; pre condition
  (
    isAContext
    assert.reportAssertion
  )
  { true } ;; post condition
)
assert
define
\stopJoylolCode

\startTestCase[should succeed if an object is a context]
\startJoylolTest
thisContext
assert.isAContext
\stopJoylolTest
\stopTestCase

\startTestCase[should succeed if an object is not a context]
\startJoylolTest
(
  1234
  assert.isAContext
) assert.shouldFail
\stopJoylolTest
\stopTestCase
\stopTestSuite

\startTestSuite[assert.isADictionary]

\startJoylolCode
(
  isADictionary
  { true } ;; pre condition
  (
    isADictionary
    assert.reportAssertion
  )
  { true } ;; post condition
)
assert
define
\stopJoylolCode

\startTestCase[should succeed if an object is a dictionary]
\startJoylolTest
thisNaming
assert.isDictionary
\stopJoylolTest
\stopTestCase

\startTestCase[should succeed if an object is not a dictionary]
\startJoylolTest
(
  1234
  assert.isDictionary
) assert.shouldFail
\stopJoylolTest
\stopTestCase
\stopTestSuite

\starttyping
\startTestSuite[assert.isADictNode]

\startJoylolCode
\stopJoylolCode

\startTestCase[should succeed if an object is]
\startJoylolTest
\stopJoylolTest
\stopTestCase

\startTestCase[should succeed if an object is not]
\startJoylolTest
\stopJoylolTest
\stopTestCase
\stopTestSuite
\stoptyping

\setCHeaderStream{private}
\startCHeader
extern void registerJoylolTestWords(
  JoyLoLInterp* jInterp
);
\stopCHeader
\setCHeaderStream{public}

\startCCode
void registerJoylolTestWords(
  JoyLoLInterp* jInterp
) {
  assert(jInterp);

}
\stopCCode

\startJoylolCode
;; a final line
\stopJoylolCode
