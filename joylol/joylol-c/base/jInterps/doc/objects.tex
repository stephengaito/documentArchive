% A ConTeXt document [master document: jInterps.tex]

\section[title=JObjs]
\setCHeaderStream{public}
\setCCodeStream{objects}

Since the implementation of any particular CoAlgebraic extension will of 
necessity make use of instance specific data stored in a 
\type{JObj}ect, we \emph{begin} by providing the implementation of 
\type{JObj}. Together with any extension specific part, all of our 
\type{JObj}s have the same three part base structure. 

\subsection[title=Type part] The first (type) part is a \type{CoAlgebra*} 
pointer to the data structure which represents the CoAlgebra for which the 
\type{JObj} is an instance. This CoAlgebra pointer ensures that the 
implementation code knows what the given instance, is an instance 
\emph{for}, as well as what it can \emph{do}. 

\subsection[title=Tag part] The second (tag) part is an unsigned integer 
index into the \type{JoyLoLInterp}'s vector of registered 
\type{CoAlgebra}s. While the tags of all the required \type{CoAlgebra}ic 
extensions have been defined above and are, hence, \quote{well known}. 
Tags for all non-required extensions are assigned as the extension is 
registered with the interpreter. 

\startCHeader
#ifdef __LP64__
typedef uint32_t TagType;
#else
typedef uint16_t TagType;
#endif
\stopCHeader

\subsection[title=Flag part] The third (flag) part is a collection of bits 
to provide useful meta-flags associated with a particula instance. At 
least one of these meta-flags are reserved by the JoyLoL interpreter to 
signal ongoing garbage collection. Since any object is potentially part of 
a cyclic structure, these meta-flags ensure garbage collection does not 
fall into infinite cycles. 

All non-reserved meta-flags may be used by the implementation of a 
CoAlgebra extension for its own internal purposes. Typically meta-flags 
might be used to signal how to interpret the data stored in the third 
(data) part of the object. For the \type{Naturals} CoAlgebraic extension, 
a meta-flag will be used to signal that the object's data part is a 
pointer to a Gnu Multi-precision integer, rather than to a double word 
integer. This allows significant speed optimizations in the typical cases, 
but allows for full data representations in rare but important cases. 

All reserved meta-flags will be located in low order bits of the flag data 
word. This ensures that any CoAlgebraic extension which makes use of 
meta-flags can simply rotate the reserved flags off the end of the word 
before making use of the non-reserved flags. In particular a CoAlgebraic 
extension \emph{could} interpret its flags as an integer or pointer. Such 
interpretations are private to each extension, and should \emph{not} be 
relied upon by code which is not part of the code's own extension. 

As \quote{global} meta-flags we reserve the following three 
\emph{low-order} bits together with a Mask of all three bits and the 
number of reserved bits to shift (left). 

\startCHeader
#define MARK_SWEEP_FLAG     0x1L
#define RESERVED_FLAG_MASK  0x1L
#define RESERVED_FLAG_SHIFT 1

#ifdef __LP64__
typedef uint32_t FlagsType;
#else
typedef uint16_t FlagsType;
#endif
\stopCHeader

We can now define the \emph{base} \type{JObj} as:

\startCHeader
typedef struct joylol_object_struct {
  JClass    *type;
  TagType    tag;
  FlagsType  flags; // an arbitrary collection of bits
} JObj;

#define asType(aLoL)  (((JObj*)(aLoL))->type)
#define asTag(aLoL)   (((JObj*)(aLoL))->tag)
#define asFlags(aLoL) (((JObj*)(aLoL))->flags)

#define isFlagSet(aLoL, bitMask)        \
  (((JObj*)(aLoL))->flags & (bitMask))

#define isFlagClear(aLoL, bitMask)          \
  ((~(((JObj*)(aLoL))->flags)) & (bitMask))

#define checkObj(jInterp, theObj, theTag) \
  (                                       \
    assert(jInterp),                      \
    assert(theObj),                       \
    assert(theTag),                       \
    assert((((JObj*)(theObj))->type) ==   \
      ((jInterp)->coAlgs[theTag]))        \
  )
\stopCHeader

\section[title=isAtom isPair]

A common requirement is to determine whether or not a given 
\type{JObj} is an \quote{atom} or a \quote{pair}. Quite simply we 
define anything that is not an instance of \type{Pairs} an \quote{atom}.

\startCHeader
#define isAtom(anObject) \
  ((anObject) && (asTag(anObject) != PairsTag))
  
#define isPair(anObject) \
  ((anObject) && (asTag(anObject) == PairsTag))
\stopCHeader

\section[title=Object Memory]

Since a JoyLoL interpreter is essentially a list processor, any JoyLoL 
program will create (and orphan) a very large number of list \type{Pairs} 
over the course of a computation. This means that we need to make the 
allocation and eventual garbage collection of orphaned \type{JObjs} 
as efficient as possible.

We do this by defining a \type{ObjectMemory} as a pair of pointers. The 
first pointer points to a linked list of free objects. The second pointer 
points to a linked list of \type{ObjectBlock}s. 

\startCHeader
typedef struct object_block_struct ObjectBlock;

//typedef struct object_memory_struct {
//  JObj*    freeObjects;
//  ObjectBlock* rootObjectBlock;
//} ObjectMemory;

typedef struct object_block_struct {
  size_t       objectSize;
  void*        block;
  ObjectBlock* nextBlock;
} ObjectBlock;
\stopCHeader

\startTestSuite[addObjectBlock]

\setCHeaderStream{private}
\startCHeader
#define OBJECT_BLOCK_SIZE 1024

extern void addObjectBlock(JClass *theClass);
\stopCHeader
\setCHeaderStream{public}

\startCCode
void addObjectBlock(JClass* theClass) {
  assert(theClass);
  assert(theClass->jInterp);
  
  size_t objectSize = theClass->objectSize;

  DEBUG(theClass->jInterp, "addObjectBlock > %p [%s] %zu %zu\n", 
    theClass, theClass->name, objectSize, (size_t)OBJECT_BLOCK_SIZE);
  
  // obtain a new object block
  ObjectBlock* aNewObjectBlock = 
    (ObjectBlock*)calloc(1, sizeof(ObjectBlock));
  assert( IS_MEM_ALIGNED(aNewObjectBlock) );

  aNewObjectBlock->objectSize = objectSize;
  aNewObjectBlock->nextBlock  = NULL;
  
  // integrate this new object block into the linked list of
  // object blocks
  if ( theClass->rootObjectBlock ) {
    ObjectBlock *lastObjectBlock = theClass->rootObjectBlock;
    while ( lastObjectBlock->nextBlock ) {
      lastObjectBlock = lastObjectBlock->nextBlock;
    }
    assert(lastObjectBlock->nextBlock == NULL);
    lastObjectBlock->nextBlock = aNewObjectBlock;
  } else {
    theClass->rootObjectBlock = aNewObjectBlock;
  }
  DEBUG(theClass->jInterp, "addObjectBlock = %p %p\n", 
    theClass, aNewObjectBlock);

  // make sure this object block has some JObjs
  aNewObjectBlock->block = 
    calloc(OBJECT_BLOCK_SIZE, objectSize);
  assert( aNewObjectBlock->block );
  assert( IS_MEM_ALIGNED(aNewObjectBlock->block) );
  
  // add these new JObjs to the free list
  void* nextObject = aNewObjectBlock->block;
  for (size_t i = 1 ; i < OBJECT_BLOCK_SIZE ; i++) {
    assert( IS_MEM_ALIGNED(nextObject) );
    *(void**)nextObject = nextObject + objectSize;
    nextObject += objectSize;
  }
  *(void**)nextObject =
    (void*)theClass->freeObjects;
  theClass->freeObjects = aNewObjectBlock->block;
  DEBUG(theClass->jInterp, "addObjectBlock < %p %p %p\n",
    theClass, aNewObjectBlock, aNewObjectBlock->block);
}
\stopCCode

\startTestCase[must add new object block]
\startCTest
  // create the first list block and make sure it is 
  // properly integrated into linked list of list blocks

  AssertPtrNotNull(jInterp);
  JClass *theClass = jInterp->coAlgs[JInterpsTag];
  AssertPtrNotNull(theClass);
  
  AssertPtrNull(theClass->rootObjectBlock);
  AssertPtrNull(theClass->freeObjects);

  addObjectBlock(theClass);
  AssertPtrNotNull(theClass->rootObjectBlock);
  AssertPtrNotNull(theClass->rootObjectBlock->block);
  AssertPtrNull(theClass->rootObjectBlock->nextBlock);
  AssertPtrNotNull(theClass->freeObjects);

  // check to make sure freeObjects list is correctly linked
  void* nextObject  = theClass->rootObjectBlock->block;
  size_t objectSize = theClass->rootObjectBlock->objectSize;
  AssertIntEquals(objectSize, theClass->objectSize);
  
  for ( size_t i = 1 ; i < OBJECT_BLOCK_SIZE ; i++, nextObject += objectSize ) {
    AssertIntTrue(IS_MEM_ALIGNED(nextObject));
    AssertIntZero(asTag(nextObject));
    AssertIntZero(asFlags(nextObject));
    AssertPtrEquals( asType(nextObject), nextObject + objectSize);
  }
  AssertPtrNull(*((void**)nextObject));
  AssertPtrEquals(theClass->freeObjects,
    theClass->rootObjectBlock->block);

  // add another object block
  void* oldFreeObjects = theClass->freeObjects;
  addObjectBlock(theClass);
  AssertPtrNotNull(theClass->rootObjectBlock->nextBlock);
  AssertPtrNotNull(theClass->rootObjectBlock->nextBlock->block);
  AssertPtrNull(theClass->rootObjectBlock->nextBlock->nextBlock);
  AssertPtrNotNull(theClass->freeObjects);

  // check to make sure freeObjects list is correctly linked
  nextObject = theClass->rootObjectBlock->nextBlock->block;

  for (size_t i = 1 ; i < OBJECT_BLOCK_SIZE ; i++, nextObject += objectSize ) {
    AssertIntTrue(IS_MEM_ALIGNED(nextObject));
    AssertIntZero(asTag(nextObject));
    AssertIntZero(asFlags(nextObject));
    AssertPtrEquals(asType(nextObject), nextObject + objectSize);
  }
  AssertPtrEquals(asType(nextObject), oldFreeObjects);
  AssertPtrEquals(theClass->freeObjects,
    theClass->rootObjectBlock->nextBlock->block);
\stopCTest
\stopTestCase
\stopTestSuite

\subsection[title=Garbage collection]

Given that we explicitly implement memory pools for all fixed sized 
structures, the highest churn will be in the creation and release of small 
strings through interactions directly with the user and with the template 
engine. Since all strings used by JoyLoL will be part of the 
\type{Symbols} CoAlgebraic extension, we have complete control over the 
use of pointers to strings. This suggests that we allocate all 
\quote{small strings} using our own implementation of a collection of 
string pools. This allows us to periodically compact under-utilized pools 
which are highly fragmented by copying strings to either completely new 
string pools or inside existing string pools. By having a series of string 
pools we have the ability to provide the pools with a generational 
structure so that \quote{older} pools would tend to fragment less. 

However before we implement this strategy we should begin by instrumenting 
the use of small strings so we can gain an understanding of the spectrum of 
string sizes. 

\quote{Large strings} will tend to be produced as the output of either the 
user interaction or the template engine. We can reduce churn in these 
large strings by ensuring the user interaction interface and template 
engine either uses a list of strings (which the interface/engine 
implicitly concatenates), or uses an explicit string buffer. Initially any 
such string buffer can be based upon GNU Lib C's open_memstream interface, 
though we could directly implement our own string buffers should that be 
needed. 

\setCHeaderStream{private}
\startCHeader
extern void collectGarbage(JClass *theClass);
\stopCHeader
\setCHeaderStream{public}

\startCCode
void collectGarbage(JClass *theClass) {
  assert(theClass);
  
  DEBUG(theClass->jInterp, "collectGarbage %p\n", theClass);
  //
  // add a garbage collection mark-sweep here
  //
  // we will use a tri/quad colour mark-sweep algorithm
  // similar to that used by LuaJIT v3.0
  // see: http://wiki.luajit.org/New-Garbage-Collector
  // see: https://en.wikipedia.org/wiki/Tracing_garbage_collection
  // our "grey" list will be simply scanning a given 
  // object block for currently grey markings... and
  // we keep a pointer to the current object block and
  // where in the block we last checked for grey markings.
  // this means we can have an incremental mark/trace cycle
  // which can be run in small increments in each call to eval.
  //
}
\stopCCode

\startTestSuite[newObject]

To allocate a new \type{JObj} the first free object is taken from 
the linked list of free objects. If there are no remaining free objects, 
then we first attempt to collect any garbage and then if this fails to get 
any new free objects, a new \type{ObjectBlock} is allocated together with 
its collection of free objects. 

\startCHeader
#define newObject(jInterp, aTag)                              \
  (                                                           \
    assert(getJInterpClass(jInterp)->newObjectFunc),          \
    (getJInterpClass(jInterp)->newObjectFunc)(jInterp, aTag)  \
  )
\stopCHeader

\setCHeaderStream{private}
\startCHeader
extern JObj* newObjectImpl(
  JoyLoLInterp* jInterp,
  TagType aTag
);
\stopCHeader
\setCHeaderStream{public}

\startCCode
JObj* newObjectImpl(
  JoyLoLInterp* jInterp,
  TagType aTag
) {
  assert(jInterp);
  DEBUG(jInterp, "newObjectImpl(start) %p %zu\n", jInterp, (size_t)aTag);
  
  assert(jInterp->coAlgs);
  assert(aTag < jInterp->numCoAlgs);
  JClass *theClass = jInterp->coAlgs[aTag];
  assert(theClass);
  assert(theClass->tag == aTag);

  DEBUG(jInterp, "newObjectImpl freeObjects %p objectBlock %p\n",
    theClass->freeObjects,
    theClass->rootObjectBlock
  ); 

  // ensure there are some free objects
  if ( ! theClass->freeObjects ) 
    collectGarbage(theClass);
  if ( ! theClass->freeObjects )
    addObjectBlock(theClass);

  assert(theClass->freeObjects);

  JObj* aNewObject   = theClass->freeObjects;
  theClass->freeObjects = (JObj*)(aNewObject->type);

  asType(aNewObject)  = jInterp->coAlgs[aTag];
  asTag(aNewObject)   = aTag;
  asFlags(aNewObject) = 0;
  
  DEBUG(jInterp, "newObjectImpl(done) %p %zu\n", aNewObject, (size_t)aTag);
  return aNewObject;
}
\stopCCode

\startTestCase[Allocate one new JObj]
\startCTest
  AssertPtrNotNull(jInterp);
  AssertPtrNotNull(jInterp->coAlgs);
  TagType aTag = JInterpsTag;
  JClass *theClass = jInterp->coAlgs[aTag];
  AssertPtrNotNull(theClass);
  
  // a block has already been assigned from the last test
  // so we simply throw it away...
  theClass->freeObjects = NULL;
  theClass->rootObjectBlock = NULL;

  // get one new object to ensure our lazily
  // initialized object memory structures 
  // are initialized...
  JObj* aNewObject = newObject(jInterp, aTag);
  
  AssertPtrNotNull(theClass->freeObjects);
  AssertPtrNotNull(theClass->rootObjectBlock);
  AssertPtrNotNull(aNewObject);
  AssertPtrNotNull(asType(aNewObject));
  AssertPtrEquals(asType(aNewObject), theClass);
  AssertIntEquals(asTag(aNewObject), aTag);
  AssertIntZero(asFlags(aNewObject));
  AssertPtrEquals(aNewObject,
    theClass->rootObjectBlock->block);

  // now get one more object to ensure
  // we can properly deal with object blocks/memory
  
  JObj* oldFreeObjects =
    theClass->freeObjects;
  JObj* newfreeObjects = 
    (JObj*)theClass->freeObjects->type;

  aNewObject = newObject(jInterp, JInterpsTag);

  AssertPtrNotNull(aNewObject);
  AssertPtrNotNull(asType(aNewObject));
  AssertPtrEquals(asType(aNewObject), jInterp->coAlgs[JInterpsTag]);
  AssertIntEquals(asTag(aNewObject), JInterpsTag);
  AssertIntZero(asFlags(aNewObject));
  AssertPtrEquals(oldFreeObjects, aNewObject);
  AssertPtrEquals(newfreeObjects, theClass->freeObjects);
\stopCTest
\stopTestCase

\startTestCase[Allocate lots of new JObjs]

\startCTest
  AssertPtrNotNull(jInterp);

  JObj* aNewObject = NULL;
  for ( size_t i = 0; i < 3*OBJECT_BLOCK_SIZE; i++ ) {
    aNewObject = newObject(jInterp, JInterpsTag);
  }
  AssertPtrNotNull(aNewObject);
  AssertPtrNotNull(asType(aNewObject));
  AssertPtrEquals(asType(aNewObject), jInterp->coAlgs[JInterpsTag]);
  AssertIntEquals(asTag(aNewObject), JInterpsTag);
  AssertIntZero(asFlags(aNewObject));
\stopCTest
\stopTestCase
\stopTestSuite
