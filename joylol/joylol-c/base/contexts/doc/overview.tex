% A ConTeXt document [master document: contexts.tex]

\startsection[title=Goals]

JoyLoL Contexts form the kernel of the trampolined interpreter for JoyLoL. 

A Context consists of the data and process stacks organized as two lists 
of lists. The evaluation or interpretation of JoyLoL in the context of a 
Context consists of taking the top item off the process stack and 
\quote{performing} it. \quote{Performing} a given JoyLoL word, will result 
in changes being made to both the data and process stacks. A Context is 
interpreted in an eval loop, until there are no more JoyLoL words on the 
process stack. 

{\bf Question}: are threads (POSIX-threads/native-threads) identified with 
contexts? Or are contexts the equivalent of \quote{green threads} (i.e. 
\quote{threads} emulated by JoyLoL rather than identified with the 
underlying Operating System's threads)? If contexts are not identified 
with POSIX-threads then we will need to mutex contexts (green threads) as 
\quote{data} structures, since it will be possible for multiple running 
threads to alter the same context (as a data structure). If contexts are 
identified with POSIX-threads, then there is no sense to our previous 
(joyLoL20170316) commands to \quote{switch contexts}. 

\setCHeaderStream{public}
\startCHeader
typedef struct context_object_struct ContextObj;
\stopCHeader

\subsection{Thoughts}

The current implementation pops the top of the data of the old context and 
pushes it onto the *data* of the new context. 

However once something gets onto the data of a context it is "inert"... 
and won't be executed... any commands that we want to run in a different 
context need to be pushed on the process NOT the data 

What should happen with a given context's process stack is emptied? Should 
all processing (of everything) cease.. or should a context switch to its 
parent context? At the moment.... processing stops, which is I guess the 
correct continuation. 

The best way to think about context switching is as a potentially very 
large number of cooperating entities working on separate parts of a very 
large list of lists structure. The fact that they are cooperating means 
that they choose to hand-off to another context and hope to be resumed at 
some point in the "future" and restart computation where the original 
context handed off. This means that the process stack MUST NOT be altered. 
However the data stack might be and this represents the only way of 
communication between the contexts. Note that any given context can ignore 
the top item on its data stack.... 

Computations involving processes, will by necessity require cooperating 
contexts to perform useful but potentially infinite computations. 

What does switching contexts *mean* categorically?