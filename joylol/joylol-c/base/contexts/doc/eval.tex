% A ConTeXt document [master document: contexts.tex]

\section[title=Evaluation]

\startCCode
static void traceAction(
  ContextObj *aCtx,
  Symbol     *action,
  JObj       *aLoL
) {
  assert(aCtx);
  JoyLoLInterp *jInterp = aCtx->jInterp;
  assert(jInterp);
  DEBUG(jInterp, "traceAction: %p [%s] %p\n",
    aCtx, action, aLoL);
  if (aLoL && jInterp->debug) {
    assert(asType(aLoL));
    DEBUG(jInterp, "traceAction(lol): %p %p\n",
      aLoL, asType(aLoL));
  }

  StringBufferObj *aStrBuf = newStringBuffer(aCtx);
  strBufPrintf(aStrBuf, "%s: ", action);
  printLoL(aStrBuf, aLoL);
  strBufPrintf(aStrBuf, "\n");
  jInterp->writeStdOut(jInterp, getCString(aStrBuf));
  strBufClose(aStrBuf);
}
\stopCCode

\startCHeader
typedef void (EvalCommandInContext)(
  ContextObj *aCtx,
  JObj   *command
);

#define evalCommandInContext(aCtx, aCommand)      \
  (                                               \
    assert(aCtx),                                 \
    assert(getContextsClass(aCtx->jInterp)        \
      ->evalCommandInContextFunc),                \
    (getContextsClass(aCtx->jInterp)              \
      ->evalCommandInContextFunc(aCtx, aCommand)) \
  )
\stopCHeader

\setCHeaderStream{private}
\startCHeader
void evalCommandInContextImpl(
  ContextObj *aCtx,
  JObj   *command
);
\stopCHeader
\setCHeaderStream{public}

\startCCode
void evalCommandInContextImpl(
  ContextObj *aCtx,
  JObj       *command
) {
  assert(aCtx);
  JoyLoLInterp *jInterp = aCtx->jInterp;
  assert(jInterp);
  //assert(command);
  
  DEBUG(jInterp, "evalCommandInContext > %p [%s] %p\n",
    aCtx, aCtx->name, command);
  //
  // push this command onto the top of the process stack
  //
  pushCtxProcessImpl(aCtx, command);
  //
  while(aCtx->process) {
    //
    // ensure we have the most recent dictionary
    //
    DictObj *theDict = aCtx->dict;
    assert(theDict);
    
    if (aCtx->tracingOn) {
      StringBufferObj *aStrBuf = newStringBuffer(aCtx);
      strBufPrintf(aStrBuf,
        "\n----------------------------------------------------\n"
      );
      strBufPrintf(aStrBuf,
        "ctx: %s(%s)\n",
        aCtx->name, theDict->name
      );
      jInterp->writeStdOut(jInterp, getCString(aStrBuf));
      strBufClose(aStrBuf);
    }
    //
    // pop the next command off the process stack
    //
    command = popCtxProcessImpl(aCtx);
    aCtx->command = command;
    //assert(command);
    //
    if (isCFunction(command)) {
      //
      // if the command is a function.. call the function
      //
      if (isCtxCFunction(command)) {
        assert(asCtxCFunc(command));
        //
        // this is a CTX CFunction
        // (we allow it to change the current context
        //  in addition to any changes of the data and process stacks
        //  of either the old or new contexts)
        //
        if (aCtx->tracingOn) 
          traceAction(aCtx, "calling(c-ctx)", command);
        aCtx = (asCtxCFunc(command))(aCtx);
        assert(aCtx);
        //
      } else {
        assert(asCFunc(command));
        //
        // this is a normal CFunction
        // (this ONLY makes changes to the data and process stacks)
        //
        if (aCtx->tracingOn) 
          traceAction(aCtx, "calling(c)", command);
        (asCFunc(command))(aCtx);
        //
      }
    } else if (isAssertion(command)) {
      //
      // this is an assertion ...
      //   ... so assert it 
      //
      if (aCtx->tracingOn) 
        traceAction(aCtx, "asserting", command);
      //
      if (!evalAssertionInContextImpl(aCtx, (AssertionObj*)command)) {
        //
        // this assertion failed... 
        //   ... so report it
        //
        pushNullCtxDataImpl(aCtx);
        pushOnTopCtxDataImpl(aCtx, command);
        raiseException(aCtx,
          "assertion failed"
        );
      }
    } else if (!isSymbol(command)) {
      //
      // if the command is not a Symbol ...
      //  ...  push it onto the top of the data stack
      //
      if (aCtx->tracingOn)
        traceAction(aCtx, "adding(nonSym)", command);
      pushCtxDataImpl(aCtx, command);
      //
    } else {
      //
      // if the command is a symbol ...
      //  ... look up the symbol's association in the dictionary
      //
      DictNodeObj* assoc = getSymbolEntry(theDict, asSymbol(command));
      assert(assoc);
      //
      if (!assoc->value) {
        //
        // if the association is empty.. push this symbol onto the top
        // of the data stack (re-evaluating this symbol would lead to an
        // infinite loop)
        //
        if (aCtx->tracingOn)
          traceAction(aCtx, "adding(noValue)", command);
        pushCtxDataImpl(aCtx, command);
        //
      } else if (isPair(assoc->value)) {
        //
        // if the association is a LoL.. push this LoL onto the top of the
        // process stack
        //
        if (aCtx->tracingOn) {
          traceAction(aCtx, "calling(joylol)", command);
          traceAction(aCtx, "evaluating", assoc->value);
        }
        prependListCtxProcess(aCtx,
          copyLoL(jInterp, assoc->value));
        //
      } else if (isCFunction(assoc->value)) {
        //
        // if the association is a function.. call the function
        //
        if (isCtxCFunction(assoc->value)) {
          assert(asCtxCFunc(assoc->value));
          //
          // this is a CTX CFunction
          //
          if (aCtx->tracingOn) 
            traceAction(aCtx, "calling(c-ctx)", command);
          aCtx = (asCtxCFunc(assoc->value))(aCtx);
          assert(aCtx);
          //
        } else {
          assert(asCFunc(assoc->value));
          //
          // this is a normal CFunction
          //
          if (aCtx->tracingOn)
            traceAction(aCtx, "calling(c)", command);
          (asCFunc(assoc->value))(aCtx);
          //
        }
      } else {
        //
        // if the association is NOT a PairAtom or Function...
        // ... push this new ATOM onto the top of the process stack
        //
        if (aCtx->tracingOn) {
          traceAction(aCtx, "calling(joylol)", command);
          traceAction(aCtx, "evaluating", assoc->value);
        }
        pushCtxProcessImpl(aCtx, assoc->value);
        //
      }
    }
    if (aCtx->tracingOn) {
      DEBUG(jInterp, "evalCommandInContext -> tracing%s\n", "");
      StringBufferObj *aStrBuf = newStringBuffer(aCtx);
      showCtxData(aCtx, aStrBuf);
      showCtxProcess(aCtx, aStrBuf);
      jInterp->writeStdOut(jInterp, getCString(aStrBuf));
      strBufClose(aStrBuf);
      DEBUG(jInterp, "evalCommandInContext <- tracing%s\n", "");
    }
  } // aCtx->process is empty
  DEBUG(jInterp, "evalCommandInContext < %p %p\n", aCtx, command);
}
\stopCCode

\startCHeader
typedef void (EvalContext)(
  ContextObj *aCtx
);

#define evalContext(aCtx)                   \
  (                                         \
    assert(aCtx),                           \
    assert(getContextsClass(aCtx->jInterp)  \
      ->evalContextFunc),                   \
    (getContextsClass(aCtx->jInterp)        \
      ->evalContextFunc(aCtx))              \
  )
\stopCHeader

\setCHeaderStream{private}
\startCHeader
void evalContextImpl(
  ContextObj *aCtx
);
\stopCHeader
\setCHeaderStream{public}

\startCCode
void evalContextImpl(
  ContextObj *aCtx
) {
  popCtxProcessIntoImpl(aCtx, aCommand);
  evalCommandInContext(aCtx, aCommand);
}
\stopCCode


\startTestSuite[evalAssertionInContext]

\startCHeader
typedef Boolean (EvalAssertionInContext)(
  ContextObj   *aCtx,
  AssertionObj *anAssertion
);

#define evalAssertionInContext(aCtx, anAssertion)       \
  (                                                     \
    assert(aCtx),                                       \
    assert(getContextsClass(aCtx->jInterp)              \
      ->evalAssertionInContextFunc),                    \
    (getContextsClass(aCtx->jInterp)                    \
      ->evalAssertionInContextFunc(aCtx, anAssertion))  \
  )
\stopCHeader

\setCHeaderStream{private}
\startCHeader
extern Boolean evalAssertionInContextImpl(
  ContextObj   *aCtx,
  AssertionObj *anAssertion
);
\stopCHeader
\setCHeaderStream{public}

\startCCode
Boolean evalAssertionInContextImpl(
  ContextObj   *aCtx,
  AssertionObj *anAssertion
) {
  assert(aCtx);
  JoyLoLInterp *jInterp = aCtx->jInterp;
  assert(jInterp);
  DictObj *theDict = aCtx->dict;
  assert(theDict);
  
  DEBUG(jInterp, "evalAssertionInContext > %p [%s] %p\n",
    aCtx, aCtx->name, anAssertion);

  char *metaDictName = calloc(10+strlen(theDict->name), sizeof(char));
  assert(metaDictName);
  strcat(metaDictName, "meta-");
  strcat(metaDictName, theDict->name);
  
  DictObj *metaDict = newDictionary(
    jInterp,
    metaDictName,
    theDict
  );
  
  char *metaCtxName = calloc(10+strlen(aCtx->name), sizeof(char));
  assert(metaCtxName);
  strcat(metaCtxName, "meta-");
  strcat(metaCtxName, aCtx->name);
  
  ContextObj *metaCtx = newContextImpl(
    jInterp,
    metaCtxName,
    aCtx,
    metaDict,
    aCtx->data,
    NULL
  );
  assert(metaCtx);
  metaCtx->showStack = aCtx->showStack;
  metaCtx->tracingOn = aCtx->tracingOn;
  JObj *originalTop  = aCtx->data;
  
  Boolean assertionTrue = TRUE;

  JObj *assertionList = asAssertion(anAssertion);
  while(assertionList && !(metaCtx->exceptionRaised)) {
    JObj *assertionTest = assertionList;
    if (isPair(assertionList)) {
      assertionTest = asCar(assertionList);
      assertionList = asCdr(assertionList);
    } else {
      assertionList = NULL;
    }

    if (metaCtx->tracingOn)
      traceAction(aCtx,
        "\n-------------------------\nevalAssertion(test)",
        assertionTest
      );
    prependListCtxProcess(metaCtx, assertionTest);
    popCtxProcessIntoImpl(metaCtx, aCommand);
    evalCommandInContextImpl(metaCtx, aCommand);
    
    if (metaCtx->data == originalTop) {
      // our assertion has not returned any result value
      pushNullCtxDataImpl(metaCtx);
      pushOnTopCtxDataImpl(metaCtx, assertionTest);
      raiseExceptionImpl(metaCtx,
        "assertion has not returned any result value"
        );
      metaCtx->exceptionRaised = TRUE;
      extractAP(metaCtx);
    } else if ( !isPair(metaCtx->data) ) {
      // our assertions have corrupted the data stack
      pushNullCtxDataImpl(metaCtx);
      pushOnTopCtxDataImpl(metaCtx, assertionTest);
      raiseExceptionImpl(metaCtx,
        "assertion has corrupted the data stack"
        );      
      metaCtx->exceptionRaised = TRUE;
      extractAP(metaCtx);
    } else if ( asCdr(metaCtx->data) != originalTop ) {
      // our assertions have corrupted the original context's data stack
      pushNullCtxDataImpl(metaCtx);
      pushOnTopCtxDataImpl(metaCtx, assertionTest);
      raiseExceptionImpl(metaCtx,
        "assertion has returned too many values OR corrupted the data stack"
        );
      metaCtx->exceptionRaised = TRUE;
      extractAP(metaCtx);
    } else {
      popCtxDataIntoImpl(metaCtx, assertionResult);
      if (isFalse(assertionResult)) {
        assertionTrue = FALSE;
      }
    }
  }
  
  if (metaCtx->exceptionRaised) {
    assertionTrue = FALSE;
    reportException(metaCtx);
  }
  
  if (metaDictName) free(metaDictName);
  metaDictName = NULL;
  
  if (metaCtxName) free(metaCtxName);
  metaCtxName = NULL;
  
  DEBUG(jInterp, "evalAssertionInContext < %p [%s] %p\n",
    aCtx, aCtx->name, anAssertion);
    
  return assertionTrue;
}
\stopCCode

\startTestCase[should evaluate assertions]
\startCTest
  AssertPtrNotNull(jInterp);
  ContextObj *rootCtx = jInterp->rootCtx;
  AssertPtrNotNull(rootCtx);
  DictObj *rootDict = rootCtx->dict;
  AssertPtrNotNull(rootDict);
  
  JObj *trueBoolean      = newBoolean(jInterp, TRUE);
  JObj *aTrueAssertion   = newAssertion(jInterp, trueBoolean);
  JObj *falseBoolean     = newBoolean(jInterp, FALSE);
  JObj *aFalseAssertion  = newAssertion(jInterp, falseBoolean);
  JObj *isBooleanSym     = newSymbol(jInterp, "isBoolean", "CTest", 1);
  JObj *dupSym           = newSymbol(jInterp, "dup1D", "CTest", 2);
  JObj *complexBody      = concatLists(jInterp,
    trueBoolean,
    newPair(jInterp, 
      concatLists(jInterp, dupSym, isBooleanSym),
      NULL
    )
  );
  JObj *complexAssertion = newAssertion(jInterp, complexBody);
  
  DictObj    *aDict = newDictionary(
    jInterp,
    "assertionCTestsDict",
    rootDict
  );
  ContextObj* aCtx = newContext(
    jInterp,
    "assertionCTestsCtx",
    NULL,
    aDict,
    NULL,
    NULL
  );
  AssertPtrNotNull(aCtx);
  //aCtx->showStack = TRUE;
  //aCtx->tracingOn = TRUE;
  //jInterp->debug  = TRUE;
  pushCtxData(aCtx, falseBoolean);

  Boolean result =
    evalAssertionInContextImpl(aCtx, (AssertionObj*)aTrueAssertion);
  AssertIntTrue(result);
  
  result =
    evalAssertionInContextImpl(aCtx, (AssertionObj*)aFalseAssertion);
  AssertIntFalse(result);

  result =
    evalAssertionInContextImpl(aCtx, (AssertionObj*)complexAssertion);
//  AssertIntTrue(result);
\stopCTest
\stopTestCase
\stopTestSuite

\startCHeader
typedef void (EvalTextInContext)(
  ContextObj *aCtx,
  TextObj    *aText
);

#define evalTextInContext(aCtx, aText)      \
  (                                         \
    assert(aCtx),                           \
    assert(getContextsClass(aCtx->jInterp)  \
      ->evalTextInContextFunc),             \
    (getContextsClass(aCtx->jInterp)        \
      ->evalTextInContextFunc(aCtx, aText)) \
  )
\stopCHeader

\setCHeaderStream{private}
\startCHeader
extern void evalTextInContextImpl(
  ContextObj *aCtx,
  TextObj    *aText
);
\stopCHeader
\setCHeaderStream{public}

\startCCode
void evalTextInContextImpl(
  ContextObj *aCtx,
  TextObj    *aText
) {
  assert(aCtx);
  JoyLoLInterp *jInterp = aCtx->jInterp;
  assert(jInterp);

  DEBUG(jInterp, "evalTextInContext > %p %p\n", aCtx, aText);
  while(TRUE) {
    JObj* aLoL = parseOneSymbol(aText);
    if (aCtx->showStack) {
      StringBufferObj *aStrBuf = newStringBuffer(aCtx);
      strBufPrintf(aStrBuf, "<");
      if (aLoL && isSignal(aLoL) &&
        asSignal(aLoL) == SIGNAL_END_OF_TEXT) {
        strBufPrintf(aStrBuf, " {EOT}");
      } else {
        printLoL(aStrBuf, aLoL);
      }
      strBufPrintf(aStrBuf, "\n");
      jInterp->writeStdOut(jInterp, getCString(aStrBuf));
      strBufClose(aStrBuf);
    }
    if (aLoL && isSignal(aLoL) &&
      asSignal(aLoL) == SIGNAL_END_OF_TEXT) break;
    evalCommandInContext(aCtx, aLoL);
    if (aCtx->showStack) {
      StringBufferObj *aStrBuf = newStringBuffer(aCtx);
      strBufPrintf(aStrBuf, ">");
      printLoL(aStrBuf, aCtx->data);
      strBufPrintf(aStrBuf, "\n");
      jInterp->writeStdOut(jInterp, getCString(aStrBuf));
      strBufClose(aStrBuf);
    }
    if (aCtx->exceptionRaised) break;
  }
  reportException(aCtx);
  DEBUG(jInterp, "evalTextInContext < %p %p\n", aCtx, aText);
}
\stopCCode

\startCHeader
typedef void (EvalArrayOfStringsInContext)(
  ContextObj *aCtx,
  Symbol     *someStrings[]
);

#define evalArrayOfStringsInContext(aCtx, someStrings)      \
  (                                                         \
    assert(aCtx),                                           \
    assert(getContextsClass(aCtx->jInterp)                  \
      ->evalArrayOfStringsInContextFunc),                   \
    (getContextsClass(aCtx->jInterp)                        \
      ->evalArrayOfStringsInContextFunc(aCtx, someStrings)) \
  )
\stopCHeader

\setCHeaderStream{private}
\startCHeader
extern void evalArrayOfStringsInContextImpl(
  ContextObj *aCtx,
  Symbol     *someStrings[]
);
\stopCHeader
\setCHeaderStream{public}

\startCCode
void evalArrayOfStringsInContextImpl(
  ContextObj *aCtx,
  Symbol     *someStrings[]
) {
  assert(aCtx);
  assert(aCtx->jInterp);
  DEBUG(aCtx->jInterp, "evalArrayOfStringsInContext > %p %p\n",
    aCtx, someStrings);
  assert(aCtx);
  TextObj* stringText =
    createTextFromArrayOfStrings(
      aCtx->jInterp,
      someStrings
    );
  evalTextInContext(aCtx, stringText);
  freeText(stringText);
  DEBUG(aCtx->jInterp, "evalArrayOfStringsInContext < %p %p\n",
    aCtx, someStrings);
}
\stopCCode

\startCHeader
typedef void (EvalStringInContext)(
  ContextObj *aCtx,
  Symbol     *aString
);

#define evalStringInContext(aCtx, aString)      \
  (                                             \
    assert(aCtx),                               \
    assert(getContextsClass(aCtx->jInterp)      \
      ->evalStringInContextFunc),               \
    (getContextsClass(aCtx->jInterp)            \
      ->evalStringInContextFunc(aCtx, aString)) \
  )
\stopCHeader

\setCHeaderStream{private}
\startCHeader
extern void evalStringInContextImpl(
  ContextObj *aCtx,
  Symbol     *aString
);
\stopCHeader
\setCHeaderStream{public}

\startCCode
void evalStringInContextImpl(
  ContextObj *aCtx,
  Symbol     *aString
) {
  assert(aCtx);
  assert(aCtx->jInterp);
  DEBUG(aCtx->jInterp, "evalStringInContext > %p [%s]\n", aCtx, aString);
  assert(aCtx);
  TextObj* stringText =
    createTextFromString(
      aCtx->jInterp,
      aString
    );
  evalTextInContext(aCtx, stringText);
  freeText(stringText);
  DEBUG(aCtx->jInterp, "evalStringInContext < %p [%s]\n", aCtx, aString);
}
\stopCCode