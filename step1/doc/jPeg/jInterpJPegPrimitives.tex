% A ConTeXt document [master document: ../jInterp.tex ]

\subsection[title=Primitives]

We need to implement the following \joylol\ words:

\startitemize[1]

\item \bold{\type{Fail}}

\item \bold{\type{Commit}}

\item \bold{\type{Choose}}

\item \bold{\type{RepeatAtLeast}}

\item \bold{\type{RepeatAtMost}}

\item \bold{\type{Char}}

\item \bold{\type{CharSet}}

\item \bold{\type{Any}}

\stopitemize

\TODO{We have not covered \quote{\bold{\type{not}}}, 
\quote{\bold{\type{and}}}, CharSet difference, captures, or actions on 
captures.} 

As suggested in \cite{ierusalimschy2008lpegArticle} we might implement 
captures using a \type{Capture} call with three distinct arguments: 

\startitemize[n] 

\item \type{begin n} where the actual capture begins $n$ characters before 
the current character. This capture method should be called \emph{as soon 
as} it is known that the current path will succeed. 

\item \type{end} this marks the end of a capture 

\item \type{full n} this is the same as a \type{begin n} immediately 
followed by an \type{end}. 

\stopitemize 

For all of the above \joylol\ words, the data stack must include both the 
current text structure (which must include a indication of the current 
character) as well as the current collection of captures. 

A compiler is a pipeline of co-routines. The parser might itself be a 
pipeline of co-routines, one for the lexer, one for the ultimate parser, 
but there could be numerous intermediary co-routines parsing more complex 
syntactic structures. 
