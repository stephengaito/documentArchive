% A ConTeXt document [master document: ../jInterp.tex]

\subsection[title=Basic words]

\subsubsection[title=Boolean words]
\startCCode
void isTrueWord(ContextObj *aCtx) {
  popCtxDataInto(aCtx, top);
  if (isTrueObj(top)) {
    pushBooleanCtxData(aCtx, true);
  } else
    pushBooleanCtxData(aCtx, false);
}

void isFalseWord(ContextObj *aCtx) {
  popCtxDataInto(aCtx, top);
  if (isTrueObj(top)) {
    pushBooleanCtxData(aCtx, false);
  } else
    pushBooleanCtxData(aCtx, true);
}
#define notWord ifFalseWord

void orWord(ContextObj *aCtx) {
  popCtxDataInto(aCtx, first);
  popCtxDataInto(aCtx, second);
  
  if (isTrueObj(first)) {
    pushBooleanCtxData(aCtx, true);
  } else if (isTrueObj(second)) {
    pushBooleanCtxData(aCtx, true);
  } else {
    pushBooleanCtxData(aCtx, false);
  }
}

void andWord(ContextObj *aCtx) {
  popCtxDataInto(aCtx, first);
  popCtxDataInto(aCtx, second);
  
  if (isTrueObj(first) && isFalseObj(second)) {
    pushBooleanCtxData(aCtx, true);
  } else {
    pushBooleanCtxData(aCtx, false);
  }
}

void registerBooleanWords(ContextObj *aCtx) {
  assert(isContextObj(aCtx));  
  SymbolTableObj *symTab = asCtxSymbols(aCtx);
  
  UIntegerObj *trueUInt  = newUInteger(true);
  insertSymbol(symTab, "true", asJObj(trueUInt));

  UIntegerObj *falseUInt = newUInteger(false);
  insertSymbol(symTab, "false", asJObj(falseUInt));

  registerCFunction(symTab, "isTrue",    isTrueWord);
  registerCFunction(symTab, "isFalse",   isFalseWord);
  registerCFunction(symTab, "not",       isFalseWord);
  registerCFunction(symTab, "or",        orWord);
  registerCFunction(symTab, "and",       andWord);
}
\stopCCode

\subsubsection[title=Symbol table words]

\startCCode
void lookupWord(ContextObj* aCtx) {
  assert(isContextObj(aCtx));
  popCtxDataInto(aCtx, name2lookup);

  if (!isSymbolObj(name2lookup)) {
    raiseExceptionMsg(aCtx,
      "lookup requires a symbol as top");
    return;
  }

  pushCtxData(aCtx, 
    findSymbol(asCtxSymbols(aCtx), asSymbol(name2lookup))
  );
}

void registerSymbolTableWords(ContextObj *aCtx) {
  assert(isContextObj(aCtx));  
  SymbolTableObj *symTab = asCtxSymbols(aCtx);

  registerCFunction(symTab, "lookup", lookupWord);
}
\stopCCode

\subsubsection[title=Output words]

\startCCode
static void doWrite(
  ContextObj      *aCtx,
  Boolean          newLine
) {
  assert(isContextObj(aCtx));
  
  popCtxDataInto(aCtx, aString);
  if (!isSymbolObj(aString)) {
    raiseExceptionMsg(aCtx,
      "write expected a symbol as aString");
    return;
  }

  assert(isStringBufferObj(asRenderBuffer(aCtx)));
  strBufPrintf(asRenderBuffer(aCtx), "%s", asSymbol(aString));
  if (newLine) strBufPrintf(asRenderBuffer(aCtx), "%s", "\n");
}

static void writeWord(ContextObj *aCtx) {
  doWrite(aCtx, false);
}

static void writeNLWord(ContextObj *aCtx) {
  doWrite(aCtx, true);
}

static void writeObjWord(ContextObj* aCtx) {
  assert(isContextObj(aCtx));
  
  popCtxDataInto(aCtx, lol);
  popCtxDataInto(aCtx, depth);
  
  if (!isUIntegerObj(depth)) {
    raiseExceptionMsg(aCtx,
      "printLoL expected a UInteger as depth");
    return;
  }
  size_t depthInt = asUInteger(depth);
    
  assert(isStringBufferObj(asRenderBuffer(aCtx)));
  printJObj(asRenderBuffer(aCtx), lol, "", depthInt);
}

static void printRenderBuffer(ContextObj* aCtx) {
  assert(isContextObj(aCtx));
  assert(isStringBufferObj(asRenderBuffer(aCtx)));
  
  printf("%s", getCString(asRenderBuffer(aCtx)));
}
\stopCCode

\startCCode
void registerOutputWords(ContextObj *aCtx) {
  assert(isContextObj(aCtx));
  SymbolTableObj *symTab = asCtxSymbols(aCtx);
  assert(isSymbolTableObj(symTab));
  
  registerCFunction(symTab, "writeStr",    writeWord);
  registerCFunction(symTab, "writeNL",     writeNLWord);
  registerCFunction(symTab, "writeObj",    writeObjWord);
  registerCFunction(symTab, "printBuffer", printRenderBuffer);
}
\stopCCode

\subsubsection[title=UInterger words]

\startCCode
static void addWord(ContextObj* aCtx) {
  assert(isContextObj(aCtx));

  popCtxDataInto(aCtx, top1);
  popCtxDataInto(aCtx, top2);
  if (!isUIntegerObj(top1) || !isUIntegerObj(top2)) {
    raiseExceptionMsg(aCtx,
      "addition requires that the top two stack elements are UIntegers"
    );
    return;
  }
  UIntegerObj* sum = newUInteger(asUInteger(top1) + asUInteger(top2));
  pushCtxData(aCtx, asJObj(sum));
}
\stopCCode

\startCCode
static void subtractWord(ContextObj* aCtx) {
  assert(isContextObj(aCtx));

  popCtxDataInto(aCtx, top1);
  popCtxDataInto(aCtx, top2);
  if (!isUIntegerObj(top1) || !isUIntegerObj(top2)) {
    raiseExceptionMsg(aCtx,
      "subtraction requires that the top two stack elements are UIntegers"
    );
    return;
  }
  UInteger uIntDiff = 0;
  if (asUInteger(top2) < asUInteger(top1)) {
    uIntDiff = asUInteger(top1) - asUInteger(top2);
  }
  UIntegerObj* difference = newUInteger(uIntDiff);
  pushCtxData(aCtx, asJObj(difference));
}
\stopCCode

\startCCode
static void subtractReverseWord(ContextObj* aCtx) {
  assert(isContextObj(aCtx));

  popCtxDataInto(aCtx, top1);
  popCtxDataInto(aCtx, top2);
  if (!isUIntegerObj(top1) || !isUIntegerObj(top2)) {
    raiseExceptionMsg(aCtx,
      "(reverse) subtraction requires that the top two stack elements are UIntegers"
    );
    return;
  }
  UInteger uIntDiff = 0;
  if (asUInteger(top1) < asUInteger(top2)) {
    uIntDiff = asUInteger(top2) - asUInteger(top1);
  }
  UIntegerObj* difference = newUInteger(uIntDiff);
  pushCtxData(aCtx, asJObj(difference));
}
\stopCCode

\startCCode
static void multiplyWord(ContextObj* aCtx) {
  assert(isContextObj(aCtx));
  
  popCtxDataInto(aCtx, top1);
  popCtxDataInto(aCtx, top2);
  if (!isUIntegerObj(top1) || !isUIntegerObj(top2)) {
    raiseExceptionMsg(aCtx,
      "multiplication requires that the top two stack elements are UIntegers"
    );
    return;
  }
  UIntegerObj* product = newUInteger(asUInteger(top1) * asUInteger(top2));
  pushCtxData(aCtx, asJObj(product));
}
\stopCCode

\startCCode
static void quotientWord(ContextObj* aCtx) {
  assert(isContextObj(aCtx));

  popCtxDataInto(aCtx, top1);
  popCtxDataInto(aCtx, top2);
  if (!isUIntegerObj(top1) || !isUIntegerObj(top2)) {
    raiseExceptionMsg(aCtx,
      "integer quotient requires that the top two stack elements are UIntegers"
    );
    return;
  }
  if (asUInteger(top2) == 0) {
    raiseExceptionMsg(aCtx,
      "integer quotient requires that the denominator is not zero"
    );
    return;
  }
  UIntegerObj* product = newUInteger(asUInteger(top1) % asUInteger(top2));
  pushCtxData(aCtx, asJObj(product));
}

static void quotientReverseWord(ContextObj* aCtx) {
  assert(isContextObj(aCtx));

  popCtxDataInto(aCtx, top1);
  popCtxDataInto(aCtx, top2);
  if (!isUIntegerObj(top1) || !isUIntegerObj(top2)) {
    raiseExceptionMsg(aCtx,
      "integer quotient requires that the top two stack elements are UIntegers"
    );
    return;
  }
  if (asUInteger(top1) == 0) {
    raiseExceptionMsg(aCtx,
      "integer quotient requires that the denominator is not zero"
    );
    return;
  }
  UIntegerObj* product = newUInteger(asUInteger(top2) % asUInteger(top1));
  pushCtxData(aCtx, asJObj(product));
}

static void remainderWord(ContextObj* aCtx) {
  assert(isContextObj(aCtx));

  popCtxDataInto(aCtx, top1);
  popCtxDataInto(aCtx, top2);
  if (!isUIntegerObj(top1) || !isUIntegerObj(top2)) {
    raiseExceptionMsg(aCtx,
      "integer remainder requires that the top two stack elements are UIntegers"
    );
    return;
  }
  if (asUInteger(top2) == 0) {
    raiseExceptionMsg(aCtx,
      "integer remainder requires that the denominator is not zero"
    );
    return;
  }
  UIntegerObj* product = newUInteger(asUInteger(top1) / asUInteger(top2));
  pushCtxData(aCtx, asJObj(product));
}

static void remainderReverseWord(ContextObj* aCtx) {
  assert(isContextObj(aCtx));

  popCtxDataInto(aCtx, top1);
  popCtxDataInto(aCtx, top2);
  if (!isUIntegerObj(top1) || !isUIntegerObj(top2)) {
    raiseExceptionMsg(aCtx,
      "integer remainder requires that the top two stack elements are UIntegers"
    );
    return;
  }
  if (asUInteger(top1) == 0) {
    raiseExceptionMsg(aCtx,
      "integer remainder requires that the denominator is not zero"
    );
    return;
  }
  UIntegerObj* product = newUInteger(asUInteger(top2) / asUInteger(top1));
  pushCtxData(aCtx, asJObj(product));
}
\stopCCode

\startCCode
static void lessThanUIntWord(ContextObj* aCtx) {
  assert(isContextObj(aCtx));

  popCtxDataInto(aCtx, top1);
  popCtxDataInto(aCtx, top2);
  
  UInteger result = false;
  if (isUIntegerObj(top1) &&
      isUIntegerObj(top2) &&
      (asUInteger(top1) < asUInteger(top2))) result = true;
  DEBUG(ContextDebug, "equalNat: %zu\n", result);
  pushCtxData(aCtx, top2);
  pushCtxData(aCtx, top1);
  pushCtxData(aCtx, asJObj(newUInteger(result)));
}
\stopCCode

\startCCode
static void lessThanEqualUIntWord(ContextObj* aCtx) {
  assert(isContextObj(aCtx));

  popCtxDataInto(aCtx, top1);
  popCtxDataInto(aCtx, top2);
  UInteger result = false;
  if (isUIntegerObj(top1) &&
      isUIntegerObj(top2) &&
      (asUInteger(top1) <= asUInteger(top2))) result = true;
  DEBUG(ContextDebug, "equalNat: %zu\n", result);
  pushCtxData(aCtx, top2);
  pushCtxData(aCtx, top1);
  pushCtxData(aCtx, asJObj(newUInteger(result)));
}
\stopCCode

\startCCode
static void greaterThanUIntWord(ContextObj* aCtx) {
  assert(isContextObj(aCtx));

  popCtxDataInto(aCtx, top1);
  popCtxDataInto(aCtx, top2);
  UInteger result = false;
  if (isUIntegerObj(top1) &&
      isUIntegerObj(top2) &&
      (asUInteger(top1) > asUInteger(top2))) result = true;
  DEBUG(ContextDebug, "equalNat: %zu\n", result);
  pushCtxData(aCtx, top2);
  pushCtxData(aCtx, top1);
  pushCtxData(aCtx, asJObj(newUInteger(result)));
}
\stopCCode

\startCCode
static void greaterThanEqualUIntWord(ContextObj* aCtx) {
  assert(isContextObj(aCtx));
  
  popCtxDataInto(aCtx, top1);
  popCtxDataInto(aCtx, top2);
  UInteger result = false;
  if (isUIntegerObj(top1) &&
      isUIntegerObj(top2) &&
      (asUInteger(top1) >= asUInteger(top2))) result = true;
  DEBUG(ContextDebug, "equalNat: %zu\n", result);
  pushCtxData(aCtx, top2);
  pushCtxData(aCtx, top1);
  pushCtxData(aCtx, asJObj(newUInteger(result)));
}
\stopCCode

\startCCode
static void equalUIntWord(ContextObj* aCtx) {
  assert(isContextObj(aCtx));

  popCtxDataInto(aCtx, top1);
  popCtxDataInto(aCtx, top2);
  UInteger result = false;
  if (isUIntegerObj(top1) &&
      isUIntegerObj(top2) &&
      (asUInteger(top1) == asUInteger(top2))) result = true;
  DEBUG(ContextDebug, "equalNat: %zu\n", result);
  pushCtxData(aCtx, top2);
  pushCtxData(aCtx, top1);
  pushCtxData(aCtx, asJObj(newUInteger(result)));
}
\stopCCode

\startCCode
static void isZeroWord(ContextObj* aCtx) {
  assert(isContextObj(aCtx));

  popCtxDataInto(aCtx, top);
  UInteger result = false;
  if (isUIntegerObj(top) &&
      (asUInteger(top) == 0)) result = true;
  pushCtxData(aCtx, top);
  pushCtxData(aCtx, asJObj(newUInteger(result)));
}
\stopCCode

\startCCode
static void isUIntegerWord(ContextObj* aCtx) {
  assert(isContextObj(aCtx));

  popCtxDataInto(aCtx, top);
  UInteger result = false;
  if (isUIntegerObj(top)) result = true;
  pushCtxData(aCtx, top);
  pushCtxData(aCtx, asJObj(newUInteger(result)));
}
\stopCCode

\startCCode
void registerUIntegerWords(ContextObj *aCtx) {
  assert(isContextObj(aCtx));
  SymbolTableObj *symTab = asCtxSymbols(aCtx);
  assert(isSymbolTableObj(symTab));
  
  registerCFunction(symTab, "isUInteger", isUIntegerWord);
  registerCFunction(symTab, "+",          addWord);
  registerCFunction(symTab, "-",          subtractWord);
  registerCFunction(symTab, "-rev",       subtractReverseWord);
  registerCFunction(symTab, "*",          multiplyWord);
  registerCFunction(symTab, "/",          quotientWord);
  registerCFunction(symTab, "/rev",       quotientReverseWord);
  registerCFunction(symTab, "%",          remainderWord);
  registerCFunction(symTab, "%rev",       remainderReverseWord);
  registerCFunction(symTab, "<UInt",      lessThanUIntWord);
  registerCFunction(symTab, "<=UInt",     lessThanEqualUIntWord);
  registerCFunction(symTab, ">UInt",      greaterThanUIntWord);
  registerCFunction(symTab, ">=UInt",     greaterThanEqualUIntWord);
  registerCFunction(symTab, "=UInt",      equalUIntWord);
  registerCFunction(symTab, "isZero",     isZeroWord);
}
\stopCCode

\subsubsection[title=List words]

\startCCode
static void isNilWord(ContextObj* aCtx) {
  assert(isContextObj(aCtx));

  popCtxDataInto(aCtx, top);
  Boolean result = false;
  if (!top) result = true;
  pushCtxData(aCtx, top);
  pushCtxData(aCtx, asJObj(newUInteger(result)));
}
\stopCCode

\startCCode
static void isPairWord(ContextObj* aCtx) {
  assert(isContextObj(aCtx));

  popCtxDataInto(aCtx, top);
  Boolean result = false;
  if (isPairObj(top)) result = true;
  pushCtxData(aCtx, top);
  pushCtxData(aCtx, asJObj(newUInteger(result)));
}
\stopCCode

\startCCode
static void isAtomWord(ContextObj* aCtx) {
  assert(isContextObj(aCtx));

  popCtxDataInto(aCtx, top);
  Boolean result = false;
  if (isAtomObj(top)) result = true;
  pushCtxData(aCtx, top);
  pushCtxData(aCtx, asJObj(newUInteger(result)));
}
\stopCCode

\startCCode
static void pushEmptyList(ContextObj *aCtx) {
  assert(isContextObj(aCtx));
  
  pushNullCtxData(aCtx);
}
\stopCCode

\startCCode
void registerListWords(ContextObj *aCtx) {
  assert(isContextObj(aCtx));
  SymbolTableObj *symTab = asCtxSymbols(aCtx);
  assert(isSymbolTableObj(symTab));
  
  registerCFunction(symTab, "isNil",     isNilWord);
  registerCFunction(symTab, "isList",    isPairWord);
  registerCFunction(symTab, "isPair",    isPairWord);
  registerCFunction(symTab, "isAtom",    isAtomWord);
  registerCFunction(symTab, "emptyList", pushEmptyList);
}
\stopCCode

\section[title=Symbol words]

\startCCode
static void isSymbolWord(ContextObj* aCtx) {
  assert(isContextObj(aCtx));

  popCtxDataInto(aCtx, top);
  Boolean result = false;
  if (isSymbolObj(top)) result = true;
  pushCtxData(aCtx, top);
  pushCtxData(aCtx, asJObj(newUInteger(result)));
}
\stopCCode

\startCCode
void equalSymbolWord(ContextObj* aCtx) {
  assert(isContextObj(aCtx));

  popCtxDataInto(aCtx, top1);
  popCtxDataInto(aCtx, top2);
  Boolean result = false;
  if (isSymbolObj(top1) && isSymbolObj(top2)) {
    if (strcmp(asSymbol(top1), asSymbol(top2)) == 0) result = true;
  }
  pushCtxData(aCtx, asJObj(newUInteger(result)));
}
\stopCCode

\startCCode
void registerSymbolWords(ContextObj *aCtx) {
  assert(isContextObj(aCtx));
  SymbolTableObj *symTab = asCtxSymbols(aCtx);
  assert(isSymbolTableObj(symTab));

  SymbolObj *newLineObj = newSymbol("\n", "newLine", 0);
  insertSymbol(symTab, "newLine", asJObj(newLineObj));

  registerCFunction(symTab, "isSymbol", isSymbolWord);
  registerCFunction(symTab, "=Sym",     equalSymbolWord);
}
\stopCCode

\startTestSuite[registerBasicWords]

\startCHeader
void registerBasicWords(ContextObj *aCtx);
\stopCHeader

\startCCode
void registerBasicWords(ContextObj *aCtx) {
  registerBooleanWords(aCtx);
  registerSymbolTableWords(aCtx);
  registerOutputWords(aCtx);
  registerUIntegerWords(aCtx);
  registerListWords(aCtx);
  registerSymbolWords(aCtx);
}
\stopCCode

\startTestCase[should register the basic words]

\startCHeader
void equalSymbolWord(ContextObj *aCtx);
\stopCHeader

\startCTest
  ContextObj *aCtx = newContext(NULL, NULL, "testRegisterBasicWords");
  AssertIntTrue(isContextObj(aCtx));
  
  SymbolTableObj *symTab = asCtxSymbols(aCtx);
  AssertIntTrue(isSymbolTableObj(symTab));
  AssertPtrNull(findSymbol(symTab, "true"));
  registerBasicWords(aCtx);
  JObj *shouldBeTrue = findSymbol(symTab, "true");
  AssertPtrNotNull(shouldBeTrue);
  AssertIntTrue(isUIntegerObj(shouldBeTrue));
  AssertIntEquals(asUInteger(shouldBeTrue), true);
  
  JObj *shouldBeEqualSym = findSymbol(symTab, "=Sym");
  AssertPtrNotNull(shouldBeEqualSym);
  AssertIntTrue(isCFunctionObj(shouldBeEqualSym));
  AssertPtrEquals(asCFunction(shouldBeEqualSym), equalSymbolWord);
\stopCTest

\stopTestCase
\stopTestSuite