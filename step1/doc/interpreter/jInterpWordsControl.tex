% A ConTeXt document [master document: contexts.tex]

\section[title=Context control code]

\subsection[title=Operators]

\subsection[title=Combinators]

\startCCode

static void pushListWord(ContextObj* aCtx) {
  popCtxDataInto(aCtx, top);
  popCtxDataInto(aCtx, aList);
  top = asJObj(newPair(top, NULL));
  aList = asJObj(concatLists(top, aList));
  pushCtxData(aCtx, aList);
}

static void popListWord(ContextObj* aCtx) {
  popCtxDataInto(aCtx, aList);
  popListInto(aCtx, aList, top);
  pushCtxData(aCtx, aList);
  pushCtxData(aCtx, top);
}
\stopCCode

\startCHeader
void wrapWord(ContextObj *aCtx);
\stopCHeader

\startCCode
void wrapWord(ContextObj* aCtx) {
  popCtxDataInto(aCtx, top);
  top = asJObj(newPair(top, NULL));
  pushCtxData(aCtx, top);
}
\stopCCode

\startCHeader
void prependWord(ContextObj *aCtx);
\stopCHeader

\startCCode
void prependWord(ContextObj* aCtx) {
  popCtxDataInto(aCtx, top);
  popCtxDataInto(aCtx, second);
  JObj* result = asJObj(concatLists(top, second));
  pushCtxData(aCtx, result);
}
\stopCCode

\startCCode
static void appendWord(ContextObj* aCtx) {
  popCtxDataInto(aCtx, top);
  popCtxDataInto(aCtx, second);
  JObj* result = asJObj(concatLists(second, top));
  pushCtxData(aCtx, result);
}
\stopCCode

\startCCode
static void extractWord(ContextObj* aCtx) {
  popCtxDataInto(aCtx, top);
  prependListCtxData(aCtx, top);
}
\stopCCode

\startCCode
static void interpretWord(ContextObj* aCtx) {
  popCtxDataInto(aCtx, top);
  prependListCtxProcess(aCtx, top);
}
\stopCCode

\startCCode
void ifteWord(ContextObj* aCtx) {
  popCtxDataInto(aCtx, topElse);
  popCtxDataInto(aCtx, topThen);
  popCtxDataInto(aCtx, topIf);
  pushCtxProcess(aCtx, topElse); // save for later
  pushCtxProcess(aCtx, topThen); // save for later
  pushSymbolCtxProcess(aCtx, "ifteCont");
  prependListCtxProcess(aCtx, topIf); // execute the if condition
}
\stopCCode

\startCCode
static void ifteContWord(ContextObj* aCtx) {
  assert(isContextObj(aCtx));

  popCtxDataInto(aCtx, topIf);
  popCtxProcessInto(aCtx, topThen);
  popCtxProcessInto(aCtx, topElse);
  if (isTrueObj(topIf)) {
    prependListCtxProcess(aCtx, topThen);
  } else {
    prependListCtxProcess(aCtx, topElse);
  }
}
\stopCCode

\startCCode
static void forWord(ContextObj* aCtx) {
  assert(isContextObj(aCtx));

  popCtxProcessInto(aCtx, nextCommand);
  if (symbolIs(nextCommand, "forDone")) {
    // this for loop is done
    DEBUG(DebugMask, "forWord DONE%s\n", "");
  } else {
    DEBUG(DebugMask, "forWord continue%s\n", "");
    pushSymbolCtxProcess(aCtx, "for");
    prependListCtxProcess(aCtx, nextCommand);
  }
}
\stopCCode

\startCCode
static void forDoneWord(ContextObj* aCtx) {
  // ignore
}
\stopCCode

\startCCode
static void lispInterpretWord(ContextObj* aCtx) {
  popCtxDataInto(aCtx, top);
  if (!isPairObj(top)) {
    raiseExceptionMsg(aCtx,
      "listInterpret expected a pair as top");
    return;
  }
  JObj* operationName = asCar(top);
  JObj* operationBody = asCdr(top);
  pushCtxProcess(aCtx, operationName);
  pushCtxProcess(aCtx, operationBody);
}
\stopCCode

\startCCode
static void lispForWord(ContextObj* aCtx) {
  assert(isContextObj(aCtx));

  popCtxProcessInto(aCtx, nextCommand);
  if (symbolIs(nextCommand, "forDone")) {
    // this lisp for loop is done
    DEBUG(DebugMask, "lispForAP DONE%s\n", "");
  } else {
    DEBUG(DebugMask, "lispForAP continue%s\n", "");
    if (!isPairObj(nextCommand)) {
      raiseExceptionMsg(aCtx,
        "listFor (continue) expected a pair as nextCommand");
      return;
    }
    JObj* operationName = asCar(nextCommand);
    JObj* operationBody = asCdr(nextCommand);
    pushSymbolCtxProcess(aCtx, "lispFor");
    pushCtxProcess(aCtx, operationName);
    pushCtxProcess(aCtx, operationBody);
  }
}
\stopCCode

\startCCode
static void doneWord(ContextObj* aCtx) {
  assert(isContextObj(aCtx));
  aCtx->process = NULL;
}
\stopCCode

\startCCode
static void clearContextWord(ContextObj* aCtx) {
  assert(isContextObj(aCtx));
  aCtx->data    = NULL;
  aCtx->process = NULL;
}
\stopCCode

\subsection[title=Exception handling]

\startCCode
static void tryWord(ContextObj* aCtx) {
  popCtxDataInto(aCtx, handlerExp);
  popCtxDataInto(aCtx, tryExp);
  pushCtxProcess(aCtx, handlerExp);
  pushSymbolCtxProcess(aCtx, "tryHandler");
  prependListCtxProcess(aCtx, tryExp);
}
\stopCCode

\startCCode
static void raiseWord(ContextObj* aCtx) {
  assert(isContextObj(aCtx));
  aCtx->exceptionRaised = true;
  popCtxDataInto(aCtx, raiseExp);
  pushSymbolCtxProcess(aCtx, "findFirstTryHandler");
  prependListCtxProcess(aCtx, raiseExp);
}
\stopCCode

\startCCode
static void raiseIfFalseWord(ContextObj* aCtx) {
  assert(isContextObj(aCtx));

  popCtxDataInto(aCtx, condition);
  popCtxProcessInto(aCtx, raiseExp);
  if (isFalseObj(condition)) {
    pushSymbolCtxProcess(aCtx, "findFirstTryHandler");
    prependListCtxProcess(aCtx, raiseExp);
  }
}
\stopCCode

\startCCode
ContextObj *findFirstTryHandlerWord(ContextObj* aCtx) {
  assert(isContextObj(aCtx));

  if (aCtx->process) {
    popCtxProcessInto(aCtx, aCommand);
    if (symbolIs(aCommand, "tryHandler")) {
      aCtx->exceptionRaised = false;
      popCtxProcessInto(aCtx, handlerExp);
      prependListCtxProcess(aCtx, handlerExp);
      return aCtx;
    }
    pushSymbolCtxProcess(aCtx, "findFirstTryHandler");
    return aCtx;
  }

  ContextObj* parentCtx = aCtx->parent;
  if (!parentCtx) {
    aCtx->exceptionRaised = true;
    return aCtx;
  }

  assert(isContextObj(parentCtx));
  
  popCtxDataInto(   aCtx,      oldContextTop );
  pushCtxData(      parentCtx, oldContextTop );
  
  wrapWord(parentCtx);
  
  pushSymbolCtxProcess( parentCtx, "raise"       );

  if (aCtx->tracingOn) {
    StringBufferObj *aStrBuf = newStringBuffer(aCtx);
    strBufPrintf(aStrBuf,
      "findFirstTryHandler(switchCtx)\n oldCtx: %s\n newCtx: %s\n",
      aCtx->name, parentCtx->name
    );
    strBufPrintf(aStrBuf, "top = ");
    printJObj(aStrBuf, oldContextTop, "", 20);
    strBufPrintf(aStrBuf, "\n");
    printf("%s", getCString(aStrBuf));
    strBufClose(aStrBuf);
    free(aStrBuf);
  }

  return parentCtx;
}
\stopCCode

\startCCode
static void tryHandlerWord(ContextObj* aCtx) {
  popCtxProcessInto(aCtx, handlerExp);
}
\stopCCode

\startCHeader
extern void raiseExceptionImpl(
  ContextObj *aCtx,
  Symbol     *message
);

#define raiseException(aCtx, message) \
  raiseExceptionImpl(aCtx, message)

#define raiseExceptionMsg(aCtx, message)  \
  pushNullCtxData(aCtx);                  \
  raiseExceptionImpl(aCtx, message)
\stopCHeader

\startCCode
void raiseExceptionImpl(
  ContextObj *aCtx,
  Symbol     *message
) {
  assert(isContextObj(aCtx));

  if (!message) message = "unknown";

  Symbol *file = "unknown(command)";
  size_t  line = 0;
  if (isSymbolObj(aCtx->command)) {
    file = asFile(aCtx->command);
    line = asLine(aCtx->command);
  }

  DEBUG(DebugMask, "raiseException %p [%s] (%s) <%s> %zu\n",
    aCtx, aCtx->name, message, file, line
  );

  wrapWord(aCtx);
  pushSymbolCtxData(aCtx, file);
  prependWord(aCtx);
  pushUIntegerCtxData(aCtx, line);
  prependWord(aCtx);
  pushSymbolCtxData(aCtx, message);
  prependWord(aCtx);
  pushCtxData(aCtx, asJObj(aCtx));
  prependWord(aCtx);
  wrapWord(aCtx);
  pushSymbolCtxProcess(aCtx, "raise");
  
  if (aCtx->tracingOn) {
    DEBUG(DebugMask, "raiseException -> tracing%s\n", "");
    StringBufferObj *aStrBuf = newStringBuffer(aCtx);
    strBufPrintf(aStrBuf, "d>>");
    printJObj(aStrBuf, aCtx->data, "", 20);
    strBufPrintf(aStrBuf, "\n");
    strBufPrintf(aStrBuf, "p>>");
    printJObj(aStrBuf, aCtx->process, "", 20);
    strBufPrintf(aStrBuf, "\n");
    printf("%s", getCString(aStrBuf));
    strBufClose(aStrBuf);
    free(aStrBuf);
    DEBUG(DebugMask, "raiseException <- tracing%s\n", "");
  }
}
\stopCCode

\startCHeader
extern void reportException(ContextObj* aCtx);
\stopCHeader

\startCCode
void reportException(
  ContextObj* aCtx
) {
  assert(isContextObj(aCtx));

  DEBUG(DebugMask, "reportException %p [%s] %zu\n",
    aCtx, aCtx->name, aCtx->exceptionRaised
  );
  if (!aCtx->exceptionRaised) return;

  extractWord(aCtx);

  popCtxDataInto(aCtx, expCtxObj);
  ContextObj *expCtx = aCtx;
  if (isContextObj(expCtxObj)) {
    expCtx = asContextObj(expCtxObj);
  }
  
  popCtxDataInto(aCtx, expMsg);
  popCtxDataInto(aCtx, expLine);
  popCtxDataInto(aCtx, expFile);
  popCtxDataInto(aCtx, expStack);
  StringBufferObj *aStrBuf = newStringBuffer();
  strBufPrintf(aStrBuf, "\nUNHANDLED EXCEPTION:\n");
  printJObj(aStrBuf, expMsg, "", 20);
  strBufPrintf(aStrBuf, "\n\n     in file: ");
  if (isSymbolObj(expFile)) {
    strBufPrintf(aStrBuf, "%s", asSymbol(expFile));
  } else {
    printJObj(aStrBuf, expFile, NULL, 20);
  }
  strBufPrintf(aStrBuf,   "\n     on line: ");
  printJObj(aStrBuf, expLine, NULL, 20);
  strBufPrintf(aStrBuf, "\n\n   wordStack: ");
  printJObj(aStrBuf, expStack, "", 20);
  strBufPrintf(aStrBuf,   "\n     context: %s", expCtx->name);
  strBufPrintf(aStrBuf,   "\n   dataStack: ");
  printJObj(aStrBuf, aCtx->data, "", 20);
  strBufPrintf(aStrBuf,   "\nprocessStack: ");
  printJObj(aStrBuf, aCtx->process, "", 20);
  strBufPrintf(aStrBuf, "\n");
  printf("%s", getCString(aStrBuf));
  strBufClose(aStrBuf);
  free(aStrBuf);
  aCtx->exceptionRaised = false;
}
\stopCCode

\subsection[title=Support]
\startCCode
static void defineWord(ContextObj* aCtx) {
  assert(isContextObj(aCtx));  
  popCtxDataInto(aCtx, name);
  popCtxDataInto(aCtx, definition);

  if (!isSymbolObj(name)) {
//    pushNullCtxData(aCtx);
//    pushOnTopCtxData(aCtx, name);
//    pushOnTopCtxData(aCtx, definition);
    raiseExceptionMsg(aCtx,
      "define requires a symbol as wordDefinition top"
    );
    return;
  }
  insertSymbol(asCtxSymbols(aCtx), asSymbol(name), definition);
}
\stopCCode

\starttyping
\startCCode
static void defineContextWord(ContextObj* aCtx) {
  assert(isContextObj(aCtx));
  JoyLoLInterp *jInterp = aCtx->jInterp;
  assert(jInterp);
  
  popCtxDataIntoImpl(aCtx, contextName);
  popCtxDataIntoImpl(aCtx, contextNamedIn);
  popCtxDataIntoImpl(aCtx, contextNamingScope);
  popCtxDataIntoImpl(aCtx, contextData);
  popCtxDataIntoImpl(aCtx, contextProcess);
  if (!isSymbol(contextName)) {
    pushNullCtxDataImpl(aCtx);
    pushOnTopCtxDataImpl(aCtx, contextProcess);
    pushOnTopCtxDataImpl(aCtx, contextData);
    pushOnTopCtxDataImpl(aCtx, contextNamingScope);
    pushOnTopCtxDataImpl(aCtx, contextNamedIn);
    pushOnTopCtxDataImpl(aCtx, contextName);
    raiseException(aCtx,
      "defineContext required a symbol as top");
    return;
  }
  if (!isDictionary(contextNamedIn)) {
    pushNullCtxDataImpl(aCtx);
    pushOnTopCtxDataImpl(aCtx, contextProcess);
    pushOnTopCtxDataImpl(aCtx, contextData);
    pushOnTopCtxDataImpl(aCtx, contextNamingScope);
    pushOnTopCtxDataImpl(aCtx, contextNamedIn);
    pushOnTopCtxDataImpl(aCtx, contextName);
    raiseException(aCtx,
      "defineContext required a dictionary as second");
    return;
  }
  if (!isDictionary(contextNamingScope)) {
    pushNullCtxDataImpl(aCtx);
    pushOnTopCtxDataImpl(aCtx, contextProcess);
    pushOnTopCtxDataImpl(aCtx, contextData);
    pushOnTopCtxDataImpl(aCtx, contextNamingScope);
    pushOnTopCtxDataImpl(aCtx, contextNamedIn);
    pushOnTopCtxDataImpl(aCtx, contextName);
    raiseException(aCtx,
      "defineContext required a dictionary as third");
    return;
  }
  ContextObj* newCtx = newContext(
    jInterp,
    asSymbol(contextName),
    aCtx,
    (DictObj*)contextNamingScope,
    contextData,
    contextProcess
  );
  assert(newCtx);
  defineContextIn(
    jInterp,
    (DictObj*)contextNamedIn,
    asSymbol(contextName),
    newCtx
  );
}
\stopCCode

\startCCode
static void newContextWord(ContextObj* aCtx) {
  assert(isContextObj(aCtx));
  JoyLoLInterp *jInterp = aCtx->jInterp;
  assert(jInterp);

  popCtxDataIntoImpl(aCtx, contextName);
  popCtxDataIntoImpl(aCtx, contextNamingScope);
  popCtxDataIntoImpl(aCtx, contextData);
  popCtxDataIntoImpl(aCtx, contextProcess);
  if (!isSymbol(contextName)) {
    pushNullCtxDataImpl(aCtx);
    pushOnTopCtxDataImpl(aCtx, contextProcess);
    pushOnTopCtxDataImpl(aCtx, contextData);
    pushOnTopCtxDataImpl(aCtx, contextNamingScope);
    pushOnTopCtxDataImpl(aCtx, contextName);
    raiseException(aCtx,
      "newContext requires a symbol as top");
    return;
  }
  if (!isDictionary(contextNamingScope)) {
    pushNullCtxDataImpl(aCtx);
    pushOnTopCtxDataImpl(aCtx, contextProcess);
    pushOnTopCtxDataImpl(aCtx, contextData);
    pushOnTopCtxDataImpl(aCtx, contextNamingScope);
    pushOnTopCtxDataImpl(aCtx, contextName);
    raiseException(aCtx,
      "newContext requires a dictionary as second");
    return;
  }
  ContextObj* newCtx = newContext(
    jInterp,
    asSymbol(contextName),
    aCtx,
    (DictObj*)contextNamingScope,
    contextData,
    contextProcess
  );
  assert(newCtx);
  pushCtxDataImpl(aCtx, (JObj*)newCtx);
}
\stopCCode

\startCCode
static void thisContextWord(ContextObj* aCtx) {
  assert(isContextObj(aCtx));
  DEBUG(aCtx->jInterp, "thisContextAP > %p\n", aCtx);
  pushCtxDataImpl(aCtx, (JObj*)aCtx);
  DEBUG(aCtx->jInterp, "thisContextAP < %p\n", aCtx);
}
\stopCCode

\startCCode
static ContextObj* switchCtxWord(ContextObj* aCtx) {
  assert(isContextObj(aCtx));

  popCtxDataIntoImpl(aCtx, newContext);
  popCtxDataIntoImpl(aCtx, oldContextTop);
  if (!isContext(newContext)) {
    raiseExceptionMsg(aCtx,
      "switchCtx required a context as top");
    return aCtx;
  }
  
  //
  // switch to newContext
  //
  DEBUG(aCtx->jInterp, "switchCtxAP -> switching origCtx: %s\n",
    aCtx->name);
  ContextObj* newCtx = (ContextObj*)newContext;
  assert(newCtx);

  pushCtxDataImpl(newCtx, oldContextTop);
  newCtx->tracingOn = aCtx->tracingOn;
  DEBUG(aCtx->jInterp, "switchCtxAP <- switching newCtx: %s\n",
            newCtx->name);
  //
  return newCtx;
}
\stopCCode

\startCCode
static void defineNamingWord(ContextObj* aCtx) {
  assert(isContextObj(aCtx));
  JoyLoLInterp *jInterp = aCtx->jInterp;
  assert(jInterp);
  
  popCtxDataIntoImpl(aCtx, dictName);
  popCtxDataIntoImpl(aCtx, dictNamedIn);
  popCtxDataIntoImpl(aCtx, dictNamingScope);
  if (!isSymbol(dictName)) {
    pushNullCtxDataImpl(aCtx);
    pushOnTopCtxDataImpl(aCtx, dictNamingScope);
    pushOnTopCtxDataImpl(aCtx, dictNamedIn);
    pushOnTopCtxDataImpl(aCtx, dictName);
    raiseException(aCtx,
      "defineNaming required a symbol as top");
    return;
  }
  if (!isDictionary(dictNamedIn)) {
    pushNullCtxDataImpl(aCtx);
    pushOnTopCtxDataImpl(aCtx, dictNamingScope);
    pushOnTopCtxDataImpl(aCtx, dictNamedIn);
    pushOnTopCtxDataImpl(aCtx, dictName);
    raiseException(aCtx,
      "defineNaming requires a dictionary as second");
    return;
  }
  if (!isDictionary(dictNamingScope)) {
    pushNullCtxDataImpl(aCtx);
    pushOnTopCtxDataImpl(aCtx, dictNamingScope);
    pushOnTopCtxDataImpl(aCtx, dictNamedIn);
    pushOnTopCtxDataImpl(aCtx, dictName);
    raiseException(aCtx,
      "defineNaming requires a dictionary as third");
    return;
  }
  DictObj* newDict =
    newDictionary(
      jInterp,
      asSymbol(dictName),
      (DictObj*)dictNamingScope
    );
  assert(newDict);
  defineNamingIn(
    jInterp,
    (DictObj*)dictNamedIn,
    asSymbol(dictName),
    newDict
  );
}
\stopCCode

\startCCode
static void newNamingWord(ContextObj* aCtx) {
  assert(isContextObj(aCtx));
  JoyLoLInterp *jInterp = aCtx->jInterp;
  assert(jInterp);

  popCtxDataIntoImpl(aCtx, dictName);
  popCtxDataIntoImpl(aCtx, dictNamingScope);
  if (!isSymbol(dictName)) {
    pushNullCtxDataImpl(aCtx);
    pushOnTopCtxDataImpl(aCtx, dictNamingScope);
    pushOnTopCtxDataImpl(aCtx, dictName);
    raiseException(aCtx,
      "defineNaming requires a symbol as top");
    return;
  }
  if (!isDictionary(dictNamingScope)) {
    pushNullCtxDataImpl(aCtx);
    pushOnTopCtxDataImpl(aCtx, dictNamingScope);
    pushOnTopCtxDataImpl(aCtx, dictName);
    raiseException(aCtx,
      "defineNaming requires a dictionary as second");
    return;
  }

  DictObj* newDict =
    newDictionary(
      jInterp,
      asSymbol(dictName),
      (DictObj*)dictNamingScope
    );
  assert(newDict);
  pushCtxDataImpl(aCtx, (JObj*)newDict);
}
\stopCCode

\startCCode
static void localizeNamingWord(ContextObj* aCtx) {
  assert(isContextObj(aCtx));
  JoyLoLInterp *jInterp = aCtx->jInterp;
  assert(jInterp);
  
  popCtxDataIntoImpl(aCtx, localNaming);
  if (!isSymbol(localNaming)){
    raiseExceptionMsg(aCtx,
      "localizeNaming requires a symbol as top");
    return;
  }
  
  DictObj* newDict = 
    newDictionary(
      jInterp,
      asSymbol(localNaming),
      aCtx->dict
    );
  assert(newDict);
  aCtx->dict = newDict;
}
\stopCCode

\startCCode
static void thisNamingWord(ContextObj* aCtx) {
  assert(isContextObj(aCtx));
  pushCtxDataImpl(aCtx, (JObj*)(aCtx->dict));
}
\stopCCode
\stoptyping

\startCCode
static void undefineWord(ContextObj* aCtx) {
  assert(isContextObj(aCtx));
  
  popCtxDataInto(aCtx, nameList);

  if (!isPairObj(nameList)) {
    raiseExceptionMsg(aCtx,
      "undefine requires a list quoted symbol as second");
    return;
  }
  
  popListInto(aCtx, nameList, name);
  
  if (!isSymbolObj(name)) {
//    pushNullCtxDataImpl(aCtx);
//    pushOnTopCtxDataImpl(aCtx, namingScope);
//    pushOnTopCtxDataImpl(aCtx, name);
    raiseExceptionMsg(aCtx,
      "undefine requires a list quoted symbol as second");
    return;
  }
  deleteSymbol(asCtxSymbols(aCtx), asSymbol(name));
  }
\stopCCode


\startCHeader
void breakPointWord(ContextObj *aCtx);
\stopCHeader

\startCCode
void breakPointWord(ContextObj *aCtx) {
  assert(isContextObj(aCtx));
  asCtxBreakPoint(aCtx) = true;
  if (aCtx->tracingOn) printf("breakPoint: breaking\n");
}
\stopCCode

\startCHeader
void breakPointOn(ContextObj *aCtx, Symbol *aSymbol);
\stopCHeader

\startCCode
void breakPointOn(ContextObj *aCtx, Symbol *aSymbol) {
  assert(isContextObj(aCtx));
  if (!aSymbol) return;
  deleteSymbol(asCtxSymbols(aCtx), aSymbol);
  registerCFunction(asCtxSymbols(aCtx), aSymbol, breakPointWord);
}
\stopCCode

\startTestSuite[registerControlWords]

\startCHeader
void registerControlWords(ContextObj *aCtx);
\stopCHeader

\startCCode
void registerControlWords(ContextObj *aCtx) {
  assert(isContextObj(aCtx));
  SymbolTableObj *symTab = asCtxSymbols(aCtx);
  assert(isSymbolTableObj(symTab));
 
  registerCFunction(symTab, "pushList",       pushListWord);
  registerCFunction(symTab, "popList",        popListWord);
  registerCFunction(symTab, "wrap",           wrapWord);
  registerCFunction(symTab, "prepend",        prependWord);
  registerCFunction(symTab, "append",         appendWord);
  registerCFunction(symTab, "extract",        extractWord);
  registerCFunction(symTab, "i",              interpretWord);
  registerCFunction(symTab, "interpret",      interpretWord);
  registerCFunction(symTab, "ifte",           ifteWord);
  registerCFunction(symTab, "ifteCont",       ifteContWord);
  registerCFunction(symTab, "for",            forWord);
  registerCFunction(symTab, "forDone",        forDoneWord);
  registerCFunction(symTab, "lispInterpret",  lispInterpretWord);
  registerCFunction(symTab, "lispFor",        lispForWord);
  registerCFunction(symTab, "define",         defineWord);
//  registerCFunction(symTab, "defineContext",  defineContextWord);
//  registerCFunction(symTab, "newContext",     newContextWord);
//  registerCFunction(symTab, "thisContext",    thisContextWord);
//  registerCFunction(symTab, "defineNaming",   defineNamingWord);
//  registerCFunction(symTab, "newNaming",      newNamingWord);
//  registerCFunction(symTab, "localizeNaming", localizeNamingWord);
//  registerCFunction(symTab, "thisNaming",     thisNamingWord);
  registerCFunction(symTab, "undefine",       undefineWord);
  registerCFunction(symTab, "done",           doneWord);
  registerCFunction(symTab, "clear",          clearContextWord);
  registerCFunction(symTab, "try",            tryWord);
  registerCFunction(symTab, "raise",          raiseWord);
  registerCFunction(symTab, "raiseIfFalse",   raiseIfFalseWord);
  registerCFunction(symTab, "tryHandler",     tryHandlerWord);
  registerCFunction(symTab, "breakPoint",     breakPointWord);
  
  registerCtxCFunction(symTab, "findFirstTryHandler", findFirstTryHandlerWord);
//  extendCtxJoyLoLInRoot(jInterp, "switchCtx",           "", switchCtxWord,           "");
}
\stopCCode

\startTestCase[should register control words]

\startCHeader
void ifteWord(ContextObj *aCtx);
ContextObj *findFirstTryHandlerWord(ContextObj *aCtx);
\stopCHeader

\startCTest
  ContextObj *aCtx = newContext(NULL, NULL, "testRegisterControlWords");
  AssertIntTrue(isContextObj(aCtx));
  
  SymbolTableObj *symTab = asCtxSymbols(aCtx);
  AssertIntTrue(isSymbolTableObj(symTab));
  AssertPtrNull(findSymbol(symTab, "ifte"));
  
  registerControlWords(aCtx);
  JObj *shouldBeIFTE = findSymbol(symTab, "ifte");
  AssertIntTrue(isCFunctionObj(shouldBeIFTE));
  AssertPtrEquals(asCFunction(shouldBeIFTE), ifteWord);
  
  JObj *shouldBeTryHandler = findSymbol(symTab, "findFirstTryHandler");
  AssertIntTrue(isCtxCFunctionObj(shouldBeTryHandler));
  AssertPtrEquals(asCtxCFunction(shouldBeTryHandler), findFirstTryHandlerWord);
\stopCTest
\stopTestCase
\stopTestSuite