% A ConTeXt document [master document: stringBuffers.tex]

\subsection[title=String Buffers]

\startCHeader
typedef struct string_buffer_struct {
  size_t        tag;
  FILE         *memFile;
  char         *buffer;
  size_t        bufSize;
} StringBufferObj;

#define isStringBufferObj(anObj) ((anObj) && (((JObj*)(anObj))->tag == StringBufferTag))
#define asStringBufferObj(anObj) ((StringBufferObj*)(anObj))
#define asMemFile(aLoL)   (((StringBufferObj*)(aLoL))->memFile)
#define asBuffer(aLoL)    (((StringBufferObj*)(aLoL))->buffer)
#define asBufSize(aLoL)   (((StringBufferObj*)(aLoL))->bufSize)
\stopCHeader

\startTestSuite[newStringBuffer]

\startCHeader
StringBufferObj* newStringBuffer();
\stopCHeader

\startCCode
StringBufferObj* newStringBuffer() {
  StringBufferObj* newObj =
    (StringBufferObj*)calloc(1, sizeof(StringBufferObj));
  newObj->tag     = StringBufferTag;
  newObj->memFile = NULL;
  newObj->buffer  = NULL;
  newObj->bufSize = 0;
  return newObj;
}
\stopCCode

\startCHeader
void strBufClose(StringBufferObj* aStrBuf);
\stopCHeader

\startCCode
void strBufClose(StringBufferObj* aStrBuf) {
  if (!isStringBufferObj(aStrBuf)) return;

  if (asMemFile(aStrBuf)) fclose(asMemFile(aStrBuf));
  if (asBuffer(aStrBuf))  free(asBuffer(aStrBuf));
  asMemFile(aStrBuf) = NULL;
  asBuffer(aStrBuf)  = NULL;
  asBufSize(aStrBuf) = 0;
}
\stopCCode

\startCHeader
extern void strBufReOpen(StringBufferObj* aStrBuf);
\stopCHeader

\startCCode
void strBufReOpen(StringBufferObj* aStrBuf) {
  if (!isStringBufferObj(aStrBuf)) return;

  if (asMemFile(aStrBuf)) strBufClose(aStrBuf);
  asMemFile(aStrBuf) = 
    open_memstream(&asBuffer(aStrBuf), &asBufSize(aStrBuf));
}
\stopCCode

\startCHeader
Symbol *getCString(StringBufferObj *aStrBuf);
\stopCHeader

\startCCode
Symbol *getCString(StringBufferObj *aStrBuf) {
  if (!isStringBufferObj(aStrBuf)) return NULL;
  if (!asMemFile(aStrBuf)) strBufReOpen(aStrBuf);
  fflush(asMemFile(aStrBuf));
  return asBuffer(aStrBuf);
}
\stopCCode

\startTestCase[should create some new stringBuffers]

\startCTest
  StringBufferObj* aStrBuf = newStringBuffer();
  AssertIntTrue(isStringBufferObj(aStrBuf));
  AssertPtrNull(asMemFile(aStrBuf));
  AssertPtrNull(asBuffer(aStrBuf));
  AssertIntZero(asBufSize(aStrBuf));
  strBufClose(aStrBuf);
  AssertPtrNull(asMemFile(aStrBuf));
  AssertPtrNull(asBuffer(aStrBuf));
  AssertIntZero(asBufSize(aStrBuf));
  AssertStrEquals(getCString(aStrBuf), "");
  AssertPtrNotNull(asBuffer(aStrBuf));
  AssertIntZero(asBufSize(aStrBuf));
  strBufClose(aStrBuf);
  AssertPtrNull(asMemFile(aStrBuf));
  AssertPtrNull(asBuffer(aStrBuf));
  AssertIntZero(asBufSize(aStrBuf));
  strBufPrintf(aStrBuf, "a test string");
  AssertPtrNotNull(asMemFile(aStrBuf));
  AssertStrEquals(getCString(aStrBuf), "a test string");
  AssertPtrNotNull(asBuffer(aStrBuf));
  AssertIntEquals(asBufSize(aStrBuf), 13);
  strBufClose(aStrBuf); // need to release the FILE*
  free(aStrBuf);
\stopCTest
\stopTestCase
\stopTestSuite

\startTestSuite[strBufPrintf]

\startCHeader
Boolean strBufPrintf(
  StringBufferObj   *aStrBuf,
  const char        *format, 
  ...
);
\stopCHeader

\startCCode
Boolean strBufPrintf(
  StringBufferObj   *aStrBuf,
  const char        *format,
  ...
) {
  if (!isStringBufferObj(aStrBuf)) return false;
  if (!asMemFile(aStrBuf)) strBufReOpen(aStrBuf);
  
  va_list printfArgs;
  va_start(printfArgs, format);
  int numChars = vfprintf(asMemFile(aStrBuf), format, printfArgs);
  va_end(printfArgs);
  return (0 <= numChars);
}
\stopCCode

\startTestCase[should printf to a sting buffer]
\startCTest
  StringBufferObj* aStrBuf = newStringBuffer();
  strBufPrintf(aStrBuf, "a test [%s]", "an inner string");
  AssertStrEquals(getCString(aStrBuf), "a test [an inner string]");
\stopCTest
\stopTestCase
\stopTestSuite

\startTestSuite[printing stringBuffers]

\startCHeader
void stringBufferPrinter(
  StringBufferObj *aStrBuf,
  JObj            *aLoL,
  Symbol          *indent,
  size_t           timeToLive
);
\stopCHeader

\startCCode
void stringBufferPrinter(
  StringBufferObj *aStrBuf,
  JObj            *aLoL,
  Symbol          *indent,
  size_t           timeToLive
) {
  assert(isStringBufferObj(aStrBuf));
  assert(isStringBufferObj(aLoL));

  if (indent) {
    strBufPrintf(aStrBuf, "%s<strBuf:%s>\n", 
      indent,
      getCString(asStringBufferObj(aLoL)));
  } else {
    strBufPrintf(aStrBuf, "%s ", 
      getCString(asStringBufferObj(aLoL)));
  }
}
\stopCCode

\startTestCase[should print stringBuffers]

\startCTest
  StringBufferObj* aTestStrBuf = newStringBuffer();
  AssertIntTrue(isStringBufferObj(aTestStrBuf));
  
  strBufPrintf(aTestStrBuf, "test string");
  
  StringBufferObj* aStrBuf = newStringBuffer();
  AssertIntTrue(isStringBufferObj(aStrBuf));
  
  stringBufferPrinter(
    aStrBuf, asJObj(aTestStrBuf), "  ", 20);
  AssertStrMatches(getCString(aStrBuf), "<strBuf:test string>");
  AssertStrMatches(getCString(aStrBuf), "\n");
  
  strBufReOpen(aStrBuf);
  stringBufferPrinter(aStrBuf, asJObj(aTestStrBuf), NULL, 20);
  AssertStrEquals(getCString(aStrBuf), "test string ");
  
  strBufClose(aStrBuf);
  strBufClose(aTestStrBuf);
  free(aStrBuf);
  free(aTestStrBuf);
\stopCTest
\stopTestCase
\stopTestSuite

