% A ConTeXt document [master document: joyLoLMinus.tex ]

\subsection[title=Symbol Table]

\startCHeader
typedef struct symbolEntry_struct {
  Symbol *symbol;
  JObj   *jObj;
} SymbolEntry;

typedef struct symbolTable_struct {
  size_t       tag;
  SymbolEntry *symbols;
  size_t       numSymbols;
  size_t       maxSymbols;
  Symbol      *name;
} SymbolTableObj;

#define SymbolTable_GrowthSize 10

#define isSymbolTableObj(anObj) \
  ((anObj) && ((((JObj*)(anObj))->tag) == SymbolTableTag))
#define asSymbolTableObj(anObj) ((SymbolTableObj*)(anObj))
#define asSymbolTableName(anObj) (((SymbolTableObj*)(anObj))->name)
\stopCHeader

\startTestSuite[newSymbolTable]

\startCHeader
SymbolTableObj *newSymbolTable(Symbol *name);
\stopCHeader

\startCCode
SymbolTableObj *newSymbolTable(Symbol *name) {
  SymbolTableObj *newTable =
    (SymbolTableObj*)calloc(1, sizeof(SymbolTableObj));
  assert(newTable);
  newTable->tag        = SymbolTableTag;
  newTable->symbols    = NULL;
  newTable->numSymbols = 0;
  newTable->maxSymbols = 0;
  newTable->name       = strdup(name);
  return newTable;
}
\stopCCode

\startTestCase[should create a new symbolTable]

\startCTest
  Symbol *aName = "TestSymTab";
  SymbolTableObj *aSymTab = newSymbolTable(aName);
  AssertIntTrue(isSymbolTableObj(aSymTab));
  AssertPtrNull(aSymTab->symbols);
  AssertIntZero(aSymTab->numSymbols);
  AssertIntZero(aSymTab->maxSymbols);
  AssertStrEquals(aSymTab->name, aName);
  AssertPtrNotEquals(aSymTab->name, aName);
\stopCTest
\stopTestCase
\stopTestSuite

\startTestSuite[findSymbolIndex]

\startCHeader
size_t findSymbolIndex(
  SymbolTableObj *aTable,
  Symbol         *aSymbol
);
\stopCHeader

\startCCode
size_t findSymbolIndex(SymbolTableObj *aTable,  Symbol *aSymbol) {
  assert(aTable);
  assert(aSymbol);
  
  size_t firstIndex = 0;
  size_t lastIndex  = aTable->numSymbols;
  
  while(1) {
    assert(0 <= firstIndex);
    assert(firstIndex <= lastIndex);
    assert(lastIndex  <= aTable->maxSymbols);
    
    if (firstIndex == lastIndex) {
      return firstIndex;
    } else if (firstIndex + 1 == lastIndex) {
      int cmp = strcmp(aTable->symbols[firstIndex].symbol, aSymbol);
      if (0 <= cmp) {
        return firstIndex;
      } else {
        return lastIndex;
      }
    } else {
      size_t probeIndex = (firstIndex + lastIndex) >> 1; // divide by two
      assert(firstIndex <= probeIndex);
      assert(probeIndex <= lastIndex);
      int cmp = strcmp(aTable->symbols[probeIndex].symbol, aSymbol);
      if (cmp == 0) {
        return probeIndex;
      } else if (cmp < 0) {
        firstIndex = probeIndex;
      } else {
        lastIndex = probeIndex;
      }
    }
  }  
}
\stopCCode

\startTestCase[should not find index of all required symbols]

\startCTest
  SymbolTableObj *aSymTab = newSymbolTable("TestSymTab1");
  AssertIntTrue(isSymbolTableObj(aSymTab));
  
  AssertIntEquals(findSymbolIndex(aSymTab, "TestSym"), 0);
  UIntegerObj *int42 = newUInteger(42);
  UIntegerObj *int43 = newUInteger(43);
  UIntegerObj *int44 = newUInteger(44);
  
  insertSymbol(aSymTab, "first",  asJObj(int42));
  insertSymbol(aSymTab, "second", asJObj(int43));
  insertSymbol(aSymTab, "third",  asJObj(int44));
  
  AssertIntEquals(findSymbolIndex(aSymTab, "first"),  0);
  AssertIntEquals(findSymbolIndex(aSymTab, "second"), 1);
  AssertIntEquals(findSymbolIndex(aSymTab, "third"),  2);
  
  AssertIntEquals(findSymbolIndex(aSymTab, "aBeforeAllSymbols"),       0);
  AssertIntEquals(findSymbolIndex(aSymTab, "gBetweenFirstAndSecond"),  1);
  AssertIntEquals(findSymbolIndex(aSymTab, "szBetweenSecondAndThird"), 2);
  AssertIntEquals(findSymbolIndex(aSymTab, "zAfterAllSymbols"),        3);
\stopCTest
\stopTestCase
\stopTestSuite

\startTestSuite[findSymbol]

\startCHeader
JObj *findSymbol(
  SymbolTableObj *aTable,
  Symbol         *aSymbol
);
\stopCHeader

\startCCode
JObj *findSymbol(SymbolTableObj *aTable, Symbol *aSymbol) {  
  assert(aTable);
  assert(aSymbol);
  
  size_t symbolIndex = findSymbolIndex(aTable, aSymbol);
  assert(0 <= symbolIndex);
  assert(symbolIndex <= aTable->numSymbols);
  assert(symbolIndex <= aTable->maxSymbols);
    
  if (symbolIndex == aTable->numSymbols) return NULL;
  
  int cmp = strcmp(aTable->symbols[symbolIndex].symbol, aSymbol);
  
  if (cmp != 0) return NULL;
  
  return aTable->symbols[symbolIndex].jObj;
}
\stopCCode

\startTestCase[should find the correct (first) symbol table entry]

\startCTest
  SymbolTableObj *aSymTab = newSymbolTable("TestSymTab2");
  AssertIntTrue(isSymbolTableObj(aSymTab));
  
  AssertIntEquals(findSymbolIndex(aSymTab, "TestSym"), 0);
  UIntegerObj *int42 = newUInteger(42);
  UIntegerObj *int43 = newUInteger(43);
  UIntegerObj *int44 = newUInteger(44);
  
  insertSymbol(aSymTab, "first",  asJObj(int42));
  insertSymbol(aSymTab, "second", asJObj(int43));
  insertSymbol(aSymTab, "third",  asJObj(int44));
  
  AssertPtrEquals(findSymbol(aSymTab, "first"),  asJObj(int42));
  AssertPtrEquals(findSymbol(aSymTab, "second"), asJObj(int43));
  AssertPtrEquals(findSymbol(aSymTab, "third"),  asJObj(int44));
  
  AssertPtrNull(findSymbol(aSymTab, "aBeforeAllSymbols"));
  AssertPtrNull(findSymbol(aSymTab, "gBetweenFirstAndSecond"));
  AssertPtrNull(findSymbol(aSymTab, "szBetweenSecondAndThird"));
  AssertPtrNull(findSymbol(aSymTab, "zAfterAllSymbols"));
\stopCTest
\stopTestCase
\stopTestSuite

\startTestSuite[increaseMaxSymbols]

\startCHeader
void increaseMaxSymbols(SymbolTableObj *aTable);
\stopCHeader

\startCCode
void increaseMaxSymbols(SymbolTableObj *aTable) {
  assert(aTable);
  SymbolEntry *oldSymbols = aTable->symbols;
  size_t newMaxSymbols    = aTable->maxSymbols + SymbolTable_GrowthSize;
  SymbolEntry *newSymbols =
    (SymbolEntry*)calloc(newMaxSymbols, sizeof(SymbolEntry));
  assert(newSymbols);
  if (oldSymbols) {
    memcpy(
      newSymbols,
      oldSymbols,
      ((aTable->maxSymbols) * sizeof(SymbolEntry))
    );
    free(oldSymbols);
    oldSymbols = NULL;
  }
  aTable->symbols    = newSymbols;
  aTable->maxSymbols = newMaxSymbols;
  assert(aTable->symbols);
  assert(aTable->numSymbols <= aTable->maxSymbols);
}
\stopCCode

\startTestCase[should increase the maximum number of symbols in the table]

\startCTest
  SymbolTableObj *aSymTab = newSymbolTable("TestSymTab3");
  AssertIntTrue(isSymbolTableObj(aSymTab));
  
  AssertPtrNull(aSymTab->symbols);
  AssertIntZero(aSymTab->numSymbols);
  AssertIntZero(aSymTab->maxSymbols);
  
  increaseMaxSymbols(aSymTab);
  AssertPtrNotNull(aSymTab->symbols);
  AssertIntZero(aSymTab->numSymbols);
  AssertIntEquals(aSymTab->maxSymbols, SymbolTable_GrowthSize);

  UIntegerObj *int42 = newUInteger(42);
  UIntegerObj *int43 = newUInteger(43);
  UIntegerObj *int44 = newUInteger(44);
  
  insertSymbol(aSymTab, "first",  asJObj(int42));
  insertSymbol(aSymTab, "second", asJObj(int43));
  insertSymbol(aSymTab, "third",  asJObj(int44));
  
  AssertIntEquals(findSymbolIndex(aSymTab, "first"),  0);
  AssertIntEquals(findSymbolIndex(aSymTab, "second"), 1);
  AssertIntEquals(findSymbolIndex(aSymTab, "third"),  2);

  AssertPtrEquals(findSymbol(aSymTab, "first"),  asJObj(int42));
  AssertPtrEquals(findSymbol(aSymTab, "second"), asJObj(int43));
  AssertPtrEquals(findSymbol(aSymTab, "third"),  asJObj(int44));

  increaseMaxSymbols(aSymTab);
  AssertPtrNotNull(aSymTab->symbols);
  AssertIntEquals(aSymTab->numSymbols, 3);
  AssertIntEquals(aSymTab->maxSymbols, 2*SymbolTable_GrowthSize);

  AssertIntEquals(findSymbolIndex(aSymTab, "first"),  0);
  AssertIntEquals(findSymbolIndex(aSymTab, "second"), 1);
  AssertIntEquals(findSymbolIndex(aSymTab, "third"),  2);

  AssertPtrEquals(findSymbol(aSymTab, "first"),  asJObj(int42));
  AssertPtrEquals(findSymbol(aSymTab, "second"), asJObj(int43));
  AssertPtrEquals(findSymbol(aSymTab, "third"),  asJObj(int44));
\stopCTest
\stopTestCase
\stopTestSuite

\startTestSuite[insertSymbol]

\startCHeader
void insertSymbol(
  SymbolTableObj *aTable,
  Symbol         *aSymbol,
  JObj *anObj
);
\stopCHeader

\startCCode
void insertSymbol(SymbolTableObj *aTable, Symbol *aSymbol, JObj *anObj) {
  assert(aTable);
  assert(aSymbol);
  
  size_t symbolIndex = findSymbolIndex(aTable, aSymbol);
  assert(0 <= symbolIndex);
  assert(symbolIndex <= aTable->numSymbols);
  assert(symbolIndex <= aTable->maxSymbols);
  
  DEBUG(SymbolTableDebug,
    "insertSymbol: index: %zd  num: %zd max: %zd\n",
    symbolIndex,
    aTable->numSymbols,
    aTable->maxSymbols
  );
  DEBUG(SymbolTableDebug,
    "insertSymbol: inserting: [%s]<%s>: %p\n",
    aSymbol,
    getJObjName(anObj),
    anObj
  );
  if (symbolIndex < aTable->numSymbols) {
    DEBUG(SymbolTableDebug,
      "insertSymbol: found: [%s]\n",
      aTable->symbols[symbolIndex].symbol
    );
  } else {
    DEBUG(SymbolTableDebug, "%s", "insertSymbol: found: no symbols\n");
  }
  
  if (symbolIndex == aTable->numSymbols) {
    if (aTable->maxSymbols <= aTable->numSymbols) increaseMaxSymbols(aTable);
    aTable->symbols[aTable->numSymbols].symbol = strdup(aSymbol);
    aTable->symbols[aTable->numSymbols].jObj   = anObj;
    aTable->numSymbols++;
    assert(aTable->numSymbols <= aTable->maxSymbols);
    return;
  }
  
  int cmp = strcmp(aTable->symbols[symbolIndex].symbol, aSymbol);
  assert(0 <= cmp);
  if (cmp == 0) return; // symbol has already been inserted
  if (cmp != 0) { // aSymbol is less than the current index.. so make space
    if (aTable->maxSymbols <= aTable->numSymbols) increaseMaxSymbols(aTable);
    SymbolEntry *newSymbol = aTable->symbols + aTable->numSymbols;
    SymbolEntry *curSymbol = newSymbol - 1;
    SymbolEntry *insSymbol = aTable->symbols + symbolIndex;
    for (; insSymbol < newSymbol ; ) {
      newSymbol->symbol = curSymbol->symbol;
      newSymbol->jObj   = curSymbol->jObj;
      newSymbol--;
      curSymbol--;
    }
  }
  aTable->symbols[symbolIndex].symbol = strdup(aSymbol);
  aTable->symbols[symbolIndex].jObj   = anObj;
  aTable->numSymbols++;
  assert(aTable->numSymbols <= aTable->maxSymbols);
}
\stopCCode

\startTestCase[should insert a new symbol]

\startCTest
  SymbolTableObj *aSymTab = newSymbolTable("TestSymTab4");
  AssertIntTrue(isSymbolTableObj(aSymTab));

  JObj* jObjs[100];
  char aSym[5];
  aSym[0] = ' ';
  aSym[1] = 'S';
  aSym[2] = 'y';
  aSym[3] = 'm';
  aSym[4] = 0;
  for(char aChar = 'Z'; 'A' <= aChar; aChar--) {
    aSym[0]      = aChar;
    jObjs[(UInteger)aChar] = asJObj(newUInteger((UInteger)aChar));
    insertSymbol(aSymTab, aSym, jObjs[(UInteger)aChar]);
  }
  
  AssertPtrEquals(findSymbol(aSymTab, "KSym"), jObjs['K']);
  AssertPtrEquals(findSymbol(aSymTab, "OSym"), jObjs['O']);
  AssertPtrEquals(findSymbol(aSymTab, "DSym"), jObjs['D']);
  AssertPtrEquals(findSymbol(aSymTab, "USym"), jObjs['U']);
  
  UIntegerObj *int42 = newUInteger(42);
  insertSymbol(aSymTab, "KSym", asJObj(int42));
  AssertPtrEquals(findSymbol(aSymTab, "KSym"), jObjs['K']);
  insertSymbol(aSymTab, "KASym", asJObj(int42));
  AssertPtrEquals(findSymbol(aSymTab, "KASym"), asJObj(int42));
\stopCTest
\stopTestCase
\stopTestSuite

\startTestSuite[deleteSymbol]

\startCHeader
void deleteSymbol(
  SymbolTableObj *aTable,
  Symbol         *aSymbol
);
\stopCHeader

\startCCode
void deleteSymbol(SymbolTableObj *aTable, Symbol *aSymbol) {
  assert(aTable);
  assert(aSymbol);
  
  size_t symbolIndex = findSymbolIndex(aTable, aSymbol);
  assert(0 <= symbolIndex);
  assert(symbolIndex <= aTable->numSymbols);
  assert(symbolIndex <= aTable->maxSymbols);

  if (symbolIndex == aTable->numSymbols) return; // nothing to delete
  
  int cmp = strcmp(aTable->symbols[symbolIndex].symbol, aSymbol);
  if (cmp != 0) return; // nothing to delete
  SymbolEntry *newSymbol = aTable->symbols + symbolIndex;
  SymbolEntry *curSymbol = newSymbol + 1;
  SymbolEntry *lastSymbol = aTable->symbols + aTable->numSymbols;
  for (; curSymbol < lastSymbol ; ) {
    newSymbol->symbol = curSymbol->symbol;
    newSymbol->jObj   = curSymbol->jObj;
    newSymbol++;
    curSymbol++;
  }
  aTable->symbols[aTable->numSymbols].symbol = NULL;
  aTable->numSymbols--;
  assert(0 <= aTable->numSymbols);
}
\stopCCode

\startTestCase[should delete the symbol entry corresponding to a symbol]

\startCTest
  SymbolTableObj *aSymTab = newSymbolTable("TestSymTab5");
  AssertIntTrue(isSymbolTableObj(aSymTab));
  
  UIntegerObj *int42 = newUInteger(42);
  UIntegerObj *int43 = newUInteger(43);
  UIntegerObj *int44 = newUInteger(44);
  
  insertSymbol(aSymTab, "first",  asJObj(int42));
  insertSymbol(aSymTab, "second", asJObj(int43));
  insertSymbol(aSymTab, "third",  asJObj(int44));
  
  AssertIntEquals(findSymbolIndex(aSymTab, "first"),  0);
  AssertIntEquals(findSymbolIndex(aSymTab, "second"), 1);
  AssertIntEquals(findSymbolIndex(aSymTab, "third"),  2);

  deleteSymbol(aSymTab, "second");
  AssertPtrNull(findSymbol(aSymTab,"second"));
  AssertIntEquals(findSymbolIndex(aSymTab, "first"),  0);
  AssertIntEquals(findSymbolIndex(aSymTab, "third"),  1);
\stopCTest
\stopTestCase
\stopTestSuite

\startTestSuite[registerCFunction]

\startCHeader
void registerCFunction(
  SymbolTableObj *aTable,
  Symbol         *symbol,
  CFunction      *func
);
\stopCHeader

\startCCode
void registerCFunction(
  SymbolTableObj *aTable,
  Symbol         *symbol,
  CFunction      *func
) {
  assert(aTable);
  assert(symbol);
  assert(func);
  
  CFunctionObj *newObj = newCFunction(func);
  insertSymbol(aTable, symbol, asJObj(newObj));
}
\stopCCode

\startTestCase[should register a CFunction]
\startCTest
  void testCFunction(ContextObj* aCtx){
    // do nothing
  }

  SymbolTableObj *aSymTab = newSymbolTable("TestSymTab6");
  AssertIntTrue(isSymbolTableObj(aSymTab));

  registerCFunction(aSymTab, "testCFunction", testCFunction);
  JObj *aCFunc = findSymbol(aSymTab, "testCFunction");
  AssertIntTrue(isCFunctionObj(aCFunc));
  AssertPtrEquals(asCFunction(aCFunc), testCFunction);
\stopCTest
\stopTestCase
\stopTestSuite

\startTestSuite[registerCtxCFunction]

\startCHeader
void registerCtxCFunction(
  SymbolTableObj *aTable,
  Symbol         *symbol,
  CtxCFunction   *func
);
\stopCHeader

\startCCode
void registerCtxCFunction(
  SymbolTableObj *aTable,
  Symbol         *symbol,
  CtxCFunction   *func
) {
  assert(aTable);
  assert(symbol);
  assert(func);
  
  CtxCFunctionObj *newObj = newCtxCFunction(func);
  insertSymbol(aTable, symbol, asJObj(newObj));
}
\stopCCode

\startTestCase[should register a CtxCFunction]

\startCTest
  ContextObj *testCtxCFunction(ContextObj* aCtx){
    // do nothing but return aCtx
    return aCtx;
  }

  SymbolTableObj *aSymTab = newSymbolTable("TestSymTab7");
  AssertIntTrue(isSymbolTableObj(aSymTab));
  
  registerCtxCFunction(aSymTab, "testCtxCFunction", testCtxCFunction);
  JObj *aCtxCFunc = findSymbol(aSymTab, "testCtxCFunction");
  AssertIntTrue(isCtxCFunctionObj(aCtxCFunc));
  AssertPtrEquals(asCtxCFunction(aCtxCFunc), testCtxCFunction);
\stopCTest
\stopTestCase
\stopTestSuite

\startTestSuite[symbolTablePrinter]

\startCHeader
void symbolTablePrinter(
  StringBufferObj *aStrBuf,
  JObj            *anObj,
  Symbol          *indent,
  size_t           timeToLive
);

//void printTableToFile(
//  SymbolTableObj *aTable,
//  FILE           *aFile,
//  Symbol         *indent
//);

\stopCHeader

\startCCode
void symbolTablePrinter(
  StringBufferObj *aStrBuf,
  JObj            *anObj,
  Symbol          *indent,
  size_t           timeToLive
) {
  DEBUG(PrintingDebug, "symbolTablePrinter %p\n", anObj);
  assert(isSymbolTableObj(anObj));
  
  SymbolTableObj *aTable = asSymbolTableObj(anObj);
  
  if (indent) {
    char* newIndent = appendSymbols(indent, "  ");
    for( size_t i = 0; i < aTable->numSymbols; i++) {
      strBufPrintf(
        aStrBuf,
        "%ssymTab[%zd] = [%s]\n",
        indent,
        i,
        aTable->symbols[i].symbol
      );
      printJObj(
        aStrBuf,
        aTable->symbols[i].jObj,
        newIndent,
        timeToLive
      );
    }
    free(newIndent);
  } else {
    for( size_t i = 0; i < aTable->numSymbols; i++) {
      strBufPrintf(
        aStrBuf,
        "<symTab[%zd] = [%s] ",
        i,
        aTable->symbols[i].symbol
      );
      printJObj(
        aStrBuf,
        aTable->symbols[i].jObj,
        NULL,
        timeToLive
      );
      strBufPrintf(aStrBuf, "%s", "> ");
    }
  }
}
\stopCCode

\startTestCase[should print the symbol table]

\startCTest
  SymbolTableObj *aSymTab = newSymbolTable("TestSymTab8");
  AssertIntTrue(isSymbolTableObj(aSymTab));

  UIntegerObj *int42 = newUInteger(42);
  UIntegerObj *int43 = newUInteger(43);
  UIntegerObj *int44 = newUInteger(44);
  
  insertSymbol(aSymTab, "first",  asJObj(int42));
  insertSymbol(aSymTab, "second", asJObj(int43));
  insertSymbol(aSymTab, "third",  asJObj(int44));
  
  StringBufferObj *aStrBuf = newStringBuffer();
  AssertIntTrue(isStringBufferObj(aStrBuf));

  symbolTablePrinter(aStrBuf, asJObj(aSymTab), "  ", 20);
  AssertStrMatches(getCString(aStrBuf), "symTab%[");
  AssertStrMatches(getCString(aStrBuf), "%] = %[");
  AssertStrMatches(getCString(aStrBuf), "%[first%]");
  AssertStrMatches(getCString(aStrBuf), "%[second%]");
  AssertStrMatches(getCString(aStrBuf), "%[third%]");
  AssertStrMatches(getCString(aStrBuf), "42");
  AssertStrMatches(getCString(aStrBuf), "43");
  AssertStrMatches(getCString(aStrBuf), "44");
  AssertStrMatches(getCString(aStrBuf), "\n");

  strBufReOpen(aStrBuf);
  symbolTablePrinter(aStrBuf, asJObj(aSymTab), NULL, 20);
  AssertStrEquals(getCString(aStrBuf), "<symTab[0] = [first] 42 > <symTab[1] = [second] 43 > <symTab[2] = [third] 44 > ");
  AssertStrDoesNotMatch(getCString(aStrBuf), "\n");
  
\stopCTest
\stopTestCase
\stopTestSuite