% A ConTeXt document [master document: ../jInterp.tex ]

\subsection[title=Pairs]

\startCHeader
typedef struct pair_struct {
  size_t tag;
  JObj  *car;
  JObj  *cdr;
} PairObj;

#define asPairObj(anObj) ((PairObj*)(anObj))
#define isPairObj(anObj) ((anObj) && (((anObj)->tag) == PairTag))
#define isAtomObj(anObj) ((anObj) && (((anObj)->tag) != PairTag))
#define asCar(anObj)     (((PairObj*)(anObj))->car)
#define asCdr(anObj)     (((PairObj*)(anObj))->cdr)
\stopCHeader

\startTestSuite[newPair]

\startCHeader
PairObj *newPair(JObj *car, JObj *cdr);
\stopCHeader

\startCCode
PairObj *newPair(JObj *car, JObj *cdr) {
  PairObj *newObj =
    (PairObj*)calloc(1, sizeof(PairObj));
  newObj->tag = PairTag;
  newObj->car = car;
  newObj->cdr = cdr;
  return newObj;
}
\stopCCode

\startTestCase[should create a new pair]

\startCTest
  JObj *lolA = asJObj(newPair(NULL, NULL));
  JObj *lolB = asJObj(newPair(NULL, NULL));
  JObj *lol  = asJObj(newPair(lolA, lolB));
  AssertIntTrue(isPairObj(lol));
  AssertIntEquals(getJObjTag(lol), PairTag);
  AssertPtrEquals(asCar(lol), lolA);
  AssertPtrEquals(asCdr(lol), lolB);  
\stopCTest
\stopTestCase
\stopTestSuite


\startTestSuite[car and cdr]
\startTestCase[should return the car and cdr]

\startCTest
  JObj *pairA = asJObj(newPair(NULL, NULL));
  JObj *pairB = asJObj(newPair(NULL, NULL));
  JObj *aPair = asJObj(newPair(pairA, pairB));
  AssertPtrEquals(asCar(aPair), pairA);
  AssertPtrEquals(asCdr(aPair), pairB);
\stopCTest
\stopTestCase
\stopTestSuite

\startCHeader
#define popListInto(aCtx, aList, aJObjVar)                \
  JObj* aJObjVar = NULL;                                  \
  if (isPairObj(aList)) {                                 \
    aJObjVar = asCar(aList);                              \
    aList    = asCdr(aList);                              \
  }                                                       \
  if (aCtx->tracingOn) {                                  \
    StringBufferObj *aStrBuf = newStringBuffer();         \
    strBufPrintf(aStrBuf, "%s.%s = ", #aList, #aJObjVar); \
    printJObj(aStrBuf, aJObjVar, "", 20);                 \
    strBufPrintf(aStrBuf, "\n");                          \
    printf("%s", getCString(aStrBuf));                    \
    strBufClose(aStrBuf);                                 \
    free(aStrBuf);                                        \
  }
\stopCHeader

\startTestSuite[listLength]

\startCHeader
UInteger listLength(JObj *aList);
\stopCHeader

\startCCode
UInteger listLength(JObj *aList) {
  if (!aList) return 0;
  
  if (!isPairObj(aList)) return 1;
  
  UInteger numEntries = 0;
  while(isPairObj(aList)) {
    aList = asCdr(aList);
    numEntries++;
  }
  return numEntries;
}
\stopCCode

\startTestCase[should return the length of a list]

\startCTest
  AssertIntZero(listLength(NULL));
  AssertIntEquals(listLength(asJObj(newUInteger(42))), 1);

  JObj *aList =
    asJObj(newPair(NULL, 
      asJObj(newPair(NULL,
        asJObj(newPair(NULL, NULL))
      ))
    ));
  
  AssertIntEquals(listLength(aList), 3);
\stopCTest
\stopTestCase
\stopTestSuite

\startTestSuite[concatLists]

\startCHeader
JObj* concatLists(JObj *firstList, JObj *secondList);
\stopCHeader

\startCCode
JObj* concatLists(
  JObj *firstList,
  JObj *secondList
) {
  JObj *result = NULL;
 
  if (isPairObj(firstList)) {
    result = firstList;
  } else {
    result = asJObj(newPair(firstList, NULL));
  }
 
  if (!secondList) return result;
  // ensure that the second list is a LIST
  if (!isPairObj(secondList)) {
    secondList = asJObj(newPair(secondList, NULL));
  }

  if (!asCar(result)) return secondList;
 
  // find end of firstList/result
  JObj* lolList = result;
  while(isPairObj(asCdr(lolList))) {
    lolList = asCdr(lolList);
  }

  // ensure that if firstList/result ends in a non-pair we make it a pair
  if (asCdr(lolList) && !isPairObj(asCdr(lolList))) {
    asCdr(lolList) = asJObj(newPair(asCdr(lolList), NULL));
    assert(asCdr(lolList));
    lolList = asCdr(lolList);
  }

  // place secondList at the end of firstList/result
  assert(!asCdr(lolList));
  asCdr(lolList) = secondList;
 
  return result;
}
\stopCCode

\startTestCase[should concatenate two LoLs]
\startCTest
  StringBufferObj *aStrBuf = newStringBuffer();
  
  JObj* result = concatLists(NULL, NULL);
  AssertIntTrue(isPairObj(result));
  printJObj(aStrBuf, result, NULL, 20);
  AssertStrEquals(getCString(aStrBuf), "(  )  ");
  strBufClose(aStrBuf);
  
  JObj *trueBool  = asJObj(newUInteger(1));
  result = concatLists(trueBool, NULL);
  AssertIntTrue(isPairObj(result));
  AssertIntTrue(isUIntegerObj(asCar(result)));
  AssertIntEquals(asUInteger(asCar(result)), 1);
  AssertPtrNull(asCdr(result));
  printJObj(aStrBuf, result, NULL, 20);
  AssertStrEquals(getCString(aStrBuf), "( 1 )  ");
  strBufClose(aStrBuf);
  
  JObj *falseBool = asJObj(newUInteger(0));
  result = concatLists(trueBool, falseBool);
  AssertIntTrue(isPairObj(result));
  AssertIntTrue(isUIntegerObj(asCar(result)));
  AssertIntEquals(asUInteger(asCar(result)), 1);
  AssertIntTrue(isPairObj(asCdr(result)));
  AssertIntTrue(isUIntegerObj(asCar(asCdr(result))));
  AssertIntEquals(asUInteger(asCar(asCdr(result))), 0);
  printJObj(aStrBuf, result, NULL, 20);
  AssertStrEquals(getCString(aStrBuf), "( 1 ) ( 0 )  ");
  strBufClose(aStrBuf);
  
  JObj* firstList = result;
  result = concatLists(result, NULL);
  AssertPtrEquals(firstList, result);
  printJObj(aStrBuf, result, NULL, 20);
  AssertStrEquals(getCString(aStrBuf), "( 1 ) ( 0 )  ");
  strBufClose(aStrBuf);
  
  result = concatLists(firstList, trueBool);
  AssertPtrNotNull(result);
  AssertPtrEquals(firstList, result);
  AssertPtrNotNull(asCar(result));
  AssertIntTrue(isUIntegerObj(asCar(result)));
  AssertIntEquals(asUInteger(asCar(result)), 1);
  AssertPtrNotNull(asCdr(result));
  AssertIntTrue(isPairObj(asCdr(result)));
  AssertIntTrue(isUIntegerObj(asCar(asCdr(result))));
  printJObj(aStrBuf, result, NULL, 20);
  AssertStrEquals(getCString(aStrBuf), "( 1 ) ( 0 ) ( 1 )  ");
  strBufClose(aStrBuf);
\stopCTest
\stopTestCase
\stopTestSuite

\startTestSuite[copyLoL]

\startCHeader
JObj* copyLoL(JObj* aLoL);
\stopCHeader

\startCCode
JObj* copyLoL(JObj* aLoL) {
  if (!aLoL) return NULL;
  if (isAtomObj(aLoL)) return aLoL;

  return asJObj(newPair(
    copyLoL(asCar(aLoL)),
    copyLoL(asCdr(aLoL))
  ));
}
\stopCCode

\startTestCase[should make a correct copy]

\startCTest
  JObj *bool   = asJObj(newUInteger(1));
  JObj *pairA  = asJObj(newPair(bool, NULL));
  JObj *pairB  = asJObj(newPair(pairA, bool));
  JObj *aPair0 = asJObj(newPair(pairA, pairB));
  JObj *aPair1 = copyLoL(aPair0);
  AssertIntTrue(isUIntegerObj(bool));
  AssertIntTrue(isPairObj(pairA));
  AssertIntTrue(isPairObj(pairB));
  AssertIntTrue(isPairObj(aPair0));
  AssertIntTrue(isPairObj(aPair1));

  AssertIntTrue(isPairObj(asCar(aPair1)));
  AssertPtrNotEquals(asCar(aPair1), pairA);

  AssertIntTrue(isUIntegerObj(asCar(asCar(aPair1))));
  AssertPtrEquals(asCar(asCar(aPair1)), bool);
  AssertIntEquals(asUInteger(asCar(asCar(aPair1))), 1);

  AssertPtrNull(asCdr(asCar(aPair1)));

  AssertIntTrue(isPairObj(asCdr(aPair1)));
  AssertPtrNotEquals(asCdr(aPair1), pairB);

  AssertIntTrue(isPairObj(asCar(asCdr(aPair1))));

  AssertIntTrue(isUIntegerObj(asCdr(asCdr(aPair1))));
  AssertPtrEquals(asCdr(asCdr(aPair1)), bool);
  AssertIntEquals(asUInteger(asCdr(asCdr(aPair1))), 1);
\stopCTest
\stopTestCase
\stopTestSuite

\startTestSuite[pairPrinter]

\startCHeader
void pairPrinter(
  StringBufferObj *aStrBuf,
  JObj            *anObj,
  Symbol          *indent,
  size_t          timeToLive
);
\stopCHeader

\startCCode
void pairPrinter(
  StringBufferObj *aStrBuf,
  JObj            *anObj,
  Symbol          *indent,
  size_t           timeToLive
) {
  DEBUG(DebugMask, "pairPrinter %p\n", anObj);
  assert(isStringBufferObj(aStrBuf));
  assert(isPairObj(anObj));
  
  if (indent) {
    strBufPrintf(aStrBuf, "%s( \n", indent);
    Symbol *newIndent = appendSymbols(indent, "  ");
    printJObj(aStrBuf, asCar(anObj), newIndent, timeToLive - 1);
    free((char*)newIndent);
    strBufPrintf(aStrBuf, "%s)\n", indent);
    printJObj(aStrBuf, asCdr(anObj), indent, timeToLive - 1);
  } else {
    strBufPrintf(aStrBuf, "%s", "( ");
    printJObj(aStrBuf, asCar(anObj), indent, timeToLive - 1);
    strBufPrintf(aStrBuf, "%s", ") ");
    printJObj(aStrBuf, asCdr(anObj), indent, timeToLive - 1);
  }
  
}
\stopCCode

\startTestCase[should print pairs]

\startCTest
  PairObj *pairA = newPair(NULL, NULL);
  PairObj *pairB = newPair(NULL, NULL);
  PairObj* aNewPair = newPair(asJObj(pairA), asJObj(pairB));
  AssertIntTrue(isPairObj(aNewPair));
  StringBufferObj *aStrBuf = newStringBuffer();
  printJObj(aStrBuf, asJObj(aNewPair), NULL, 20);
  AssertStrEquals(getCString(aStrBuf), "( (  )  ) (  )  ");
  strBufClose(aStrBuf);
  free(aStrBuf);
\stopCTest
\stopTestCase
\stopTestSuite

