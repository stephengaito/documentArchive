% A ConTeXt document [master document: ../jInterp.tex ]

\subsection[title=Contexts]

In order to keep things simple we amalgamate the \joylol\ Context with the 
Parser Machine (a specialized Context) and the Renderer Machine (another 
specialized Context). 

\startCHeader
typedef struct symbolTable_struct SymbolTableObj;
typedef struct context_struct ContextObj;

typedef struct context_struct {
  size_t           tag;
  JObj            *data;
  JObj            *process;
  JObj            *command;
  Symbol          *name;
  SymbolTableObj  *symbolTable;
  Boolean          parserCommitted;
  Boolean          parserFailed;
  Symbol          *parserText;
  Symbol          *parserTextName;
  size_t           parserTextSize;
  size_t           parserTextPosition;
  SymbolTableObj  *parserSymbolTable;
  JObj            *parserBTPs;
  StringBufferObj *renderBuffer;
  Boolean          tracingOn;
  Boolean          debug;
  Boolean          breakPoint;
  size_t           showDepth;
  Boolean          exceptionRaised;
  ContextObj      *parent;
} ContextObj;

#define isContextObj(anObj) ((anObj) && (((ContextObj*)(anObj))->tag == ContextTag))
#define asContextObj(anObj) ((ContextObj*)(anObj))
#define asCtxName(anObj) (((ContextObj*)(anObj))->name)
#define asCtxData(anObj) (((ContextObj*)(anObj))->data)
#define asCtxProcess(anObj) (((ContextObj*)(anObj))->process)
#define asCtxCommand(anObj) (((ContextObj*)(anObj))->command)
#define asCtxSymbols(anObj) (((ContextObj*)(anObj))->symbolTable)
#define asCtxBreakPoint(anObj) (((ContextObj*)(anObj))->breakPoint)
#define asParserCommitted(anObj) (((ContextObj*)(anObj))->parserCommitted)
#define asParserFailed(anObj) (((ContextObj*)(anObj))->parserFailed)
#define asParserText(anObj) (((ContextObj*)(anObj))->parserText)
#define asParserTextName(anObj) (((ContextObj*)(anObj))->parserTextName)
#define asParserTextSize(anObj) (((ContextObj*)(anObj))->parserTextSize)
#define asParserTextPosition(anObj) (((ContextObj*)(anObj))->parserTextPosition)
#define resetParserTextPosition(anObj) \
  asParserTextPosition(anObj) = 0
#define getParserCurChar(anObj) \
  (*((char*)(asParserText(anObj)+asParserTextPosition(anObj))))
#define advanceParserTextPosition(anObj) asParserTextPosition(anObj)++
#define asParserSymbols(anObj) (((ContextObj*)(anObj))->parserSymbolTable)
#define asParserBTPs(anObj) (((ContextObj*)(anObj))->parserBTPs)
#define asRenderBuffer(anObj) (((ContextObj*)(anObj))->renderBuffer)
#define asCtxExceptionRaised(anObj) (((ContextObj*)(anObj))->exceptionRaised)
#define resetCtxExceptionRaised(anObj) \
  asCtxExceptionRaised(anObj) = 0
\stopCHeader

\startTestSuite[newContext]

\startCHeader
ContextObj *newContext(JObj *data, JObj *process, Symbol *name);
\stopCHeader

\startCCode
ContextObj *newContext(JObj *data, JObj *process, Symbol *name) {
  ContextObj *newObj =
    (ContextObj*)calloc(1, sizeof(ContextObj));
  newObj->tag                = ContextTag;
  newObj->data               = data;
  newObj->process            = process;
  newObj->command            = NULL;
  newObj->name               = strdup(name);
  Symbol *tableName          = appendSymbols(name, "Symbols");
  
  assert(tableName);
  newObj->symbolTable        = newSymbolTable(tableName);
  free((char*)tableName);
  
  newObj->parserCommitted    = false;
  newObj->parserFailed       = false;
  newObj->parserText         = NULL;
  newObj->parserTextSize     = 0;
  newObj->parserTextPosition = 0;
  newObj->parserSymbolTable  = NULL;
  newObj->renderBuffer       = NULL;
  newObj->tracingOn          = false;
  newObj->debug              = false;
  newObj->breakPoint         = false;
  newObj->showDepth          = 5;
  newObj->exceptionRaised    = false;
  newObj->parent             = NULL;
  return newObj;
}
\stopCCode

\startTestCase[should create a new context]

\startCTest
  JObj *data       = asJObj(newPair(NULL, NULL));
  JObj *process    = asJObj(newPair(NULL, NULL));
  ContextObj* aCtx = newContext(data, process, "TestCtx");
  AssertIntTrue(isContextObj(aCtx));
  AssertIntEquals(getJObjTag(aCtx), ContextTag);
  AssertStrEquals(asCtxName(aCtx), "TestCtx");
  AssertPtrEquals(asCtxData(aCtx), data);
  AssertPtrEquals(asCtxProcess(aCtx), process);
  AssertIntFalse(asCtxBreakPoint(aCtx));
  AssertIntTrue(isSymbolTableObj(asCtxSymbols(aCtx)));
  AssertIntFalse(asParserFailed(aCtx));
  AssertPtrNull(asParserText(aCtx));
  AssertIntZero(asParserTextSize(aCtx));
  AssertIntZero(asParserTextPosition(aCtx));
  AssertPtrNull(asParserSymbols(aCtx));
  AssertPtrNull(asRenderBuffer(aCtx));
  AssertIntFalse(aCtx->tracingOn);
  AssertIntFalse(aCtx->debug);
  AssertPtrNull(asCtxCommand(aCtx));
  AssertIntFalse(aCtx->exceptionRaised);
  AssertPtrNull(aCtx->parent);
\stopCTest
\stopTestCase
\stopTestSuite

\startTestSuite[initializeParser]

\startCHeader
void initializeParser(ContextObj *aCtx, Symbol *parserName);
\stopCHeader

\startCCode
void initializeParser(ContextObj *aCtx, Symbol *parserName) {
  assert(isContextObj(aCtx));
  asParserSymbols(aCtx) = newSymbolTable(parserName);
}
\stopCCode

\startTestCase[should create the parser symbol table]

\startCTest
  ContextObj *aCtx = newContext(NULL, NULL, "testCtxParser");
  AssertIntTrue(isContextObj(aCtx));
  
  AssertPtrNull(asParserSymbols(aCtx));
  initializeParser(aCtx, "testParser");
  AssertIntTrue(isSymbolTableObj(asParserSymbols(aCtx)));
  AssertStrEquals(asSymbolTableName(asParserSymbols(aCtx)), "testParser");
\stopCTest
\stopTestCase
\stopTestSuite


\startTestSuite[parseString]

\startCHeader
void parseString(ContextObj *aCtx, Symbol *aStringToParse);
\stopCHeader

\startCCode
void parseString(ContextObj *aCtx, Symbol *aStringToParse) {
  assert(isContextObj(aCtx));
  
  size_t aStrSize = 0;
  if (aStringToParse) aStrSize = strlen(aStringToParse);
    
  asParserText(aCtx)         = aStringToParse;
  asParserTextSize(aCtx)     = aStrSize;
  asParserTextPosition(aCtx) = 0;
  asParserCommitted(aCtx)    = false;
  asParserFailed(aCtx)       = false;
}
\stopCCode

\startTestCase[should load a string into parser text]

\startCTest
  ContextObj *aCtx = newContext(NULL, NULL, "testParseString");
  AssertIntTrue(isContextObj(aCtx));
  
  AssertPtrNull(asParserText(aCtx));
  AssertIntZero(asParserTextSize(aCtx));
  parseString(aCtx, "");
  AssertPtrNotNull(asParserText(aCtx));
  AssertIntEquals(asParserTextSize(aCtx), 0);
  AssertStrMatches(asParserText(aCtx), "");
  
  asParserText(aCtx)     = NULL;
  asParserTextSize(aCtx) = 0;
  AssertPtrNull(asParserText(aCtx));
  AssertIntZero(asParserTextSize(aCtx));
  parseString(aCtx, "This is a test");
  AssertPtrNotNull(asParserText(aCtx));
  AssertIntEquals(asParserTextSize(aCtx), 14);
  AssertStrMatches(asParserText(aCtx), "This is a test");
\stopCTest
\stopTestCase
\stopTestSuite

\startTestSuite[parseFile]

\startCHeader
void parseFile(ContextObj *aCtx, Symbol *aTextFilePath);
\stopCHeader

\startCCode
void parseFile(ContextObj *aCtx, Symbol *aTextFilePath) {
  assert(isContextObj(aCtx));
  
  if (!aTextFilePath || !aTextFilePath[0]) return;
  
  FILE *textFile = fopen(aTextFilePath, "r");
  
  if (!textFile) return;
  
  fseek(textFile, 0L, SEEK_END);
  size_t textSize = ftell(textFile);
  rewind(textFile);
  
  if (textSize < 1) return;
  
  char *textBuffer = (char*)calloc(textSize+1, sizeof(char));
  if (!textBuffer) return;
  
  size_t bytesRead = fread(textBuffer, sizeof(char), textSize, textFile);

  fclose(textFile);
  
  if (bytesRead != textSize) {
    free(textBuffer);
    return;
  }

  asParserText(aCtx)     = textBuffer;
  asParserTextSize(aCtx) = textSize;
}
\stopCCode

\startTestCase[should load file into parser text]

\startCTest
  ContextObj *aCtx = newContext(NULL, NULL, "testParserText");
  AssertIntTrue(isContextObj(aCtx));
  
  AssertPtrNull(asParserText(aCtx));
  AssertIntZero(asParserTextSize(aCtx));
  parseFile(aCtx, "doc/objects/testContextParseFile.txt");
  AssertPtrNotNull(asParserText(aCtx));
  AssertIntEquals(asParserTextSize(aCtx), 105);
  
  AssertStrMatches(asParserText(aCtx), "This is a");
  AssertStrMatches(asParserText(aCtx), "last");
  AssertStrMatches(asParserText(aCtx), "([<{}>])");
\stopCTest
\stopTestCase
\stopTestSuite

\startTestSuite[initializeRenderer]

\startCHeader
void initializeRenderer(ContextObj *aCtx);
\stopCHeader

\startCCode
void initializeRenderer(ContextObj *aCtx) {
  assert(isContextObj(aCtx));
  
  asRenderBuffer(aCtx) = newStringBuffer();
}
\stopCCode

\startTestCase[should create the render buffer]

\startCTest
  ContextObj *aCtx = newContext(NULL, NULL, "testCtxRenderer");
  AssertIntTrue(isContextObj(aCtx));
  
  AssertPtrNull(asRenderBuffer(aCtx));
  initializeRenderer(aCtx);
  AssertIntTrue(isStringBufferObj(asRenderBuffer(aCtx)));
\stopCTest
\stopTestCase
\stopTestSuite

\startTestSuite[writeRenderedText]

\startCHeader
size_t writeRenderedText(ContextObj *aCtx, Symbol *aTextFilePath);
\stopCHeader

\startCCode
size_t writeRenderedText(ContextObj *aCtx, Symbol *aTextFilePath) {
  assert(isContextObj(aCtx));

  Symbol *renderedText = getCString(asRenderBuffer(aCtx));
  
  if (!renderedText || !renderedText[0]) return 0;
  
  size_t renderedTextSize = strlen(renderedText);

  if (renderedTextSize < 1) return 0;
  
  if (!aTextFilePath || !aTextFilePath[0]) return 0;
  
  FILE *textFile = fopen(aTextFilePath, "w");
  
  if (!textFile) return 0;

  size_t bytesWritten =
    fwrite(renderedText, sizeof(char), renderedTextSize, textFile);
  
  fclose(textFile);
  
  return bytesWritten;
}
\stopCCode

\startTestCase[should write rendered text]

\startCTest
  ContextObj *aCtx = newContext(NULL, NULL, "testRenderText");
  AssertIntTrue(isContextObj(aCtx));
  
  initializeRenderer(aCtx);
  
  StringBufferObj *aStrBuf = asRenderBuffer(aCtx);
  
  strBufPrintf(aStrBuf, "%s", "This is a test of rendering\n");
  
  Symbol *testFilePath = "doc/objects/testContextRenderFile.txt";
  size_t bytesWritten  = writeRenderedText(aCtx, testFilePath);
  AssertIntEquals(bytesWritten, 28);
  
  char *testBuffer = calloc(bytesWritten+1, sizeof(char));
  AssertPtrNotNull(testBuffer);
  
  FILE *renderFile = fopen(testFilePath, "r");
  size_t bytesRead = fread(testBuffer, sizeof(char), bytesWritten, renderFile);
  fclose(renderFile);
  AssertIntEquals(bytesRead, bytesWritten);
  AssertStrEquals(getCString(asRenderBuffer(aCtx)), testBuffer);
  free(testBuffer);
\stopCTest
\stopTestCase
\stopTestSuite

\startTestSuite[contextPrinter]

\startCHeader
void contextPrinter(
  StringBufferObj *aStrBuf,
  JObj            *anObj,
  Symbol          *indent,
  size_t           timeToLive
);
\stopCHeader

\startCCode
void contextPrinter(
  StringBufferObj *aStrBuf,
  JObj            *anObj,
  Symbol          *indent,
  size_t           timeToLive
) {
  DEBUG(PrintingDebug, "contextPrinter %p\n",
  anObj);
  assert(isContextObj(anObj));
  if (indent) {
    char *newIndent = appendSymbols(indent, "    ");
    strBufPrintf(aStrBuf, "%s<context:%s>\n", indent, asCtxName(anObj));
    strBufPrintf(aStrBuf, "%s  <data[%zu]>\n",
      indent, listLength(asCtxData(anObj)));
    //printJObj(aStrBuf, asCtxData(anObj), newIndent, 20);
    strBufPrintf(aStrBuf, "%s  <process[%zu]>\n",
      indent, listLength(asCtxProcess(anObj)));
    //printJObj(aStrBuf, asCtxProcess(anObj), newIndent, 20);
    free(newIndent);
  } else {
    strBufPrintf(aStrBuf, "<context:%s ", asCtxName(anObj));
    strBufPrintf(aStrBuf, "  <data[%zu]: ",
      listLength(asCtxData(anObj)));
    //printJObj(aStrBuf, asCtxData(anObj), NULL, 20);
    strBufPrintf(aStrBuf, "> <process[%zu]: ",
      listLength(asCtxProcess(anObj)));
    //printJObj(aStrBuf, asCtxProcess(anObj), NULL, 20);
    strBufPrintf(aStrBuf, "> >");
  }
}
\stopCCode

\startTestCase[should print a context object]

\startCTest
  JObj *data       = asJObj(newPair(NULL, NULL));
  JObj *process    = asJObj(newPair(NULL, NULL));
  ContextObj* aCtx = newContext(data, process, "TestCtx");
  AssertIntTrue(isContextObj(aCtx));
  
  StringBufferObj *aStrBuf = newStringBuffer();
  AssertIntTrue(isStringBufferObj(aStrBuf));
  
  contextPrinter(aStrBuf, asJObj(aCtx), "  ", 20);
  //AssertStrMatches(getCString(aStrBuf), "null");
  AssertStrMatches(getCString(aStrBuf), "<context:TestCtx>")
  AssertStrMatches(getCString(aStrBuf), "<data%[1%]>")
  AssertStrMatches(getCString(aStrBuf), "<process%[1%]>")
  AssertStrMatches(getCString(aStrBuf), "\n")

  strBufReOpen(aStrBuf);
  contextPrinter(aStrBuf, asJObj(aCtx), NULL, 20);
  //AssertStrEquals(getCString(aStrBuf), "<context:TestCtx   <data: (  )  > <process: (  )  > >");
  AssertStrEquals(getCString(aStrBuf), "<context:TestCtx   <data[1]: > <process[1]: > >");
  AssertStrDoesNotMatch(getCString(aStrBuf), "\n");
  
  strBufClose(aStrBuf);
  free(aStrBuf);
\stopCTest
\stopTestCase
\stopTestSuite