% A ConTeXt document [master document: ../jInterp.tex ]

\subsection[title=Contexts]

\startCHeader
typedef struct symbolTable_struct SymbolTableObj;

typedef struct context_struct {
  size_t          tag;
  JObj           *data;
  JObj           *process;
  JObj           *command;
  Symbol         *name;
  SymbolTableObj *symbolTable;
  Boolean         tracingOn;
  Boolean         debug;
  size_t          showDepth;
} ContextObj;

ContextObj *newContext(JObj* data, JObj *process, Symbol *name);
void contextPrinter(JObj *anObj, Symbol *indent, size_t timeToLive);
#define isContextObj(anObj) ((anObj) && (((ContextObj*)(anObj))->tag == ContextTag))
#define asContextObj(anObj) ((ContextObj*)(anObj))
#define asCtxName(anObj) (((ContextObj*)(anObj))->name)
#define asCtxData(anObj) (((ContextObj*)(anObj))->data)
#define asCtxProcess(anObj) (((ContextObj*)(anObj))->process)
\stopCHeader

\startCCode
ContextObj *newContext(JObj *data, JObj *process, Symbol *name) {
  ContextObj *newObj =
    (ContextObj*)calloc(1, sizeof(ContextObj));
  newObj->tag       = ContextTag;
  newObj->data      = data;
  newObj->process   = process;
  newObj->command   = NULL;
  newObj->name      = strdup(name);
  newObj->tracingOn = false;
  newObj->debug     = false;
  return newObj;
}

void contextPrinter(JObj *anObj, Symbol *indent, size_t timeToLive) {
  DEBUG(DebugMask, "contextPrinter %p\n", anObj);
  assert(isContextObj(anObj));
  char *newIndent = appendSymbols(indent, "    ");
  printf("%s<context:%s>\n", indent, asCtxName(anObj));
  printf("%s  <data>\n", indent);
  printJObj(asCtxData(anObj), newIndent, 20);
  printf("%s  <process>\n", indent);
  printJObj(asCtxProcess(anObj), newIndent, 20);
}
\stopCCode
