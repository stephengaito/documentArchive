% A ConTeXt document [master document: ../jInterp.tex ]

\subsection[title=C-Functions]

\startCHeader
typedef void (CFunction)(ContextObj*);

typedef struct cFunction_struct {
  size_t     tag;
  CFunction *func;
} CFunctionObj;

#define isCFunctionObj(anObj) ((anObj) && ((((JObj*)(anObj))->tag) == CFunctionTag))
#define asCFunctionObj(anObj) ((CFunctionObj*)(anObj))
#define asCFunction(anObj) (((CFunctionObj*)(anObj))->func)
\stopCHeader

\startTestSuite[newCFunction]

\startCHeader
CFunctionObj *newCFunction(CFunction* func);
\stopCHeader

\startCCode
CFunctionObj *newCFunction(CFunction *func) {
  CFunctionObj *newObj = 
    (CFunctionObj*)calloc(1, sizeof(CFunctionObj));
  newObj->tag  = CFunctionTag;
  newObj->func = func;
  return newObj;
}
\stopCCode

\startTestCase[should create a new CFunction]

\startCTest
  void testCFunction(ContextObj* aCtx){
    // do nothing
  }

  CFunctionObj* aCFunc = newCFunction(testCFunction);
  AssertIntTrue(isCFunctionObj(aCFunc));
  AssertIntEquals(getJObjTag(aCFunc), CFunctionTag);
  AssertPtrEquals(asCFunction(aCFunc), testCFunction);
\stopCTest
\stopTestCase
\stopTestSuite

\startTestSuite[cFunctionPrinter]

\startCHeader
void cFunctionPrinter(
  StringBufferObj *aStrBuf,
  JObj            *anObj,
  Symbol          *indent,
  size_t           timeToLive
);
\stopCHeader

\startCCode
void cFunctionPrinter(
  StringBufferObj *aStrBuf,
  JObj            *anObj,
  Symbol          *indent,
  size_t           timeToLive
) {
  DEBUG(DebugMask, "cFunctionPrinter %p\n", anObj);
  assert(isCFunctionObj(anObj));
  printf("%s<cFunc:%p>\n", indent, asCFunction(anObj));
}
\stopCCode

\startTestCase[should print a CFunction]

\skipTestCase
\stopTestSuite

\subsection[title=Context changing C-Functions]

\startCHeader
typedef ContextObj* (CtxCFunction)(ContextObj*);

typedef struct ctxCFunction_struct {
  size_t        tag;
  CtxCFunction *func;
} CtxCFunctionObj;

#define isCtxCFunctionObj(anObj) ((anObj) && (((anObj)->tag) == CtxCFunctionTag))
#define asCtxCFunctionObj(anObj) ((CtxCFunction*)(anObj))
#define asCtxCFunction(anObj)    (((CtxCFunctionObj*)(anObj))->func)
\stopCHeader

\startTestSuite[newCtxCFunction]

\startCHeader
CtxCFunctionObj *newCtxCFunction(CtxCFunction* func);
\stopCHeader

\startCCode
CtxCFunctionObj *newCtxCFunction(CtxCFunction *func) {
  CtxCFunctionObj *newObj =
    (CtxCFunctionObj*)calloc(1, sizeof(CtxCFunctionObj));
  newObj->tag  = CtxCFunctionTag;
  newObj->func = func;
  return newObj;
}
\stopCCode

\startTestCase[should create a new CtxCFunction]

\startCTest
  ContextObj *testCtxCFunction(ContextObj* aCtx){
    // do nothing but return aCtx
    return aCtx;
  }
  
  CtxCFunctionObj* aCtxCFunc = newCtxCFunction(testCtxCFunction);
  AssertIntTrue(isCtxCFunctionObj(aCtxCFunc));
  AssertIntEquals(getJObjTag(aCtxCFunc), CtxCFunctionTag);
  AssertPtrEquals(asCtxCFunction(aCtxCFunc), testCtxCFunction);
\stopCTest
\stopTestCase
\stopTestSuite

\startTestSuite[ctxCFunctionPrinter]

\startCHeader
void ctxCFunctionPrinter(
  StringBufferObj *aStrBuf,
  JObj            *jObj,
  Symbol          *indent,
  size_t           timeToLive
);
\stopCHeader

\startCCode
void ctxCFunctionPrinter(
  StringBufferObj *aStrBuf,
  JObj            *anObj,
  Symbol          *indent,
  size_t           timeToLive
) {
  DEBUG(DebugMask, "ctxCFunctionPrinter %p\n", anObj);
  assert(isCtxCFunctionObj(anObj));
  printf("%s<ctxCFunc:%p>\n", indent, asCtxCFunction(anObj));
}
\stopCCode

\startTestCase[should print a ctxCFunction]
\skipTestCase
\stopTestSuite