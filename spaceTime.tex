%% Space-Time

\chapter{Space-time: directed simplicies, refinements and space-time paths}

\section{Introduction}

In the previous section we explored the structures required to provide a \emph{time} focused 
foundations of computation. In this section we expand these time focused structures into 
their corresponding fully space-time structures.

We base our analysis of (directed) space on directed \emph{trans-finite} simplicies.
Recall that \emph{un-directed} simplicial structures are, categorically, defined as the
collection of pre-sheaves on the small category of finite (positive) ordinals. See for
example, the work of Grandis, \cite{grandis2001symSimpSets}, and
\cite{grandis2001fundamentalFunctorsSimplicial}. For a good \emph{concrete} introduction
to the categorical concept of pre-sheaves, see \cite{reyesReyesZolfaghari2004presheaves}.

We use \emph{directed} simplicies \emph{because} we wish to understand the ``higher
dimensional'' \emph{causal} structure of space-time.

Equally importantly, directed simplicies provide a higher dimensional generalisation of
the ubiquitously used concept of ``matrix'' outside of explicitly ``Euclidean vector
spaces''. Essentially a groupoid (that is a un-directed 1-category) \emph{is} a sparse
matrix. This analogy is critical when we want to study higher dimensional analogues of
``Markov Matrices'' and their corresponding ``Markov Chains''.

\TODO{word-smith these introductions together}

We are interested in \ndDeltaC{a}{b}, \sNdDeltaC{a}{b}, \DeltaC{a}{b}, \sDeltaC{a}{b}.

In this chapter, our aim is to define the discrete structures we will use to \emph{model}
our differential topologies.  These are simplicial \emph{structures} which are related to
the simplicial sets used by algebraic topologists.  However we \emph{explicitly} do not
limit ourselves to either finite dimensions \emph{or} to, potentially, assuming that these
structures are \emph{sets}.  With wild abandon, we will allow simplicial structures which
\emph{might} entail Russell's paradox.  Instead we will, essentially, use techniques from
Algebraic Set Theory (AST), to delineate those appropriately ``tame'' simplicial
sub-structures which prohibit Russell's paradox.  \TODO{DO we really need/mean this?
Doesn't category theory (see nLab discussion of set theory and ETCS) essentially
automatically tame Russell's paradox? Or does it just push his paradox into a slightly
more complex environment?  I suspect we could state everything as ``presheaves'' not into
\setC{} but into some larger category of classes/collections which \emph{should} be a
topos but I am unsure how that would change the overall theoretical tools that I want to
use. SO we will really only work with small categories and leave the full YACT theory for
later generations to finalize in all of its glory.}

\section{The small categories of \texorpdfstring{$\kappa$}{kappa}-Simplexes} 

In a Categorical setting, such as we are using, the most appropriate way to define
simplicial sets is via the Topos of functor presheaves, \opFuncCat{\DeltaC{}{}}{\setC{}}
(alternatively \altOpFuncCat{\DeltaC{}{}}{\setC{}}), see for example \cite[Chapter
1]{goerssJardin1999SimplicialHomotopyTh}, \cite[Section
VII.5]{macLane1971categoriesWorkingMathematician}, \cite[section
2.1]{verity2005complicialSets}, \cite[Section I.2]{may1967simplicialObjectsInAlgTopo},
\cite[Chapter 2]{gabrielZisman1967homotopyTh} \TODO{add the numerous papers}.

We will actually base our transfinite generalization on Dominic Verity's exposition,
\cite[section 2.1]{verity2005complicialSets}. The essential difference between the
expositions of Goerss and Jardin's versus that of Verity is that Goerss and Jardin's
exposition uses a more purely Categorical notation (which avoids the use of internal set
theoretic notation), while Verity's exposition uses a more classical notation by
essentially working in the internal set theory of \setC{}.  For those who are not familiar
with simplicial topology, Friedman's expository account of simplicial sets,
\cite{friedman2008illustratedIntroduction}, provides a highly recommended introduction to 
this theory
with a good number of visual explanations of what is going on both combinatorially as well
as geometrically.

It is significant that Goerss and Jardin's exposition makes frequent, mainly illustrative,
use of the adjunction between the categories of Simplicial Sets, \simpC{}, and Topology,
\topologyC{} (and in particular the topological space of ``Euclidean space'', $\Reals^{n}$
for some $n$).  At this point in our exposition (construction?) we have not defined
``Euclidean space''.  From our point of view, Euclidean spaces are particular types of
approximation structures in the category of Simplicity Sets, \simpC{}.  This is
\emph{because} we are primarily interested in how space-time is ``constructed'' for the
experience of a beastie, and do not assume its existence\footnote{We will however make use
of Euclidean spaces when we come to prove the \emph{existence} of sufficiently complex
simplicial sets}.

As stated in the introduction we are actually interested in including, at least, countably
infinite dimensional simplicies. To do this we need to generalize the usual definitions of
simplicial sets to that of \emph{transfinite} simplicial sets.  We do this by essentially
simply re-interpreting Dominc Verity's definition of \emph{finite} dimensional simplicial
sets in a transfinite setting.  Essentially all of the additional work is done by
generalizing the simplicial face and degeneracy maps to include these maps to and from
limit ordinals as well as ``just'' finite ordinals.

\section{Ordinals}

\begin{definition}
A pograph, \gC{}, is \define{totally ordered} if there is exactly one morphism between any
two objects. If \gC{} is a pograph, then any initial or terminal objects it might have are
unique. An initial object of \gC{} is the \define{least element} of \gC{}. A terminal
object of \gC{} is the \define{greatest element} of \gC{}.  A totally ordered graph,
\gC{}, is \define{well-ordered} if every non-empty full subgraph of \gC{}, has a least
element.  Given a totally-ordered graph, \gC{}, and an object $a \in \objects{\gC{}}$, the
\define{section}, $\gC{}_a$, is the full subgraph of \gC{}, who's objects are defined by
$\gC{}_a = \set{x \in \objects{\gC{}} \suchThat x \leq a \AND x \neq a}$.
\end{definition}

\TODO{We should probably be careful with our definition of totally ordered here.  We might
define totally ordered to be a pograph for which any \emph{distinct unordered} pair of
objects has a morphism.  If we drop the distinctness, then we essentially get a pocat. 
The critical distinction here is which types of graph maps are "natural"... if we disallow
morphisms between the same object, then we force all graph maps to be \emph{strict}, if we
allow morphisms between the same object, then we allow non-strict graph maps.  Which
definition do we really want?}

Category theory is all about relationships, in the form of morphisms, between objects.  In
categorical \emph{theory}, the objects themselves have no explicit structure other than
that detectable from the morphisms between the objects.  In \emph{applications} of
category theory, the objects of a category often have considerable explicit structure, for
example, the category of Groups, or the category of Topological Spaces.  One of the
recurring themes in our work is the interplay between relationships \emph{inside}
structured objects and the relationships \emph{between} these structured objects.  Our
first primary example of this is in our definition of ordinal graphs in the graph of all
totally ordered graphs.
\begin{definition}
The \define{graph of totally ordered graphs}, \totalOrderedC{}, is defined by
\begin{description}
\item[objects] are totally ordered graphs
\item[morphisms] are graph maps between totally ordered graphs.
\end{description}
\end{definition}
Note that graph maps between totally ordered graphs are the graphical form of the order
preserving maps of (set based) posets. We will study them in depth in the next section.

Before we can define the ordinals, we need to understand co-limits of objects in
\totalOrderedC{}.  This would be simpler if we already had the whole structure of Category
and Topos theory, something we won't have until much later in this mathematical narrative.
 However the actual construction is fairly simple and moreover adds to our understanding
of Category theory. The co-limits in \graphC{} are simply the ``disjoint union'' of an
appropriate indexing set of the component graphs.  Unfortunately, in \totalOrderedC{},
there is \emph{exactly} one morphism between any pair of objects, so the simple ``disjoint
union'' construction is insufficient for our needs.  Instead we can define ordinals if we
understood the successor construction, the order relation between ordinals, and the union
(directed co-limit) of predecessor ordinals.  \TODO{Point out that this is really the
definition of the von Neumann ordinals in a graphical context.}

\begin{definition}
Consider a totally ordered graph, \tC{}. The \define{successor graph}, \successor{\tC{}},
of \tC{}, is
\begin{description}
\item[objects] the objects of \tC{}, \objects{\tC{}}, together with \tC{},
\item[morphisms] the morphisms of \tC{} between any two objects of \tC{} , together with
the morphism $a \leq \tC{}$  for each object, $a \in \objects{\tC{}}$.
\end{description}
\end{definition}
By construction, \successor{\tC{}} is a totally ordered graph.

\begin{definition}
Consider a pair of graphs, \gC{} and \mathCategory{G'}.  Then \define{\gC{} is a
predecessor of \mathCategory{G'}}, denoted $\gC \leq \mathCategory{G'}$ iff \gC{} is a
full subgraph of \mathCategory{G'}.
\end{definition}
Again, by construction, we know that $\tC{} \leq \successor{\tC}$.  More importantly since
a full subgraph of a full subgraph is a full subgraph (via the composition of graph maps),
we know that $\tC{} \leq \successor{\successor{\cdots \successor{\tC{}}}}$.

\begin{definition}
Consider a totally ordered collection of totally ordered graphs. Then, (not surprisingly)
it is a totally ordered graph. \TODO{Is this not just a tautology? FIX THIS!  What we
really want to do here is to give an explicit construction of the directed co-limit.}
\end{definition}

\begin{definition}
The \define{graph of countable ordinals}, \omegaC{}, is defined by
\begin{description}
\item[objects] the totally ordered graphs, \successor{\successor{\cdots
\successor{\zeroC{}}}}, \item[morphisms] $a \leq b$ iff $a$ is a full subgraph of $b$ for
$a, b \in \objects{\omegaC{}}$.
\end{description}
\TODO{We really have not given a good description of the collection of objects (how many
s's are allowed?).}
\end{definition}
We can picture \omegaC{} as the following totally ordered graph:
\begin{cTikzPicture}
\coordinate (v0) at (0,0);
\coordinate (v1) at (2,0);
\coordinate (v2) at (4,0);
\coordinate (v-n-1) at (6,0);
\coordinate (v-n) at (8,0);
%
\node[above] at (v0)    {\zeroC{}};
\node[above] at (v1)    {\successor{\zeroC{}}};
\node[above] at (v2)    {\successor{\successor{\zeroC{}}}};
\node[above] at (v-n-1) {\successor{\successor{\cdots\successor{\zeroC{}}}}};
%\node[above] at (v-n)   {};
%
\fill (v0) circle[radius=1pt];
\fill (v1) circle[radius=1pt];
\fill (v2) circle[radius=1pt];
\fill (v-n-1) circle[radius=1pt];
%\fill (v-n) circle[radius=1pt];
%
\begin{scope}[->,shorten <=4pt,shorten >=4pt]
\path (v0) edge (v1);
\path (v1) edge (v2);
\path (v2) edge[dotted] (v-n-1);
\path (v-n-1) edge[dotted] (v-n);
\end{scope}
\end{cTikzPicture}

The following is a highly impredicative definition of the ordinals.  We could easily use
the ideas in \cite{joyalMoerdijk1995algSetTh} or even Martin-L\"{o}f's type theory, see
for example \cite{dybjer1991inductiveSets}, to provide a fully predicative definition for
at least the countable ordinals that we require.
\begin{definition}
(\cite{cameron1999a})  An \define{ordinal} is a well-ordered graph, \gC{}, for which
$\gC{}_a = a$ for all $a \in \objects{\gC{}}$.  \TODO{We need a better symbol for section
which we really should name tosection (total ordered section) so that we don't conflict
with the categorical notion of a section.} In other words each object in an ordinal,
\gC{}, is the totally ordered graph of all of its predecessors. \define{Zero}, \zeroC{},
which is defined to be the graph with one object, is the smallest ordinal.  An ordinal
which is the union (directed co-limit in \totalOrderedC{}) of all of its predecessors
$\lambda = \underset{\alpha < \lambda}{\union} \alpha$ is a \define{limit ordinal}. A
non-zero ordinal that is not a limit ordinal is a \define{successor ordinal}. A
\define{cardinal} is an ordinal $\alpha$ with the property that there is no bijective
graph map between $\alpha$ and any section of $\alpha$. \TODO{Need to define bijective
graph maps! Do we really need the definition of cardinal?!}
\end{definition}

\begin{definition}
For any pair of ordinals, $\alpha$ and $\beta$, \define{ordinal addition}, $\alpha +
\beta$, is recursively defined as follows:
\begin{enumerate}
\item $\alpha + \zeroC{} = \alpha$,
\item if $\beta$ is a successor ordinal, then $\alpha + \successor{\beta} =
\successor{\alpha + \beta}$,
\item if $\beta$ is a limit ordinal, then $\alpha + \beta = \underset{\eta <
\beta}{\union} \; \alpha + \eta$.
\end{enumerate}
\end{definition}
It is important to realize that ordinal addition is not commutative, in particular, note
that $n + \omegaC{} = \omegaC{}$ but that $\omegaC{} + n \neq \omegaC{}$. \TODO{We really
need to show that $\underset{\eta < \omegaC}{\union} \; n + \eta$ is isomorphic to
\omegaC{}. }

\TODO{ We need to show all of the following: ( \cite[Chapter 2: Ordinal
Numbers]{jech2003setTheory})
\begin{enumerate}
\item Any full subgraph of a well-ordered graph is well-ordered.
\item Any well-ordered graph, $X$, is isomorphic to a unique ordinal, $\ordinal{X}$.
\item For any pair of ordinals, $\alpha < \beta$ if and only if $\alpha \in \objects{\beta}$.
\item For any ordinal, $\objects{\alpha} = \set{ \beta \suchThat \beta < \alpha}$.
\item The successor of an ordinal, $\alpha$, is $\successor{\alpha} = \alpha + 1 = \alpha
\union \set{ \alpha }$. \TODO{This notation needs explaining or changing!}
\item If an ordinal $\alpha = \successor{\beta} = \beta + 1$ for some ordinal $\beta$ then
$\alpha$ is a \define{successor ordinal}.
\item If an ordinal $\lambda$ is not a successor ordinal for any $\beta < \lambda$, then
$\lambda$ is a \define{limit ordinal}.   (Note that it is convenient for us to define $0$
as a limit ordinal).
\item For any pair of ordinals, $\alpha$ and $\beta$, \define{ordinal addition} is defined 
as follows:
\begin{enumerate}
\item $\alpha + 0 = \alpha$
\item $\alpha + \successor{\beta} = \successor{\alpha + \beta}$
\item if $\beta$ is a limit ordinal, then $\alpha + \beta = \union_{\eta < \beta} \alpha +
\eta$
\TODO{use Jech's limit definitions -- actually we need to define our own graphical versions!}
\end{enumerate}
\end{enumerate}
}

Throughout this section we work with ordinals inside the Topos, \setC{}.  We note that
every surjection in \setC{} is a split epimorphism. This is equivalent to the internal set
theory of \setC{} allowing the Axiom of Choice, which similarly is equivalent to the
ordinals begin well ordered.  Since we are working in \setC{}, we may phrase our arguments
using classical set theory.

\begin{definition} (\cite[Definition 1, Section 2.1]{verity2005complicialSets})
In \setC{}, fix $\kappa$ as a fixed \emph{small} cardinal, such as $\omega$.  
The \define{Category of symmetric $\kappa$-simplexes}, denoted \sDeltaC{}{}, is 
\begin{enumerate}
\item \textbf{objects} the set of ordinals, $0 \leq \alpha \leq \kappa$, (in \setC{})
considered as well-ordered sets, $\woSet{ \alpha } = \set{ 0 < 1 < \cdots < \alpha} =
s(\alpha)$,
\item \textbf{morphisms} the set of maps $\gamma : \woSet{ \alpha } \mapsTo \woSet{ \beta
}$ (in \setC{}), from the well-ordered set \woSet{\alpha} to the well-ordered set
\woSet{\beta}.
\end{enumerate}
The \define{Category of non-degenerate symmetric $\kappa$-simplexes}, denoted
\sNdDeltaC{}{}, is the subcategory of \sDeltaC{}{} consisting of all objects of
\sDeltaC{}{} together with only the monomorphisms of \sDeltaC{}{}.

The \define{Category of $\kappa$-simplexes}, denoted \DeltaC{}{}, is the subcategory of
\sDeltaC{}{} consisting of all of the objects of \sDeltaC{}{} together with only the order
preserving morphisms of \sDeltaC{}{}.

The \define{Category of non-degenerate $\kappa$-simplexes}, denoted \ndDeltaC{}{}, is the
subcategory of \DeltaC{}{} consisting of all of the objects of \DeltaC{}{} together with
only the \emph{strictly} order preserving morphisms of \DeltaC{}{}.
\end{definition}

For finite $n$, the graph \woSet{n} can be depicted as:
\begin{cTikzPicture}
\coordinate (v0) at (0,0);
\coordinate (v1) at (1.5,0);
\coordinate (v2) at (3,0);
\coordinate (v-n-1) at (4.5,0);
\coordinate (v-n) at (6,0);
%
\node[above] at (v0)    {$0$};
\node[above] at (v1)    {$1$};
\node[above] at (v2)    {$2$};
\node[above] at (v-n-1) {$n-1$};
\node[above] at (v-n)   {$n$};
%
\fill (v0) circle[radius=1pt];
\fill (v1) circle[radius=1pt];
\fill (v2) circle[radius=1pt];
\fill (v-n-1) circle[radius=1pt];
\fill (v-n) circle[radius=1pt];
%
\begin{scope}[->,shorten <=4pt,shorten >=4pt]
\path (v0) edge (v1);
\path (v1) edge (v2);
\path (v2) edge[dotted] (v-n-1);
\path (v-n-1) edge (v-n);
\end{scope}
\end{cTikzPicture}


The first and most important task is to understand the structure and compositional
decomposition of the hom sets of \sDeltaC{}{}, \DeltaC{}{} and \ndDeltaC{}{}.  We begin by
essentially following \cite[Notation 3 and Observations 7 and 8]{verity2005complicialSets}. 
We note that, the usual simplicial face maps are not able to describe face maps which
``cross'' limit ordinals.  Instead of being able to define all face maps in terms of
``dropping'' a single vertex, we must work by ``dropping'' sets of verticies.

Before we begin, consider an orientation preserving map $m : \woSet{\alpha} \mapsTo
\woSet{\beta}$ then:
\begin{enumerate}
\item Let \image{m} denote the subset of \woSet{\beta} given by:
\begin{equation*}
	\image{m} = \set{ \gamma \in \woSet{\beta} \suchThat \thereExists \delta \in \woSet{\alpha} \forWhich m(\delta) = \gamma }
\end{equation*}
\item Let \cImage{m} denote the subset of \woSet{\beta} given by:
\begin{equation*}
	\cImage{m} = \woSet{\beta} \withOut \image{m}
\end{equation*}
Note that \cImage{m} is the subset of \woSet{\beta} which the morphism $m$ has ``dropped''.
\item Let \kernel{m} denote the subset of \woSet{\alpha} given by:
\begin{equation*}
	\kernel{m} = \set{ s(\gamma) \in \woSet{\alpha} \suchThat m(\gamma) = m(s(\gamma)) }
\end{equation*}
\end{enumerate}

\subsection{Properties of maps in \texorpdfstring{\sNdDeltaC{}{}}{sNdDelta} and
\texorpdfstring{\sDeltaC{}{}}{ndDelta}}

The following lemma is almost certainly in the category theory folklore\footnote{If I
could read French Mathematics, it is almost certainly to be found in SGA4
\cite{artinGrothendieckVerdierBourbakiDeligneSaintDonat1963SGA41},
\cite{artinGrothendieckVerdierDeligneSaintDonat1963SGA42} or
\cite{artinGrothendieckVerdierDeligneSaintDonat1963SGA43}, since I believe I have seen
references to Grothendieck making use of the category of symmetric simplexes,
\sDeltaC{}{}.}. The idea for the definitions of symmetric and non-degenerate symmetric
simplexes comes from the PhD thesis of John Shrimpton, \cite[Diagram on page
29]{shrimpton1989graphsSymmetryCatMethods}, elements of which were later published in
\cite{brownMorrisShrimptonWensley2008graphCat}. The importance of these categories of
symmetric and non-degenerate symmetric simplexes, other than possible references to their
use by Grothendieck, is that the ``standard'' Algebraic Topology definitions of
orientations of simplicial structures is wholly inadequate for our needs.

\begin{lemma} (Combing lemma)
Every morphism, $m : \woSet{\alpha} \mapsTo \woSet{\beta}$, in \sDeltaC{}{} can be
factored into an automorphism, $n$ of \woSet{\alpha} followed by an order preserving map,
$\hat{m} : \woSet{\alpha} \mapsTo \woSet{\beta}$ (i.e. a morphism of \DeltaC{}{}).
\end{lemma}
\begin{proof} 
Essentially we are combing a collection of threads one per element of \woSet{\alpha}.  As
such we can hold either end fixed and ``comb'' the threads straight. Since the kernel of
the morphism $m$, \kernel{m}, need not be empty it is prudent to hold the codomain
(\woSet{\beta}) fixed and comb back to the domain.

We will use (transfinite) induction on the elements of \woSet{\alpha}. The required
automorphism will be the product of a permutation for each element of \woSet{\alpha}. 
Since at each step of the induction, all previous ``threads'' have been ordered, each
subsequent permutation is independent of all previous permutations, so that no cycles are
longer than a two cycle.  This is important for the induction over any limit ordinals in
\woSet{\alpha}. the ordered product of the permutations at each induction step.

At each step, $0 \leq \gamma \leq \alpha$, we choose a permutation so that the appropriate
product of the inverses of the collection of permutations for all $0 \leq \lambda <
\gamma$ with the original morphism, $m$, is, when restricted to \set{\lambda \suchThat
\lambda < \gamma} is an order preserving map. We can then choose a permutation which
permutes $\gamma$ with the origin of the least element of $\tuple[]{m(\lambda)}_{\gamma
\leq \lambda \leq \alpha}$ (if there are multiple origins for such least elements choose
one).
\end{proof}

\begin{corollary}
Every morphism, $m : \woSet{\alpha} \mapsTo \woSet{\beta}$, in \sNdDeltaC{}{} can be
factored into an automorphism $n$ of \woSet{\alpha} followed by a \emph{strictly} order
preserving map, $\hat{m} : \woSet{\alpha} \mapsTo \woSet{\beta}$ (i.e. a morphism of
\ndDeltaC{}{}).
\end{corollary}
\begin{proof}
Since automorphisms are by definition monomorphisms, if $\hat{m}$ were not a monomorphism,
then the product, $m = \hat{m} \compose n$ would not be a monomorphism and hence not a
morphism of \sNdDeltaC{}{}.
\end{proof}

\begin{corollary}
Every morphism of either \sDeltaC{}{} or \sNdDeltaC{}{} (respectively) can be decomposed,
in possibly many ways, as the product of an automorphism of its domain followed by either
an order preserving or a strictly order preserving (respectively) map from its domain to
its codomain, followed by an automorphism of its codomain.
\end{corollary}

\subsection{Properties of coface maps in \texorpdfstring{\DeltaC{}{}}{Delta} and
\texorpdfstring{\ndDeltaC{}{}}{ndDelta}}

\begin{enumerate}
\item The injective maps in \DeltaC{}{} are referred to as \define{coface maps}.
\item Being injective, coface maps are \emph{strictly} order preserving and hence are also
morphisms in \ndDeltaC{}{}.  Conversely any \emph{strictly} order preserving map is
injective.  This means that \ndDeltaC{}{} only contains the injective or coface morphisms
of \DeltaC{}{}.
\item Consider $\alpha \leq \beta \leq \kappa$, and $U \subset \woSet{\beta}$ for which
$\woSet{\beta} \withOut U \isomorphic \woSet{\alpha}$. Note that $U$ and all of its
subsets are well-ordered since they are subsets of \woSet{\beta}, and hence are all
isomorphic to unique ordinals. We define the simplicial map $\coFace{\alpha}{\beta}{U} :
\woSet{\alpha} \mapsTo \woSet{\beta}$ by
	\begin{equation*}
		\coFace{\alpha}{\beta}{U}(i) = \ordinal{\set{ j \in U \suchThat j \leq i}} + i
	\end{equation*}
Note that since ordinal addition in \setC{} is noncommutative, the choice of order in the
above sum is significant.  We have chosen this order so that
$\image{\coFace{\alpha}{\beta}{U}} = \woSet{\beta}\withOut U$ (or alternatively,
$\cImage{\coFace{\alpha}{\beta}{U}} = U$) where the important case to check is, for
example, $\alpha < \omega \leq \beta$ with $\omega \notin U$.  Equally importantly, we
know that $\kernel{\coFace{\alpha}{\beta}{U}} = \emptySet{}$.
\item\label{property:unique.coface} If $D : \woSet{\alpha} \mapsTo \woSet{\beta}$ is an
injective order-preserving map, then $D = \coFace{\alpha}{\beta}{U}$ where $U = \cImage{D}
= \woSet{\beta} \withOut \image{D}$ (this generalizes \cite[page
4]{may1967simplicialObjectsInAlgTopo}).
\item For all $\alpha \leq \gamma \leq \beta \leq \kappa, \; U \subset \woSet{\beta}, \; W
\subset \woSet{\beta}$ and $V \subset \woSet{\gamma}$, if $\ordinal{\woSet{\beta} \withOut
W} = \ordinal{\woSet{\gamma}}, \; \ordinal{\woSet{\gamma} \withOut V} =
\ordinal{\woSet{\alpha}}$ and $U = \tuple[\coFace{\gamma}{\beta}{W}]{V} \union W$ then
$\coFace{\alpha}{\beta}{U} = \coFace{\gamma}{\beta}{W} \compose
\coFace{\alpha}{\gamma}{V}$.
\item It is instructive to note that, for finite $n$, $\coFace{n}{s(n)}{\set{j}}$ is the
usual coface map $D^n_j$.
\item For each ordinal, $\alpha$ and $\beta \in \woSet{\alpha}$ the simplicial map
$\epsilon^{\alpha}_{\beta} : \woSet{0} \mapsTo \woSet{\alpha}$ given by
$\epsilon^{\alpha}_{\beta}(0) = \beta$ is the $\beta$-vertex map of \woSet{\alpha}.
\end{enumerate}

\subsection{Properties of codegeneracy maps in \texorpdfstring{\DeltaC{}{}}{Delta}}

\begin{enumerate}
\item The surjective maps in \DeltaC{}{} are referred to as \define{codegeneracy maps}.

\item Consider $\beta \leq \alpha \leq \kappa$, and $V \subset \woSet{\alpha}$ for which
$\woSet{\alpha} \withOut V \isomorphic \woSet{\beta}$. Note that $\woSet{\alpha}\withOut
V$ and all of its subsets are well-ordered since they are subsets of \woSet{\alpha}, and
hence they are all isopmorphic to unique ordinals. We define the simplicial map
$\coDegeneracy{\alpha}{\beta}{V} : \woSet{\alpha} \mapsTo \woSet{\beta}$ by
	\begin{equation*}
		\coDegeneracy{\alpha}{\beta}{V}(i) = \ordinal{\set{ j \in \woSet{\alpha}\withOut V
		\suchThat j < i}}
	\end{equation*}
We note that $\kernel{\coDegeneracy{\alpha}{\beta}{V}} = V$ and that
$\image{\coDegeneracy{\alpha}{\beta}{V}} = \woSet{\beta}$.

\item\label{property:unique.codegeneracy} If $S : \woSet{\alpha} \mapsTo \woSet{\beta}$ is
a surjective order-preserving map, then $S = \coDegeneracy{\alpha}{\beta}{\kernel{S}}$
(this generalizes \cite[page 4]{may1967simplicialObjectsInAlgTopo}).

\item For all $\beta \leq \gamma \leq \alpha \leq \kappa, \; U \subset \woSet{\gamma}, \;
W \subset \woSet{\alpha}$ and $V \subset \woSet{\alpha}$ if $\ordinal{\woSet{\alpha}
\withOut W} = \ordinal{\woSet{\gamma}}, \; \ordinal{\woSet{\gamma} \withOut U} =
\ordinal{\woSet{\beta}}$ and $V = \tuple[\tuple{\coDegeneracy{\alpha}{\gamma}{W}}^{-1}]{U}
\union W$, then $\coDegeneracy{\alpha}{\beta}{V} = \coDegeneracy{\gamma}{\beta}{U}
\compose \coDegeneracy{\alpha}{\gamma}{W}$.

\item It is instructive to note that, for finite $n$, $\coDegeneracy{n}{s(n)}{\set{j}}$ is
the usual codegeneracy map $S^n_j$.

\item For each ordinal, $\alpha$ the simplicial map $\eta^{\alpha} : \woSet{\alpha}
\mapsTo \woSet{0}$ given by $\eta^{\alpha}(\beta) = 0$ is the $\alpha$-degeneracy map of
\woSet{\alpha}.
\end{enumerate}

\subsection{Properties of general maps in \texorpdfstring{\DeltaC{}{}}{Delta} and
\texorpdfstring{\ndDeltaC{}{}}{ndDelta}}

\begin{lemma}\label{lemma:factorization}
Every order preserving map, $m : \woSet{\alpha} \mapsTo \woSet{\beta}$, factors into a
unique composite $m = m^f \compose m^d$.
\end{lemma}
\begin{proof}
Let $\gamma = \ordinal{\image{m}} = \ordinal{\woSet{\alpha}\withOut\kernel{m}}$ then
$\gamma \leq \alpha$ and $\gamma \leq \beta$.  Define the map $m^d : \woSet{\alpha}
\mapsTo \woSet{\gamma}$ by $m^d = \coDegeneracy{\alpha}{\gamma}{\kernel{m}}$.  Similarly
define the map $m^f : \woSet{\gamma} \mapsTo \woSet{\beta}$ by $m^f =
\coFace{\gamma}{\beta}{\woSet{\beta}\withOut\image{m}}$.  Then we have $m = m^f \compose
m^d$.
 
By properties \ref{property:unique.coface} and \ref{property:unique.codegeneracy} of
respectively the coface and codegeneracy maps, these are the only maps, $m^f$ and $m^d$
with the respectively required image, $\image{m}$, and kernel, $\kernel{m}$.  Hence this
decomposition is unique.
\end{proof}

\begin{corollary}
If $\alpha \leq \beta \leq \kappa$, every strictly order preserving map, $m :
\woSet{\alpha} \mapsTo \woSet{\beta}$ is uniquely $m =
\coFace{\alpha}{\beta}{\woSet{\beta}\withOut\image{m}}$.
\end{corollary}

In order to understand the combinatorial structure of
\homomorphisms{\DeltaC{}{}}{\woSet{\alpha}, \woSet{\beta}}, which is the set of all order
preserving maps between a pair of objects, \woSet{\alpha} and \woSet{\beta}, in
\DeltaC{}{}, we need to add a bit more notation. Let \powerSet[c]{\woSet{\alpha}} denote
the set of complements, with respect to \woSet{\alpha}, of subsets of \woSet{\alpha}. 
Since we are working in \setC{}, we know that \powerSet[c]{\woSet{\alpha}} is trivially
isomorphic to \powerSet{\woSet{\alpha}} since every subset in \setC{} is complemented. 
Note that this is only true in a Boolean Topos such as \setC{}.

Consider $\alpha, \beta \leq \kappa$.  We note that as sets of ordinals, \woSet{\alpha}
and \woSet{\beta} are subsets of \woSet{\kappa}.  Let
\kappaPullBackCoPowerSet{\alpha}{\beta} denote the pullback of $\ordinal{\cdot} :
\powerSet[c]{\woSet{\alpha}} \mapsTo \woSet{\kappa}$ and $\ordinal{\cdot} :
\powerSet[c]{\woSet{\beta}} \mapsTo \woSet{\kappa}$.  That is,
\kappaPullBackCoPowerSet{\alpha}{\beta} is the unique object of \setC{} which makes the
following diagram commute:
\begin{cTikzPicture}
\matrix (m) [comDiagM]
{ \kappaPullBackCoPowerSet{\alpha}{\beta} & \powerSet[c]{\woSet{\alpha}} \\
  \powerSet[c]{\woSet{\beta}}             & \woSet{\kappa} \\ };
\path[comDiagP]
(m-1-1) edge node[above] {$ \pi_{\alpha} $} (m-1-2)
(m-1-1) edge node[left]      {$ \pi_{\beta} $}   (m-2-1)
(m-1-2) edge node[right]   { \ordinal{\cdot} } (m-2-2)
(m-2-1) edge node[below] { \ordinal{\cdot} } (m-2-2);
\end{cTikzPicture}
Alternatively in the internal set theory of \setC{} we have
$$ \kappaPullBackCoPowerSet{\alpha}{\beta} = \set{ (V, U) \in \powerSet{\alpha} \times
\powerSet{\beta} \suchThat \ordinal{\woSet{\alpha} \withOut V} = \ordinal{\woSet{\beta}
\withOut U}} $$

\begin{lemma}\label{lemma:combStructDeltaHom}
$$\homomorphisms{\DeltaC{}{}}{\woSet{\alpha}, \woSet{\beta}} \isomorphic_{\setC}
\kappaPullBackCoPowerSet{\alpha}{\beta}$$
\end{lemma}\begin{proof}
The proof of lemma \ref{lemma:factorization} provides the forward mapping, in \setC{},
from \homomorphisms{\DeltaC{}{}}{\woSet{\alpha}, \woSet{\beta}} to
\kappaPullBackCoPowerSet{\alpha}{\beta}.

Conversely, for each pair of ordinals, $\alpha$ and $\beta$, and pairs of sets, $U \subset
\woSet{\beta}$ and $V \subset \woSet{\alpha}$ for which $\ordinal{\woSet{\beta}\withOut U}
= \ordinal{\woSet{\alpha}\withOut V} = \gamma$ we have a unique order preserving map $m :
\woSet{\alpha} \mapsTo \woSet{\beta}$ defined by $m = d^{\gamma,\beta}_U \compose
s^{\alpha,\gamma}_V$ for which $\image{m} = \woSet{\beta}\withOut U$ and $\kernel{m} = V$.
\end{proof}

\begin{corollary}
$$\homomorphisms{\ndDeltaC{}{}}{\woSet{\alpha}, \woSet{\beta}} \isomorphic_{\setC} 
\tuple[\pi^{-1}_{\alpha}]{\emptySet} \isomorphic_{\setC}
\ndKappaPullBackCoPowerSet{\alpha}{\beta}$$
\end{corollary}
\begin{proof}
Since any morphism, $m \in \homomorphisms{\ndDeltaC{}{}}{\woSet{\alpha}, \woSet{\beta}}$,
is strictly order preserving and hence injective, we know that $\kernel{m} = \emptySet$. 
Tracing through the proof of lemma \ref{lemma:combStructDeltaHom} as well as the commuting
diagram in the definition of \kappaPullBackCoPowerSet{\alpha}{\beta} with $\kernel{m} = V
= \emptySet \subset \woSet{\alpha}$ provides that required isomorphisms in \setC{}.
\end{proof}

If we are given a map which is the composition of a coface followed by a codegeneracy, that 
is the ``wrong way around'', what is its unique coface/codegeneracy factorization given by 
lemma \ref{lemma:factorization}?  Consider $\alpha, \beta, \gamma \leq \kappa, \; U \subset 
\woSet{\gamma}$ and $V \subset \woSet{\gamma}$ for which $\alpha \leq \gamma, \; 
\ordinal{\woSet{\alpha}} = \ordinal{\woSet{\gamma}\withOut U}$ and $\beta \leq \gamma, \; 
\ordinal{\woSet{\beta}} = \ordinal{\woSet{\gamma}\withOut V}$, then the morphism 
$$m^{\alpha,\gamma,\beta}_{U,U} = \coDegeneracy{\gamma}{\beta}{V} \compose 
\coFace{\alpha}{\gamma}{U} : \woSet{\alpha} \mapsTo \woSet{\beta}$$ is well defined. 

Let $\tilde{U} = \image{m^{\alpha,\gamma,\beta}_{V,U}}$, then:
\begin{align*}
\tilde{U} & = \image{m^{\alpha,\gamma,\beta}_{V,U}} \\
  & = \image{ \coDegeneracy{\gamma}{\beta}{V} \compose \coFace{\alpha}{\gamma}{U}} \\
  & = \tuple[\coDegeneracy{\gamma}{\beta}{V}]{\image{\coFace{\alpha}{\gamma}{U}}} \\
  & = \tuple[\coDegeneracy{\gamma}{\beta}{V}]{\woSet{\gamma} \withOut U}
\end{align*}

Similarly, let $\tilde{V} = \kernel{m^{\alpha,\gamma,\beta}_{V,U}}$, then
\begin{align*}
\tilde{V} & = \kernel{m^{\alpha,\gamma,\beta}_{V,U}} \\
  & = \kernel{ \coDegeneracy{\gamma}{\beta}{V} \compose \coFace{\alpha}{\gamma}{U}} \\
  & = 
  \tuple[\tuple{\coFace{\alpha}{\gamma}{U}}^{-1}]{\kernel{\coDegeneracy{\gamma}{\beta}{V}}} 
  \\
  & = \tuple[\tuple{\coFace{\alpha}{\gamma}{U}}^{-1}]{V}
\end{align*}

Finally let, $\tilde{\gamma} = \ordinal{\image{m^{\alpha,\gamma,\beta}_{V,U}}} = 
\ordinal{\woSet{\alpha}\withOut \kernel{m^{\alpha,\gamma,\beta}_{V,U}}}$, then
\begin{equation*}
\coDegeneracy{\gamma}{\beta}{V} \compose \coFace{\alpha}{\gamma}{U} = 
\coFace{\tilde{\gamma}}{\beta}{\tilde{U}} \compose 
\coDegeneracy{\alpha}{\tilde{\gamma}}{\tilde{V}}
\end{equation*}

Consider a morphism, $m : \woSet{\alpha} \mapsTo \woSet{\beta}$, of \DeltaC{}{}.
\begin{enumerate}
\item If $\beta < \alpha \leq \kappa$ then \kernel{m} is non empty.  This means that there 
are no injective order preserving maps from \woSet{\alpha} to \woSet{\beta}, and hence 
\homomorphisms{\ndDeltaC{}{}}{\woSet{\alpha}, \woSet{\beta}} is empty.
%
\item If $\alpha = \beta$ then $m = \identity{\woSet{\alpha}}$ is the only \emph{strictly} 
order preserving morphism from \woSet{\alpha} to itself, and hence 
$\homomorphisms{\ndDeltaC{}{}}{\woSet{\alpha},\woSet{\alpha}} = 
\set{\identity{\woSet{\alpha}}}$
%
\item If $\alpha < \beta \leq \kappa$ then \cImage{m} is non empty and hence there are no 
surjective order preserving maps from \woSet{\alpha} to \woSet{\beta}.
\end{enumerate}

\subsection{Idempotents of \ndDeltaC{}{}, \DeltaC{}{}, \sDeltaC{}{}}

Following \cite[Chapter 5]{reyesReyesZolfaghari2004presheaves} we are interested in the 
generic figures of each of \ndDeltaC{}{}, \DeltaC{}{} and \sDeltaC{}{}.  These generic 
figures are essentially defined by the idempotents of each category. \TODO{The only 
idempotents of \ndDeltaC{}{} are the identity morphisms of each \woSet{\alpha}.  The 
idempotents of \DeltaC{}{} are the different ways that simplexes of higher dimension can be 
(singularly) embedded in simplexes of lower dimension, while those of \sDeltaC{}{} are the 
symmetrization of the idempotents of \DeltaC{}{}. (See 
\cite[page90]{reyesReyesZolfaghari2004presheaves}).  For both \DeltaC{}{} and \sDeltaC{}{}, 
cannonical inclusion of the original category in its respective Cauchy completion is full 
and faithful and essentially surjective by arguments similar to that given in the discussion 
to \cite[Propostion 5.2.1]{reyesReyesZolfaghari2004presheaves} and just before \cite[Remark 
5.2.5]{reyesReyesZolfaghari2004presheaves}.  This means that both categories and their 
respective Cauchy completions are equivalent as categories.}

\TODO{Introduce and define the Cauchy completion of a small category using ``splitting'' of 
idempotents as discussed in \cite[Section 5.2]{reyesReyesZolfaghari2004presheaves}.}

\section{The graph of automorphisms of \texorpdfstring{$\Delta$}{Delta}}

Our aim in this section is to follow the work of John Shrimpton (\cite[Chapter 
5]{shrimpton1989graphsSymmetryCatMethods}, \cite[Sections 3, 
4]{brownMorrisShrimptonWensley2008graphCat}, see also \cite[Section 2]{brown1994symmetry}) 
to define the group-graph of automorphisms a graph, in our case the group-graph of 
automorphisms of \DeltaC{}{}.

\section{The graph of ordinals}

The definition of the collection of test-$1$-paths is fairly straight forward. Fix the
first (non-zero) limit ordinal $\omega$. For any $0 \leq n \leq \omega$ let \woSet{n}
denote the totally ordered graph defined as
\begin{description}
\item[objects] there is an object for each $0 \leq i < 1+n$,
\item[morphisms] for each object $0 \leq i < 1+n$ there is the morphism $\hat{m}_i : i 
\mapsTo i$, if $0 \leq 1+i < 1+n$ then there is  also the morphism $m_i : i \mapsTo i+1$.
\end{description}
together with the obvious \domain{\cdot}, \coDomain{\cdot} and \Identity{\cdot} mappings.  
It is important to note that ordinal addition is not commutative, see Appendix 
\ref{chap:ordinals}.  In particular $1 + \omega = \omega$.  This means that our choice of 
order in the condition, $0 \leq i < 1+n$ is significant. For finite $n$, the graph \woSet{n} 
can be depicted as:
\begin{cTikzPicture}
\coordinate (v0) at (0,0);
\coordinate (v1) at (1.5,0);
\coordinate (v2) at (3,0);
\coordinate (v-n-1) at (4.5,0);
\coordinate (v-n) at (6,0);
%
\node[above] at (v0)    {$0$};
\node[above] at (v1)    {$1$};
\node[above] at (v2)    {$2$};
\node[above] at (v-n-1) {$n-1$};
\node[above] at (v-n)   {$n$};
%
\fill (v0) circle[radius=1pt];
\fill (v1) circle[radius=1pt];
\fill (v2) circle[radius=1pt];
\fill (v-n-1) circle[radius=1pt];
\fill (v-n) circle[radius=1pt];
%
\begin{scope}[->,shorten <=4pt,shorten >=4pt]
\path (v0) edge (v1);
\path (v1) edge (v2);
\path (v2) edge[dotted] (v-n-1);
\path (v-n-1) edge (v-n);
\end{scope}
\end{cTikzPicture}

Consider a pair of test paths, \woSet{\alpha} and \woSet{\beta} for $\alpha, \beta \leq 
\omega$.  With out loss of generality we can assume that $\alpha \leq \beta$. \TODO{There is 
are $\beta - \alpha + 1$ inclusion graph maps from \woSet{\alpha} into \woSet{\beta}.  Since 
we have included the degenerate identity morphism, there are also all of the surjections 
contained in \DeltaC{}{}.  This asymmetry shows that our current definition of Graph is 
incomplete. The more convenient definition of graph is as a category, or alternatively we 
are too permissive in our definition of simplicial set and really only want delta sets.  }

\TODO{Show that any two test paths can be combined into a composite test path.  Show that 
this composition is associative. Show that all test paths can be generated by composition of 
the test path \woSet{1}.}

The graph, \mathCategory{T1P}, of test-$1$-paths is defined as
\begin{description}
\item[objects] for any $0 \leq n \leq \omega$, the graphs \woSet{n} defined above,
\item[morphisms] any graph map between a pair of objects, \woSet{n} and \woSet{m},
\end{description}
with the obvious \domain{\cdot}, \coDomain{\cdot} and \Identity{\cdot} mappings.  

While this is an explicit construction, when we get far enough in our exposition to be able 
to define the following concepts, we will see that the important properties of 
\mathCategory{T1P} are
\begin{enumerate}
\item for each object, \woSet{n}, considered as a graph, is connected,
\item for each object, \woSet{n}, considered as a graph, the fundmental groupoid, 
\fundGroupoid{\woSet{n}}, is trivial,
\item for each object, \woSet{n}, considered as a graph, is unbranched,
\item for each object, \woSet{n}, can always be extended (i.e. can be embedded in a 
``larger'' object \woSet{m}), 
\item \mathCategory{T1P} is a \emph{poset},  (in fact it is a Domain),
\item (the located \mathCategory{T1P} is shift invariant \TODO{define this shift map}),
\item if there is a morphism $f : \woSet{n} \mapsTo \woSet{n}$ then there exists a unique 
morphism, $f' : \woSet{n} \mapsTo \woSet{m}$ (\woSet{n} factors through ?? (conversely 
factors through ??) \TODO{this is really not complete but I am not yet sure how best to say 
it},  \TODO{This effectively show that all objects are ``flat'' --- that is there is only 
one way to complete/extend them},
\item if there are morphisms $f : \woSet{n} \mapsTo \woSet{m}$ and $ f' : \woSet{m} \mapsTo 
\woSet{m'}$ then there is a unique composition \TODO{again I do not (yet) know how to say 
this correctly.} \TODO{similarly this shows, in some sense, the ``flatness'' of the objects}
\end{enumerate}
\TODO{explicitly name this graph, as a Symbolic Dynamics.}

\TODO{QUESTION: how do we define the Symbolic Dynamics notion of overlapping words and or 
magical words.}

\TODO{We want to stress the collections of paths rather than simply composition.  The 
example of Symbolic Dynamics non-subshifts of finite type such as minimal shifts, 
\cite{lindMarcus1995a}, is key here.  There is unlikely to be \emph{any} ``composition'' 
operator for minimal shifts.... however they \emph{are} interesting collections of paths.  
Composition operators will be definable for the equivalent of ``subshifts of finite type'' 
which are essentially algebraic algebras defined by generators (words) and relations 
(constraints on which words can be composed).}

\section{Introduction: The category of simplicial sets}

\TODO{We need a better intro here!}

\TODO{Discuss the fact that working in \setC{} means we can use the local set theory with 
\emph{just} points (i.e. \setC{} has enough points). Should also discuss subsets in Topos 
and \setC{}.}

\section{The category of simplicial sets}

\begin{definition}
A \define{(transfinite) simplicial set}, $X$, is an object in the functor category, 
\opFuncCat{\DeltaC{}{}}{\setC{}}, or alternatively a covariant functor $X : 
\opposite{\DeltaC{}{}} \mapsTo{} \setC{}$ (or a contravariant functor $X : \DeltaC{}{} 
\mapsTo \setC{}$).  Explicitly it is defined by:
\begin{enumerate}
\item For each object, \woSet{\alpha} in \DeltaC{}{}, there is an associated set 
$\tuple[X]{\woSet{\alpha}} = X_{\woSet{\alpha}} = X_{\alpha}$ in \setC{}.
\item For each morphism, $m : \woSet{\alpha} \mapsTo \woSet{\beta}$, in \DeltaC{}{}, there 
is an associated map $\tuple[X]{m} = X_m : X_{\beta} \mapsTo X_{\alpha}$ in \setC{} for 
which the following diagram commutes:
%
\begin{cTikzPicture}
\matrix (m) [comDiagM]
{ \woSet{\alpha} & X_{\alpha} \\
  \woSet{\beta}  & X_{\beta} \\ };
\path[comDiagP]
(m-1-1) edge node[above] { $X$ }   (m-1-2)
(m-1-1) edge node[left]  { $m$ }   (m-2-1)
(m-2-2) edge node[right] { $X_m$ } (m-1-2)
(m-2-1) edge node[below] { $X$ }   (m-2-2);
\end{cTikzPicture}
%
\item If $m : \woSet{\alpha} \mapsTo \woSet{\beta}$ and $m' : \woSet{\beta} \mapsTo 
\woSet{\gamma}$ are composable morphisms in \DeltaC{}{} then $X_m : X_{\beta} \mapsTo 
X_{\alpha}$ and $X_{m'} : X_{\gamma} \mapsTo X_{\beta}$ are composable maps and 
$X_{m\compose m'} = X_{m'} \compose X_m$ in \setC{}. This means that the following diagram 
commutes:
%
\begin{cTikzPicture}
\matrix (m) [comDiagM]
{ \woSet{\alpha} & X_{\alpha} \\
  \woSet{\beta}  & X_{\beta}  \\ 
  \woSet{\gamma} & X_{\gamma} \\ 
};
\path[comDiagP]
(m-1-1) edge node[above] { $X$ }      (m-1-2)
(m-2-1) edge node[above] { $X$ }      (m-2-2)
(m-3-1) edge node[below] { $X$ }      (m-3-2)
%
(m-1-1) edge                node[right]  { $m$ }             (m-2-1)
(m-2-1) edge                node[right]  { $m'$ }            (m-3-1)
(m-1-1) edge[bend right=60] node[left]   { $m' \compose m$ } (m-3-1)
%
(m-2-2) edge                node[left]  { $X_m$ }                 (m-1-2)
(m-3-2) edge                node[left]  { $X_{m'}$ }              (m-2-2)
(m-3-2) edge[bend right=60] node[right] { $X_m \compose X_{m'}$ } (m-1-2);
\end{cTikzPicture}
%
\item For each coface map, \coFace{\alpha}{\beta}{U} in \DeltaC{}{}, the image via the 
functor $X$, $\tuple[X]{\coFace{\alpha}{\beta}{U}} = X_{\coFace{\alpha}{\beta}{U}} = 
\face{\alpha}{\beta}{U}$, is a \define{face} map in $X$.
\item For each codegeneracy map, \coDegeneracy{\alpha}{\beta}{V} in \DeltaC{}{}, the image 
via the functor $X$, $\tuple[X]{\coDegeneracy{\alpha}{\beta}{V}} = 
X_{\coDegeneracy{\alpha}{\beta}{V}} = \degeneracy{\alpha}{\beta}{V}$, is a 
\define{degeneracy} map in $X$.
\item For each morphism, $m : \woSet{\alpha} \mapsTo \woSet{\beta}$, with its unique 
coface/codegeneracy factorization, $m = 
\coFace{\gamma}{\beta}{\woSet{\beta}\withOut\image{m}} \compose 
\coDegeneracy{\alpha}{\gamma}{\kernel{m}}$ we have a correspondingly unique degeneracy/face 
factorization, 
\begin{align*}
\tuple[X]{m} & = \tuple[X]{\coFace{\gamma}{\beta}{\woSet{\beta}\withOut\image{m}} \compose 
                 \coDegeneracy{\alpha}{\gamma}{\kernel{m}}} \\
             & = \tuple[X]{\coDegeneracy{\alpha}{\gamma}{\kernel{m}}} \compose 
                 \tuple[X]{\coFace{\gamma}{\beta}{\woSet{\beta}\withOut\image{m}}}  \\
             & = \degeneracy{\alpha}{\gamma}{\kernel{m}} \compose 
                 \face{\gamma}{\beta}{\woSet{\beta}\withOut \image{m}}
\end{align*}
\end{enumerate}
For each $\alpha \leq \kappa$, the elements of $X_{\alpha}$ are 
\define{$\alpha$-simplicies}. \TODO{Define dimension of a simplex to be the $\alpha$ for 
which it is contained in $X_{\alpha}$.}
\end{definition}

\begin{definition} \TODO{Comment on naturality and refer to \cite[Chapter 
7]{awodey2006catTh}}
A \define{(transfinite) simplicial map}, $f : X \mapsTo Y$ is defined by:
\begin{enumerate}
\item For each object, \woSet{\alpha}, in \DeltaC{}{}, there is an associated map 
$f_{\woSet{\alpha}} = f_{\alpha} : X_{\alpha} \mapsTo Y_{\alpha}$.
\item If $m : \woSet{\alpha} \mapsTo \woSet{\beta}$ is a morphism (order preserving map) in 
\DeltaC{}{} then $f_{\alpha} \compose X_m = Y_m \compose f_{\beta}$, that is the diagram on 
the right commutes:
%
\begin{cTikzPicture}
\matrix (m) [comDiagM]
{ \woSet{\alpha} & & X_{\alpha} & Y_{\alpha} \\
  \woSet{\beta}  & & X_{\beta}  & Y_{\beta}  \\ };
\begin{scope}[comDiagP]
\path
(m-1-1) node (delta)   {} (m-2-1)
(m-1-3) node (simplex) {} (m-2-3);
\path
(m-1-1) edge node (delta)   {}               (m-2-1)
%
(m-1-3) edge node[above]    { $f_{\alpha}$ } (m-1-4)
(m-2-3) edge node (simplex) {}               (m-1-3)
(m-2-4) edge node[right]    { $Y_m$ }        (m-1-4)
(m-2-3) edge node[below]    { $f_{\beta}$ }  (m-2-4)
%
(delta) edge[densely dotted, shorten <=5pt, shorten >=20pt] (simplex);
%
\node[left] at (delta)   { $m$ };
\node[left] at (simplex) { $X_m$ };
\end{scope}
\end{cTikzPicture}
%
\end{enumerate}
\end{definition}

\begin{lemma}
The composition of two simplicial maps is a simplicial map.
\end{lemma}
\begin{proof}
\TODO{complete this proof! DO WE REALLY NEED THIS?}
\end{proof}

\TODO{We need to discuss the corresponding functor category of non-degenerate simplicial 
sets (Freidman calls Delta sets, \cite[Section 2.3]{friedman2008illustratedIntroduction}).  
Morphisms 
of \funcCat{\ndDeltaC{}{}}{\setC{}} are not as ``flexible'' as morphisms of 
\funcCat{\DeltaC{}{}}{\setC{}}. The map identify ``whole'' simplicies but they may not 
identify subsimplicies (faces) unless the pair of containing simplicies are also identified. 
Certainly there is an adjoint pair of functors between the categories of simplicial sets and 
non-degenerate simplicial sets, and that, on objects, this adjoint pair is an isomorphism, 
and that every morphism of non-degenerate simplicial sets extends (freely) to a morphism of 
simplicial sets.  However on morphisms this mapping is injective but not surjective.}

\begin{definition}
The category of \define{simplicial sets}, \simpC{}, is the functor category, 
\funcCat{\opposite{\DeltaC{}{}}}{\setC{}}.  Explicitly, the objects of \simpC{} are 
(covariant) functors $X : \opposite{\DeltaC{}{}} \mapsTo \setC{}$ and the morphisms of 
\simpC{} are natural transformations between two functors, $X, Y : \opposite{\DeltaC{}{}} 
\mapsTo \setC{}$.
\end{definition}

\begin{definition}
Consider a simplicial map, $f : X \mapsTo Y$.  The \define{image of a simplicial map}, 
\image{f}, is defined by:
\begin{enumerate}
\item For each object, \woSet{\alpha}, in \DeltaC{}{}, there is a set 
$f_{\alpha}(X_{\alpha}) \subset Y_{\alpha}$.
\item For each morphism, $m : \woSet{\alpha} \mapsTo \woSet{\beta}$, in \DeltaC{}{}, there 
is a restriction map, $\restrictedTo{Y_m}{f_{\beta}(X_{\beta})} : f_{\beta}(X_{\beta}) 
\mapsTo f_{\alpha}\compose X_m (X_{\beta}) \subset f_{\alpha}(X_{\alpha})$ 
\end{enumerate}
\end{definition}

\begin{lemma}
The image of a simplicial map is a simplicial set.
\end{lemma}
\begin{proof} Consider a simplicial map, $f : X \mapsTo Y$.
\begin{enumerate}
\item For each $0$-simplex, \woSet{\alpha}, in \DeltaC{}{}, let $\image{f}_{\alpha} = 
f_{\alpha}(X_{\alpha})$.
\item For each $1$-simplex, $m : \woSet{\alpha} \mapsTo \woSet{\beta}$, in \DeltaC{}{}, let 
$\image{f}_m = \restrictedTo{Y_m}{f_{\beta}(X_{\beta})}$.
\end{enumerate}
All we need to show is that composable $m$ and $m'$ are still composable... \TODO{show this!}
\end{proof}

\section{The Yoneda embedding: the standard simplicies}

\TODO{We need to define the subset of non-degenerate morphisms between simplicial sets. We 
need to note that for \DeltaC{\alpha}{} for, $\alpha < \beta \leq \kappa$, \emph{all} 
morphisms in \homomorphisms{\DeltaC{}{}}{\woSet{\beta}, \woSet{\alpha}} are 
\emph{degenerate}. For $\alpha \leq \kappa$, a simplicial set is an 
\define{$\alpha$-simplicial set}, if $X_{\beta} = \emptySet$ for all $\alpha < \beta$.  This 
last statement is incorrect.  We need to define non-degenerate simplicies and then define an 
$\alpha$-simplicial set to be one for which all (sub)$\beta$-simplicies for $\alpha < \beta 
\leq \kappa$ are degenerate. Define dimension of a non-degenerate simplex to be the $\alpha$ 
for which...}

\TODO{Need to work in Goerss and Jardine's Example 1.7 at this point.  And provide explicit 
definitions of degenerate and non-degenerate and elemental simplicies.}  \TODO{The concept 
of collapsing scheme looks useful to our work, see \cite{citterio2001a}} \TODO{Allegretti's 
work, \cite{allegretti2008simplicialSets} provides an interesting link between Van Kempen's 
theorem, groupoids and classification in the context of our work.}

Since it is key to most of our subsequent work, we now work through Goerss and Jardine's 
Example 1.7, \cite[page 6]{goerssJardin1999SimplicialHomotopyTh}, in some detail.  This is 
essentially a standard application of the Categorical concepts of representable Functors, 
the Yoneda Lemma, and the Yoneda Embedding.  A good exposition of these ideas can be found 
in \cite[Section 4.5]{barrWells1995catTh} or \cite[Chapter 8]{awodey2006catTh}.  
\TODO{Provide other expositions.}

Consider $\alpha \leq \kappa$.  Since \woSet{\alpha} \emph{is} a set in \setC{}, we know 
that \homomorphisms{\DeltaC{}{}}{\cdot, \woSet{\alpha}} is a (covariant) functor from 
\opposite{\DeltaC{}{}} to \setC{}, and hence is an object of 
\opFuncCat{\DeltaC{}{}}{\setC{}}, which in turn means that it is a simplicial set. We define 
$\DeltaC{\alpha}{} = \homomorphisms{\DeltaC{}{}}{\cdot, \woSet{\alpha}}$ to be the 
\define{$\alpha$-standard simplex}. 

In Categorical terms $ \DeltaC{\alpha}{} = \tuple[y]{\woSet{\alpha}}= 
\homomorphisms{\DeltaC{}{}}{\cdot, \woSet{\alpha}} $ is the image of the Yondea 
\emph{Embedding}, $ y : \DeltaC{}{} \mapsTo \opFuncCat{\DeltaC{}{}}{\setC{}} $.  It is a 
representable functor which is represented by the object \woSet{\alpha} in \setC{}.  
Explicitly:
\begin{enumerate}
\item For an object, \woSet{\beta} in \DeltaC{}{},  
$\tuple[\homomorphisms{\DeltaC{}{}}{\cdot, \woSet{\alpha}}]{\woSet{\beta}} = 
\homomorphisms{\DeltaC{}{}}{\woSet{\beta}, \woSet{\alpha}}$ which since \setC{} is a locally 
small category, is a set in \setC{}. 

In particular, we know from our work in \DeltaC{}{}, this is the \emph{set} of order 
preserving morphisms from the \emph{set} \woSet{\beta} to the \emph{set} \woSet{\alpha} 
which is isomorphic, in \setC{}, to the pullback, \kappaPullBackCoPowerSet{\beta}{\alpha}. 

\item For a morphism, $m : \woSet{\beta} \mapsTo \woSet{\delta}$ in \DeltaC{}{}, 
$\tuple[\homomorphisms{\DeltaC{}{}}{\cdot, \woSet{\alpha}}]{m} = 
\homomorphisms{\DeltaC{}{}}{m, \woSet{\alpha}} : \homomorphisms{\DeltaC{}{}}{\woSet{\delta}, 
\woSet{\alpha}} \mapsTo \homomorphisms{\DeltaC{}{}}{\woSet{\beta}, \woSet{\alpha}}$. 

In particular for a morphism $m' : \woSet{\delta} \mapsTo \woSet{\alpha}$, we have that 
$\tuple[\homomorphisms{\DeltaC{}{}}{m, \woSet{\alpha}}]{m'} = m' \compose m$ which is a 
morphism in \homomorphisms{\Delta}{\woSet{\beta}, \woSet{\alpha}} as required.

Combinatorially, we can also consider \homomorphisms{\DeltaC{}{}}{m, \woSet{\alpha}} as a 
map between the pullbacks, \kappaPullBackCoPowerSet{\delta}{\alpha}, and 
\kappaPullBackCoPowerSet{\beta}{\alpha}.
If the above $m$ and $m'$ have their unique coface/codegeneracy factorizations, $$m = 
\coFace{\gamma}{\beta}{\woSet{\beta}\withOut\image{m}} \compose 
\coDegeneracy{\delta}{\gamma}{\kernel{m}}$$ and $$m' = 
\coFace{\gamma'}{\delta}{\woSet{\delta}\withOut\image{m'}} \compose 
\coDegeneracy{\alpha}{\gamma'}{\kernel{m'}}$$ respectively, then, working in \DeltaC{}{}, we 
have 
\begin{align*}
m'' & = m' \compose m \\
      & = \coFace{\gamma'}{\delta}{\woSet{\delta}\withOut\image{m'}} \compose 
      \coDegeneracy{\alpha}{\gamma'}{\kernel{m'}}
            \compose 
            \coFace{\gamma}{\beta}{\woSet{\beta}\withOut\image{m}} \compose 
            \coDegeneracy{\delta}{\gamma}{\kernel{m}}
\end{align*}
\TODO{SORT THIS OUT!  -- It should be ``easy'', but my brain is dead!  To do this correctly 
and to raise the combinatorial profile of this work we need a notation for these pullbacks!}
\end{enumerate}

Similarly we can define $ \ndDeltaC{\alpha}{} = \homomorphisms{\ndDeltaC{}{}}{\cdot, 
\woSet{\alpha}}$ to be the \define{$\alpha$-standard non-degenerate simplex}. Again, in 
Categorical terms $ \ndDeltaC{\alpha}{} = \tuple[y_{nd}]{\woSet{\alpha}}= 
\homomorphisms{\ndDeltaC{}{}}{\cdot, \woSet{\alpha}} $ is the image of the Yondea 
\emph{Embedding}, $ y_{nd} : \ndDeltaC{}{} \mapsTo \opFuncCat{\ndDeltaC{}{}}{\setC{}} $.  It 
is a representable functor which is represented by the object \woSet{\alpha} in \setC{}.  
Explicitly:
\begin{enumerate}
\item For an object, \woSet{\beta} in \ndDeltaC{}{},  
$\tuple[\homomorphisms{\ndDeltaC{}{}}{\cdot, \woSet{\alpha}}]{\woSet{\beta}} = 
\homomorphisms{\ndDeltaC{}{}}{\woSet{\beta}, \woSet{\alpha}}$ which since \setC{} is a 
locally small category, is a set in \setC{}. 

In particular, we know from our work in \ndDeltaC{}{}, this is the \emph{set} of 
\emph{strictly} order preserving morphisms from the \emph{set} \woSet{\beta} to the 
\emph{set} \woSet{\alpha} which is isomorphic, in \setC{}, to the pullback, 
\ndKappaPullBackCoPowerSet{\beta}{\alpha}. 
\item For a morphism, $m : \woSet{\beta} \mapsTo \woSet{\delta}$ in \ndDeltaC{}{}, 
$\tuple[\homomorphisms{\ndDeltaC{}{}}{\cdot, \woSet{\alpha}}]{m} = 
\homomorphisms{\ndDeltaC{}{}}{m, \woSet{\alpha}} : 
\homomorphisms{\ndDeltaC{}{}}{\woSet{\delta}, \woSet{\alpha}} \mapsTo 
\homomorphisms{\ndDeltaC{}{}}{\woSet{\beta}, \woSet{\alpha}}$. 

In particular for a morphism $m' : \woSet{\delta} \mapsTo \woSet{\alpha}$,  we have that 
$\tuple[\homomorphisms{\ndDeltaC{}{}}{m, \woSet{\alpha}}]{m'} = m' \compose m$ which is a 
morphism in \homomorphisms{\ndDeltaC{}{}}{\woSet{\beta}, \woSet{\alpha}} as required.

Combinatorially, we can also consider \homomorphisms{\ndDeltaC{}{}}{m, \woSet{\alpha}} as a 
map between the pullbacks, \ndKappaPullBackCoPowerSet{\delta}{\alpha}, and 
\ndKappaPullBackCoPowerSet{\beta}{\alpha}. If the above $m$ and $m'$ have their unique 
coface/codegeneracy factorizations, $$m = 
\coFace{\gamma}{\beta}{\woSet{\beta}\withOut\image{m}}$$ and $$m' = 
\coFace{\gamma'}{\delta}{\woSet{\delta}\withOut\image{m'}}$$ respectively, then, working in 
\DeltaC{}{}, we have 
\begin{align*}
m'' & = m' \compose m \\
     & = \coFace{\gamma'}{\delta}{\woSet{\delta}\withOut\image{m'}}
            \compose 
            \coFace{\gamma}{\beta}{\woSet{\beta}\withOut\image{m}} \\
       & = \coFace{?}{?}{?}
\end{align*}
\TODO{SORT THIS OUT!  -- It should be ``easy'', but my brain is dead!  To do this correctly 
and to raise the combinatorial profile of this work we need a notation for these pullbacks!}
\end{enumerate}

\begin{definition}
A \define{simplex category} (respectively a \define{non-degenerate simplex category}) is 
given by
\begin{enumerate}
\item Objects: 
\end{enumerate}
\end{definition}

\TODO{We need to expand upon Goerss and Jardine's classifying maps example.  Then note 
define a simplical set to be an $\alpha$-simplicial set if, for $\alpha < \beta \leq 
\kappa$, the classifying maps, \ndHomomorphisms{\simpC{}}{\DeltaC{\beta}{}, X}, is empty.}

For a given simplicial set, $F$ in \simpC{}, the Yoneda \emph{Lemma} implies that, for each 
$\alpha \leq \kappa$, the simplicial maps whose domains are the $\alpha$-standard simplex, 
$\DeltaC{\alpha}{} \mapsTo F$, classifies the $\alpha$-simplicies of $F$ in that there is a 
natural isomorphism between $$ \homomorphisms{\simpC{}}{\DeltaC{\alpha}{}, F} \isomorphic 
F_{\alpha} $$ Given its importance to our subsequent work, we explicitly expand out what is 
essentially the proof of the Yondea Lemma in our specific case as follows. See \cite[Lemma 
8.2]{awodey2006catTh} for a good detailed general proof of the Yondea Lemma.

We are interested in defining a mapping $\eta_{\woSet{\alpha},F} : 
\homomorphisms{\DeltaC{}{}}{\woSet{\alpha},F} \mapsTo \tuple[F]{\woSet{\alpha}}$. Fix both a 
simplicial set $F$ and $\alpha \leq \kappa$. We begin by considering an arbitrary simplicial 
map, $\theta : \DeltaC{\alpha}{} \mapsTo F$ in \homomorphisms{\simpC{}}{\DeltaC{\alpha}{}, 
F}. That is, $\theta$ is a natural transformation from the functor $\DeltaC{\alpha}{} = 
\homomorphisms{\DeltaC{}{}}{\cdot, \woSet{\alpha}} : \opposite{\DeltaC{}{}} \mapsTo 
\woSet{\alpha}$ \emph{to} the functor $F : \opposite{\DeltaC{}{}} \mapsTo \setC{}$. Since we 
are interested in showing a mapping into the set $\tuple[F]{\woSet{\alpha}} = F_{\alpha}$ we 
should consider $\tuple[\DeltaC{\alpha}{}]{\woSet{\alpha}} = 
\homomorphisms{\DeltaC{}{}}{\woSet{\alpha},\woSet{\alpha}}$ which is the set of order 
preserving maps from \woSet{\alpha} to itself.  However the identity map, 
\identity{\woSet{\alpha}}, \emph{is} the only one such order preserving mapping. 

\TODO{Complete this discussion using \cite[Lemma 8.2]{awodey2006catTh}.}
\TODO{We need to work out in detail the work of \cite[Lemma 
2.1]{goerssJardin1999SimplicialHomotopyTh} which shows essentially that a simplicial set, 
$X$, is the ``sum'' of its $\DeltaC{\alpha}{} \mapsTo X$ ``parts'' (see also 
\cite[Proposition 8.10]{awodey2006catTh}).} \TODO{nLab:subfunctor 

A subfunctor of a functor G:C?D is a pair (F,i) where F:C?D is a functor and i:F?G is a 
natural transformation such that its components i M:F(M)?G(M) are monic. In fact one often 
by a subfunctor means just an equivalence class of such monic natural transformations; 
compare subobject.

In a concrete category with images one can choose a representative of a subfunctor where the 
components of i are genuine inclusions of the underlying sets; then a subfunctor is just a 
natural transformation whose components are inclusions. The naturality in terms of concrete 
inclusions just says that for all f:c?d, F(f)=G(f)? F(c). If the set-theoretic circumstances 
allow consideration of a category of functors, then a subfunctor is a subobject in such a 
category.

A subfunctor (F,i) of the identity id C:C?C in a category with images is an often used case: 
it amounts to a natural assignment c?F(c)?ic of a subobject to each object c in C. For 
concrete categories with images then F(f)=f? F(c).

A subfunctor of a representable functor Hom(?,x) is precisely a sieve over the representing 
object x.

  Revised on August 29, 2009 20:32:10 
  by Toby Bartels}
  
\section{A bestiary of simplicial sets}  

With this accumulated knowledge, we can provide the following ``picture'' of \DeltaC{2}{}

\begin{cTikzPicture}
\coordinate (O) at (0,0);

\coordinate (A0) at (0,2);
\coordinate (A1) at (0,4);
\coordinate (A2) at ($ (A0) + (O)!2cm!60:(A0) $);
\coordinate (A3) at ($ (A0) + (O)!2cm!-60:(A0) $);

\coordinate (B0) at ($ (O)!2cm!120:(A0) $);
\coordinate (B1) at ($ (O)!4cm!120:(A0) $);
\coordinate (B2) at ($ (B0) + (O)!2cm!60:(A0) $);
\coordinate (B3) at ($ (B0) - (O)!2cm!(A0) $);

\coordinate (C0) at ($ (O)!2cm!-120:(A0) $);
\coordinate (C1) at ($ (O)!4cm!-120:(A0) $);
\coordinate (C2) at ($ (C0) + (O)!2cm!-60:(A0) $);
\coordinate (C3) at ($ (C0) - (O)!2cm!(A0) $);

\draw (A0) edge node (AB0) {} (B0);
\draw (B0) edge node (BC0) {} (C0);
\draw (C0) edge node (CA0) {} (A0);
\draw (A2) edge node (AB1) {} (B2);
\draw (B3) edge node (BC1) {} (C3);
\draw (C2) edge node (CA1) {} (A3);

\fill[color=black!5] (A0) -- (B0) -- (C0) -- cycle;

\node[above]       at (A1) {\set{a}};
\node[below left]  at (B1) {\set{b}};
\node[below right] at (C1) {\set{c}};

\node[above left]  at (AB1) {\set{a,b}};
\node[above right] at (CA1) {\set{a,c}};
\node[below]       at (BC1) {\set{b,c}};

\node at (O) {\set{a,b,c}};

\foreach \x in { (A0), (A1), (A2), (A3), (B0), (B1), (B2), (B3), (C0), (C1), (C2), (C3) } {
  \fill \x circle[radius=1pt];
}

\begin{scope}[->, shorten >=4pt, shorten <=4pt] 
\path (A0) edge node[auto]      {\face{0}{2}{\set{b,c}}} (A1);
\path (B0) edge node[auto]      {\face{0}{2}{\set{a,c}}} (B1);
\path (C0) edge node[auto,swap] {\face{0}{2}{\set{a,b}}} (C1);
\path (A2) edge node[auto]      {\face{0}{1}{\set{b}}} (A1);
\path (A3) edge node[auto,swap] {\face{0}{1}{\set{c}}} (A1);
\path (B2) edge node[auto,swap] {\face{0}{1}{\set{a}}} (B1);
\path (B3) edge node[auto]      {\face{0}{1}{\set{c}}} (B1);
\path (C2) edge node[auto]      {\face{0}{1}{\set{a}}} (C1);
\path (C3) edge node[auto,swap] {\face{0}{1}{\set{b}}} (C1);
\end{scope}
\begin{scope}[<-, shorten >=2pt, shorten <=3pt] 
\path (AB1) edge node[auto] {\face{1}{2}{\set{c}}} (AB0);
\path (BC1) edge node[auto] {\face{1}{2}{\set{a}}} (BC0);
\path (CA1) edge node[auto] {\face{1}{2}{\set{b}}} (CA0);
\end{scope}

\end{cTikzPicture}

As we will see below the boundary of \DeltaC{2}{}, denoted $\boundary{\DeltaC{2}{}}$, is

\begin{cTikzPicture}
\coordinate (O) at (0,0);

\coordinate (A0) at (0,2);
\coordinate (A1) at (0,4);

\coordinate (B0) at ($ (O)!2cm!120:(A0) $);
\coordinate (B1) at ($ (O)!4cm!120:(A0) $);

\coordinate (C0) at ($ (O)!2cm!-120:(A0) $);
\coordinate (C1) at ($ (O)!4cm!-120:(A0) $);

\draw (A0) edge node[auto,swap] {\set{a,b}} (B0);
\draw (B0) edge node[auto,swap] {\set{b,c}} (C0);
\draw (C0) edge node[auto,swap] {\set{a,c}} (A0);

\node[above]       at (A1) {\set{a}};
\node[below left]  at (B1) {\set{b}};
\node[below right] at (C1) {\set{c}};

\foreach \x in { (A0), (A1), (B0), (B1), (C0), (C1) } {
  \fill \x circle[radius=1pt];
}

\begin{scope}[->, shorten >=4pt, shorten <=4pt] 
\path (A0) edge[bend left=30]  node[auto]      {\face{0}{1}{\set{b}}} (A1);
\path (A0) edge[bend right=30] node[auto,swap] {\face{0}{1}{\set{c}}} (A1);
\path (B0) edge[bend left=30]  node[auto]      {\face{0}{1}{\set{c}}} (B1);
\path (B0) edge[bend right=30] node[auto,swap] {\face{0}{1}{\set{a}}} (B1);
\path (C0) edge[bend left=30]  node[auto]      {\face{0}{1}{\set{a}}} (C1);
\path (C0) edge[bend right=30] node[auto,swap] {\face{0}{1}{\set{b}}} (C1);
\end{scope}

\end{cTikzPicture}

Similarly, we will see that, the following simplicial set, whose vertices can be labelled by 
the integer pairs in $ \Integers{} \times \Integers{}$, has no boundary
\begin{cTikzPicture}

\foreach \x in {-2, -1, 0, 1} {
  \foreach \y in {-2, -1, 0, 1} {
    \fill (\x, \y) circle[radius=1pt];
    \draw (\x, \y) -- ($ (\x, \y) + (1,0) $);
    \draw (\x, \y) -- ($ (\x, \y) + (0,1) $);
    \draw (\x, \y) -- ($ (\x, \y) + (1,1) $);
  }
  
  \fill (\x, 2) circle[radius=1pt];
  \draw (\x, 2) -- ($ (\x, 2) + (1,0) $);
  \draw[densely dotted] (\x, 2) -- ($ (\x, 2) + (0,0.5) $);
  \draw[densely dotted] (\x, 2) -- ($ (\x, 2) + (0.5,0.5) $);
  \draw[densely dotted] ($ (\x, -2) + (1, -0.5) $) -- ($ (\x, -2) + (1, 0) $);
  \draw[densely dotted] ($ (\x, -2) + (0.5, -0.5) $) -- ($ (\x, -2) + (1, 0) $);
}

\foreach \y in {-2, -1, 0, 1} {
  \fill (2, \y) circle[radius=1pt];
  \draw (2, \y) -- ($ (2, \y) + (0,1) $);
  \draw[densely dotted] (2, \y) -- ($ (2, \y) + (0.5,0) $);
  \draw[densely dotted] (2, \y) -- ($ (2, \y) + (0.5,0.5) $);
  \draw[densely dotted] ($ (-2, \y) - (0.5, 0) $) -- (-2, \y);
  \draw[densely dotted] ($ (-2, \y) - (0.5, 0.5) $) -- (-2, \y); 
}

\fill (2, 2) circle[radius=1pt];
\draw[densely dotted] (2,2) -- (2.5,2.5);
\draw[densely dotted] (2,2) -- (2.5,2);
\draw[densely dotted] (2,2) -- (2,2.5);

\draw[densely dotted] (-2,2) -- (-2.5,2);
\draw[densely dotted] (-2,2) -- (-2.5,1.5);

\draw[densely dotted] (-2,-2) -- (-2,-2.5);
\end{cTikzPicture}

Finally, to remind ourselves that simplical sets need \emph{not} be nice regular 
``manifolds'', consider the following simplicial set formed by joining two copies of 
\DeltaC{2}{} with one copy of each of \DeltaC{3}{}, \DeltaC{1}{} and \DeltaC{0}{} together 
with a copy of \boundary{\DeltaC{2}{}} (the boundary of \DeltaC{2}{}).

\begin{cTikzPicture}
\coordinate (A) at (-0.8, 0);
\coordinate (B) at (0,-1);
\coordinate (C) at (0,1);
\coordinate (D) at (0.8, 0);
\coordinate (E) at (-1,-1);
\coordinate (F) at (2,0);
\coordinate (G) at (-1,1);
\coordinate (H) at (1,1);
\coordinate (I) at (-2,1);
\coordinate (J) at (-1,2);

\foreach \x in { (A), (B), (C), (D), (E), (F), (G), (H), (I), (J) } {
  \fill \x circle[radius=1pt];
}

\draw[fill=black!5] (A) -- (B) -- (C) -- cycle;
\draw[fill=black!5] (B) -- (D) -- (C) -- cycle;
\draw[densely dotted] (A) -- (D);
\draw[fill=black!5] (A) -- (E) -- (B) -- cycle;
\draw (A) -- (G) -- (C);
\draw (D) -- (F);
\draw[fill=black!5] (G) -- (I) -- (J) -- cycle;
\end{cTikzPicture}
where the tetrahedron in the center, because it is a copy of \DeltaC{3}{}, is ``filled''.

Again we will see that the boundary of this simplicial set is

\begin{cTikzPicture}
\coordinate (A) at (-0.8, 0);
\coordinate (B) at (0,-1);
\coordinate (C) at (0,1);
\coordinate (D) at (0.8, 0);
\coordinate (E) at (-1,-1);
\coordinate (F) at (2,0);
\coordinate (G) at (-1,1);
\coordinate (H) at (1,1);
\coordinate (I) at (-2,1);
\coordinate (J) at (-1,2);

\foreach \x in { (A), (B), (C), (D), (E), (F), (G), (I), (J) } {
  \fill \x circle[radius=1pt];
}

\draw[fill=black!5] (A) -- (B) -- (C) -- cycle;
\draw[fill=black!5] (B) -- (D) -- (C) -- cycle;
\draw[densely dotted] (A) -- (D);
\draw (A) -- (E) -- (B) -- cycle;
\draw (G) -- (I) -- (J) -- cycle;
\end{cTikzPicture}
where the tetrahedron in the center, which is now a copy of \boundary{\DeltaC{3}{}}, is 
``hollow''.

As another example of a simplicial set consider
\begin{cTikzPicture}
\coordinate (A) at (-1,0);
\coordinate (B) at (1,0);
\fill (A) circle[radius=1pt];
\fill (B) circle[radius=1pt];
\path (A) edge[bend left=30] (B);
\path (A) edge[bend right=30] (B);
\end{cTikzPicture}
This example shows that we can not consider each simplex as a subset of the $0$-simplicies.

\TODO{We now need to define the sets of degenerate and non-degenerate simplicies. Which 
allows us to define a simplicial set to be an $\alpha$-simplicial set if all simplicies of 
dimension $\alpha < \beta \leq \kappa$ or greater are degenerate. Define the dimension of a 
simplical set to be the largest $\alpha$ for which there exists a non-degenerate 
$\alpha$-simplex.} 
\begin{remark}
Not surprisingly, \DeltaC{}{} is a $1$-simplicial set.  Explicitly, \DeltaC{}{0} is the set 
of $0$-simplicies, \set{\woSet{\alpha} \suchThat \alpha \leq \kappa}.  \DeltaC{}{1} is the 
set of $1$-simplicies, or order preserving maps, between pairs of $0$-simplicies, 
\woSet{\alpha} and \woSet{\beta}. Finally $\DeltaC{}{\gamma} = \emptySet$ for all $1 < 
\gamma \leq \kappa$.  Since there are no $\gamma$-simplicies for $1 < \gamma \leq \kappa$, 
we only need to exhibit the simplicial maps corresponding to $\coFace{0}{1}{\set{0}}, 
\coFace{0}{1}{\set{1}}, \coDegeneracy{1}{0}{\set{1}}$.  For which we have:
\begin{enumerate}
\item $\DeltaC{}{\coFace{0}{1}{\set{0}}}(m:\woSet{\alpha} \mapsTo \woSet{\beta}) = 
\woSet{\beta}$ (``domain'')
\item $\DeltaC{}{\coFace{0}{1}{\set{1}}}(m:\woSet{\alpha} \mapsTo \woSet{\beta}) = 
\woSet{\alpha}$ (``codomain'')
\item $\DeltaC{}{\coDegeneracy{1}{0}{\set{1}}}(\woSet{\alpha}) = \identity{\woSet{\alpha}}$ 
(``identity map'')
\end{enumerate}
\TODO{is this complete?!  It is no longer correct... correct this for our current notion of 
$1$-simplicial set.}
\end{remark}

\TODO{We want to introduce the fundamental group of a simplical set.  We want to introduce 
the covering space of a simplicial set.  We eventually want to introduce the (co)homology of 
a simplicial set.}

\section{Closure operators}

\section{Boundary operators}

\TODO{We need to define the boundary operator $\boundary : \simpC{} \mapsTo \simpC{}$.  
Really this definition boils down to the question ``how are we going to \emph{use} the 
boundary operator?''  A simplex is \emph{not} on the boundary of a simplicial set \emph{if} 
it is surrounded by other simplicies.  If it is internal to a collapsible sub-simplicial 
set. If it is internal to a locally contractible sub-simplicial set.}



\TODO{We need to find an (finite) algebra free way of defining the boundary operators.  We 
can do this by defining subfunctors and faces to be injective and surjective subfunctors.  
We can then discuss the poset of subfunctors and those subfunctors that are ``shared'' are 
``internal'' and hence not part of the boundary.  We may be able to define free words as a 
logic of relationships and all of the different possible models which respect those 
relationships.}
\TODO{We really want to get rid of this section entirely!  We really want to define the 
boundary operator without the use of abelian groups so that we do not need strictly finite 
sums!  This SHOULD be do able!}

\begin{definition}
\TODO{Define simplicial abelian group as the free group over X... al la 
\cite{goerssJardin1999SimplicialHomotopyTh} and then embedd X into this abelian group.  
Probably really want $\Integers_3$ rather than \Integers{} or $\Integers_2$. Define grade as 
}.
The \define{$n$-simplicial boundary operator}, $\partial_n : X_{n+1} \mapsTo X_n$, is 
\begin{equation}
\partial = \partial_n = \sum_{i=0}^n (-1)^i d_i
\end{equation}
\end{definition}

\begin{definition}
An $n$-simplex, $x \in X_n$, is a \define{face} if it is the image, $x = d_i(x')$, of an 
$(n+1)$-simplex, $x' in X_{n+1}$ for some $0 \leq i \leq n+1$.  An $n$-simplex, $x \in X_n$, 
is \define{degenerate} if it is the image, $x = s_i(x')$, of an $(n-1)$-simplex, $x' \in 
X_{n-1}$ for some $0 \leq i \leq n+1$. An $n$-simplex, $x \in X_n$, is a \define{boundary} 
if it is the image, $x = \partial_n(x')$, of an $(n+1)$-simplex, $x' \in X_{n+1}$.
\end{definition}

\begin{definition}
Fix a nondegenerate simplex, $x \in X$.  The \define{Hom-set for $\partial x$ with respect 
to $X$} is $\homomorphisms{X}{\partial x} = \set{ y \in X \suchThat \partial y = \partial 
x}$.
\end{definition}

\begin{remark}
A Hom-set, \homomorphisms{X}{\partial x}, is the set of all ways, within the simplicial set 
$X$, in which the boundary $\partial x$ can be ``filled in''. The reason for calling this 
the Hom-set will become clear when we come to consider the simplicial generalization of 
Category Theory in the next section. Essentially Theorem \TODO{which Theorem} will show that 
a Category is the ``ghost'' of a completed simplicial set.
\end{remark}

\TODO{need to define and provide lemmas for intersection, union of simplicies}
\TODO{need to define proto hom set of simplicies which have the same boundary}

\TODO{show that any $\gamma$-simplicial set can be embedded into a complete free 
(transfinite) simplicial set.  However what is the free set built out of a simplicial set 
which has simplicies of all orders?  Should it just be the free simplicial set on the $X_0$? 
Or should it respect the lack of simplicies at various orders? If we do not respect the lack 
of simplicies than any free simplicial set is just a free simplicial set on the 
$0$-simplicies.... is this what we want? The free/completion of a simplicial set is just the 
adjoint to the forgetful functor which forgets some of the simplicial sets structure (for 
example above some ``level''). This is in fact incorrect by in principle correct.  We want 
to define free simplicial sets using the universal algebraic approach and in topos theory 
this would be done with a monad operator from \simpC{} to itself which is a closure/interior 
operator.  This is NOT what the boundary operator is ;-( }

\begin{definition}
Given a set $X_0$, the free complete simplicitial set generated by $X_0$ is....
\end{definition}

\begin{lemma}
Given a simplicial set, X, its free completion is....
\end{lemma}

\begin{example}
A graph as a simplicial set.... \TODO{there is a minimal and a maximal version given by 
whether or not a given proto-$n$-simplex is an $n$-simplex for $1 < n$.  This suggests that 
we should have a ``Free'' functor associated to the forgetful functor which forgets all 
$n$-simplicies for $1 < n$}.
\end{example}

\section{Simplicial Approximation Structures}

\TODO{Need to discuss the limit (or is it colimit?) structure in \simpC{}.  In particular we 
want to understand how one ``simpler'' simplicial set embeds into another more ``detailed'' 
simplicial set. We are probably leading up to the concept of bisimulation and so will need 
to define open maps similarly to \cite{joyalNielsenWinskel1996bisimulation} and 
\cite{joyalMoerdijk1995algSetTh}.  In particular it will be the ideal structures associated 
to SAS's that will be interesting.}

\TODO{We can define a canonical closure operator (and hence a canonical topology on a given 
simplicial set) by including all non-degenerate simplicies above the set of verticies of a 
given simplicial set (and then filling in all requierd degenerate simplicies -- is this just 
the free completion?).}

\TODO{With SAS's we can define dimensioned interior operators (roughly due to the closure 
operators discussed above) which map \simpC{} into itself.  However these interior operators 
are really naturally defined as an operator from  \simpApproxC{} to itself.  At a specific 
approximation (this is really a Topos concept in the Topos of \symDynC{}), the interior 
simplical set is a retract of the simplicies which ``touch'' the boundary of the image of 
the closure operator. Or alternatively a sub-simplicial set is open if its closure is a 
proper sub-simplicial set (or is it distinct from the boundary operator?)}

\section{Categorical Actions}

\begin{definition}
A functor, $C \mapsTo S$, is \define{functorial} if for each commutative diagram
\begin{cTikzPicture}
\matrix (m) [comDiagM]
{ \DeltaC{k}{} & S \\
   \DeltaC{n}{} & C \\ };
\path[comDiagP, injection]
(m-1-1) edge (m-1-2)
             edge (m-2-1)
(m-2-1) edge (m-2-2)
(m-2-2) edge[map] node[right] {$w$} (m-1-2);
\end{cTikzPicture}
there exists a unique monic functor, $h : \Delta^n \mapsTo S$ which makes the diagram
\begin{cTikzPicture}
\matrix (m) [comDiagM]
{ \DeltaC{k}{} & S \\
   \DeltaC{n}{} & C \\ };
\path[comDiagP, injection]
(m-1-1) edge (m-1-2)
             edge (m-2-1)
(m-2-1) edge[densely dotted] node[inFront] {$h$} (m-1-2)
             edge node[below] {$i$} (m-2-2)
(m-2-2) edge[map] node[right] {$w$} (m-1-2);
\end{cTikzPicture}
commute.  \TODO{This is too strong.  We only want the top triangle to commute and in the 
bottom triangle for the injection of $h$ to be into the image of $w \compose i$}.
\end{definition}

\begin{lemma}
A functor, $C \mapsTo S$, is functorial if it is functorial with respect to $k = 0$ (in the 
above definition).
\end{lemma}
\begin{proof}
\TODO{prove this!}
\end{proof}

\begin{conjecture}
Every functor is functorial.
\end{conjecture}
\begin{proof}
\TODO{prove this!}
\end{proof}

\begin{definition}
The \define{action of a category}, $C$, on a set, $S$, is a functorial functor $a : C 
\mapsTo S$ \cite[Action of a Category on a set, version 2]{nLab}. An \define{action is free} 
if for each $c \in \objects{C}$, and $f \in \homomorphisms{C}{\partial c}$ then $a(f) = 
a(\identity{c})$ implies $f = \identity{c}$.
\end{definition}

\TODO{What we are calling an action should really be called a selection (or something 
similar).  We \emph{can} define an action as the collection of ``pointed'' simplicies of the 
selection.  That is for each simplex of the selection there are the $0$-simplicies of the 
selection simplex which represents the action.  In the $1$-category case we can (and do) 
identify the $0$-simplicies with the selected $1$-simplex since there is only one ``other'' 
$0$-simplex.  This then gives us a more traditional action on the $0$-simplicies}

\TODO{provide a more catgorical definition of free -- not so evil}

\TODO{We can actually provide a definition of action which is closer to the current 
definition of action by considering the ``set of actions'' ``mapping'' from the current 
$0$-simplex to each of the ``other'' $0$-simplicies of the ``selected'' $n$-simplex. This 
then gives us an ``action'' on the set of $0$-simplicies which is very similar to the 
current definition of action.}

\section{Category of Words}

\TODO{We really have to narrow a definition of word.  In reality words need not be free 
(i.e. homotopic to a point) but rather may have non-trivial homotopy groups.  Do we know 
that transfinite simplicial sets always have simple covers?  If so then we really have a 
category of covered words.  We then want to understand how to join words in the cover and 
obtain an meaningful joined word in the ``base''. }

\TODO{Define cartesian fibration see \cite[Section7.2]{brown2006TopologyGroupoids} and 
\cite[Definition 2.1]{streicher2008fibredCat}}

\begin{definition} \cite[??]{brown2006TopologyGroupoids} \\
Fix a pair of (small) categories, $C$ and $W$.  The category $W$ \define{covers}{} $C$ if 
there exists a functor $w : W \mapsTo C$ for which $w$ has the right lifting property with 
respect to the monic inclusion of $\Delta^0$ into $\Delta^n$ for any $n$, given by the 
diagram:
\begin{cTikzPicture}
\matrix (m) [comDiagM]
{ \DeltaC{0}{} & W \\
   \DeltaC{n}{} & C \\ };
\path[comDiagP, injection]
(m-1-1) edge (m-1-2)
             edge (m-2-1)
(m-2-1) edge[map,densely dotted] node[inFront] {$h$} (m-1-2)
             edge (m-2-2)
(m-1-2) edge[map] node[right] {$w$} (m-2-2);
\end{cTikzPicture}
\TODO{do we want $h$ monic?}
\end{definition}

\begin{lemma}
Every cover has an associate action on its domain induced by its image.  Every action on a 
set has an associated cover by the set on the action's domain.
\end{lemma}
\begin{proof}
\TODO{prove this!}
\TODO{does this exhibit categorical action of functor image on the domain of a functor}
\end{proof}

\TODO{Show that every cover is a Kan fibration  -- do this by showing that if you have 
commuting diagrams (as above) with standard simplicies which intersect then the total 
diagram commutes}

\begin{lemma}
Fix a triple of (small) categories, $C$, $W$ and $W'$.  Assume that $W$ covers $C$ via the 
functor $w : W \mapsTo C$.  If there exists a functor $w' : W' \mapsTo C$, then there exists 
a functor $h : W' \mapsTo W$ which makes the following diagram commute:
\begin{cTikzPicture}
\matrix (m) [comDiagM]
{      & W \\
   W' & C \\ };
\path[comDiagP, map]
(m-2-1) edge node[above left] {$h$} (m-1-2)
(m-1-2) edge node[right] {$w$} (m-2-2)
(m-2-1) edge node[below]  {$w'$} (m-2-2);
\end{cTikzPicture}
If $W'$ is connected then \image{w} is connected.
\end{lemma}
\begin{proof}
This follows \cite[Lemma 10.3.1]{brown2006TopologyGroupoids}. \TODO{do we need $W'$ to cover 
its $w'$-image?} \TODO{can we state that the $h$-image of $W'$ is connected in $W$? -- Is 
this the reason we need connectedness in $W'$?} \TODO{Do we really want to state that $h$ is 
monic?}
\end{proof}

\TODO{Do we need to state and prove lemmas similar to \cite[Lemmas 10.3.2 and 
10.3.3]{brown2006TopologyGroupoids}?}

\begin{lemma}
If $W$ is a cover of $C$ via $w : W \mapsTo C$, then there exists an action of $C$ on $W_0$.
\end{lemma}
\begin{proof}
\TODO{we need to show that there exists a functor $a : C \mapsTo \powerSet{W_0}$, in this 
case we follow \cite[Section 10.4]{brown2006TopologyGroupoids}}
\end{proof}

\TODO{Do we need to state and prove lemmas similar to \cite[Lemma 
10.4.2]{brown2006TopologyGroupoids} (which proves the existance of covers and in particular 
free covers)?}

\begin{definition}
Fix a (small) category, $C$.  A \define{bare word} over $C$, is a (small) category, $W$, 
together with a functor, $w : W \mapsTo C$ which is a free cover of the $w$-image of $W$ in 
$C$.  \TODO{define subword (do I need universality?)}. A \define{specific located word} over 
$C$, is a triple, \tuple{p, w, w'}, of bare words over $C$ for which $p$ and $w$ are 
subwords of $w'$ and $p$ is isomorphic to $\Delta^0$.  A \define{located word} is an 
equivilance class of specific located words where \tuple{p, w, w'} is equivilant to 
\tuple{\hat{p}, \hat{w}, \hat{w'}} iff $p = \hat{p}, w = \hat{w}$ and there exists a bare 
word, $w''$ for which $w$ and $\hat{w'}$ are intersecting subwords of $w''$.  \TODO{this is 
too strict -- we probably only care about $p = p'$ and $w = w'$ up to some equivilance or 
isomorphism not strict identity!  -- this is realated to the problem that we do not have a 
nice ``lattice'' structure to the ``points'' underlying this ``space'' so we really do not 
have a nice ``bundle'' structure.}
\end{definition}

\TODO{\cite[Section 10.2]{brown2006TopologyGroupoids} defines a groupoid cover as a 
bijection of the local star at an object.  A cartesian fibration \cite[Definition 
2.1]{streicher2008fibredCat} is defined in such a way that all compositions in the base lift 
to compositions in the fiber.  Are these two definitions equivilant in our case?}

\TODO{comment/prove that covers are cartesian fibrations which are unique}.
