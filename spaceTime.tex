%% Space-Time

\chapter{Space-time: directed simplicies, refinements and space-time paths}

In the previous section we explored the structures required to provide a \emph{time} focused 
foundations of computation. In this section we expand these time focused structures into 
their corresponding fully space-time structures.

We base our analysis of (directed) space on directed \emph{trans-finite} simplicies.
Recall that \emph{un-directed} simplicial structures are, categorically, defined as the
collection of pre-sheaves on the small category of finite (positive) ordinals. See for
example, the work of Grandis, \cite{grandis2001symSimpSets}, and
\cite{grandis2001fundamentalFunctorsSimplicial}. For a good \emph{concrete} introduction
to the categorical concept of pre-sheaves, see \cite{reyesReyesZolfaghari2004presheaves}.

We use \emph{directed} simplicies \emph{because} we wish to understand the ``higher
dimensional'' \emph{causal} structure of space-time.

Equally importantly, directed simplicies provide a higher dimensional generalisation of
the ubiquitously used concept of ``matrix'' outside of explicitly ``Euclidean vector
spaces''. Essentially a groupoid (that is a un-directed 1-category) \emph{is} a sparse
matrix. This analogy is critical when we want to study higher dimensional analogues of
``Markov Matrices'' and their corresponding ``Markov Chains''.
