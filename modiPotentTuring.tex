% LaTeX source for the modiPotent Turing document
%

\documentclass[a4paper,openany]{amsbook}
\usepackage{disitt}
\usepackage{disitt-symbols}
\usepackage{mdframed}
\newmdenv[linecolor=white,backgroundcolor=gray!10]{infobox}
\newenvironment{myQuote}{\begin{quotation}}{\end{quotation}}
\surroundwithmdframed[linecolor=white,backgroundcolor=gray!10]{myQuote}

\begin{document}
\frontmatter
\sloppy

\title[Modi-Potent-Turing]{Turing Machines for Modi-Potent beings}
%% author: stg
\author{Stephen Gaito}
\address{PerceptiSys Ltd, 21 Gregory Ave, Coventry, CV3 6DJ, United Kingdom}%
\email{stephen@perceptisys.co.uk}%
\urladdr{http://www.perceptisys.co.uk}


%% version summary
\thanks{Created: 2015-11-23}
\thanks{Git commit \gitReferences{} (\gitAbbrevHash{}) commited on \gitAuthorDate{} by \gitAuthorName{}}
\thanks{AMS-\LaTeX{}'ed on \today{}.}

%% Copyrights
\thanks{\textbf{Copyright: \copyright{} Stephen Gaito, PerceptiSys Ltd \the\year{}; Some rights reserved}}
\thanks{\textbf{This work is licensed under a Creative Commons Attribution-ShareAlike 4.0 International License.}}

\subjclass[2010]{Primary unknown; Secondary unknown} %
\keywords{Keyword one, keyword two etc.}%

\begin{abstract}
We provide a ``classical'' description of the Modi-Potent Turing Machine.
\end{abstract} 
\maketitle 
\tableofcontents 
\mainmatter


\section{Introduction}

In this paper we explore a more uniform definition of Turing machine which will
differ simply in the power of the maximal ordinal used by the Turing machine. In
this paper we do this using ``classical'' socially checked proofs using
``classical'' set theory. You are welcome to choose your favourite axiomization
using first order predicate logic. In subsequent papers we will found
mathematics using the non-``classical'' computation which is equivilant to our
current definition of Modi-potent Turing Machine.

Of particular importance in either definition, is the power of the Axiom of
Choice(s) used. For our purposes there are three versions of the Axiom of
Choice, in increasing power:
%
\begin{enumerate}
\item Prime Ideal Theorem (PIT)
\item Axiom of Choice (AoC)
\item Global Axiom of Choice (GAoC)
\end{enumerate}

The Prime Ideal Theorem (PIT) is used to define the ``result'' of computation at
limit ordinals, which corresponds to creation of limit points or alternatively
the closure, via ``ideal points'' of any computation.

The Axiom of Choice (AoC) itself, is used to define sequential computation, or
equivalently, to ensure the existence of lists of any required ordinal size.

While we will not use the Global Axiom of Choice (GAoC) in any essential way in
our work, we will see that the GAoC is required to ensure the existence of
co-algebraic lists, when we work, in subsequent papers, with Co-Algebraic
structures.

\bibliographystyle{amsalpha}
\bibliography{modiPotentTuring}

\end{document}

