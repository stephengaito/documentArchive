% LaTeX source for the trans-finite Turing document
%

%\documentclass[a4paper,openany]{amsbook}
\documentclass[a4paper,openany]{amsart}
\usepackage{disitt}
\usepackage{disitt-symbols}
\usepackage{mdframed}
\newmdenv[linecolor=white,backgroundcolor=gray!10]{infobox}
\newenvironment{myQuote}{\begin{quotation}}{\end{quotation}}
\surroundwithmdframed[linecolor=white,backgroundcolor=gray!10]{myQuote}

\begin{document}
%\frontmatter
\sloppy

\title[Modi-Potent-Turing]{Trans-finite Turing Machines for Modi-Potent beings}
%% author: stg
\author{Stephen Gaito}
\address{PerceptiSys Ltd, 21 Gregory Ave, Coventry, CV3 6DJ, United Kingdom}%
\email{stephen@perceptisys.co.uk}%
\urladdr{http://www.perceptisys.co.uk}


%% version summary
\thanks{Created: 2016-10-12}
\thanks{Git commit \gitReferences{} (\gitAbbrevHash{}) commited on \gitAuthorDate{} by \gitAuthorName{}}
\thanks{AMS-\LaTeX{}'ed on \today{}.}

%% Copyrights
\thanks{\textbf{Copyright: \copyright{} Stephen Gaito, PerceptiSys Ltd \the\year{}; Some rights reserved}}
\thanks{\textbf{This work is licensed under a Creative Commons Attribution-ShareAlike 4.0 International License.}}

\subjclass[2010]{Primary unknown; Secondary unknown} %
\keywords{Keyword one, keyword two etc.}%

\begin{abstract}
We provide a ``classical'' description of the hierarchy of Modi-Potent Turing Machines.
\end{abstract} 
\maketitle 
\tableofcontents 
%\mainmatter


\section{Introduction}

This paper is a prequel to our cycle of papers aiming to provide a rigorous mathematical
theory of what limited, ``Modi-potent'', beings can understand of Reality. One of the
central tenets of this ``Theory of Reality'' is that the correct foundations of
Mathematics is not first order logic but rather computation. However to recover all of
classical mathematics inside these computational foundations, we need a theory of
computation which is more powerful than classical (non oracle) Turing machines allow. This
paper provides definitions of these more powerful, trans-finite Turing machines in a
classical Set Theory setting. Subsequent papers will redo these definitions in the context
of computational foundations.

There have been numerous previous attempts to define ``more powerful'' Turing machines,
some of which, for example, oracle or ``o-machines'', date back to Turing's original work.
More recently the area of ``hypercomputation'' has attracted some interest, some of which
is rather controversial. Ord, \cite{ord2006hyperComputation}, provides a good review of
much of this work\footnote{For a high-level overview with some additional citations, see
the Wikipedia articles, \cite{wikipedia2015hyperComputation} and
\cite{wikipedia2015superRecursiveAlgorithm}, as well as the nLab article,
\cite{nLab2015hyperComputation}}. Some authors claim that there (may) exist physical
implementations of hypercomputational devices, \emph{we make \emph{no} such claims}.
Davis, \cite{davis2004hyperComputation}, provides a summary of some of the arguments
against these claims. Instead, our interest comes from our desire to capture
non-constructive classical mathematics in a computational setting.

\TODO{\cite{hamkinsLewis1998inifiniteTimeTuringMachines}}

Our aim in this paper is to explicitly explore a more uniform definition of Turing machine
which will differ simply in the power of the maximal ordinal used by the Turing machine.
We do this using ``classical'' socially checked proofs using ``classical'' set theory. You
are welcome to choose your favourite axiomization using first order predicate logic. In
subsequent papers we will found mathematics using a form of trans-finite computation which
is equivalent to our current definition of trans-finite Turing Machine. 

Classically, the interest and hence the definitions of both computation and set theory,
have been focused upon computations which \emph{halt} and sets which are
\emph{well-founded}. Our computational foundations will admit that there are some
interesting computations and structures which do not halt or are non-well-founded. In
categorical terms the halting computations and well-founded sets are \emph{algebraic},
while the non-halting processes and non-well-founded structures are \emph{co-algebraic}. 
While we will not make any use of co-algebraic structures in this paper, our subsequent 
papers will make essential use of these co-algebraic structures.

Of particular importance for either the classical or compuatational definitions of
trans-finite compuatation, is the power of the Axiom of Choice(s) used. For our purposes
there are three versions of the Axiom of Choice, in increasing power: %
\begin{enumerate}
\item Prime Ideal Theorem (PIT)
\item Axiom of Choice (AoC)
\item Global Axiom of Choice (GAoC)
\end{enumerate}

The Prime Ideal Theorem (PIT) is used to define the ``result'' of computation at
limit ordinals, which corresponds to creation of limit points or alternatively
the closure, via ``ideal points'' of any computation.

The Axiom of Choice (AoC) itself, is used to define sequential computation, or
equivalently, to ensure the existence of lists of any required ordinal size.

While we will not use the Global Axiom of Choice (GAoC) in any essential way in
our work, we will see that the GAoC is required to ensure the existence of
co-algebraic lists, when we work, in subsequent papers, with Co-Algebraic
structures.

\bibliographystyle{amsalpha}
\bibliography{modiPotentTuring}

\end{document}

