% Computation

Our express aim is to provide a \emph{computational} foundation for mathematics.
To do this we will use a number fundamental concepts or realizations:
\begin{enumerate}
\item	Lists/sequences are the fundamental ``objects''. Sets and logic are
derivative concepts.

\item	Categories are central to the definition of computation. They form the
natural structures with which to ``generalize'' Turing's tabular definition of a
computational machine or more recent labelled state transition machine
descriptions at the heart of Turing's definition of computation. More
importantly they provide the natural structures with which to describe
\emph{provably correct} computation.

\item	Turing's choice-machines (c-machines; non-deterministic machines) are more
fundamental than his automatic-machines (a-machines; determinisitic machines).
The von Neuman architecture can be viewed as a c-machine overlayed with a form
of (modal) Hennessy-Milner logic and McCarthy's LISP to ensure deterministic
computation. Compare the `Wizard Book's description of the arithmetic data-paths
and central processing units of Register Machines, \cite[Chapter
5]{abelsonSussmanSussman1996structureInterpretationComputerPrograms}.

\item	Provably correct computation depends upon the triumvirate of Denotational,
Operational and Axiomatic Semantics pioneered by Scott, Strachey, Plotkin, Floyd
and Hoare in the late 1960's, 1970's and early 1980's. These semantics roughly
correspond to Tarski's semantical model theory, Gentzen's
natural-deduction/proof-theory, and (very roughly) Church's typed
$\lambda$-calculus respectively from the 1930's and 1940's.

\item	To provide a computational foundation for mathematics, we seek the
\emph{largest} fixed point of the syntax/semantics  That is we seek a syntax
which \emph{is} its own semantics. This means that non-well founded structures,
potentially non-terminating processes and co-algebras are \emph{central} to this
exposition. Ironically, the past 100 years of the foundations of mathematical
thought has consistently striven to banish non-well founded structures as the
basis of, among others, Russell's paradox. In fact, we see the \emph{human
interpretation} of the linguistic meaning of what is essentially the logical
liar paradox as the source of these paradoxes.

\item	The primary objective of the working mathematician is to \emph{prove} that
``useful'' computations are correct. Different mathematical communities use
different (computational) languages focused upon different definitions of the
concept of ``usefulness''. The objective of the community of foundational
mathematicians is to provide the tools required to prove that these various
(mathematical) (computational) languages \emph{are} correct.

\end{enumerate}

Current mathematical practice is to provide socially checked linguistic proofs
in a style, which, we all assume, could be translated into a completely logical
notation. There are, at present, a number of initiatives to replace these
socially checked proofs with proof outlines constructed using ``interactive
theorem provers'' such as ACL2, Agda, Coq, HOL, Isabelle, or Mizar. All of these
systems use one or other system of logic (first, second or higher-order). 
Our objective is to provide computations which are proven correct. 

Hilbert's original goal was to ``remain in Cantor's paradise'' by proving that
the axioms of a general foundation for mathematics (at one level) were
\emph{consistent} using more restrictive ``finitist'' arguments at a higher
meta-level. To recover the \emph{whole} of current mathematical practice using a
\emph{computational} definition of rigour, we will be forced to make use of some
definition of trans-finite computation, which is, not surprisingly, generally
assumed to be beyond the abilities of finite-beings such as ourselves and our
computational devices. This means that \emph{at some (meta-)level}, we must,
like Hilbert, identify and prove correct a finte-computation which proves all
``lower'' level computations to be correct.

\section{Semi-formal overview}

Current mathematical practice in the foundations of mathematics, is to accept
socially checked finite proofs at the meta-mathematical level. Our aim is to
provide provably correct computational proofs at \emph{all} levels.
\emph{However} we are forced to \emph{begin} by providing a socially agreed
semi-formal overview.

Again, all current mathematical descriptions of the foundations of mathematics
are essentially circular. Unfortunately our description is no exception. While
we intend to provide complete formal definitions of all of the concepts we use,
to get started, we will have to defer the fully formal definition of a number of
key concepts until we have ``built'' sufficient formal structure to be able to
provide these formal definitions. We assume we have some definition of the
finite natural numbers, as well as a defintion of a mapping (function) from the
finite natural numbers to themselves. We also assume that we have definitions of
collection, partial-order, well-order and a well-order endowed with a
predecessor mapping.

\subsection{Syntax}

When we are done we will have a well-ordering, $\mathfrak{W}(\bottom, \le)$
endowed with a predecessor mapping which indexes a collection of
meta-mathematical syntaxes. The lowest syntax, $S_{\bottom}$, corresponding to
the ``object language'', consists exclusively of the symbols: `(', `)', and `.`.
The syntaxes at any higher level may contain additional symbols. For each
syntax, $S_{\tau}$, except $S_{\bottom}$, there is a distinguished symbol,
`$\$_{\tau}$', as well as a distinguished collection of symbols,
$\mathfrak{I}_{\tau}$, of identifiers and for each $\bottom < \tau$, $\$_{\tau}$
is an identifier. There is an associated mapping, $|\cdot | : S^* \mapsTo
\mathfrak{W}$ from the collection of symbol texts to the well-ordering.

One of the key concepts in the various semantics is that of the state
(environment or valuation). A state is a mapping, $\sigma : \mathfrak{I} \mapsTo
\Universe{}{}$, from the collection of identifiers to \Universe{}{} where
$|\sigma(x)| \le |x|$.

We begin by defining our (initially) finite-computational universe,
\Universe{}{}, which we call ``Plato's playground'', to include all structures
in the largest fixed point of the following two rules:

\begin{prooftree}
\AxiomC{}
\RightLabel{empty-list}
\UnaryInfC{()}
\end{prooftree}

\begin{prooftree}
\AxiomC{x}
\AxiomC{y}
\RightLabel{pair}
\BinaryInfC{(x . y)}
\end{prooftree}

Liguistically we interpret these two rules to mean:
\begin{itemize}
\item	The empty list is a member of the universe.

\item	If $x$ and $y$ are lists in the universe, then the ``pair'', $(x.y)$, of
$x$ and $y$ is also in the universe.

\end{itemize}

We explicitly accept that any member of the universe, \Universe{}{}, might not
be finite. That is, any member of the universe might be, what computer
scientists call, a ``stream'', and in deed it might be a ``stream'' of
``streams''.

A \emph{denotational semantics} is provided by a mapping, $[ \cdot ]_{\cdot} :
\Universe{}{} \crossProduct \Sigma \mapsTo \Universe{}{}$, from an element of
\Universe{}{} together with a given state, $\sigma$, to a specific element of a
collection of semantic entities. Correspondingly, the denotational semantics of
the first rule is that the \emph{syntactical form} $()$ (the marks in the sand
`$($' and `$)$') together with any environment (including the empty environment)
maps to the $()$ element in the \Universe{}{}. In a more mathematical notation,
$[()]_{\sigma} = ()$. Similarly the denotational semantics of the second rule is
that given an environment, $\sigma$, which contains a mapping for the variables,
$x$ and $y$, the syntactic form $(x , y)$, is mapped to the element of
\Universe{}{} given by $[ ( x . y ) ]_{\sigma} = ( [x]_{\sigma} . [y]_{\sigma} )
= ( \sigma(x) . \sigma(y) )$. Note that for these two rules and hence for any
syntactic structure in Plato's playground, the denotational semantics of the
syntax \emph{is} the syntax, and hence Plato's playground is the fixed-point of
the denotational semantics.

\emph{Operational semantics} is provided by a 

\emph{Axiomatic semantics} is provided by

\section{Booting Mathematics}

Bootstrapping a computer at power on is in practice a very tedious process,
involving loading ultra-simple code from well know locations in the CPU's local
address space, which then, in turn, loads slightly more complex code from well
know locations on some external disk, which, in turn, loads ...

Building the first ever compiler for a language with no other compilation tools
is equally a very tedious process, one which, given modern cross-compilation
techniques, has, almost certainly, not been done in decades.

``Bootstrapping'' Mathematics involves an equally tedious process. We must
identify specific mappings which are simple enough for an average mathematician
to understand (and more importantly ``believe'' reasonably ``true'') but which
is complex enough to in turn define more complex mathematical structures.

\section{Mud's eye view}

In this section we pursue a mud's eye view of the construction of Plato's
playground. While it does not represent the best way to conceptually understand
the \emph{meaning} of Plato's playground, it \emph{does} satisfy the requirement
of exhibiting an \emph{explicit finitist} construction.

The typical activity of the mathematical community is to build language
extensions of the foundational structure. So at any one time Plato's playground
will be the minimal language in a partially ordered collection of currently
useful/interesting linguistic extensions. A language extension consists of the
specification of a syntax/semantics mapping which is a fixed point on the
minimal language of Plato's playground itself.

The conceptually simplest linguistic extension, since it is closest to one of
the inspirations of Plato's playground, is the LISP extension. Following LISP
convention, \cite{mcCarthy1960lisp}, we refer to Plato's playground itself as
the foundational language of F-expressions, while the LISP extension itself
consists of the additional symbolic langauge of S-expressions. In fact ``the''
LISP extension is actually a family of mutually consistent extensions. One which
adds lists (S$_{\text{L}}$-expressions), one which adds the natural numbers
(S$_{\text{N}}$-expressions), one which adds the lambda calculus
(S$_{\lambda}$-expressions), and so on, as need or interest arises.

For the F-expressions, we allow ourselves three symbols `(', `)',
'.'\footnote{To make things readable, as humans we are also allowed to use any
amount of whitespace (in ASCII Unix notation the characters `\textvisiblespace'
(space), `\textbackslash t' (tab), `\textbackslash n' (newline), `\textbackslash
r' (return) ). All whitespace is ignored. Following LISP conventions we are also
allowed one line comments, in that anything \emph{including and after} the
character `;' (semicolon) is ignored up to the end of the current line.}.

For the S-expressions, we allow ourselves any of the printable ASCII
characters\footnote{We could in principle allow any collection of printable
characters such as, for example, the UTF-8 characters. However, for current
purposes the ASCII characters are sufficient and yield a simpler core
description.}.

The S$_{\text{L}}$, S$_{\text{N}}$, and S$_{\lambda}$-expressions all inherit
the same collection of characters as the S-expressions, but provide additional
semantics not present in the base S-expressions.

