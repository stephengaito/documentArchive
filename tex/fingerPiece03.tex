% building Plato's engine

While few mathematicians realise it, the primary activity of Mathematics is to
create (Domain Specific) Computational Languages, otherwise known as Programming
Languages. Those scientists and engineers who \emph{use} mathematics, use it to
\emph{compute} useful values given a \emph{mathematical} model and
various observations. Even ``pure'' mathematicians \emph{compute} the truth of a
proposition. The (current) foundations of Mathematics are predicated on the
assumption that the truth of any mathematical statement can be proven using
first order logic together with the axioms of set theory. Proof theory is really
the study of how these proofs \emph{compute} the \emph{truth} of a statement.

The \emph{problem} with the current foundations of mathematics is that, given
the `Liar paradox', \emph{computing truth} is problematic, since the
\emph{semantic meaning of truth} can be encoded in the syntactic structure of
any sufficiently interesting theory based on first order logic. This is the
substance of both Russell's paradox as well as G\"odel's proof that first order
theories can not be both consistent and complete.

As practised, mathematics is really a \emph{collection} of loosely inter-related
languages. The ``purpose'' of set theory is to provide a ``universe'' of
mathematical objects within which ``normal'' mathematics can be understood to
exist. We accept the existence and (relative) consistency of these languages by
(ultimately) interpreting them in set theory, which is itself just one of the
many languages. In fact there are a number of different languages which we
``accept'' as ``set theory''. Since all of these languages, set theory or not,
are ultimately based upon logic and the computation of truth we can only ever
``prove'' relative consistency of one language by interpreting it in another. In
particular, we ``accept'' the consistency of one set theory by interpreting it
in another set theory.

Instead of computing the truth of the existence of mathematical objects using
the axioms of the first order set theory, we show that we can directly
\emph{compute} these objects using a small collection of intuitively simple
computational primitives. In fact there are a number of equivalent small
collections of computational primitives we could choose from. The type-less
$\lambda$-calculus is certainly one of these.

In order to ``compute'' the \emph{whole} of classical mathematics, at least one
of these simple computational primitives will have to be as ``powerful'' as the
Axiom of Choice. We will show that \emph{transfinite} computation is equivalent
to the Axiom of Choice. The simplest way to define an intuitively meaningful
notion of transfinite computation, is to use transfinite lists, alternatively
known as sequences. While the type-less $\lambda$-calculus can certainly define
both lists and numbers, we prefer to use a variant of Pure Lisp since lists and
the manipulation of lists are ``first class'' computational primitives in Lisp.
``Pure (1.5) Lisp'', \cite{mcCarthy1960lisp}, uses recursion in the form of the
``lambda'' operator and application. We choose to replace recursion with
iteration in the form of the ``while/do'' operator.

To define transfinite computation we will need to begin by defining the
ordinals. For each ordinal there is a corresponding concept of computation up to
that ordinal. We will define the transfinite ordinals essentially in the same
way that Cantor did. \TODO{choose correct reference}

Given that it is generally assumed that we, as finite beings, are only capable
of finite computation, how do we know if a transfinite computation is
``correct''? Theoretical Computer Science has three answers to this problem:
denotational, operational and axiomatic semantics, based roughly on Tarski's
semantic interpretation of truth, Gentzen's proof theory, and Type-theory
respectively. While the computational versions of semantics ultimately depend
upon mathematical set theory, we will use these concepts to define a fixed point
interpretation of our variant of Pure Lisp, and hence a \emph{computational}
foundation for Mathematics. Essentially we will follow Hilbert's original
program but replace logic with transfinite computation.

\begin{myQuote}
As every self respecting Computer Scientist knows, Plato's universe does not
consist of a collection of ``Sets'' but rather a ``collection'' of Lists of
Lists. \quad ``Before the Set, was the List''
\end{myQuote}

Our objective is to compute a universe of objects, Plato's Wilderness, suitable
to provide the foundations of Mathematics. We do this by defining Plato's
Programming Language, (PPL).

\section{Criteria for any Foundation of Mathematics}

Following Hatcher, \cite{hatcher1982logicalFoundationsMath}, we posit the
following list of critera any Foundation of Mathematics must satisfy:

\begin{enumerate}
\item \textbf{A foundation of mathematics must be adequate for a reasonably large
portion of mathematics.}

\noindent If we ``assume'' something as powerful as the Axiom of Choice, we will
be able to recover the whole of current mathematical set theory (smaller than
the first strongly inaccessible cardinal).

\item \textbf{A foundation must derive from some intuitively natural principles.}

\noindent The foundational (co-)data language \emph{is} the List of Lists. The
only operations on this language are LISP's car, cdr, cons, together with
failure and repetition.

\item \textbf{The basic principles and primitive (undefined) notions should be
as economical as possible.}

\noindent See above.

\item \textbf{The foundation must be consistent.}

\noindent For a computational foundation, consistency is given since we are
\emph{computing} the (standard) model of the formal theory. What is more
important is that the formal theory is fully abstract, that is the denotation
and operational semantics are equivalent, in logical terms, the formal theory is
both sound and complete.

\item \textbf{The foundation should be expressed (or expressible) as a formal system.}

\noindent In computational terms, each formal system is a (programming) language
complete with a syntax as well as denotational, operational and axiomatic
semantics. Every such formal system \emph{must} have (at least) a denotational
interpretation into the foundational (co-)data language.

\item \textbf{The construction of everyday mathematics in the system should be
``natural'' and ``orderly''.}

\noindent With the proposed computational foundations of mathematics, there is a
collection of formal theories each of which have (at least) an interpretation in
the (co-)data language. Each of these formal theories should be tailored to the
mathematical field in question, so by definition they should represent
``natural'' and ``orderly'' systems in which to conduct mathematics in a given
field.

\end{enumerate}

\section{Plato's Wilderness: Base Case: The Playground}

Before we define transfinite computation, we will begin with the slightly
``easier'' finite computation.  We loosely follow LISP 1.5's example,
\cite{mcCarthy1960lisp}, by organising our Plato language as a tower of
different types of expressions, the bottom layer being the object expressions
and the top layer being meta-expressions. We also loosely follow Mike Gordon's
1973 PhD thesis, \cite{gordon1973semanticsPureLisp}, to work through the
semantics of our form of Plato's language.


\begin{cTikzPictureWorking}

\fill[orange!20!white] (0,0) -- (4,0) -- (4,4) -- (0,4)
  -- (0,3) -- (3,3) -- (3,1) -- (0,1) -- (0,0);

\path (0,0) edge (4,0);
\path (0,0) edge (0,4);
\path (4,0) edge (4,4);
\path (0,4) edge (4,4);

\path (0,1) edge (3,1);
\path (3,1) edge (3,2);
\path (0,2) edge (3,2);

\path (0,3) edge (3,3);
\path (3,2) edge (3,3);

\path (3,1.5) edge[dashed] (4,1.5);

\node at (1.5,1.5) {L-expression};
\node at (1.5,2.5) {M-expression};
\node at (2,3.7)   {Interpreter - Process};
\node at (2,3.25)  {(Co-algebra)};
\node at (2,0.7)   {Interpreter - Memory};
\node at (2,0.25)  {(Algebra)};

\end{cTikzPictureWorking}

The processor part of the interpreter, could be a human, a computer, or a Lisp
machine. The memory part of the interpreter, could be sand, paper, an abacus,
computer RAM, or Lisp machine list memory.

An L-expression is that subset of all Lisp S-exrepssions which have no atoms
other than NIL, and hence are pure Lists of Lists.

Our objective is to build an interpreter, like John McCarthy's original Lisp
programming language, \cite{mcCarthy1960lisp,
mcCarthyAbrahamsEdwardsHartLevin1965lispManual}. We will use ideas from the two
excellent texts teaching Programming Language Design using the PLT Racket
dialect of Lisp,
\cite{krishnamurthi2007programmingLanguagesApplicationInterpretation,
krishnamurthi2012programmingLanguagesApplicationInterpretation,
friedmanWand2008essentialsProgrammingLanguages, racket2016racket}, as well as
the Graph Programming Language (GP), pioneered by Detlef Plump's group at the
University of York, \cite{steinert2007graphProgramming,
manningPlump2008yorkMachine, plump2009graphProgramming,
plumpSteinert2010semanticsGraphProgramming, plump2012graphProgramming}.

Our interpreter is based upon the York Abstract Machine (YAM),
\cite{manningPlump2008yorkMachine}. While the YAM is a bytecode interpreter, we
will simply interpret to either Racket/Lisp or Haskell. We will implement
Prolog/Haskell style pattern matching by implementing a unification subsystem,
\cite[section
30.5.1]{krishnamurthi2007programmingLanguagesApplicationInterpretation}, as well
as Prolog backtracking search,
\cite[34.1.1]{krishnamurthi2007programmingLanguagesApplicationInterpretation}.

We will also provide proof of correctness of all computation using Denotational,
Operational, Axiomatic and Weakest Precondition based semantics,
\cite{gunter1992semainticProgrammingLanguages,
winskel1993formalSemanticsProgrammingLanguages, gries1981scienceProgramming,
scott1970theoryComputation, plotkin1981structuralOperationalSemantics,
hoare1969axiomaticSemantics, dijkstra1975guardedCommandsFormalDerivation}. More
specifically these ideas have been worked out for Graph Programming by Detlef
Plump's group, \cite{steinert2007graphProgramming,
plumpSteinert2010semanticsGraphProgramming, poskittPlump2010hoareLogic,
poskittPlump2010hoareCalculus}.

\section{(co-)data language}

We begin by providing the syntax of our diSimplex (co-)data language following the
standard set by \cite{friedmanWand2008essentialsProgrammingLanguages}.

\newcommand{\MExp}[1]{& \textnormal{MExp:} & #1  \nonumber \\}
\newcommand{\denExp}[2]{\ensuremath{ \lbrack \! \lbrack #1 \rbrack \! \rbrack #2}}
\newcommand{\denEql}[3]{\ensuremath{ \lbrack \! \lbrack #1 \rbrack \! \rbrack #2 = #3}}
\newcommand{\Denotation}[1]{& \textnormal{Den:} & #1 \\}
\newcommand{\opExp}[3]{\ensuremath{\langle #1 , #2 \rangle \rightarrow #3}}
\newcommand{\Operational}[1]{& \textnormal{Oper:} & %
 \begin{minipage}{6cm} \begin{prooftree} #1 \end{prooftree} \end{minipage} \nonumber \\}
\newcommand{\Axiomatic}[1]{& \textnormal{Axiom:} & %
 \begin{minipage}{6cm} \begin{prooftree} #1 \end{prooftree} \end{minipage} \nonumber }

\subsection{List expressions}

\begin{bnf}\label{null-list}
 \bnfProd{ListExp}{ \bnfTD{(} \bnfSP \bnfTD{)} \nonumber }
 \MExp{( \bnfSP \bnfTD{nil} \bnfSP )}
 \Denotation{\denEql{\bnfSP(\bnfSP)\bnfSP}{\sigma}{()}}
 \Operational{
  \AxiomC{}
  \RightLabel{$(\bnfSP)$}
  \UnaryInfC{\opExp{()}{\sigma}{()}}}
 \Axiomatic{
  \AxiomC{}
  \RightLabel{$(\bnfSP)$}
  \UnaryInfC{()}
 }
\end{bnf}

\begin{bnf}\label{cons-list}
 \bnfProd{ListExp}{ \bnfTD{(} \bnfSP \bnfPN{ListExp} \bnfSP \bnfPN{ListExp} \bnfSP \bnfTD{)} \nonumber }
 \MExp{( \bnfSP \bnfTD{cons} \bnfSP \bnfTD{list1} \bnfSP \bnfTD{list2} \bnfSP )}
 \Denotation{\denEql{\bnfSP\bnfTD{cons}\bnfSP\bnfTD{list1}\bnfSP\bnfTD{list2}\bnfSP}
  {\sigma}{(\bnfSP \denExp{\bnfTD{list1}}{\sigma} \bnfSP
   \denExp{\bnfTD{list2}}{\sigma} \bnfSP)}}
 \Operational{
  \AxiomC{\opExp{list1}{\sigma}{\lambda_1}}
  \AxiomC{\opExp{list2}{\sigma}{\lambda_2}}
  \RightLabel{$\bnfTD{cons}\bnfSP\bnfTD{list1}\bnfSP\bnfTD{list2}$}
  \BinaryInfC{\opExp{\bnfTD{cons}\bnfSP\bnfTD{list1}\bnfSP\bnfTD{list2}}{\sigma}
                   {(\bnfSP \lambda_1\bnfSP\lambda_2\bnfSP)}}
 }
 \Axiomatic{
  \AxiomC{$\lambda_1$}
  \AxiomC{$\lambda_2$}
  \RightLabel{$\bnfTD{cons}\bnfSP\bnfTD{list1}\bnfSP\bnfTD{list2}$}
  \BinaryInfC{$(\bnfSP\lambda_1\bnfSP\lambda_2\bnfSP)$}
 }
\end{bnf}

\begin{bnf}\label{car-list}
 \bnfProd{ListExp}{ \bnfTD{car} \bnfSP \bnfPN{ListExp} \nonumber }
 \MExp{( \bnfSP \bnfTD{car} \bnfSP \bnfTD{list1} \bnfSP )}
 \Denotation{\denEql{\bnfSP\bnfTD{car}\bnfSP\bnfTD{list1}\bnfSP}
  {\sigma}{\begin{cases}
  \lambda_1 \text{ if } \opExp{list1}{\sigma}{(\bnfSP\lambda_1\bnfSP\lambda_2\bnfSP)} \\
  \bot \text{ otherwise }
  \end{cases}}}
 \Operational{
  \AxiomC{\opExp{list1}{\sigma}{(\bnfSP\lambda_1\bnfSP\lambda_2\bnfSP)}}
  \RightLabel{$\bnfTD{car}\bnfSP\bnfTD{list1}$}
  \UnaryInfC{\opExp{\bnfSP\bnfTD{car}\bnfSP\bnfTD{list1}}{\sigma}{\lambda_1}}
 }
 \Axiomatic{
  \AxiomC{ $ ( \bnfSP \lambda_1 \bnfSP \lambda_2 \bnfSP ) $ }
  \RightLabel{$\bnfTD{car}\bnfSP\bnfTD{list1}$}
  \UnaryInfC{ $ \lambda_1 $ }
 }
\end{bnf}

\begin{bnf}\label{cdr-list}
 \bnfProd{ListExp}{ \bnfTD{cdr} \bnfSP \bnfPN{ListExp} \nonumber }
 \MExp{( \bnfSP \bnfTD{cdr} \bnfSP \bnfTD{list1} \bnfSP )}
 \Denotation{\denEql{\bnfSP\bnfTD{cdr}\bnfSP\bnfTD{list1}\bnfSP}
  {\sigma}{\begin{cases}
   \lambda_2 \text{ if } \opExp{list1}{\sigma}{(\bnfSP\lambda_1\bnfSP\lambda_2\bnfSP)} \\
   \bot \text{ otherwise }
  \end{cases}}}
 \Operational{
  \AxiomC{\opExp{list1}{\sigma}{(\bnfSP\lambda_1\bnfSP\lambda_2\bnfSP)}}
  \RightLabel{$\bnfTD{cdr}\bnfSP\bnfTD{list1}$}
  \UnaryInfC{\bnfSP\opExp{\bnfTD{car}\bnfSP\bnfTD{list1}\bnfSP}
   {\sigma}{\lambda_2}}
 }
 \Axiomatic{
  \AxiomC{$(\bnfSP\lambda_1\bnfSP\lambda_2\bnfSP)$}
  \RightLabel{$\bnfTD{cdr}\bnfSP\bnfTD{list1}$}
  \UnaryInfC{$\lambda_2$}
 }
\end{bnf}

\subsection{Command expressions}

\begin{bnf}
 \bnfProd{Proof}{ \bnfTD{Begin} \bnfSP \bnfPN{Expression} \bnfSP \bnfTD{End} \nonumber }
 \MExp{( \bnfSP \bnfTD{proof} \bnfSP \bnfTD{exp1} \bnfSP )}
 \Denotation{something}
% \Operational{something}
 \Axiomatic{
  \AxiomC{}
  \RightLabel{empty-ctx}
  \UnaryInfC{\cJudgement{\cdot}}
 }
\end{bnf}

\begin{bnf}
 \bnfProd{Expression}{\bnfPN{List} \nonumber }
 \MExp{( \bnfSP \bnfTD{list} \bnfSP \bnfTD{list1} \bnfSP )}
 \Denotation{something}
% \Operational{something}
% \Axiomatic{something}
\end{bnf}

\begin{bnf}
 \bnfProd{Expression}{ \bnfTD{null?} \bnfSP \bnfPN{List} \nonumber }
 \MExp{( \bnfSP \bnfTD{null?} \bnfSP \bnfTD{list1} \bnfSP )}
 \Denotation{something}
% \Operational{something}
% \Axiomatic{something}
\end{bnf}

\begin{bnf}
 \bnfProd{Expression}{ \bnfTD{if}   \bnfSP \bnfPN{Expression} \bnfSP
                       \bnfTD{then} \bnfSP \bnfPN{Expression} \bnfSP
                       \bnfTD{else} \bnfSP \bnfPN{Expression} \nonumber }
 \MExp{( \bnfSP \bnfTD{if} \bnfSP \bnfTD{exp1} \bnfSP \bnfTD{exp2} \bnfSP \bnfTD{exp3} \bnfSP )}
 \Denotation{something}
% \Operational{something}
% \Axiomatic{something}
\end{bnf}

\begin{bnf}
 \bnfProd{Expression}{ \bnfPN{Identifier} \nonumber }
 \MExp{( \bnfSP \bnfTD{var} \bnfSP \bnfTD{var1} \bnfSP )}
 \Denotation{something}
% \Operational{something}
% \Axiomatic{something}
\end{bnf}

\begin{bnf}
 \bnfProd{Expression}{ \bnfTD{let} \bnfSP \bnfPN{Identifier} \bnfSP
                       \bnfTD{=}   \bnfSP \bnfPN{Expression} \nonumber }
 \MExp{( \bnfSP \bnfTD{let} \bnfSP \bnfTD{var1} \bnfSP \bnfTD{exp1} \bnfSP )}
 \Denotation{something}
% \Operational{something}
% \Axiomatic{something}
\end{bnf}


