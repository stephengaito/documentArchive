% building Plato's engine

As every self respecting Computer Scientist knows, Plato's universe does not
consist of a collection of ``Sets'' but rather a ``collection'' of Lists of
Lists.

``Before the Set, was the List''

\begin{bnf*}
 \bnfProd{List}{ ( \bnfSP ) }
 \bnfAlt{ ( \bnfSP \bnfPN{List} \bnfSP \bnfTD{.} \bnfSP \bnfPN{List} \bnfSP )}
\end{bnf*}

Alternatively in Kleene star format as essentially used in Lisp/Scheme:

\begin{bnf*}
 \bnfProd{List}{ ( \bnfSP \bnfKS{ \bnfPN{List} }{*} \bnfSP ) }
\end{bnf*}

\begin{racket}
(println "Hello from Plato's engine!")
\end{racket}


\begin{cTikzPictureWorking}

\fill[orange!20!white] (0,0) -- (4,0) -- (4,4) -- (0,4)
  -- (0,3) -- (3,3) -- (3,1) -- (0,1) -- (0,0);

\path (0,0) edge (4,0);
\path (0,0) edge (0,4);
\path (4,0) edge (4,4);
\path (0,4) edge (4,4);

\path (0,1) edge (3,1);
\path (3,1) edge (3,2);
\path (0,2) edge (3,2);

\path (0,3) edge (3,3);
\path (3,2) edge (3,3);

\path (3,1.5) edge[dashed] (4,1.5);

\node at (1.5,1.5) {L-expression};
\node at (1.5,2.5) {M-expression};
\node at (2,3.7)   {Interpreter - Process};
\node at (2,3.25)  {(Co-algebra)};
\node at (2,0.7)   {Interpreter - Memory};
\node at (2,0.25)  {(Algebra)};

\end{cTikzPictureWorking}

The processor part of the interpreter, could be a human, a computer, or a Lisp
machine. The memory part of the interpreter, could be sand, paper, an abacus,
computer RAM, or Lisp machine list memory.

An L-expression is that subset of all Lisp S-exrepssions which have no atoms
other than NIL, and hence are pure Lists of Lists.

Our objective is to build an interpreter, like John McCarthy's original Lisp
programming language, \cite{mcCarthy1960lisp,
mcCarthyAbrahamsEdwardsHartLevin1965lispManual}. We will use ideas from the two
excellent texts teaching Programming Language Design using the PLT Racket
dialect of Lisp,
\cite{krishnamurthi2007programmingLanguagesApplicationInterpretation,
krishnamurthi2012programmingLanguagesApplicationInterpretation,
friedmanWand2008essentialsProgrammingLanguages, racket2016racket}, as well as
the Graph Programming Language (GP), pioneered by Detlef Plump's group at the
University of York, \cite{steinert2007graphProgramming,
manningPlump2008yorkMachine, plump2009graphProgramming,
plumpSteinert2010semanticsGraphProgramming, plump2012graphProgramming}.

Our interpreter is based upon the York Abstract Machine (YAM),
\cite{manningPlump2008yorkMachine}. While the YAM is a bytecode interpreter, we
will simply interpret to either Racket/Lisp or Haskell. We will implement
Prolog/Haskell style pattern matching by implementing a unification subsystem,
\cite[section
30.5.1]{krishnamurthi2007programmingLanguagesApplicationInterpretation}, as well
as Prolog backtracking search,
\cite[34.1.1]{krishnamurthi2007programmingLanguagesApplicationInterpretation}.

We will also provide proof of correctness of all computation using Denotational,
Operational, Axiomatic and Weakest Precondition based semantics,
\cite{gunter1992semainticProgrammingLanguages,
winskel1993formalSemanticsProgrammingLanguages, gries1981scienceProgramming,
scott1970theoryComputation, plotkin1981structuralOperationalSemantics,
hoare1969axiomaticSemantics, dijkstra1975guardedCommandsFormalDerivation}. More
specifically these ideas have been worked out for Graph Programming 1by Detlef
Plump's group, \cite{steinert2007graphProgramming,
plumpSteinert2010semanticsGraphProgramming, poskittPlump2010hoareLogic,
poskittPlump2010hoareCalculus}.