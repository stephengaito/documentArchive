% LaTeX source for the zfcIsConsistent document
%

\documentclass[a4paper,openany]{amsart}
\usepackage[utf8]{inputenc}
\usepackage[english]{babel}
\usepackage{disitt}
\usepackage{disitt-symbols}
\usepackage[backend=biber,style=alphabetic,citestyle=alphabetic]{biblatex}
\addbibresource{zfcIsConsistent.bib}
\usepackage{mdframed}
\newmdenv[linecolor=white,backgroundcolor=gray!10]{infobox}
\newenvironment{myQuote}{\begin{quotation}}{\end{quotation}}
\surroundwithmdframed[linecolor=white,backgroundcolor=gray!10]{myQuote}

\begin{document}

\sloppy

\title[ZFC is absolutely consistent]{A computational proof of the absolute consistency of ZFC}
%% author: stg
\author{Stephen Gaito}
\address{PerceptiSys Ltd, 21 Gregory Ave, Coventry, CV3 6DJ, United Kingdom}%
\email{stephen@perceptisys.co.uk}%
\urladdr{http://www.perceptisys.co.uk}


%% version summary
\thanks{Created: 2016-10-12}
\thanks{Git commit \gitReferences{} (\gitAbbrevHash{}) commited on \gitAuthorDate{} by \gitAuthorName{}}
\thanks{AMS-\LaTeX{}'ed on \today{}.}

%% Copyrights
\thanks{\textbf{Copyright: \copyright{} Stephen Gaito, PerceptiSys Ltd \the\year{}; Some rights reserved}}
\thanks{\textbf{This work is licensed under a Creative Commons Attribution-ShareAlike 4.0 International License.}}

\subjclass[2010]{Primary unknown; Secondary unknown} %
\keywords{Keyword one, keyword two etc.}%

\begin{abstract}
We show that the first order Zermelo-Fraenkel set theory with the Axiom of
Choice has a model in a countable computational structure.
\end{abstract} 
\maketitle 
\tableofcontents 


\section{Introduction}

Assuming Zermelo-Fraenkel set theory with the Axiom of Choice (ZFC), the
``downward'' L\"owenheim-Skolem Theorem, \cite[Theorem 12.1, page
157]{jech2003setTheory} implies that ZFC itself, as a first order theory, has a
\emph{countable} model. The objective of this paper is to exhibit an explicit
countable model of ZFC using purely computational structures, and hence
\emph{not} making use of ZFC itself. This provides a non-set-theoretic model of
ZFC which, given its countable computational structure, could be, in theory,
computed by any powerful enough computer given sufficient time.

\printbibliography
\end{document}

