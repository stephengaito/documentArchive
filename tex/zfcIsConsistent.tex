% LaTeX source for the zfcIsConsistent document
%

\documentclass[a4paper,openany]{amsart}
\usepackage[utf8]{inputenc}
\usepackage[english]{babel}
\usepackage{disitt}
\usepackage{disitt-symbols}
\usepackage{disitt-code}
\usepackage[backend=biber,style=alphabetic,citestyle=alphabetic]{biblatex}
\addbibresource{zfcIsConsistent.bib}
\usepackage{mdframed}
\newmdenv[linecolor=white,backgroundcolor=gray!10]{infobox}
\newenvironment{myQuote}{\begin{quotation}}{\end{quotation}}
\surroundwithmdframed[linecolor=white,backgroundcolor=gray!10]{myQuote}

\begin{document}

\sloppy

\title[ZFC is absolutely consistent]
{A computational proof of the absolute consistency of ZFC}
%% author: stg
\author{Stephen Gaito}
\address{PerceptiSys Ltd, 21 Gregory Ave, Coventry, CV3 6DJ, United Kingdom}%
\email{stephen@perceptisys.co.uk}%
\urladdr{http://www.perceptisys.co.uk}


%% version summary
\thanks{Created: 2016-10-12}
\thanks{Git commit \gitReferences{} (\gitAbbrevHash{}) commited on \gitAuthorDate{} by \gitAuthorName{}}
\thanks{AMS-\LaTeX{}'ed on \today{}.}

%% Copyrights
\thanks{\textbf{Copyright: \copyright{} Stephen Gaito, PerceptiSys Ltd \the\year{}; Some rights reserved}}
\thanks{\textbf{This work is licensed under a Creative Commons Attribution-ShareAlike 4.0 International License.}}

\subjclass[2010]{Primary unknown; Secondary unknown} %
\keywords{Keyword one, keyword two etc.}%

\begin{abstract}
We show that the first order Zermelo-Fraenkel set theory with the Axiom of
Choice has a model in a countable computational structure.
\end{abstract} 
\maketitle 
\tableofcontents 

\section{Introduction}

Assuming Zermelo-Fraenkel set theory with the Axiom of Choice (ZFC), the
``downward'' L\"owenheim-Skolem Theorem, \cite[Theorem 12.1, page
157]{jech2003setTheory} implies that ZFC itself, as a first order theory, has a
\emph{countable} model. The objective of this paper is to exhibit an explicit
countable model of ZFC using purely computational structures, and hence
\emph{not} making use of ZFC itself. This provides a non-set-theoretic model of
ZFC which, given its countable computational structure, could be, in theory,
computed by any powerful enough computer given sufficient time.

From a computational point of view we intend to build a compiler which compiles
strings in a suitable dialect of first order logic with equality and the set
membership relation, $\in$, into a suitable computational data structure. This
``suitable'' data structure is based upon a ``classical'' list of lists. The
simplest naive approach would be to map the cumulative hierarchy of sets in ZFC,
\cite[Lemma 6.3]{jech2003setTheory}, into a similar cumulative hierarchy of
lists. Given that ``classically'' computational structures are at most
countable, this naive approach will not work. The next simplest approach, which
we \emph{do} take, is to index the cumulative computational structures by the
ordinals using Cantor’s Normal Form, \cite[Theorem 2.26 ]{jech2003setTheory}.

\subsection{Approach}

Few proofs in mathematics are complete in that they rarely map out \emph{every}
detail of \emph{every} step. The proof contained in this paper will be rather
more explict than any other current proof, in that it \emph{will} be a
computation which \emph{can be} computed on \emph{the reader's} computer.
However to be absolutely complete we should also map this computational process
back to, for example, an \emph{explicit} Turing machine. We will do the next
best thing, the computational code contained in this paper will be in the Lisp
dialect Scheme's sub-dialect Guile, which is, for our purposes, a relatively
slight, but useable, modification of Church's type free Lambda Calculus. An
encyclopedic introduction to the syntax and semantics of the type free Lambda
Calculus can be found in Barendregt's \cite{barendregt2012lambdaCalculus}.

Since our primary computational structure will be lists of lists, Lisp in
general and Scheme in particular is a good practical choice of computational
infrastructure. Of the Lisp dialects, Scheme is minimalistic with a clean
implementation of recursion where functions are first class values. Guile,
\cite{guile2014online, guile2014reference}, is a comprehensive modern
open-source implementation of Scheme.

\section{Cumulative computational structures}

\begin{guile}
(write (command-line))
(newline)
(display "Hello, world!")
(newline)
\end{guile}

\section{The language of First Order Logic with Equality and $\in$}

\printbibliography
\end{document}

