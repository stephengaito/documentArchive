% Peano Arithmetic

In this section we count. We use a List of Lists form of Cantor's normal form,
\cite[Theorem 2.26]{jech2003setTheory}, to represent the Ordinals, \Ordinal{}.

\begin{bnf*}
 \bnfProd{Ordinal}{ ( \bnfSP \bnfKS{\bnfPN{OrdinalSumand}}{*} \bnfSP )}
 \bnfProd{OrdinalSumand}{ 
   ( \bnfSP 
     ( \bnfSP \bnfPN{Ordinal} \bnfSP ) \bnfSP
     ( \bnfSP \bnfPN{Natural} \bnfSP ) \bnfSP
   ) }
 \bnfProd{Natural}{ ( \bnfSP \bnfKS{( \bnfSP )}{*} \bnfSP ) }
\end{bnf*}

\begin{racket}
;; Peano Arithmetic

(println "Hello from Peano Arithmetic!")

#|
(define zero (lambda () ( list ) ) )

(define omega (lambda () ) )

(define isZero? (lambda (n) (null? n)))

;;(define isOmega? (lambda (n) (zero? n)))

(define suc (lambda (n) ( list* '() n) ) )

(define pred (lambda (n) (cdr n) ) )
|#

(define zero (lambda () 'zero ) )

(define isZero? (lambda (n) (eq? n 'zero)))

(define suc
  (lambda (n)
    ( if (isZero? n)
      '('zero (()))
      ( if (isZero? (cdr n))
        (list (car n) '(()))
        (list (car n) (list* '() (cadr n)))
      )
    )
  )
)

(define one (lambda () (suc (zero))))

(define omega (lambda () (list* (one) (zero))))
\end{racket}