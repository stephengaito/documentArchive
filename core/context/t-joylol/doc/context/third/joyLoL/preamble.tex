% A ConTeXt document [master document: joylol.tex]

\startchapter[title=Preamble]

\prependMkIVCode{default}
\startMkIVCode
%D \module
%D   [     file=t-joylol,
%D      version=2017.05.10,
%D        title=\CONTEXT\ User module,
%D     subtitle=The JoyLoL prgramming language for \ConTeXt\,
%D       author=Stephen Gaito,
%D         date=\currentdate,
%D    copyright=PerceptiSys Ltd (Stephen Gaito),
%D        email=stephen@perceptisys.co.uk,
%D      license=MIT License]

%C Copyright (C) 2017 PerceptiSys Ltd (Stephen Gaito)
%C
%C Permission is hereby granted, free of charge, to any person obtaining a
%C copy of this software and associated documentation files (the
%C "Software"), to deal in the Software without restriction, including
%C without limitation the rights to use, copy, modify, merge, publish,
%C distribute, sublicense, and/or sell copies of the Software, and to
%C permit persons to whom the Software is furnished to do so, subject to
%C the following conditions:
%C
%C The above copyright notice and this permission notice shall be included
%C in all copies or substantial portions of the Software.
%C
%C THE SOFTWARE IS PROVIDED "AS IS", WITHOUT WARRANTY OF ANY KIND, EXPRESS
%C OR IMPLIED, INCLUDING BUT NOT LIMITED TO THE WARRANTIES OF
%C MERCHANTABILITY, FITNESS FOR A PARTICULAR PURPOSE AND NONINFRINGEMENT.
%C IN NO EVENT SHALL THE AUTHORS OR COPYRIGHT HOLDERS BE LIABLE FOR ANY
%C CLAIM, DAMAGES OR OTHER LIABILITY, WHETHER IN AN ACTION OF CONTRACT,
%C TORT OR OTHERWISE, ARISING FROM, OUT OF OR IN CONNECTION WITH THE
%C SOFTWARE OR THE USE OR OTHER DEALINGS IN THE SOFTWARE.

% begin info
%
% title   : JoyLoL CoAlgebra definitions
% comment : Provides structured document and code generation
% status  : under development, mkiv only
%
% end info

\unprotect

\ctxloadluafile{t-joylol}
\stopMkIVCode

\startMkIVCode
\protect \endinput
\stopMkIVCode

\createMkIVCodeFile%
  {default}%
  {../../../../../build/t-joylol.mkiv}%
  {A ConTeXt MkIV module}

\prependLuaCode{default}
\startLuaCode
-- This is the lua code associated with t-joylol.mkiv

if not modules then modules = { } end modules ['t-joylol'] = {
    version   = 1.000,
    comment   = "joylol programming language - lua",
    author    = "PerceptiSys Ltd (Stephen Gaito)",
    copyright = "PerceptiSys Ltd (Stephen Gaito)",
    license   = "MIT License"
}

thirddata        = thirddata        or {}
thirddata.joylol = thirddata.joylol or {}

local joylol   = thirddata.joylol

-- local coAlgs     = thirddata.joyLoLCoAlgs
-- coAlgs.theCoAlg  = {}
-- local theCoAlg   = coAlgs.theCoAlg


local tInsert = table.insert
local tConcat = table.concat
local tRemove = table.remove
local tSort   = table.sort
local sFmt    = string.format
local sMatch  = string.match
local toStr   = tostring

-- **Problem**: we can not assume that a user *has* a compiled and working 
-- C based JoyLoL. This is the "Bootstrapping (Compiler)" problem (see 
-- Wikipedia). We solve this problem by writing a minimal joyLoL 
-- interpreter in Lua. 

-- SO we first check to see if the joyLoL (C shared libraries) exists and 
-- can be required, if it can not be loaded, we load the joyLoLMinLua 
-- version instead. 

-- The following conditional require is adapted from: shuva's answer to 
--  "How to check if a module exists in Lua?"
-- see: http://stackoverflow.com/a/22686090

local hasJoyLoL,joyLoL = pcall(require, "joyLoL/joyLoL")
if not hasJoyLoL then
  interfaces.writestatus("joyLoL",
    "Could NOT load joyLoL... loading mininal Lua version instead.")
  joyLoL = require 'joyLoLMinLua/joyLoL'
end

coAlgs.joyLoL = joyLoL

local pushData, pushProcess = joyLoL.pushData, joyLoL.pushProcess
local pushProcessQuoted = joyLoL.pushProcessQuoted
local popData, popProcess   = joyLoL.popData, joyLoL.popProcess
local newList, newDictionary = joyLoL.newList, joyLoL.newDictionary
local jEval = joyLoL.eval

interfaces.writestatus("joyLoL", joyLoL.version())

-----------------------------------------------------------------------------
-- NOTE the following uses raw JoyLoL code to collect the coAlgebra's 
-- literate code description.

-- To understand this code.... **think categorically**

-- In JoyLoL a particular object in the category *is* the structure of the 
-- data stack, while a particular arrow in the category *is* the process 
-- stack.

-- To understand what these arrows are doing... you read the JoyLoL code 
-- in reverse order (from a 'jEval' up). 
-----------------------------------------------------------------------------

local function addNewDict(aCtx, dictName)
  pushProcess(aCtx, 'addToDict')
  newDictionary(aCtx)
  pushProcessQuoted(aCtx, dictName)
end

local function addNewList(aCtx, listName)
  pushProcess(aCtx, 'addToDict')
  newList(aCtx)
  pushProcess(aCtx, listName)
end

local function addStrToListNamed(aCtx, aStr, listName)
  pushProcess(aCtx, 'popData')
  pushProcess(aCtx, 'appendToEndList')
  pushProcessQuoted(aCtx, aStr) -- need to explicitly quote this string
  pushProcess(aCtx, 'lookupInDict')
  pushProcessQuoted(aCtx, listName)
end

function coAlgs.newCoAlg(coAlgName)
  texio.write_nl('newCoAlg: ['..coAlgName..']')
  theCoAlg               = {}
  theCoAlg.name          = coAlgName
  theCoAlg.ctx           = joyLoL.newContext()
  theCoAlg.hasJoyLoLCode = false;
  theCoAlg.hasLuaCode    = false;
  theCoAlg.hasCHeader    = false;
  theCoAlg.hasCCode      = false;
  -- 
  local aCtx = theCoAlg.ctx
  pushProcess(aCtx, 'addToDict')
  pushProcessQuoted(aCtx, coAlgName)
  pushProcessQuoted(aCtx, 'coAlgName')
  addNewList(aCtx, 'dependsOn')
  addNewList(aCtx, 'wordOrder')
  addNewDict(aCtx, 'words')
  newDictionary(aCtx)
  jEval(aCtx)
  --
  -- add the new word: "global"
  --
  coAlgs.newWord('global')
end

function coAlgs.addDependency(dependencyName)
  texio.write_nl('addDependency: ['..dependencyName..']')
  local aCtx = theCoAlg.ctx
  pushProcess(aCtx, 'popData')
  pushProcess(aCtx, 'appendToEndList')
  pushProcessQuoted(aCtx, dependencyName)
  pushProcess(aCtx, 'lookupInDict')
  pushProcess(aCtx, 'dependsOn')
  jEval(aCtx)
end

local function writeCodeFile(aCtx, coAlgName, templateName, filePath, fileExt)
  local outFilePath = string.format('%s/%s.%s', filePath, coAlgName, fileExt)
  local outFile = io.open(outFilePath, 'w')
  outFile:write(pp.write(theCoAlg))
  pushProcess(aCtx, 'render')
  pushProcess(aCtx, 'getTemplate')
  pushProcessQuoted(aCtx, templateName)
  jEval(aCtx)
  local renderedBaseTemplate = popData(aCtx)
  texio.write_nl(renderedBaseTemplate)
  outFile:write('\n')
  outFile:write(renderedBaseTemplate)
  outFile:write('\n')
  outFile:close()
end

function coAlgs.createCoAlg()
  texio.write_nl("createCoAlg...")
  if not theCoAlg then return end
  local coAlgName = theCoAlg.name or 'unknown'
  texio.write_nl(string.format('creating JoyLoL CoAlgebra: [%s]', coAlgName))
  --
  local aCtx = theCoAlg.ctx
  coAlgs.loadTemplates(aCtx) -- contains a jEval
  --
  if theCoAlg.hasCHeader    then writeCodeFile(aCtx, coAlgName, 'cHeader',    'build', 'h')   end
  if theCoAlg.hasCCode      then writeCodeFile(aCtx, coAlgName, 'cCode',      'build', 'c')   end
  if theCoAlg.hasLuaCode    then writeCodeFile(aCtx, coAlgName, 'luaCode',    'build', 'lua') end
  if theCoAlg.hasJoyLoLCode then writeCodeFile(aCtx, coAlgName, 'joyLoLCode', 'build', 'joy') end
  --
  texio.write_nl(string.format(' created JoyLoL CoAlgebra: [%s]', coAlgName))
end

function coAlgs.newWord(wordName)
  texio.write_nl('newWord: ['..wordName..']')
  theCoAlg.curWord    = wordName
  local aCtx = theCoAlg.ctx
  pushProcess(aCtx, 'popData')
  pushProcess(aCtx, 'addToDict')
  addNewList(aCtx, 'preData')
  addNewList(aCtx, 'postData')
  addNewList(aCtx, 'preProcess')
  addNewList(aCtx, 'postProcess')
  addNewList(aCtx, 'joyLoLCode')
  addNewList(aCtx, 'cHeader')
  addNewList(aCtx, 'cCode')
  addNewList(aCtx, 'luaCode')
  newDictionary(aCtx)
  pushProcessQuoted(aCtx, wordName) -- need to explicitly quote this string
  pushProcess(aCtx, 'lookupInDict')
  pushProcessQuoted(aCtx, 'words')
  addStrToListNamed(aCtx, wordName, 'wordOrder')
  jEval(aCtx)
end

function coAlgs.endWord()
  theCoAlg.curWord = 'global'
end

function coAlgs.addPreDataStackDescription(name, condition)
  texio.write_nl('addPreDataStackDescription: ['..name..']['..condition..']')
  local aCtx = theCoAlg.ctx
  pushProcess(aCtx, 'popData')
  pushProcess(aCtx, 'popData')
  pushProcess(aCtx, 'popData')
  pushProcess(aCtx, 'appendToEndList')
  pushProcess(aCtx, 'addToDict')
  pushProcessQuoted(aCtx, condition)
  pushProcessQuoted(aCtx, 'condition')
  pushProcess(aCtx, 'addToDict')
  pushProcessQuoted(aCtx, name)
  pushProcessQuoted(aCtx, 'name')
  newDictionary(aCtx) -- new dictionary for this condition
  pushProcess(aCtx, 'lookupInDict')
  pushProcessQuoted(aCtx, 'preData')
  pushProcess(aCtx, 'lookupInDict')
  pushProcessQuoted(aCtx, theCoAlg.curWord)
  pushProcess(aCtx, 'lookupInDict')
  pushProcess(aCtx, 'words')
  jEval(aCtx)
end

function coAlgs.addPostDataStackDescription(condition)
  texio.write_nl('addPostDataStackDescription: ['..condition..']')
  local aCtx = theCoAlg.ctx
  pushProcess(aCtx, 'popData')
  pushProcess(aCtx, 'popData')
  pushProcess(aCtx, 'popData')
  pushProcess(aCtx, 'appendToEndList')
  pushProcess(aCtx, 'addToDict')
  pushProcessQuoted(aCtx, condition)
  pushProcessQuoted(aCtx, 'condition')
  newDictionary(aCtx) -- new dictionary for this condition
  pushProcess(aCtx, 'lookupInDict')
  pushProcessQuoted(aCtx, 'postData')
  pushProcess(aCtx, 'lookupInDict')
  pushProcessQuoted(aCtx, theCoAlg.curWord)
  pushProcess(aCtx, 'lookupInDict')
  pushProcess(aCtx, 'words')
  jEval(aCtx)
end

function coAlgs.addPreProcessStackDescription(name, condition)
  texio.write_nl('addPreProcessStackDescription: ['..name..']['..condition..']')
  local aCtx = theCoAlg.ctx
  pushProcess(aCtx, 'popData')
  pushProcess(aCtx, 'popData')
  pushProcess(aCtx, 'popData')
  pushProcess(aCtx, 'appendToEndList')
  pushProcess(aCtx, 'addToDict')
  pushProcessQuoted(aCtx, condition)
  pushProcessQuoted(aCtx, 'condition')
  pushProcess(aCtx, 'addToDict')
  pushProcessQuoted(aCtx, name)
  pushProcessQuoted(aCtx, 'name')
  newDictionary(aCtx) -- new dictionary for this condition
  pushProcess(aCtx, 'lookupInDict')
  pushProcessQuoted(aCtx, 'preProcess')
  pushProcess(aCtx, 'lookupInDict')
  pushProcessQuoted(aCtx, theCoAlg.curWord)
  pushProcess(aCtx, 'lookupInDict')
  pushProcess(aCtx, 'words')
  jEval(aCtx)
end

function coAlgs.addPostProcessStackDescription(condition)
  texio.write_nl('addPostProcessStackDescription: ['..condition..']')
  local aCtx = theCoAlg.ctx
  pushProcess(aCtx, 'popData')
  pushProcess(aCtx, 'popData')
  pushProcess(aCtx, 'popData')
  pushProcess(aCtx, 'appendToEndList')
  pushProcess(aCtx, 'addToDict')
  pushProcessQuoted(aCtx, condition)
  pushProcessQuoted(aCtx, 'condition')
  newDictionary(aCtx) -- new dictionary for this condition
  pushProcess(aCtx, 'lookupInDict')
  pushProcessQuoted(aCtx, 'postProcess')
  pushProcess(aCtx, 'lookupInDict')
  pushProcessQuoted(aCtx, theCoAlg.curWord)
  pushProcess(aCtx, 'lookupInDict')
  pushProcess(aCtx, 'words')
  jEval(aCtx)
end

function coAlgs.addJoyLoLCode(bufferName)
  texio.write_nl('addJoyLoLCode: ['..bufferName..']')
  theCoAlg.hasJoyLoLCode = true
  local aCtx = theCoAlg.ctx
  pushProcess(aCtx, 'popData')
  pushProcess(aCtx, 'popData')
  pushProcess(aCtx, 'popData')
  pushProcess(aCtx, 'appendToEndList')
  pushProcessQuoted(aCtx, buffers.getcontent(bufferName))
  pushProcess(aCtx, 'lookupInDict')
  pushProcessQuoted(aCtx, 'joyLoLCode')
  pushProcess(aCtx, 'lookupInDict')
  pushProcessQuoted(aCtx, theCoAlg.curWord)
  pushProcess(aCtx, 'lookupInDict')
  pushProcess(aCtx, 'words')
  jEval(aCtx)
end

function coAlgs.addCHeader(bufferName)
  texio.write_nl('addCHeader: ['..bufferName..']')
  theCoAlg.hasCHeader = true
  local aCtx = theCoAlg.ctx
  pushProcess(aCtx, 'popData')
  pushProcess(aCtx, 'popData')
  pushProcess(aCtx, 'popData')
  pushProcess(aCtx, 'appendToEndList')
  pushProcessQuoted(aCtx, buffers.getcontent(bufferName))
  pushProcess(aCtx, 'lookupInDict')
  pushProcessQuoted(aCtx, 'cHeader')
  pushProcess(aCtx, 'lookupInDict')
  pushProcessQuoted(aCtx, theCoAlg.curWord)
  pushProcess(aCtx, 'lookupInDict')
  pushProcess(aCtx, 'words')
  jEval(aCtx)
end

function coAlgs.addCCode(bufferName)
  texio.write_nl('addCCode: ['..bufferName..']')
  theCoAlg.hasCCode = true
  local aCtx = theCoAlg.ctx
  pushProcess(aCtx, 'popData')
  pushProcess(aCtx, 'popData')
  pushProcess(aCtx, 'popData')
  pushProcess(aCtx, 'appendToEndList')
  pushProcessQuoted(aCtx, buffers.getcontent(bufferName))
  pushProcess(aCtx, 'lookupInDict')
  pushProcessQuoted(aCtx, 'cCode')
  pushProcess(aCtx, 'lookupInDict')
  pushProcessQuoted(aCtx, theCoAlg.curWord)
  pushProcess(aCtx, 'lookupInDict')
  pushProcess(aCtx, 'words')
  jEval(aCtx)
end

function coAlgs.addLuaCode(bufferName)
  texio.write_nl('addLuaCode: ['..bufferName..']')
  theCoAlg.hasLuaCode = true
  local aCtx = theCoAlg.ctx
  pushProcess(aCtx, 'popData')
  pushProcess(aCtx, 'popData')
  pushProcess(aCtx, 'popData')
  pushProcess(aCtx, 'appendToEndList')
  pushProcessQuoted(aCtx, buffers.getcontent(bufferName))
  pushProcess(aCtx, 'lookupInDict')
  pushProcessQuoted(aCtx, 'luaCode')
  pushProcess(aCtx, 'lookupInDict')
  pushProcessQuoted(aCtx, theCoAlg.curWord)
  pushProcess(aCtx, 'lookupInDict')
  pushProcess(aCtx, 'words')
  jEval(aCtx)
end

coAlgs.joyLoL = joyLoL

interfaces.writestatus('joyLoL', "loaded JoyLoL CoAlgs")
\stopLuaCode

\createLuaCodeFile%
  {default}%
  {../../../../../build/t-joylol.lua}%
  {-- A Lua file}

\prependLuaTemplate{default}
\startLuaTemplate
if not modules then modules = { } end modules ['t-joylol-coalg'] = {
    version   = 1.000,
    comment   = "JoyLoLCoAlgs - templates",
    author    = "PerceptiSys Ltd (Stephen Gaito)",
    copyright = "PerceptiSys Ltd (Stephen Gaito)",
    license   = "MIT License"
}

thirddata              = thirddata              or {}
thirddata.joyLoLCoAlgs = thirddata.joyLoLCoAlgs or {}

local coAlgs     = thirddata.joyLoLCoAlgs

local templates  = { }

templates.cHeader = [=[
This is the start of a cHeader template
{{ lookupInDict 'coAlgName }}
This is the end of a cHeader template
]=]

templates.cCode = [=[
This is the start of a cCode template
{{ lookupInDict 'coAlgName }}
This is the end of the cCode template
]=]

templates.joyLoLCode = [=[
This is the start of a joyLoLCode template
{{ lookupInDict 'coAlgName }}
This is the end of the joyLoLCode template
]=]

templates.luaCode = [=[
-- A Lua file (automatically generated)
{{ lookupInDict 'coAlgName }}
This is the end of the luaCode template
]=]

local joyLoL = coAlgs.joyLoL
local pushData, pushProcess = joyLoL.pushData, joyLoL.pushProcess
local pushProcessQuoted = joyLoL.pushProcessQuoted
local popData, popProcess   = joyLoL.popData, joyLoL.popProcess
local newList, newDictionary = joyLoL.newList, joyLoL.newDictionary
local jEval = joyLoL.eval

-----------------------------------------------------------------------------
-- NOTE the following uses raw JoyLoL code to load the templates into the 
-- context provided. 

-- To understand this code.... **think categorically**

-- In JoyLoL a particular object in the category *is* the structure of the 
-- data stack, while a particular arrow in the category *is* the process 
-- stack.

-- To understand what these arrows are doing... you read the JoyLoL code 
-- in reverse order (from a 'jEval' up). 
-----------------------------------------------------------------------------

function coAlgs.loadTemplates(aCtx)
  pushProcess(aCtx, 'addToDict')
  for aKey, aValue in pairs(templates) do
    pushProcess(aCtx, 'addToDict')
    pushProcessQuoted(aCtx, aValue)
    pushProcessQuoted(aCtx, aKey)
  end
  newDictionary(aCtx)
  pushProcessQuoted(aCtx, 'templates')
  jEval(aCtx)
end

interfaces.writestatus('joyLoL', 'loaded JoyLoL CoAlg templates')
\stopLuaTemplate

\createLuaTemplateFile%
  {default}%
  {../../../../../build/t-joylol-templates.lua}%
  {-- A Lua template file}

\prependMinJoyLoL{default}
\startMinJoyLoL
-- A Lua file

-- Copyright (C) 2017 PerceptiSys Ltd (Stephen Gaito)
--
-- Permission is hereby granted, free of charge, to any person obtaining a 
-- copy of this software and associated documentation files (the 
-- "Software"), to deal in the Software without restriction, including 
-- without limitation the rights to use, copy, modify, merge, publish, 
-- distribute, sublicense, and/or sell copies of the Software, and to 
-- permit persons to whom the Software is furnished to do so, subject to 
-- the following conditions:
--
-- The above copyright notice and this permission notice shall be included 
-- in all copies or substantial portions of the Software.
--
-- THE SOFTWARE IS PROVIDED "AS IS", WITHOUT WARRANTY OF ANY KIND, EXPRESS 
-- OR IMPLIED, INCLUDING BUT NOT LIMITED TO THE WARRANTIES OF 
-- MERCHANTABILITY, FITNESS FOR A PARTICULAR PURPOSE AND NONINFRINGEMENT. 
-- IN NO EVENT SHALL THE AUTHORS OR COPYRIGHT HOLDERS BE LIABLE FOR ANY 
-- CLAIM, DAMAGES OR OTHER LIABILITY, WHETHER IN AN ACTION OF CONTRACT, 
-- TORT OR OTHERWISE, ARISING FROM, OUT OF OR IN CONNECTION WITH THE 
-- SOFTWARE OR THE USE OR OTHER DEALINGS IN THE SOFTWARE.

local pp = require('prettyPrint')

local texlua = rawget(os,'exec')    ~= nil

local function reportError(aMessage)
  aMessage = 'ERROR: '..aMessage..debug.traceback('\n----', 2)..'\n----\n'
  if texlua then
    texio.write(aMessage)
  else
    print(aMessage)
  end
end

local joyLoL = { }
local table_insert = table.insert
local table_remove = table.remove
local table_concat = table.concat

function joyLoL.version()
  return 'JoyLoL minimal Lua version: 0.0.2 (hand coded)'
end

joyLoL.trace = false

-- We need JoyLoL contexts

function joyLoL.newContext()
  return { data = { }; process = { } }
end

function joyLoL.pushData(aCtx, anObj)
  return table_insert(aCtx.data, 1, anObj)
end

local pushData = joyLoL.pushData

function joyLoL.popData(aCtx)
  local result = nil
  if 0 < #aCtx.data then
    result = table_remove(aCtx.data, 1)
  end
  return result
end

local popData = joyLoL.popData

function joyLoL.swapData(aCtx)
  local top    = popData(aCtx)
  local second = popData(aCtx)
  pushData(aCtx, top)
  pushData(aCtx, second)
end

function joyLoL.peekData(aCtx)
  local result = nil
  if 0 < #aCtx.data then
    result = aCtx.data[1]
  end
  return result
end

local peekData = joyLoL.peekData

function joyLoL.peekNData(aCtx, anInt)
  local result = nil
  if 0 < anInt and anInt < (#aCtx.data + 1) then
    result = aCtx.data[anInt]
  end
  return result
end

function joyLoL.pushProcess(aCtx, anObj)
  return table_insert(aCtx.process, 1, anObj)
end

local pushProcess = joyLoL.pushProcess

function joyLoL.pushProcessQuoted(aCtx, aStr)
  return table_insert(aCtx.process, 1, "'"..aStr)
end

local pushProcessQuoted = joyLoL.pushProcessQuoted

function joyLoL.popProcess(aCtx)
  local result = nil
  if 0 < #aCtx.process then
    result = table_remove(aCtx.process, 1)
  end
  return result
end

local popProcess = joyLoL.popProcess

function joyLoL.swapProcess(aCtx)
  local top    = popProcess(aCtx)
  local second = popProcess(aCtx)
  pushProcess(aCtx, top)
  pushProcess(aCtx, second)
end

function joyLoL.peekProcess(aCtx)
  local result = nil
  if 0 < #aCtx.process then
    result = aCtx.process[1]
  end
  return result
end

local peekProcess = joyLoL.peekProcess

function joyLoL.peekNProcess(aCtx, anInt)
  local result = nil
  if 0 < anInt and anInt < (#aCtx.process + 1) then
    result = aCtx.process[anInt]
  end
  return result
end

function joyLoL.reportContext(aCtx)
  local aCtxStr = pp.toString(aCtx)
  pushData(aCtx, aCtxStr)
end

local function popDataStr(aCtx)
  local aStr = popData(aCtx)
  if type(aStr) ~= 'string' then
    reportError("Expected a string\naCtx: "..pp.toString(aCtx)..'\naStr: '..pp.toString(aStr))
  end
  return aStr
end

local function popDataList(aCtx)
  local aList = popData(aCtx)
  if type(aList) ~= 'table' then
    reportError("Expected a list\naCtx: "..pp.toString(aCtx)..'\naList: '..pp.toString(aList))
  end
  return aList
end

local function popDataDict(aCtx)
  local aDict = popData(aCtx)
  if type(aDict) ~= 'table' or 0 < #aDict then
    reportError("Expected a dictionary\naCtx: "..pp.toString(aCtx)..'\naDict: '..pp.toString(aDict))
  end
  return aDict
end

-- We need a JoyLoL LPeg parser which is capable of parsing a simple Lua 
-- string 

function joyLoL.nextWord(aCtx)
  local strToParse = popData(aCtx)
  if type(strToParse) == 'string' then
    local position = 1
    local whiteSpace = strToParse:match('^%s+', position)
    if whiteSpace then position = position + #whiteSpace end
    local aWord = strToParse:match('^[^%s]+', position)
    if not aWord then aWord = "" end
    position = position + #aWord
    local restOfStrToParse = strToParse:sub(position, #strToParse)
    pushData(aCtx, restOfStrToParse)
    pushData(aCtx, aWord)
  else
    pushData(aCtx, "")
    pushData(aCtx, "")
  end
end

local matchingSymbols = {
  ['('] = ')';
  ['{'] = '}';
  ['<'] = '>';
  ['['] = ']'
}

function joyLoL.parseList(aCtx)
  local aWord = popData(aCtx)
  if type(aWord) == 'string' and 0 < #aWord then
    local aList = popProcess(aCtx)
    local closingChar = peekProcess(aCtx)
    if aWord == closingChar then
      -- pop up one level of parseList recursion...
      popProcess(aCtx)       -- remove this closing char
      local prevList = popProcess(aCtx)
      table_insert(prevList, aList)
      pushProcess(aCtx, prevList)
      pushProcess(aCtx, 'parseList')
      pushProcess(aCtx, 'nextWord')
    elseif matchingSymbols[aWord] then
      -- push down one level of parseList recursion...
      pushProcess(aCtx, aList) -- replace aList back on to stack
      pushProcess(aCtx, matchingSymbols[aWord]) -- recursive list end
      pushProcess(aCtx, {})                     -- recursive list to build
      pushProcess(aCtx, 'parseList')
      pushProcess(aCtx, 'nextWord')
    else
      -- add word to list and continue parsing list
      table_insert(aList, aWord)
      pushProcess(aCtx, aList)
      pushProcess(aCtx, 'parseList')
      pushProcess(aCtx, 'nextWord')
    end
  else
    popData(aCtx) -- remove the empty rest of string
    local aList = popProcess(aCtx)
    pushData(aCtx, aList)
    popProcess(aCtx) -- remove the matching symbol
  end
end

function joyLoL.parse(aCtx)
  pushProcess(aCtx, "")  -- closingChar
  pushProcess(aCtx, {}) -- list being built
  pushProcess(aCtx, 'parseList')
  pushProcess(aCtx, 'nextWord')
end

-- We need a simple JoyLoL template engine
-- Our template engine has been inspired by:
--   https://john.nachtimwald.com/2014/08/06/using-lua-as-a-templating-engine/

function joyLoL.renderNextChunk(aCtx)
  local renderedText    = popProcess(aCtx)
  local curTemplate     = popProcess(aCtx)
  local prevJoyLoLChunk = popData(aCtx)
  
  if prevJoyLoLChunk
    and type(prevJoyLoLChunk) == 'string'
    and 0 < #prevJoyLoLChunk then
    table_insert(renderedText, prevJoyLoLChunk)
  end
  
  if type(curTemplate) == 'string' and (0 < #curTemplate) then
    if curTemplate:find('{{') then
      local position  = 1
      local textChunk = curTemplate:match('^.*{{', position)
      if textChunk then 
        local textChunkLen = #textChunk
        textChunk = textChunk:sub(1, textChunkLen-2)
        if 0 < #textChunk then table_insert(renderedText, textChunk) end
        position = position + textChunkLen
      end
      
      local joyLoLChunk = curTemplate:match('^.+}}', position)
      if joyLoLChunk then
        local joyLoLChunkLen = #joyLoLChunk
        joyLoLChunk = joyLoLChunk:sub(1, joyLoLChunkLen-2)
        position = position + joyLoLChunkLen
        curTemplate = curTemplate:sub(position, #curTemplate)
        pushProcess(aCtx, curTemplate)
        pushProcess(aCtx, renderedText)
        pushProcess(aCtx, 'renderNextChunk')
        if not joyLoLChunk:match('^%s*$') then
          pushProcess(aCtx, 'interpret')
          pushProcess(aCtx, 'parse')
          pushData(aCtx, joyLoLChunk)
        else
          pushData(aCtx, "")
        end
      end
    else -- there is no '{{' in the template
      table_insert(renderedText, curTemplate)
      pushData(aCtx, table_concat(renderedText))
    end
  else
    -- nothing more to do...
    pushData(table_concat(renderedText))
  end
end

function joyLoL.render(aCtx)
  local aTemplate = popData(aCtx)
  pushData(aCtx, "")    -- "result" of "initial" joyLoLChunk

  pushProcess(aCtx, aTemplate)
  pushProcess(aCtx, {}) -- result of renderer
  pushProcess(aCtx, 'renderNextChunk')
end

function joyLoL.getTemplate(aCtx)
  local templateName = popDataStr(aCtx)
  pushProcess(aCtx, 'popData')
  pushProcess(aCtx, 'swapData')
  pushProcess(aCtx, 'lookupInDict')
  pushProcessQuoted(aCtx, templateName)
  pushProcess(aCtx, 'lookupInDict')
  pushProcessQuoted(aCtx, 'templates')
end

-- We need to able to manipulate lists

function joyLoL.newList(aCtx)
  pushProcess(aCtx, {})
end

function joyLoL.pushOntoList(aCtx)
  local anItem = popData(aCtx)
  local aList  = popDataList(aCtx)
  table_insert(aList, 1, anItem)
  pushData(aCtx, aList)
end

function joyLoL.popFromList(aCtx)
  local aList = popDataList(aCtx)
  local anItem = table_remove(aList, 1)
  pushData(aCtx, aList)
  pushData(aCtx, anItem)
end

function joyLoL.appendToEndList(aCtx)
  local anItem = popData(aCtx)
  local aList  = popDataList(aCtx)
  table_insert(aList, anItem)
  pushData(aCtx, aList)
end

function joyLoL.removeFromEndList(aCtx)
  local aList = popDataList(aCtx)
  local anItem = table_remove(aList)
  pushData(aCtx, aList)
  pushData(aCtx, anItem)
end

function joyLoL.concatList(aCtx)
  local aSep  = popDataStr(aCtx)
  local aList = popDataList(aCtx)
  local aStr = table_concat(aList, aSep)
  pushData(aCtx, aStr)
end

-- We need to be able to manipulate dictionaries

function joyLoL.newDictionary(aCtx)
  pushProcess(aCtx, {})
end

local function peekDataDict(aCtx)
  local aDict = peekData(aCtx)
  if type(aDict) ~= 'table' or 0 < #aDict then
    reportError("Expected a dictionary\naCtx: "..pp.toString(aCtx)..'\naDict: '..pp.toString(aDict))
  end
  return aDict
end

function joyLoL.addToDict(aCtx)
  local aValue = popData(aCtx)
  local aKey   = popDataStr(aCtx)
  local aDict  = popDataDict(aCtx)
  aDict[aKey] = aValue
  pushData(aCtx, aDict)
end

function joyLoL.deleteFromDict(aCtx)
  local aKey  = popDataStr(aCtx)
  local aDict = popDataDict(aCtx)
  aDict[aKey] = nil
  pushData(aCtx, aDict)
end

function joyLoL.lookupInDict(aCtx)
  local aKey  = popDataStr(aCtx)
  local aDict = peekDataDict(aCtx)
  if aDict[aKey] then
    pushData(aCtx, aDict[aKey])
  else
    pushData(aCtx, "")
  end
end

-- We need to be able to evaluate contexts.

function joyLoL.eval(aCtx, localTrace)
  local trace = localTrace or joyLoL.trace
  while 0 < #aCtx.process do
    if trace then
      print('\n\n-----')
      print(pp.toString(aCtx))
    end
    local aCmd = popProcess(aCtx)
    if type(aCmd) == 'function' then
      if trace then print('calling: ['..pp.toString(aCmd)..']') end
      aCmd(aCtx)
    elseif type(aCmd) == 'string' and aCmd:match("^'") then
      if trace then print('adding quoted string: ['..pp.toString(aCmd)..']') end
      pushData(aCtx, aCmd:sub(2))
    elseif type(joyLoL[aCmd]) == 'function' then
      if trace then print('calling: ['..aCmd..']('..pp.toString(joyLoL[aCmd])..')') end
      joyLoL[aCmd](aCtx)
    else
      if trace then print('adding: ['..pp.toString(aCmd)..']') end
      pushData(aCtx, aCmd)
    end
  end
  if trace then print('finished eval: ['..pp.toString(aCtx.data)..']') end
  return aCtx.data
end

function joyLoL.evalString(aStr)
  local aCtx = newContext()
  pushProcess(aCtx, "interpret")
  pushProcess(aCtx, "parse")
  pushData(aCtx, aStr)
  return evalCtx(aCtx)
end

function joyLoL.interpret(aCtx)
  local joyLoLBlock = popDataList(aCtx)
  for i, aCmd in ipairs(joyLoLBlock) do
    pushProcess(aCtx, aCmd)
  end
end

return joyLoL
\stopMinJoyLoL

\createMinJoyLoLFile%
  {default}%
  {../../../../../build/joyLoLMinLua/t-joylol-templates.lua}%
  {-- A Lua file}

\setLuaCodeStream{prettyPrint}
\startLuaCode
-- A Lua file

local pp = { }

-- nil, boolean, number, string, function, userdata, thread, and table

local tInsert, tSort = table.insert, table.sort

local function compareKeyValues(a, b)
  return (a[1] < b[1])
end

local function toString(anObj, indent)
  local result = ""
  indent = indent or ""
  if type(anObj) == 'nil' then
    result = 'nil'
  elseif type(anObj) == 'boolean' then
    if anObj then result = 'true' else result = 'false' end
  elseif type(anObj) == 'number' then
    result = tostring(anObj)
  elseif type(anObj) == 'string' then
    result = '"'..anObj..'"'
  elseif type(anObj) == 'function' then
    result = tostring(anObj)
  elseif type(anObj) == 'userdata' then
    result = tostring(anObj)
  elseif type(anObj) == 'thread' then
    result = tostring(anObj)
  elseif type(anObj) == 'table' then
    local origIndent = indent
    indent = indent..'  '
    result = '{\n'
    for i, aValue in ipairs(anObj) do
      result = result..indent..toString(aValue, indent)..',\n'
    end
    local theKeyValues = { }
    for aKey, aValue in pairs(anObj) do
      if type(aKey) ~= 'number' or aKey < 1 or #anObj < aKey then
        tInsert(theKeyValues, { toString(aKey), aKey, toString(aValue, indent) })
      end
    end
    tSort(theKeyValues, compareKeyValues)
    for i, aKeyValue in ipairs(theKeyValues) do
      result = result..indent..'['..aKeyValue[1]..'] = '..aKeyValue[3]..',\n'
    end
    result = result..origIndent..'}'
  else
    os.exit('UNKNOWN TYPE')
  end
  return result
end

pp.toString = toString

return pp
\stopLuaCode

\createLuaCodeFile%
  {prettyPrint}%
  {../../../../../build/joyLoLMinLua/t-joylol-prettyPrint.lua}%
  {-- A Lua file}

\stopchapter