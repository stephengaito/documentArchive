% A ConTeXt document [master document: coreContext.tex]

\chapter[title=Preamble]

\prependMkIVCode{default}
\startMkIVCode
%D \module
%D   [     file=t-joylol,
%D      version=2017.05.10,
%D        title=\CONTEXT\ User module,
%D     subtitle=The JoyLoL prgramming language for \ConTeXt\,
%D       author=Stephen Gaito,
%D         date=\currentdate,
%D    copyright=PerceptiSys Ltd (Stephen Gaito),
%D        email=stephen@perceptisys.co.uk,
%D      license=MIT License]

%C Copyright (C) 2017 PerceptiSys Ltd (Stephen Gaito)
%C
%C Permission is hereby granted, free of charge, to any person obtaining a
%C copy of this software and associated documentation files (the
%C "Software"), to deal in the Software without restriction, including
%C without limitation the rights to use, copy, modify, merge, publish,
%C distribute, sublicense, and/or sell copies of the Software, and to
%C permit persons to whom the Software is furnished to do so, subject to
%C the following conditions:
%C
%C The above copyright notice and this permission notice shall be included
%C in all copies or substantial portions of the Software.
%C
%C THE SOFTWARE IS PROVIDED "AS IS", WITHOUT WARRANTY OF ANY KIND, EXPRESS
%C OR IMPLIED, INCLUDING BUT NOT LIMITED TO THE WARRANTIES OF
%C MERCHANTABILITY, FITNESS FOR A PARTICULAR PURPOSE AND NONINFRINGEMENT.
%C IN NO EVENT SHALL THE AUTHORS OR COPYRIGHT HOLDERS BE LIABLE FOR ANY
%C CLAIM, DAMAGES OR OTHER LIABILITY, WHETHER IN AN ACTION OF CONTRACT,
%C TORT OR OTHERWISE, ARISING FROM, OUT OF OR IN CONNECTION WITH THE
%C SOFTWARE OR THE USE OR OTHER DEALINGS IN THE SOFTWARE.

% begin info
%
% title   : JoyLoL CoAlgebra definitions
% comment : Provides structured document and code generation
% status  : under development, mkiv only
%
% end info

\unprotect

\ctxloadluafile{t-joylol}
\stopMkIVCode

\startMkIVCode
\protect \endinput
\stopMkIVCode

\prependLuaCode{default}
\startLuaCode
-- This is the lua code associated with t-joylol.mkiv

if not modules then modules = { } end modules ['t-joylol'] = {
    version   = 1.000,
    comment   = "joylol programming language - lua",
    author    = "PerceptiSys Ltd (Stephen Gaito)",
    copyright = "PerceptiSys Ltd (Stephen Gaito)",
    license   = "MIT License"
}

thirddata          = thirddata        or {}
thirddata.joylol   = thirddata.joylol or {}
local joylol       = thirddata.joylol
joylol.options     = joylol.options or {}
local options      = joylol.options

options.verbose    =
  options.verbose    or false
options.configFile =
  options.configFile or 'config'
options.userPath   =
  options.userPath   or os.getenv('HOME')..'/.joylol'
options.localPath  =
  options.localPath  or '/usr/local/lib/joylol'
options.systemPath =
  options.systemPath or '/usr/lib/joylol'

local tInsert = table.insert
local tConcat = table.concat
local tRemove = table.remove
local tSort   = table.sort
local sFmt    = string.format
local sMatch  = string.match
local toStr   = tostring

-- **Problem**: we can not assume that a user *has* a compiled and working 
-- C based JoyLoL. This is the "Bootstrapping (Compiler)" problem (see 
-- Wikipedia). We solve this problem by writing a minimal joyLoL 
-- interpreter in Lua. 

-- SO we first check to see if the joyLoL (C shared libraries) exists and 
-- can be required, if it can not be loaded, we load the joyLoLMinLua 
-- version instead. 

-- The following conditional require is adapted from: shuva's answer to 
--  "How to check if a module exists in Lua?"
-- see: http://stackoverflow.com/a/22686090

-- local hasJoyLoL,joyLoL = pcall(require, "joyLoL/joyLoL")
-- if not hasJoyLoL then
-- interfaces.writestatus("joyLoL",
--     "Could NOT load joyLoL... loading mininal Lua version instead.")
--  joyLoL = require 'joyLoLMinLua/joyLoL'
-- end
\stopLuaCode