% A ConTeXt document [master document: coreLua.tex]

\chapter[title=Lua main]

\prependLuaCode{default}

\component gitVersion-lua

\startLuaCode
-- joylol command line interpreter 

-- Start by dealing with the options

joylol = { }
joylol.options = { }
local options = joylol.options

-- default options
options.verbose    = false
options.debug      = false
options.quiet      = false
options.tracing    = false
options.loadFiles  = { }
options.loadPaths  = { }  
options.configFile = 'config'
options.userPath   = os.getenv('HOME')..'/.joylol'
options.localPath  = '/usr/local/lib/joylol'
options.systemPath = '/usr/lib/joylol'

helpText = {
  "usage: joylol [options] [files to load]",
  "",
  "options: ",
  " -h --help        prints this help text and exits",
  " -i --ignore      ignores default configuration file (~.joylol)",
  " -l --load <file> loads the file <file>",
  " -p --path <path> adds <path> to the list of load paths",
  " -q --quiet       toggles verbose off",
  " -v --verbose     toggles verbose on",
  " -t --trace       toggles tracing on",
  " -d --debug       turns on internal debugging",
  " -u --userPath    sets the user load path",
  " -L --localPath   sets the local load path",
  " -s --systemPath  sets the system load path",
  "",
  "files to load:",
  "  Any remaining options are treated as files to be loaded.",
  "  If there are no remaining options, joylol enters the read,",
  "  eval, print loop."
}

while(0 < #arg) do
  anArg = table.remove(arg, 1)
  if anArg:match('-h') or anArg:match('--help') then
    print(table.concat(helpText, '\n'))
    os.exit(0);
  elseif anArg:match('-c') or anArg:match('--config') then
    optArg = table.remove(arg, 1)
    options.configFile = optArg
  elseif anArg:match('-i') or anArg:match('--ignore') then
    options.configFile = nil
  elseif anArg:match('-u') or anArg:match('--user') then
    optArg = table.remove(arg, 1)
    options.userPath = optArg
  elseif anArg:match('-L') or anArg:match('--local') then
    optArg = table.remove(arg, 1)
    options.localPath = optArg
  elseif anArg:match('-s') or anArg:match('--system') then
    optArg = table.remove(arg, 1)
    options.systemPath = optArg
  elseif anArg:match('-l') or anArg:match('--load') then
    optArg = table.remove(arg, 1)
    table.insert(options.loadFiles, optArg)
  elseif anArg:match('-p') or anArg:match('--path') then
    optArg = table.remove(arg, 1)
    table.insert(options.loadPaths, optArg)
  elseif anArg:match('-q') or anArg:match('--quiet') then
    options.verbose = false
    options.debug   = false
    options.tracing = false
    options.quiet   = true
  elseif anArg:match('-v') or anArg:match('--verbose') then
    options.verbose = true
  elseif anArg:match('-d') or anArg:match('--debug') then
    options.debug = true
  elseif anArg:match('-t') or anArg:match('--trace') then
    options.tracing = true
  else
    --optArg = table.remove(arg, 1)
    table.insert(options.loadFiles, anArg)
  end
end

-- Now add the standard joylol CoAlg locations to
-- the Lua search paths 

local joylolPaths = {
  options.userPath..'/?.lua',
  options.localPath..'/?.lua',
  options.systemPath..'/?.lua',
  package.path
}
package.path = table.concat(joylolPaths, ';')

local joylolCPaths = {
  options.userPath..'/?.so',
  options.localPath..'/?.so',
  options.systemPath..'/?.so',
  package.path
}
package.cpath = table.concat(joylolCPaths, ';')

-- Now load joylol

if options.verbose then print('loading [joylol.core.lua]') end
joylol = require 'joylol.core.lua'
if options.verbose then print('loaded [joylol.core.lua]\n') end

-- replace the options

joylol.options = options

--  deal with loadPaths, configFile, and loadFiles

for i, aPath in ipairs(options.loadPaths) do
  if options.verbose then print('adding loadPath ['..aPath..']\n') end
  joylol.pushLoadPath(aPath)
end

if (options.configFile) then
  joylol.loadFile(options.configFile)
end

for i, aFile in ipairs(options.loadFiles) do
  if options.verbose then print('loading ['..aFile..']\n') end
  joylol.loadFile(aFile)
  if options.verbose then print('loaded ['..aFile..']\n') end
end

if #options.loadFiles < 1 then
  -- Finally run the REPL
  joylol.core.lua.runREPL();
end

\stopLuaCode