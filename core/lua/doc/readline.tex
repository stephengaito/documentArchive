% A ConTeXt document [master document: coreLua.tex]

\section[title= Readline code]
\setCHeaderStream{public}

\startCCode
// Texts are a collection of characters, which are used by the Parser
// to extract successive symbols.
//
// Texts are created on one of three backing suppliers of characters:
// 1. a single string
// 2. a NULL terminated array of strings
// 3. an external file
// 4. a readline interaction with a user
//
// In all four cases, the Parser's nextSymbol method requests successive
// **lines** of characters (deliminated by new-line-characters).
//
// It is critical, for correct interaction with the user via readline,
// that the initial line is NOT obtained until actually requested by
// the parser's nextSymbol method.
//
// It is also critical that once completed, none of the sources, get
// asked for subsequent lines.
//
// When the text has been completed, the nextLine function ensures
// that aText->curLine is NULL.
\stopCCode

\startCCode
static void clearReadlinePrompt(TextObj* aText) {
  aText->curPrompt = aText->newPrompt;
}

static void setReadlinePrompts(TextObj* aText,
                        const char* newPrompt,
                        const char* continuePrompt
) {
  if (newPrompt) aText->newPrompt = newPrompt;
  else           aText->newPrompt = ">";

  if (continuePrompt) aText->continuePrompt = continuePrompt;
  else                aText->continuePrompt = ":";

  clearReadlinePrompt(aText);
}
\stopCCode

\startCCode
static void saveReadlineHistory(TextObj* aText) {
  write_history(".joyLoL-history");
}
\stopCCode

\startCCode
static void nextLineFromReadline(TextObj* aText) {
  assert(aText);
  assert(aText->jInterp);
  DEBUG(aText->jInterp, "->nextLineFromReadline %s\n", "");
  if (!aText) return; // there is nothing we can do!

  if (aText->completed) {
    // we have reached the end of the interaction with the user
    aText->completed = TRUE;
    aText->curLine   = NULL;
    aText->curChar   = NULL;
    aText->lastChar  = NULL;
    return;
  }

  // readline returns alloc'ed memory so we free it here.
  if (aText->curLine) free((void*)aText->curLine);

  aText->curLine  = NULL;
  aText->curChar  = NULL;
  aText->lastChar = NULL;

  aText->curLine = readline (aText->curPrompt);

  if (!aText->curLine) {
    aText->completed = TRUE;
    return;
  }

  if (*aText->curLine) {
    history_set_pos(0); // start at the beginning
    if (0 <= history_search (aText->curLine, -1)) { // search backwards
      remove_history(where_history());
    } else if (0 <= history_search (aText->curLine, 1)) { // search forwards
      remove_history(where_history());
    }
    history_set_pos(0);
    add_history (aText->curLine);
  }

  aText->curChar  = aText->curLine;
  aText->lastChar = aText->curChar + strlen(aText->curChar);

  aText->curPrompt = aText->continuePrompt;
}
\stopCCode

\startCCode
static TextObj* currentReadLineText = NULL;
\stopCCode

\startCCode
static char* dictionaryWalker(const char* text, int state) {
  if (!currentReadLineText) return NULL;
  if (!state) {

    assert(currentReadLineText);
    assert(currentReadLineText->jInterp);

    currentReadLineText->curNode =
      findLUBSymbol(currentReadLineText->jInterp->dict, text);
    DEBUG(currentReadLineText->jInterp,
          "dictionaryWalker-start %p\n",
          currentReadLineText->curNode);
  }
  DictNodeObj* curNode = currentReadLineText->curNode;

  if (!curNode) return NULL;

  if (strncmp(curNode->symbol, text, strlen(text)) == 0) {
    DEBUG(currentReadLineText->jInterp,
          "dictionaryWalker %p {%s}[%s]\n",
          curNode, text, curNode->symbol);
    currentReadLineText->curNode = curNode->next;
    return strdup(curNode->symbol);
  }

  currentReadLineText->curNode = NULL;
  return NULL;
}

static char** dictionaryCompletion(const char* text, int start, int end) {
 return rl_completion_matches(text, dictionaryWalker);
}
\stopCCode

\startCCode
static TextObj* createTextFromReadline(JoyLoLInterp *jInterp) {
  
  assert(jInterp);
  DEBUG(jInterp, "createTextFromReadline %p\n", jInterp);
  
  TextObj* aText = (TextObj*)newObject(jInterp, TextsTag);
  assert(aText);
  //
  // readline specific initializations
  //
  DEBUG(jInterp, "starting readline initialization %s\n", "");
  using_history();
  read_history(".joyLoL-history");
  rl_readline_name = "joyLoL";
  rl_attempted_completion_function = dictionaryCompletion;
  setReadlinePrompts(aText, NULL, NULL);
  aText->curNode = NULL;
  aText->curCompletionText = NULL;
  aText->curCompletionLen  = 0;
  aText->nextLine = nextLineFromReadline;
  //
  // general initialization
  //

  aText->jInterp    = jInterp;
  aText->completed  = FALSE;
  aText->sym        = NULL;
  aText->curLine    = NULL;
  aText->curChar    = NULL;
  aText->lastChar   = NULL;
  //
  aText->inputFile = NULL;
  //
  aText->textLines  = NULL;
  aText->curLineNum = 0;

  assert(!currentReadLineText); // there should not be more than one
  currentReadLineText = aText;

  return aText;
}
\stopCCode

\startCHeader
extern void runREPLInContext(
  ContextObj* aCtx
);
\stopCHeader

\startCCode
void runREPLInContext(
  ContextObj* aCtx
) {
  assert(aCtx);
  JoyLoLInterp *jInterp = aCtx->jInterp;
  assert(jInterp);
  
//  if (aCtx->verbose) {
    getGitVersionInto(gitVersionKeyValues, "commitShortHash", commitHash);
    getGitVersionInto(gitVersionKeyValues, "commitDate",      commitDate);
    StringBufferObj *aStrBuf = newStringBuffer(jInterp);
    strBufPrintf(jInterp, aStrBuf, "Welcome to JoyLoL v0.1 ( %s ; %s )\n",
      commitHash,
      commitDate
    );
    jInterp->writeStdOut(jInterp, getCString(jInterp, aStrBuf));
    strBufClose(jInterp, aStrBuf);
//  }
  aCtx->showStack = TRUE;
  TextObj* aText = createTextFromReadline(jInterp);
  assert(aText);
  
  while(TRUE) {
    DEBUG(jInterp, "runREPLInContext %p reading\n", aCtx);
    JObj* aLoL = parseOneSymbol(aText);
    DEBUG(jInterp, "runREPLInContext %p read %p\n", aCtx, aLoL);
    //
    if (!aLoL) break;
    //
    evalCommandInContext(aCtx, aLoL);
    assert(!aCtx->process);
    //
    if (aCtx->showStack) {
      DEBUG(jInterp, "runREPLInContext %p printing data %p\n", aCtx, aCtx->data);
      StringBufferObj *aStrBuf = newStringBuffer(jInterp);
      strBufPrintf(jInterp, aStrBuf, ">>");
      printLoL(jInterp, aStrBuf, aCtx->data);
      strBufPrintf(jInterp, aStrBuf, "\n");
      jInterp->writeStdOut(jInterp, getCString(jInterp, aStrBuf));
      strBufClose(jInterp, aStrBuf);
      DEBUG(jInterp, "runREPLInContext %p printed data %p\n", aCtx, aCtx->data);
    }
    //
    reportException(aCtx); // report last exception if raised
  }
  saveReadlineHistory(aText);
  jInterp->writeStdOut(jInterp, "\n");
  DEBUG(jInterp, "runREPLInContext %p COMPLETED\n", aCtx);
}
\stopCCode

\starttyping
static void showVersionsAP(Context* aCtx) {
  fprintf(stdout, "\n");
  reportMainVersions(stdout);
  fprintf(stdout, "\n");
  reportLibVersions(stdout);
  fprintf(stdout, "\n");
}
\stoptyping