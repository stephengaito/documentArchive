% A ConTeXt document [master document: sortingJoyLoL.tex]

\chapter[title=Preface]

This document is one of a number of \quote{finger pieces}, each designed 
to explore a small part of the overall problem. \emph{This} finger piece 
explores the syntax and semantics of \quote{Proof} in JoyLoL. 

When using JoyLoL as a foundation for mathematics, we assert there is no 
external logic, and hence no externally valid proof theory. However we are 
still interested in \quote{proving} that a given program fragment actually 
computes what it purports to compute. 

This finger piece explores the notation and computations required to prove 
a given program fragment correct. We do this using a combination of 
Dynamic Logic, \cite{harelKozenTiuryn2000dynamicLogic}, and Modal 
$\mu$-Logic, \cite{demriGorankoLange2016temporalLogics}. The dynamic logic 
provides a focus upon the structure of the program fragments, while the 
modal $\mu$-logic provides a focus upon the dual algebraic and coalgebraic 
structure of the computation, via least and greatest fixed points 
respectively. 

We do this while exploring the relatively \quote{simple} problem of 
\quote{sorting} a list of \quote{comparable} \quote{things}. The problem 
of sorting is well developed with many different algorithms. This allows 
us to explore how different specifications distinguish between these 
different algorithms and their implementations. In particular we will 
explore Moschovakis' definition of algorithm versus implementation, 
\cite{moschovakis2012foundationsAlgorithms}.

All of the \quote{classical} algorithms for sorting, assume that the list 
is a finite, algebraic, structure. We will also be interested in how each 
sorting algorithm performs with infinite, coalgebraic, processes. 
\emph{Can processes be sorted?} 

