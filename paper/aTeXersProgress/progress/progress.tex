% A ConTeXt file

\usemodule[joylol]

\setupinteraction
  [state=start,
   color=green,
   style=bold]
% make chapter, section bookmarks visible when opening document
\placebookmarks[chapter,section,subsection][chapter,section]
\setupinteractionscreen[option=bookmark]

\startcomponent atest-file

%<a rel="license" href="http://creativecommons.org/licenses/by-sa/4.0/"><img 
%alt="Creative Commons License" style="border-width:0" 
%src="https://i.creativecommons.org/l/by-sa/4.0/88x31.png" /></a><br /><span 
%xmlns:dct="http://purl.org/dc/terms/" href="http://purl.org/dc/dcmitype/Text" 
%property="dct:title" rel="dct:type">A TeXer's Progress</span> by <a 
%xmlns:cc="http://creativecommons.org/ns#" 
%href="https://github.com/stephengaito" property="cc:attributionName" 
%rel="cc:attributionURL">Stephen Gaito</a> is licensed under a <a 
%rel="license" href="http://creativecommons.org/licenses/by-sa/4.0/">Creative 
%Commons Attribution-ShareAlike 4.0 International License</a>.<br />Based on a 
%work at <a xmlns:dct="http://purl.org/dc/terms/" 
%href="https://github.com/stephengaito/aTeXersProgress" 
%rel="dct:source">https://github.com/stephengaito/aTeXersProgress</a>. 

%\starttext
%\startdocument

\doifmode{book}{\startfrontmatter}
\placecontent

\doifmode{book}{
\stopfrontmatter
\startbodymatter
}

\startsection[title={Introduction}]

A TeXer's Progress is a personal attempt, of an existing \LaTeX\ user, to 
understand how to write mathematical texts using \ConTeXt. We will produce 
ten distinct documents. The first one, this document, will highlight the 
important points of this journey. The second, and third documents will be a 
mathematical exposition of the chaotic dynamics of the H\'enon, Lozi and 
Two-shift maps, written in \LaTeX\ as a long article and a book. The fourth 
and fifth documents will be the same mathematical exposition redone using 
\ConTeXt, again in both a long article and a book format. Finally, each of 
these previous documents will be provided in both PDF and XHTML formats. 

\stopsection

\startsection[title={Where to put things}]

The first and most important problem to solve is where to put things. 
\ConTeXt\ has four main types of files, those which define Projects, 
Products, Components and Environments. This abundance of structure is very 
well suited to writing large complex multi-book projects. However, for our 
purposes we will only make use of Product, Component and Environment files. 

Each article will consist of a couple of major sections, each of which will 
be written in their own file. The main article will then correspond to a 
\ConTeXt\ Product which specifies which sections make up the article. Each 
major section will correspond to a \ConTeXt\ Component. Environments are used 
to specify one or more article styles to be used by \ConTeXt\ when it 
lays-out the whole article. 

\ConTeXt's ability to manage complex documents is reflected in its ability to 
manage files over multiple levels of directories. If a file, such as a 
Project, Product or Environment file is not found in the "current directory", 
\ConTeXt\ will search for this file in successive \emph{parent} directories 
until the search stops at the root of the file-system. 

For our purposes we will use a simple two level structure. We will place all 
common files, such as the \ConTeXt\ Environment files, in the top level 
directory. All of the Product and Component files for given article will be 
placed in the same subdirectory. 

\stopsection

\startsection[title={A question of style}]

The key idea behind both \LaTeX\ and \ConTeXt\ is that content and format 
should be kept separate. A key difference between \LaTeX\ and \ConTeXt\ is 
the degree of difficulty for a writer to make small format changes should 
they be needed. \LaTeX\ hides most formatting parameters deeply inside class 
files which are not meant to be altered by an average \LaTeX\ user. \ConTeXt, 
on the other hand, provides a very simple default format and expects writers 
to specify much more of the format decisions in \ConTeXt\ Environment files. 

As we write we will develop a common collection of formats specified in 
\ConTeXt\ Environment files located in the top level directory. One or more of 
these environment files can then be used in each article. This provides a 
useful separation between the content of the mathematical article and the 
layout style of the article's format. It also allows the writer to provide 
local modifications to suit their own needs. 

\stopsection

\startsection[title={Using Lua}]

One of the most important benefits of \ConTeXt\ is the integration of Lua. 

\stopsection

\startsection[title={ToDo}]

\startitemize
  \item using Modules?
  \item Citations/Bibliography
  \item Titles/Author/License/GitVersion/Date
  \item ...
\stopitemize

\stopsection

\doifmode{book}{
\stopbodymatter
\startbackmatter
}

\startsection[title={Bibliography}]

\placelistofpublications

\stopsection

\doifmode{book}{\stopbackmatter}

%\stopdocument
%\stoptext

\stopcomponent