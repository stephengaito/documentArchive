% A ConTeXt document [master document: realAnalysis.tex]

\chapter[title=Computing The Universe: Fans]

%%%%%%%%%%%%%%%%%%%%%%%%%%%%%%%%%%%%%%%%%%%%%%%%%%%%%%%%%%%%%%%%%%%%%%%
\placefigure[here][fig:zeroOneTree]
{A tree of \quote{zero}s and \quote{one}s}
\bgroup\startMPcode{commDiag}

setupCommDiags ;

addObject(1, 16, "()") ;

addObject(2, 8,  "0") ;
addObject(2, 24, "1") ;

addObject(3, 4,  "00") ;
addObject(3, 12,  "01") ;
addObject(3, 20, "10") ;
addObject(3, 28, "11") ;

addObject(4, 2,  "000") ;
addObject(4, 6,  "001") ;
addObject(4, 10, "010") ;
addObject(4, 14, "011") ;
addObject(4, 18, "100") ;
addObject(4, 22, "101") ;
addObject(4, 26, "110") ;
addObject(4, 30, "111") ;

addObject(5, 1,  "");
addObject(5, 3,  "");
addObject(5, 5,  "");
addObject(5, 7,  "");
addObject(5, 9,  "");
addObject(5, 11, "");
addObject(5, 13, "");
addObject(5, 15, "");
addObject(5, 17, "");
addObject(5, 19, "");
addObject(5, 21, "");
addObject(5, 23, "");
addObject(5, 25, "");
addObject(5, 27, "");
addObject(5, 29, "");
addObject(5, 31, "");

drawRegularObjects(0.75cm, 0.4cm) ;

addArrow(1,16,  2,8,   "-", 0)()()("", 0, top) ;
addArrow(1,16,  2,24,  "-", 0)()()("", 0, top) ;

addArrow(2,8,  3,4,   "-", 0)()()("", 0, top) ;
addArrow(2,8,  3,12,  "-", 0)()()("", 0, top) ;
addArrow(2,24, 3,20,  "-", 0)()()("", 0, top) ;
addArrow(2,24, 3,28,  "-", 0)()()("", 0, top) ;

addArrow(3,4,  4,2,   "-", 0)()()("", 0, top) ;
addArrow(3,4,  4,6,   "-", 0)()()("", 0, top) ;
addArrow(3,12, 4,10,  "-", 0)()()("", 0, top) ;
addArrow(3,12, 4,14,  "-", 0)()()("", 0, top) ;
addArrow(3,20, 4,18,  "-", 0)()()("", 0, top) ;
addArrow(3,20, 4,22,  "-", 0)()()("", 0, top) ;
addArrow(3,28, 4,26,  "-", 0)()()("", 0, top) ;
addArrow(3,28, 4,30,  "-", 0)()()("", 0, top) ;

addArrow(4,2,  5,1,  "-", 0)()()("", 0, top) ;
addArrow(4,2,  5,3,  "-", 0)()()("", 0, top) ;
addArrow(4,6,  5,5,  "-", 0)()()("", 0, top) ;
addArrow(4,6,  5,7,  "-", 0)()()("", 0, top) ;
addArrow(4,10, 5,9,  "-", 0)()()("", 0, top) ;
addArrow(4,10, 5,11, "-", 0)()()("", 0, top) ;
addArrow(4,14, 5,13, "-", 0)()()("", 0, top) ;
addArrow(4,14, 5,15, "-", 0)()()("", 0, top) ;
addArrow(4,18, 5,17, "-", 0)()()("", 0, top) ;
addArrow(4,18, 5,19, "-", 0)()()("", 0, top) ;
addArrow(4,22, 5,21, "-", 0)()()("", 0, top) ;
addArrow(4,22, 5,23, "-", 0)()()("", 0, top) ;
addArrow(4,26, 5,25, "-", 0)()()("", 0, top) ;
addArrow(4,26, 5,27, "-", 0)()()("", 0, top) ;
addArrow(4,30, 5,29, "-", 0)()()("", 0, top) ;
addArrow(4,30, 5,31, "-", 0)()()("", 0, top) ;

\stopMPcode\egroup
%%%%%%%%%%%%%%%%%%%%%%%%%%%%%%%%%%%%%%%%%%%%%%%%%%%%%%%%%%%%%%%%%%%%%%%

Our objective in this chapter is to show that a \emph{process} can 
\quote{compute} the \quote{collection} of all zero-one sequences. To do 
this we exhibit a \joylol\ program, \type{zeroOne}, which computes this 
collection. 

Since this \joylol\ program is a \emph{process}, its pre and post 
conditions are very simple: \type{true} and \type{false} respectively. The 
\type{true} precondition implies that \type{zeroOne} has \emph{no} 
particular preconditions. The \type{false} postcondition implies that 
\type{zeroOne} \emph{is} a \emph{process} which never stops, since if it 
did, what ever it had computed in a finite number of steps would have been 
incorrect (\type{false}). 

The really important condition for a process, such as \type{zeroOne}, is 
its \emph{invariant}. This process invariant, will be intimately related 
to the particular \emph{way} in which we \emph{structure} the process's 
computation. For the \type{zeroOne} process, we will use a recursive 
structure which... 

Since we are considering the construction of an infinite structure, we can 
\emph{not} assume the use of a finite RAM machine, we \emph{must} use a 
purely stack based approach. We \emph{can} however use a finite collection 
of \quote{local} variables to hold intermediate sub-stack structures. 

Since we are computing a \emph{process}, if we use a recursive 
computation, then the recursion code can not depend upon any 
\emph{computation} happening in its continuation. This is because any 
continuation which the recursive code \quote{uses} will never be executed. 
\ToDo{Not true! We do use continuations below. For each level the 
continuations are completed except for the \quote{last} \type{zeroOne} 
(repetition) itself.} 

The question is how do we want to structure the evolving \emph{\joylol\ 
structure}? We want to \emph{construct} every \emph{finite} zero-one 
sequence of every \emph{length}. To do this we build a \quote{tree} (a 
list of lists) whose every node consists of a list of three items. The 
first item is the fully constructed finite list of zeros and ones, the 
\quote{current sequence}. The second item is the (sub) tree of zero-one 
sequences which all \emph{begin} with the current sequence followed by a 
\emph{zero}. The third item is the (sub) tree of zero-one sequences which 
all \emph{being} with the current sequence followed by a \emph{one}. 

Typically when recursing over a tree, we can compute in a depth-first or 
breadth-first manner. However since we are computing a \emph{process} we 
can not use a depth-first approach since we would never complete the first 
sub-tree at each level. This means that we must use a breadth-first 
approach. However, at each step, we can only compute one more level of 
each sub-tree. 

This requires us to keep \quote{back-track} points so that when the 
addition of one additional level of a given sub-tree is complete, we know 
where to \quote{back-track to} in order to start building the next level. 

As back-track point, we will partially dismantle, onto the process stack, 
the already completed tree which is being built on the data stack. 

The invariant is that at the end of a complete level of a given sub-tree, 
we have a complete (sub) tree. 

We begin by defining a local variable, \type{seq}, in the local context. 
To declare this variable, we need to provide the invariant which all 
values will satisfy. To do this, we need to agree on what a \quote{zero} 
and a \quote{one} \emph{is}. Since a null list is \type{()}, we will 
define a \quote{zero} as \type{(())} and a \quote{one} as \type{(()())} 
respectively. This means that a \quote{zero-one} \emph{sequence} is a list 
of \quote{zero}s and \quote{one}s. For example the sequence of 
\type{010110} would be denoted: \type{( (()) (()()) (()) (()()) (()()) 
(()) )}. 

In \joylol\ we define a \quote{zero} as 

\starttyping
\startJoylolCode

\stopJoylolCode
\stoptyping

So our algorithm is:

The process stack will be an alternation between partial trees, 
code-continuations. We will have the following six code-continuations:

\startitemize

\item The \type{buildSubTree} method pushes both itself and then the 
\type{goLeft} method onto the process stack. 

\starttyping
\startJoylolCode
buildSubTree()() -> ()(goLeft buildSubTree)
\stopJoylolCode
\stoptyping

\item The \type{buildOneLevel} method takes the top of the data stack and 
places it into the \type{seq} local context variable. It then builds a 
one-level sub-tree using the value of the \type{seq} variable as a prefix 
and leaves the construction on the top of the data stack. 

\starttyping
\startJoylolCode
buildOneLevel(seq:isSeqZO)() -> (
  (((seq zero) () ()) ((seq one) () ()))
)()
\stopJoylolCode
\stoptyping

\item The \type{goLeft} method takes the top of the data stack and places 
it in the \type{seq} local context variable and then checks the top of the 
stack. 

If the top of the stack is the empty list then the \type{goLeft} method 
places a copy of the first item of the top of the data stack onto the new 
top of the data stack. It then places \type{buildOneLevel}, \type{goRight} 
and then \type{joinLeftSubTree} onto the process stack. 

If the top of the data stack is \emph{not} the empty list, then the 
\type{goLeft} method takes the second item from the list at the top of the 
data stack and places this item on the top of the data stack, and then 
places \type{goLeft} and then \type{joinLeftSubTree} onto the top of the 
process stack. 

\starttyping
\startJoylolCode
goLeft(seq:isSeqZO top:isTree)() {
  if (top isEmpty) {
    -> (seq)( buildOneLevel gotRight joinLeftSubTree )
  } else {
    -> ((cadr top) top)( goLeft joinLeftSubTree )
  }
}
\stopJoylolCode
\stoptyping

\item The \type{joinLeftSubTree} method takes the top of the data stack 
and places it in the second item of the second object on the data stack. 

\starttyping
\startJoylolCode
joinLeftSubTree(leftTree:isTree top:isTree)()
-> ( ((car top)(leftTree)(cadd top)))()
\stopJoylolCode
\stoptyping

\item The \type{goRight} method takes the top of the data stack and places 
it in the \type{seq} local context variable and then checks the top of the 
stack. 

If the top of the stack is the empty list then the \type{goRight} method 
places a copy of the first item of the top of the data stack onto the new 
top of the data stack. It then places \type{buildOneLevel} and then 
\type{joinRightSubTree} on the process stack. 

If the top of the stack is \emph{not} the empty list, then the 
\type{goRight} method takes the third item from the list at the top of the 
data stack and places this item on the top of the data stack and then 
places \type{goLeft} and then \type{joinRightSubTree} onto the top of the 
process stack. 

\starttyping
\startJoylolCode
goRight(seq:isSeqZO top:isTree)() {
  if (top isEmpty) {
    -> (seq)( buildOnLevel joinRightSubTree)
  } else {
    -> ((cadd top) top)( goLeft joinRightSubTree)
  }
}
\stopJoylolCode
\stoptyping

\item The \type{joinRightSubTree} method takes the top of the data stack 
and places it in the third item of the second object on the top of the 
data stack. 

\starttyping
\startJoylolCode
joinRightSubTree (rightTree:isTree top:isTree)()
-> ( ((car top)(cadr top)(rightTree)))()
\stopJoylolCode
\stoptyping

\item Finally, the \type{buildZeroOneTree} method places 
\type{buildSubTree} onto the top of the process stack. 

\starttyping
\startJoylolCode
buildZeroOneTree()() -> ()( buildSubTree )
\stopJoylolCode
\stoptyping

\stopitemize

Having built the fan of all \quote{zero-one} sequences, we now show how to 
alter this algorithm to build the fan of all sequences of natural numbers. 

