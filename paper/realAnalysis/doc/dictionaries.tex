% A ConTeXt document [master document: realAnalysis.tex]

\chapter[title=Dictionaries: The work horses of Set and Category theory]

Dictionaries are structures which provide the ability to \quote{store} a 
collection of \quote{key}-\quote{value} pairs for later \quote{retrieval}. 
They effectively implement a form of \quote{Random access memory}, for any 
given \quote{key} you can \quote{retrieve} its associated \quote{value}. 
For computationally-bounded collections, you can also determine whether or 
not the \quote{key}-\quote{value} pair is already stored in the 
collection. For computationally-unbounded collections, this is no longer 
possible. \TODO{provide correct computational terminology}.

We will implement three types of dictionary.

\startitemize[n]

\item {\bf Linear List based dictionary}

List based dictionaries are the simplest to implement, but have only 
\quote{linear}, $O(n)$, performance. 

\item {\bf Binary Tree based dictionary}

We can implement either balanced or unbalanced trees. The balancing 
operation is only guaranteed to complete for computationally-bounded 
collections. Unbalanced trees have worst-case performance which approaches 
a simple list. Balanced trees can maintain a worst-case performance of 
$O(\ln(n))$. Unfortunately only computationally-bounded collections can be 
balanced in a finite amount of time. There will always be 
computationally-unbounded collections whose worst-case balancing time is 
infinite, and so never halts and/or whose worst-case retrieval time 
degrades to $O(n)$. This means that in the worst-case 
computationally-unbounded collections Tree based dictionaries are not 
really much better than List based dictionaries. 

There are essentially two types of \emph{balanced} Tree based dictionary, 
dependent upon the information they keep to determine how to rebalance. 
Red-Black trees, \cite{cormenLeisersonRivestStein2009introAlgorithms} 
Chapter 13, which keep only one bit of information and are balanced 
\quote{locally}, and AVL trees, 
\cite{cormenLeisersonRivestStein2009introAlgorithms} Problem 13-3, which 
keep the depth of the tree \quote{below} a node. For unbounded 
collections, this depth will potentially be infinite for each node. 

\item {\bf Hash based dictionary}

Hash based dictionaries can only be implemented for computationally-bounded 
collections of key-value pairs. 

\stopitemize

Each type has very different performance. More importantly, only the List 
and Tree based dictionaries are suitable for processes. Hash based 
dictionaries are only suitable for bounded collections of objects, 
otherwise known as finite sets. This distinction between bounded and 
unbounded collections and their associated types of dictionaries is 
central to all of theoretical mathematics. 

For bounded (computationally-finite) collections, we can implement 
dictionaries which only have one entry per key. For computationally 
unbounded (infinite processes) collections, we can only implement 
dictionaries which have (potentially) multiple entries per key. In terms 
of set theory, sets can only ever be bounded collections, unbounded 
collections are represented by \emph{multi}-sets. 

For any dictionary we will implement the following methods:

\startitemize

\item \type{insert}: Insert a new key-value pair. For 
computationally-bounded collections we guarantee that a given key will 
only exist once in the dictionary, and hence any existing value 
corresponding to the given key will be \emph{replaced}. For 
computationally-unbounded collections, this guarantee can no longer be 
maintained. This means that for computationally-unbounded collections 
there may be multiple key-value pairs with the same keys but different 
values \emph{simultaneously}. 

\item \type{getValueFor}: Search for and retrieve the value for a given 
key. For computationally-bounded collections this method is guaranteed to 
complete and hence returns \type{nil} if the dictionary does not contain 
the key. For computationally-unbounded collections this method can no 
longer guarantee to complete (halt). 

\item \type{delete}: Search for and remove the key (and any associated 
value(s)). Again, this is only guaranteed to complete for 
computationally-bounded collections.

\stopitemize

\section[title=List based dictionaries]


\subsection[title=\type{insert}]

For a computationally-unbounded dictionary, to insert a key-value pair we 
simply append it to the beginning of the list. 

\starttyping
\startJoylolCode

insert ( dict:isDict (key value) )() 
  -> ( dict (key value) )( appendToList )
  
\stopJoylolCode
\stoptyping

\subsection[title=\type{delete}]

For a computationally-unbounded dictionary, to delete a key-value pair we 
simply insert a key with a \type{nil} value. 

\starttyping
\startJoylolCode

delete ( dict:isDict key )() 
  -> ( dict (key nil) )( insert )

\stopJoylolCode
\stoptyping

\subsection[title=\type{getValueFor}]

For a computationally-unbounded dictionary, we sequentially \quote{walk} 
down the list looking for the first key-value pair with the given key, and 
if found return the value. If the dictionary does not contain the given 
key, the \type{getValueFor} process never halts. 

\starttyping
\startJoylolCode

getValueFor ( ( (key aValue) dict:isDict ) key )() 
  -> ( aValue )()

getValueFor ( ( (aKey aValue) dict:isDict ) key )()
  -> ( dict key )( getValueFor )

\stopJoylolCode
\stoptyping

\TODO{How do we \emph{specify} that we do not halt if the key does not 
exist in the collection? How do we prove this? Can we prove this?} 

\section[title=Tree based dictionaries]

\section[title=Hash based dictionaries]

