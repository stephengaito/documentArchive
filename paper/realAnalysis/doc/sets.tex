% A ConTeXt document [master document: realAnalysis.tex]

\chapter[title=Sets] 

\section[title=Set generation] 

\TODO{Discuss the subtle difference between set generation and set 
comprehension. The previous Fans chapter essentially shows that we 
\emph{can} \quote{generate} the computationally-unbounded collection of 
all \quote{classical} \quote{reals} and/or the \quote{power-set} of all 
Natural numbers. However there are many of these \quote{reals} or 
\quote{subsets}, which while they exist, can not be computationally 
described, \quote{selected} or \quote{specified} or \quote{comprehended}, 
using any \emph{bounded} computation.} 

\section[title=Set comprehension] 

\TODO{While the collection of all zero-one sequences is recursively 
generatable, \quote{most} subsets of this collection are neither 
recursive, recursively enumerable, nor corecusively enumerable. } 

\TODO{Explore/Discuss how to \quote{pipe} from a recursive generator 
through recursive, re or co-re processes to select specific instances to 
provide a computable-subset of the recursively generated super-set. While, 
for example, the power-set of the Natural Numbers is 
recursively-generatable, there are only a countable number of 
\emph{computable-subsets} of the Natural Numbers.} 

See \cite{doetsVanEijck2012HaskellLogicMathsProgramming} section 4.1 (page 
116) on set comprehension. 

