% A ConTeXt document [master document: programmingInJoyLoL.tex]

\chapter[title=Preface] 

The programming language, \type{JoyLoL} (the Joy of Lists of 
Lists\footnote{Or alternatively, the Joy of Laughing out Loud}), is a 
concatenative programming language based loosely upon Manfred von Thun's 
programming language \type{Joy}, 
\cite{vonThun1994mathematicalFoundations}, \cite{vonThun1994overview}. 
Most programming languages, whether imperative, like \type{C}, \type{C++}, 
\type{C#} or Java, or functional, like \type{Lisp}, \type{ML}, or 
\type{Haskell}, depend upon the \emph{application} of functions to 
arguments. In a concatenative programming language, all expressions behave 
as functions, and the juxtaposition of expressions denotes function 
composition. The \quote{space} of all computations of a concatenative 
language has a natural a categorical structure which we will explicitly 
exploit. 

The JoyLoL language's primary goal is, in its most austere form, to be a 
fixed point of the semantics functor. Being a fixed point of the semantic 
functor, it is its own denotational, operational and axiomatic semantics. 
This means that, in particular, it provides a sufficient foundation for 
both computational theory, as well as, if granted transfinite 
computational power equivalent to Vopěnka's principle, all of 
conventional Mathematical discourse\footnote{Transfinite computational 
power is equivalent to the Axiom of choice, \emph{so} any one who accepts 
the Axiom of Choice must accept transfinite computational power, 
conversely, any one who rejects transfinite computational power, must 
reject the Axiom of Choice.}. 

Unfortunately for any mere human, JoyLoL's most austere form, while very 
easy for computers to parse, is very difficult for humans to understand. 
This document will instead discuss programming in an augmented form of 
JoyLoL (probably best called \type{JoyLoL++}). Eventually we will show 
that this augmented form of JoyLoL is in fact a \quote{conservative 
extension} of the austere form of JoyLoL. However, for this document, we 
will simply assume that the two variants of JoyLoL are equivalent. 

Another important goal of the JoyLoL language is to provide \quote{simple} 
tools with which to \quote{program} the rather delicate extraction of 
\quote{information} from the potentially infinitely complex structure of 
processes. We make no assumption that the \quote{lists} that JoyLoL 
manipulates are finite. Transfinite lists represent processes which might 
not halt in (trans)finite \quote{time}. To facilitate this transfer of 
information we explicitly implement co-operation of coroutines. 

Coroutines in JoyLoL are implemented using the pair of concepts, Contexts 
and Continuation Passing Style (CPS) of programming. The continuation 
passing style of programming was originally developed to facilitate the 
semantics of programming languages in general, 
\cite{stracheyWadsworth1974continuations}, 
\cite{milne1974semanticsImplementations}, 
\cite{reynolds1974continuationSemantics}, 
\cite{milneStrachey1976theoryProgrammingLanguageSemantics}, 
\cite{stoy1981scottStracheyTheory}, 
\cite{gordon1979denotationalDescriptionProgLangs} and of Scheme in 
particular, \cite{sussmanSteele1975schemeInterpreter}\footnote{For a 
history of the use of Continuation Passing Style (CPS) see, 
\cite{reynolds1993continuationHistory}}. 

In JoyLoL all words are executed in a particular \quote{local} context 
which explicitly contains a pair of stacks, the data and process stacks, 
as well as an explicit naming scope in the form of a look-up dictionary. 
In JoyLoL the Continuation Passing Style of programming is represented in 
the fact that any JoyLoL word may alter either stack\footnote{In fact any 
word may alter either stack, the naming scope or the actual context 
returned to the JoyLoL interpretor.}. Altering the data stack is roughly 
equivalent to a normal programming language changing the \quote{state} of 
the computer system. Altering the \emph{process} stack changes what the 
JoyLoL interpreter \emph{does} next. This is typically \emph{not} 
explicitly allowed in most programming languages. While this extreme 
flexibility allows JoyLoL's semantics to be ultra simple, and facilitates 
the use of co-operative coroutines, it makes understanding the effect of a 
given program text considerably harder to understand for mere humans. 
Dijkstra, \cite{dijkstra1968goToHarmful}, had very good reasons for making 
the \quote{simple} remarks that: 

\startblockquote
My first remark is that, although the programmer's activity ends when he 
has constructed a correct program, the process taking place under control 
of his program is the true subject matter of his activity, for it is this 
process that has to effectuate the desired effect, it is this process that 
in its dynamic behaviour has to satisfy the desired specifications. Yet, 
once the program has been made, the \quote{making} of the corresponding 
process is delegated to the machine.
\stopblockquote

\startblockquote
My second remark is that our intellectual powers are rather geared to 
master static relations and that our powers to visualize processes 
evolving in time are relatively poorly developed. For that reason we 
should do (as wise programmers aware of our limitations) our utmost best 
to shorten the conceptual gap between the static program and the dynamic 
process, to make the correspondence between the program (spread out in 
text space) and the process (spread out in time) as trivial as possible. 
\stopblockquote

While these observations are correct for programming over finite 
structures, they are woefully restrictive when programming over 
potentially non-halting processes. This makes the problem of programming 
over processes both extremely difficult and equalling extremely 
fascinating. Be warned: \quotation{\emph{Here be dragons}}. 

If JoyLoL is potentially so hard to use, why use it? Two reasons, firstly 
it should be no harder to use than first-order logic. Secondly, 
mathematics is in a sense all about desiging new langauges in which to 
\quote{program} computations. By being semantially transparent, JoyLoL 
provides a unique language in which to write higher level, provably 
correct programming langauges for, for example, group theory, vector 
spaces, real and complex analysis, numerical analysis and mathematical 
physics. 

