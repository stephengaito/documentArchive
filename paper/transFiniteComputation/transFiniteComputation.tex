% LaTeX source for the trans-finite Turing document
%
\documentclass[a4paper,openany]{amsbook}
\usepackage[utf8]{inputenc}
\usepackage[english]{babel}
\usepackage[tikzAll]{disitt}
\usepackage{disitt-symbols}
\usepackage[backend=biber,style=alphabetic,citestyle=alphabetic]{biblatex}
\addbibresource{transFiniteComputation.bib}
\usepackage{mdframed}
\newmdenv[linecolor=white,backgroundcolor=gray!10]{infobox}
\newenvironment{myQuote}{\begin{quotation}}{\end{quotation}}
\surroundwithmdframed[linecolor=white,backgroundcolor=gray!10]{myQuote}

\begin{document}
\frontmatter
\sloppy

\title[Computation: Classical version]{Trans-finite Computation: The classical version}
%% author: stg
\author{Stephen Gaito}
\address{PerceptiSys Ltd, 21 Gregory Ave, Coventry, CV3 6DJ, United Kingdom}%
\email{stephen@perceptisys.co.uk}%
\urladdr{http://www.perceptisys.co.uk}


%% version summary
\thanks{Created: 2016-10-12}
\thanks{Git commit \gitReferences{} (\gitAbbrevHash{}) commited on \gitAuthorDate{} by \gitAuthorName{}}
\thanks{AMS-\LaTeX{}'ed on \today{}.}

%% Copyrights
\thanks{\textbf{Copyright: \copyright{} Stephen Gaito, PerceptiSys Ltd \the\year{}; Some rights reserved}}
\thanks{\textbf{This work is licensed under a Creative Commons Attribution-ShareAlike 4.0 International License.}}

\subjclass[2010]{Primary unknown; Secondary unknown} %
\keywords{Keyword one, keyword two etc.}%

\begin{abstract}
We investigate what it means for a finite being to verify trans-finite computational
algorithms.
\end{abstract} 
\maketitle 
\tableofcontents 
\mainmatter

% A ConTeXt document [master document: hilbertsProgram.tex]

\chapter[title=Introduction]

\section[title=Goals]

Every long term researcher should have a silly question, if only to keep 
them focused upon whole problems amidst all of the unending detail. Your 
ultimate destination, whether or not you get there, influences how you 
plan to get there. The problem for any researcher, is that the research 
\quote{space} is infinite dimensional. From the dazzle of choices, you 
must make \quote{one} sequence of choices which, one hopes, ultimately 
leads to your answer(s). The vantage point provided by your ultimate goal, 
influences the questions asked and hence the answers found. The choice of 
where you want to get to, does have a profound influence on how you choose 
to get there. 

My silly question is:

\startalignment[center] Why do babies babble?\stopalignment

\blank[big]

There are (at least) two aspects to this question:

\startitemize[n]

\item How do organic naive \quote{brains} build models of reality. Too 
much of classical artificial intelligence focuses on the incremental 
learning based upon the capabilities of \quote{adult} learners. However 
this misses the point of learning models of Reality \emph{ab initio}. 

\item Equally important, is the question of what constitutes an efficient 
model. Animal brains must keenly balance energy use with the 
comprehensiveness of a given model. Any animal that gets this wrong too 
much of the time, becomes someone else's lunch. 

\stopitemize 

So put simply, my objective, as a \emph{mathematician}, is to build 
efficient and mathematically rigorous models Reality. However, if you are 
going to build a mathematically rigorous theory of Reality, you must first 
provide a mathematically rigorous theory of Mathematics itself. This 
document is devoted to providing just such a rigorous foundation for 
Mathematics. 

Since the ancient Greeks, western influenced Philosophical, Scientific, 
Mathematical practice has been to equate rigour with proofs of 
\quote{Truth}. Euclid's \quote{Elements} has, for just over two millennia, 
provided the pre-eminent example of this paradigm of rigorous proof. 
However, G\"odel's two Incompleteness theorems, show that classical proofs 
through logic are unable to provide the rigorous foundations we require. 
This is the failure of Hilbert's \emph{logical} Program. 

Essentially, G\"odel's work around the early 1930's represent the first 
contributions to the theory of computation. G\"odel's theorems from this 
period concern themselves with our ability to compute \quote{Truth}, and 
the fact that such computation of \quote{Truth}, is only partially 
recursive, rather than totally recursive.

Instead of computing \quote{Truth}, is there a more useful computation 
which might provide a foundation of Mathematics? By exploring Hilbert's 
Program within a \emph{computational} framework, the objective of this 
document is to show that the answer to this question is yes. 

\section[title=Some philosophy]

In Western philosophy, since at least the time of the ancient Greeks, 
there have been a wide range of Philosophical theories of the 
\quote{Reality} of \quote{Reality}. Our objective in building a rigorous 
Mathematical theory of Reality is not to prove any of these Philosophical 
theories (in)correct. Instead our objective, and really the only one 
available \emph{Mathematically}, is to explore what a finite computational 
device, \emph{\quote{a wee beastie}}, can learn about Reality. 

\subsection[title=Hume's problem]

Ignoring Hume's \quote{Problem of Induction} for the moment, as a 
Scientist and Engineer, like any young child, I live in the belief that I 
can both learn about and, more importantly, \emph{interact} with Reality. 
To bastardize Descartes, \emph{from moment to moment, I can see that I 
have made marks in the sand, therefore I am}. 

It is naive to assert that finite beings, such as ourselves, can not learn 
to predict at least some of the future. Russell's farmyard birds, given 
their limited cognitive abilities, \emph{are rational} to expect to be 
feed daily\footnote{See chapter VI, \quote{On Induction}, in Russell's 
\quote{The Problems of Philosophy}, \cite{russell1912problemsOfPhilosophy} 
near page 98}. However, for any \emph{finite} being, there will always be 
events, some highly critical events, which are outside of that being's 
ability to know about and hence predict. This is emphatically the 
\emph{\quote{wee}} in my understanding of a \emph{\quote{wee beastie}}. 
For any finte being, there is always a more capable being, we might just 
not yet found these more capable beings. Understanding the limits of being 
finite, is the true import of Hume's Problem. 

It is equally naive to assert that a finite being can not interact with 
their environment. I \emph{can} communicate with you over both distance 
and time. We \emph{can} build (finite) computational devices. I am writing 
this document using one such device, you are no doubt reading this 
document using at least one other. So it is at least potentially 
reasonable to expect that \emph{finitist} Mathematics could be founded 
computationally. 

\TODO{What does it mean to \emph{exist} mathematically?}

\subsection[title=What is a thing?]

\blank[big]\startblockquote

From the range of the basic questions of metaphysics we shall here ask 
this one question: “What is a thing?” The question is quite old. What 
remains ever new about it is merely that it must be asked again and 
again.\footnote{Martin Heidegger, page 1, first paragraph, in 
\cite{heidegger1967whatIsAThing}, as quoted by 
\cite{doeringIsham2008thingTheoryFoundationsPhysics}.} 

\stopblockquote\blank[big]

A Zen Master would respond that there is \emph{no-thing}, there is only 
\emph{is-isness}, \emph{existence in its entirety}\footnote{Indeed a Zen 
Master would refuse to use mere words. To slightly mix philosophies, 
\quote{The Tao which can be named is not the Tao}, or again in Jewish 
tradition, God is not to be directly \emph{named}. Words differentiate 
\quote{things}, and Zen's \quote{existence-in-entirety}, the Tao and God 
are beyond all human limits to differentiate, identify or understand. We, 
as limited beings, can only \emph{experience}. This is similar to Cantor's 
expressed understanding of his Absolute Infinite Magnitudes, again, see 
Cantor's letter to Dedekind dated 1899, 
\cite{vanHeijenoort1967fregeToGodel}.}. That quarrelsome \quote{thing}, 
\emph{\quote{I}}, is only an illusion. The hardest thing any \quote{one} 
can do is to ignore the \emph{\quote{I}} in order to \quote{see} the 
\emph{\quote{is}}. The dissolution of this \emph{\quote{I}} is 
un-important in the context of the \emph{\quote{is}}. In our work, this 
point of view will be indispensable. 

However, for most of \quote{us}, such a view point is very hard to hold. 
We all play a \quote{me}-\quote{environment} game with existence. This 
view point is equally important for our work.

A \quote{quark} plays the absolute simplest of games, a 
\quote{quark}-\quote{everything-else} game. A \quote{quark}, re-acts to 
its environment. Any model of a \quote{quark} is a (fairly) simple 
S-Matrix. 

A frog's game is only slightly less simple, there is the frog, there are 
\quote{things} that are small enough to be potential food, there are 
\quote{things} that are so large they might be predators, and finally 
there are \quote{things} which might be potential mates. A frog's 
\quote{environment} has some substructure, \quote{prey}, 
\quote{predators}, and \quote{mates}. We assume that a frog's brain 
models, to a sufficiently complex level of detail, these three 
\quote{things}, however, by and large, frogs do not need to expend much 
more energy on playing any more complex games so they don't. By modelling 
only the most important categories, frogs can save energy by not building 
and maintaining complex and energetically expensive nervous 
systems\footnote{See, for example, Ewert's \emph{Motion Perception Shapes 
the Visual World of Amphibians}, 
\cite{ewert2004motionPerceptionAmphibians}. While this reference focuses 
primarily on the visual system of amphibians, it indicates an overall lack 
of need for complex models in an amphibian's nervous system. It is 
estimated, \cite{raichleGusnard2002brainEnergyBudget}, that the adult 
human brain consumes around 20\% of all calories consumed each day, yet 
the human brain only represents around 2\% of our body weight. Nervous 
systems \emph{are} relatively expensive to keep running. For most of a 
frog's needs, this additional complexity is not needed, this is a frog's 
evolutionary niche.} 

A human's game is \emph{much} more complex. We regularly, split 
\quote{our} environment into many many \quote{things}. \quote{Objects} for 
which we build wide classes of models of their behaviour and even their 
potential internal, \quote{intentional}, state of \quote{mind}. Chairs and 
mugs have different uses. Metals and glasses, have different abilities to 
be re-fashioned into useful tools. Animals have widely different 
behaviours providing useful companions or dangerous enemies. Even more 
complex, though, are our \quote{models} of other humans. Each person in 
our environment, has widely differing objectives of their \quote{own}. All 
of which we must, and, by and large, do, keep track of. For each of these 
objects we take an \emph{intentional stance}\footnote{See Daniel Dennett's 
\quote{The Intentional Stance}, \cite{dennett1987a}.}, they each have 
various, though widely, differing abilities to re-act, or intend with 
\quote{me}. These different intentional abilities are reflected in the 
overall complexity in the various models we build to represent any 
particular \quote{thing}. 

So how do we, most efficiently, build these models? Naively, we all 
\quote{know} what an object \quote{is} when we \quote{see} it. Physics 
suggests that all material things are made up of sub-atomic particles 
which are in turn made up of \quote{quarks}. It is \quote{obvious} at one 
level that \quote{I} and \quote{you} are \quote{different} things. 
However, when I shake your hand, where do \quote{my} \quote{quarks} end 
and \quote{your} \quote{quarks} begin? At the \quote{most basic} level, we 
can not separate one \quote{thing} from another \quote{thing}. A Zen 
master's view is actually deeply entwined with any complete mathematical 
model of \quote{things}. How do we reconcile these multiple levels of 
\quote{being} into one comprehensive and complete model of 
\quote{Reality}? 

Andreas D\"oring and Chris Isham in their paper, \emph{What is a thing?}, 
\cite{doeringIsham2008thingTheoryFoundationsPhysics}, suggest that Physics 
can best be captured using the variable sets point of view provided by 
Categorical Topos\footnote{See Lawvere's concept of variable sets, 
\cite{lawvere1975continuouslyVariableSets} or 
\cite{lawvereRosebrugh2003setsForMathematics}.}. For us, a Topos over a 
collection of descriptive \quote{levels}, provides the natural tool with 
which to capture the coherent variation of \quote{thing-ness} as our level 
of description varies. This suggests that the use of the structuralist 
Categorical point of view, in general, and the associated variable 
descriptive Topos point of view, in particular, will be very important to 
our work. 

\subsection[title=A Neuron's eye view]

So lets reflect for a moment on a neuron's point of view. When a neuron 
fires, it \quote{means} something, but what does it mean? Deeply embedded 
inside a complex collection of other neurons, each collecting the spike 
trains from countless other neurons. In some sense each neuron integrates 
the \quote{information} conveyed by each of these spike trains from 
up-stream neurons. What sort of information should these spike trains be 
communicating? At the very least they should be communicating some 
important value. This is what artificial neural networks model. 

\TODO{Bayesian brain, \cite{doyaIshiiPougetRao2007bayesianBrain}, Spikes, 
\cite{riekeWarlandDeRuyterVanSteveninck1999spikesNeuralCode} Pouget, 
\cite{beckPouget2007inferencesImplementationMarkov} and 
\cite{knillPouget2004bayesianUncertaintyComputation}, markov models, 
Shalizi spatial models, \cite{shalizi2001thesis}, section 10.2.1 \emph{Why 
Global States Are not Enough}, required. Markov has restrictive scope of 
description... but by recoding and using levels of description, we can 
recover the effect of history on the present and future.} 

\TODO{talk about Wally's oversight... \cite{walley1991impreciseProb}. 
There is a deep distinction between the ideal asymptotic reals and the 
finite subsequences of processes} 

\TODO{rework the following paragraphs}

So this paper generalizes the collected work of Spitters, Coquand, (see, 
for example, \cite{coquandSpitters2009integralsAndValuations})\footnote{It 
is also important to see the related work of Heunen, Landsman and 
Spitters, \cite{heunenLandsmanEtAl2009toposForAlgebraicQuantumTheory}}, 
together with Walley's work on imprecise probabilities, to produce a 
computable measure theory which a beastie could use. \TODO{Add the work of 
\cite{jackson2006phdThMeasureThSheaves} and its generalization in Isham 
\cite{doeringIsham2008thingTheoryFoundationsPhysics}, Section 8.2} 

As has become traditional in Imprecise Probability theory, in his book, 
\emph{Statistical Reasoning with Imprecise Probabilities}, 
\cite{walley1991impreciseProb}, Walley makes the upper and lower 
previsions (expectations) the primary objects of study with upper and 
lower probabilities as derived concepts. Since we will be concerned with 
probability based Markov structures we will reverse this orientation. One 
of the reasons Walley choose to work with previsions (expectations) 
instead of probabilities is because of his belief that Lower Probablities 
did not determine Lower Previsions (see section 2.7.3 page 82, 
\cite{walley1991impreciseProb}). In fact we will show below that with the 
correct definition of upper and lower measures and upper and lower 
integrals, lower probabilities do determine lower previsions. This will be 
the substance of the (Imprecise) Dedekind-Riesz Representation Theorms 
proven below. 

\TODO{paragraphs above}

\subsection[title=Mappae Mundi]

From our \quote{modern} point of view, Mappae Mundi are, at best, highly 
distorted (navigational) \quote{maps}. From the original 
\quote{cartographer's} point of view, each Mappa Mundi provides a useful 
representation of an important understanding about the 
\emph{cartographer's} \quote{world}. However, a modern notion of 
\quote{navigation} was highly \emph{un}-likely to be the primary purpose 
of any given historical Mappa Mundi. 

Research Mathematicians who habitually cross (standard) mathematical 
disciplines, of necessity, become collectors of mathematical mappa mundi. 
Each (matheamtical) mappa mundi represents an idea in a given discipline, 
which is slightly \quote{wrong} or \quote{mis-shappen} for the research 
mathematician's problem(s) at hand. The \quote{problem(s) at hand} are 
very unlikely to be well represented (if at all) on any existing 
mathematical mappa mundi. This is after all the \emph{point} of 
mathematical research.

\TODO{list my mappae mundi}

\subsection[title=Hilbert's program]

\TODO{Talk about meta-mathematics} 

Hilbert's primary objective was to provide a rigorous 
foundation for Analysis whose metamathematics is finitist. 

\TODO{discuss the concept of data versus processes. classical computation 
theory has been mostly focused upon data not proceses.} 

The classical Reals, which are required for classical Analysis, are 
uncountable. Once we have defined the Ordinals, we will find that there is 
a reasonable definition of $\lambda$-computation for each ordinal, 
$\lambda$. \emph{If} we assume a computational equivalent to the 
\quote{standard} Axiom of Choice, then we can define the transfinite 
ordinals and hence computational structures which can interpret the 
uncountable collection of the Reals (as \emph{data}). Using this 
structure, we will then be able to develop the classical theory of 
Analysis, which was Hilbert's ultimate goal. 

Alternatively, we can define the Reals as (measurement) \emph{processes}, 
with out requiring any \emph{transfinite} ordinals. The resulting theory 
of Analysis will not be quite classical, since the \emph{internal} logic 
of the collection of processes, is not classical. However, I conjecture 
that this process logic provides an explanation of the \quote{strangeness} 
of, and hence the correct foundations for, Quantum Mechanics. 

\TODO{Need to introduce collection of processes as a (co)algebraic 
collection/structure.... we distinguish different processes by observing 
them...} 
\TODO{For this work the focus upon the well-founded/data/algebraic versus 
the non-well-founded/process/co-algebraic is all pervasive. } 

\TODO{Categorical thought == structuralist point of view. Quote 
\cite{awodey2009a}.} 
\TODO{discuss reals as data versus reals as processes == imprecise Reals.} 

\section[title=Some Mathematics]

\subsection[title=A tale of two foundations]

In this document, we are \emph{explicitly} re-founding mathematics using a 
\emph{computational} as opposed to a \emph{logical} tool-set. To a 
classically trained mathematician, these computational foundations will be 
strange at first. The complexity of the foundations of any building 
prefigure the building itself. However any foundations only make sense if 
one reflects on what the building \emph{will be} rather than what the 
foundations are. It is no different in mathematics. 

To help overcome the initial strangeness of these computational 
foundations, we will provide a running commentary using (as yet) classical 
mathematical terminology. Once we have the new foundations secure we can 
translate all of classical mathematics into the new tools. However, until 
the foundations are secure, we need to carefully distinguish between 
\emph{extra-foundational commentary} using classical mathematics, 
typically, classical Category theory, and the actual re-founded 
foundations. 

\startMMundi As this paragraph shows, we will distinguish any 
\emph{extra-foundational comments} by placing them between 
\color[darkgray]{grey} angled over and under bars. The angled over bar 
will also contain the words \emph{\color[darkgray]{Classical commentary}}. 
From the beginning of the next subsection, all \emph{extra-foundational 
comments} will be carefully delineated from the re-foundations themselves. 
\stopMMundi 

\subsection[title=What \emph{does} a Mathematician do?]

\startMMundi

With a \emph{computational} foundation for Mathematics, from the point of 
view of Computer Science, the task of Mathematics, is to provide various 
specialized \emph{rigorous programming languages}. Each of these 
programming languages provides users, engineers, scientists and other 
mathematicians, languages in which complicated computations are easier to 
understand and perform. The languages of Group theory, Lie Algebras, 
Differential Topology, Number theory, and Algebraic Geometry, are just one 
scattered collection of examples. 

Any given programming language consists of a pair of a \emph{syntax} and a 
corresponding \emph{semantics}. The syntax defines which finite texts 
represent valid static descriptions of the dynamic unfolding of various 
computations. The corresponding semantics provides a compositional 
interpretation of the meaning of any given syntactic text. In the theory 
of Computer Science, the connection between a given syntax and a given 
semantic model, is a collection of \emph{GSOS} laws. From a categorical 
point of view, any collection of GSOS laws is represented by a 
distributivity law, a natural transformation, between a pair of 
endo-functors. The initial algebra of one of these endo-functors 
represents the syntax of the programming language. The final coalgebra of 
the other endo-functor represents the semantics. The natural 
transformation itself, is effectively an \emph{interpreter} for the 
programming language. 

The importance of this description is that there can be many syntaxes 
representing the various mathematical languages, areas or disciplines. 
Equally, there could be many different semantics into which a given syntax 
is interpreted. At the moment, within the logical foundation of 
mathematics, the semantics of mathematics is generally agreed to be set 
theory augmented with first-order logic. That is, it is generally assumed 
that any mathematical discipline can be transcribed or interpreted in the 
language of first-order set theory, which in turn provides a 
\quote{rigorous} meaning to statements in the original discipline. 

For a computational foundation of Mathematics, we seek a semantics in 
which various syntaxes can be easily interpreted via GSOS laws. While 
there may be many categorically equivalent semantics, including, for 
example, some form of first-order set theory, we will, in this document, 
base our semantics on Lists of Lists. We will show that these Lists of 
Lists are effectively a fixed point of the \quote{semantic interpretation 
functor}.

A critical criteria which the Lists of Lists semantics satisfies which 
almost any form of first-order set theory will not satisfy, is a 
combination of textual and conceptual simplicity. Like any axiomatic 
theory, the basic axioms must be assumed \quote{true}, or in our 
compuational case \quote{computationally correct}. Our Lists of Lists 
semantics will rely on a very small collection of computations which are, 
for Lists of Lists, both textually simple and \quote{obviously 
computationally correct}. 

In this computational interpretation, what is a proof? Classically we have 
a distinction between \emph{constructive} and \emph{(non-constructive) 
existence} proofs. Constructive proofs are, by and large, essentially 
computations. Indeed in one of the most common formalisms of constructive 
logic, Per Martin-L\"of's type theory, there is a theory of how to extract 
computational programs from the constructive type theoretic proof. 

We will argue that all non-constructive existence proofs correspond to 
searches. The proof by contradiction is essentially a proof that a given 
search algorithm will complete, we just do not know how or when. Nor can 
we provide a closed form solution, all we can provide is a specification 
of what a given solution, once found, will satisfy. 

Given that vanishingly few existing computer programs are proven 
completely correct, how do we know that a given program text computes what 
it purports to compute? At a high level, we use Hoare's system of pre and 
post conditions and show that the given program text, if started in an 
environment satisfying its preconditions will, if it halts, leave its 
environment in a condition which satisfies its postconditions. Since any 
semantic interpretation of a programming language is \emph{compositional}, 
we can recursively apply Hoare's pre and post conditions to each sub-text 
of a given program until we ultimately reach \emph{atomic} statements 
which are declared to satisfy particular pre and post conditions. This is 
not dissimilar to how we currently structure a fully formal proof in set 
theory. 

In mathematics based upon either (classical) logic or computation, we need 
a logic. Unfortunately our existing first-order logic does not \quote{deal 
with} the underlying dynamics of computation. Equally importantly, 
existing (classical) logic is deemed to \emph{be} the \quote{structure} of 
\quote{human} thought (and argumentation). That is first-order logic is 
essentially \emph{extra-mathematical} as it pre-figures mathematical 
discourse.

The logic we will use to establish the computational correctness of a 
given program text, will be the $\mu$-modal logic of an underlying 
Interpreted Transition System associated with the chosen semantic model. 
This logic is inherently designed to deal with the dynamics of 
computation. More importantly, the $\mu$-modal logic we will define, will 
be intimately related to the structure of our semantic interpretation of 
computation. Once we define our semantic interpretation, the $\mu$-modal 
logic is given. Even more importantly there are well defined algorithms 
based upon the theory of two person parity games which can \emph{compute} 
the satisfiability of the collection of pre and post conditions asserted 
about any particular program text purporting to itself compute a 
mathematical result. 

So in this computational interpretation, what do typical mathematicians 
do? Some search for new algorithms to solve new problems. Some worry about 
finding the most conceptually elegant (efficient) algorithm to program the 
proof of a given theorem. Others worry about the expressivity of the 
language they use in a given discipline. Yet others worry about the 
semantic interpretation of the particular language they use. In all of 
these cases the annotation of any given algorithm with a satisfiable 
collection of $\mu$-modal pre and post conditions is required for any 
completely rigorous mathematical result. 

\stopMMundi

\subsection[title=Cast of thousands]

\startMMundi

To help us navigate through our relatively complex \quote{story}, it will 
be useful to introduce our cast of thousands: 

\startitemize[n]

\item \bold{Transition Systems} the basis of $\mu$-modal logic for 
processes. A transition system whose transition relation is transitive, is 
a Category (and is the basis of Dynamic Logic). 

\item \bold{Categories} 

\item \bold{The category of \wflol s}: most important for our work will be 
the \emph{category} of Lists of Lists, \catWFLoL. The \emph{objects} of 
\catWFLoL\ will be the collection of \wflols. The \emph{morphisms} of 
\catWFLoL\ will be any \emph{finite sequence} of the list operators, 
\type{cons}, \type{car}, \type{cdr} and \type{nil}. Composition of 
morphisms in \catWFLoL, is simple concatenation of finite sequences of list 
operators. 

\item \bold{The category of \lol s}: most important for our work will be 
the \emph{category} of Lists of Lists, \catLoL. The \emph{objects} of 
\catLoL\ will be the collection of \lols. The \emph{morphisms} of 
\catLoL\ will be any \emph{finite sequence} of the list operators, 
\type{cons}, \type{car}, \type{cdr} and \type{nil}. Composition of 
morphisms in \catLoL, is simple concatenation of finite sequences of list 
operators. 

\item \bold{The \wflol\ functor}: $\funcWFLoL : \catWFLoL \rightarrow \catWFLoL$ 
defined by 

\placeformula[+]\startformula
  \funcWFLoL(X) = 1 + X \times X
\stopformula

\noindent For $X \in \catWFLoL$. Note that \funcWFLoL\ is an endo-functor of 
the category \catWFLoL. 

\item \bold{The \lol\ functor}: $\funcLoL : \catLoL \rightarrow \catLoL$ 
defined by 

\placeformula[+]\startformula
  \funcLoL(X) = 1 + X \times X
\stopformula

\noindent For $X \in \catLoL$. Note that \funcLoL\ is an endo-functor of 
the category \catLoL. 

\item \bold{Algebras of \funcLoL}: Given any 

\item \bold{CoAlgebras of \funcLoL}:

\item \bold{\lol functor as a monad}:

\item \bold{\lol functor as a comonad}:

\item \bold{Eilenberg-Moore category of \funcLoL\ as a monad}:

\item \bold{Eilenberg-Moore category of \funcLoL\ as a comonad}:

\item \bold{Kleisli category of \funcLoL\ as a monad}: provides the natural 
category in which to discuss the process traces, and (eventually) space-time.

\item \bold{Kleisli category of \funcLoL\ as a comonad}: ?

\item \bold{Multi-sets}: We implement \quote{sets} using \lols\ using a 
\quote{mulit-set}. Basically a multi-set is a list of elements which might 
have multiple \quote{copies} of any given element. For 
\quote{sets}/\quote{multi-sets} in \wflol\ we \emph{could} sort the list 
and remove any duplicates. However, while it is possible to sort (in the 
limit) non-well-founded objects in \lol and/or non-well-founded lists of 
objects, any finite computational approximation will by necessity be a 
multi-set. Hence we generally deal with mulit-sets instead of sets. 

\item \bold{Multi-powerset}: Generalizing powersets we get 
multi-powersets. The (co)(contra)variant (multi-)powerset, \powerSet, is a 
(co)monad. 

\item \bold{GSOS laws}: see \cite{jacobs2017coalgebras} definition 5.5.6 on 
page 323. 

\stopitemize

\stopMMundi

\subsection[title=What \emph{can} a \quote{wee beastie} do?]

Intuitionistic Mathematics, as initiated by Brouwer, has identified the 
concept of the \quote{Idealized Mathematician}. Since this document is 
focused upon the \emph{Mathematics} of the \quote{Reality} which a wee 
beastie can know, for this document, we will identify any wee beastie with 
this Idealized Mathematician, and conversely, any Idealized Mathematician, 
with this wee beastie. 

Unlike Kant, this idealization makes \emph{no} assumptions about the 
structure of time or space. One consequence of our analysis of what a wee 
beastie can know, is space-time itself. 

Following Hilbert, we will assume a number of undefined terms which will 
be defined by their use in our theory. In the following definition, using 
well known practice from Computer Science, we \emph{could} use the word 
\quote{widget} to emphasize, with Hilbert, that our undefined terms can 
represent anything which behaves in the prescribed way. Having made this 
point, we will actually use the undefinded but slightly more suggestive 
terminology of \quote{Lists of Lists} or \lols. 

\startDefinition[beastieActions]

We assume any wee beastie exists in a \quote{collection} of \quote{List of 
Lists} (\lols). With these \lols, a wee beastie can: 

\startitemize[n]

\item Atomic actions

\startitemize[n]

\item create a new \quote{\type{nil}} \lol.

\item create a new \lol\ by \quote{\type{cons}}ing any two existing 
\lols\ together. 

\item create a new \lol\ by \quote{\type{car}}ing an existing \lol. 

\item create a new \lol\ by \quote{\type{cdr}}ing an existing \lol. 

\stopitemize

\item Action Combinators

\startitemize[n]

\item The action which \quote{performs} a pair of actions one followed by 
the other. If $a$ and $b$ are two actions, then the \quote{composition} 
action is explicitly denoted as \quote{\explicitCompose{a}{b}} or more 
simply as \quote{\compose{a}{b}}. 

\item The action which \quote{performs} one or other of a pair of actions. 
There is, however, \emph{no pre-determined} reason for the choice. If $a$ 
and $b$ are two actions, then the \quote{(non-determinitsic) choice} 
action is denoted \quote{\ndChoice{a}{b}}.


\item The action which performs an action multiple times before 
\quote{completing}. However, the number of repetitions is \emph{not 
pre-determined}. If $a$ is an action, then we denote the 
\quote{(non-deterministic) repetition} action as \quote{\repets{a}}. 

\item The action which \quote{completes} if a given test\footnote{Tests 
are defined in \in{Definition}[beastieTests].} succeeds. If $t$ is a test 
then we denote the \quote{test} action by \quote{\test{t}}. 

\stopitemize

\stopitemize

\noindent This is \emph{all} that any wee beastie, or (Idealized) 
Mathematician can \emph{do} to \lols\ in its environment\footnote{In 
Volume II, we will add one more thing that a \emph{transfinite} wee 
beastie can do.}. 

\stopDefinition

\startMMundi

The atomic actions provide the \emph{only} actions that a wee beastie can 
perform \emph{\quote{directly}} on its environment. All other actions are 
formed via various action combinators from these four atomic actions and, 
via the \quote{test} action, any of the tests defined below. 

The pair of definitions, \in[beastieActions] and \in[beastieTests], are 
essentially Dynamic Logic with the addition of the least, $\mu$, and 
greatest, $\nu$, fixed point operators taken from $\mu$-modal logic. See 
\cite{harelKozenTiuryn2000dynamicLogic} and 
\cite{demriGorankoLange2016temporalLogics} respectively for a more 
in-depth discussion. 

Notice our \emph{explicit} use of John McCarthy's notation for the 
programming language LISP,
\cite{mcCarthyAbrahamsEdwardsHartLevin1965lispManual}. 

Since the definitions \in[beastieActions], \in[beastieTests], and 
\in[beastieIdentity] are co-dependent, we postpone any further commentary 
until Subsection \in[beastieReality]. 

\stopMMundi

\subsection[title=What \emph{can} a \quote{wee beastie} know?]

\startDefinition[beastieTests]

Again, we assume that any wee beastie exists in a 
\quote{collection} of \lols. With these \lols, a wee beastie 
can preform (\emph{compute}) the following tests:

\startitemize[n]

\item Atomic Tests

\startitemize[n]

\item The test which always succeeds, denoted \quote{\true} or 
\quote{\top}. 

\item The test which always fails, denoted \quote{\false} or 
\quote{\bottom}. 

\item The test which succeeds if an existing \lol\ is 
\quote{\type{nil}}, denoted \quote{\isNil}.

\stopitemize

\item Static Test Combinators

\startitemize[n]

\item The test which succeeds if given pair of other tests both succeed. 
If $s$ and $t$ are two tests, then the \quote{conjunction} test is denoted 
\quote{$s \land t$}.

\item The test which succeeds if one or more of a given pair of other 
tests succeeds. If $s$ and $t$ are two tests, the \quote{disjunction} test 
is denoted \quote{$s \lor t$}.

\item The test which succeeds if a given other test fails. If $t$ is a 
test, then the \quote{negation} test is denoted \quote{$\lnot\; t$}. 

\stopitemize

\item Dynamic Test Combinators

\startitemize[n]

\item The test which succeeds if a given other test succeeds after all 
possible computations of a given action. If $t$ is a test and $a$ is an 
action, then the \quote{necessity} test is denoted \quote{$[a]t$}. 

\item The test which succeeds if a given other test succeeds after at 
least one possible computation of a given action. If $t$ is a test and $a$ 
is an action, then the \quote{possibility} test is denoted 
\quote{$\langle a\rangle t$}. 

\item The test which succeeds if \TODO{complete the description of the 
$\mu$ operator}.

\item The test which succeeds if \TODO{complete the description of the 
$\nu$ operator}. 

\stopitemize

\stopitemize

If any of the above tests does \emph{not} succeed, then that test fails. 
This is \emph{all} that any wee beastie (or Idealized Mathematician) can 
know by testing \lols\ in its environment. 

\stopDefinition

\startMMundi

The atomic tests provide the \emph{only} tests that a wee beastie can 
perform \emph{\quote{directly}} on its environment. All other tests are 
formed via various test combinators from these three atomic tests, and, 
via the \quote{necessity} and \quote{posibility} tests, any actions 
defined above. Of these three atomic tests, only the third atomic test 
increases the wee beastie's knowledge of its environment. Neither of the 
first two atomic tests reference the \quote{state} of the wee beastie's 
environment. 

The three static test combinators correspond to the core of classical 
(static) propositional logic. Note that we do \emph{not} assume the 
classical \quote{principle of the excluded middle}, hence we must define 
both the \quote{logical} \quote{conjunction} (\quote{and}) as well as the 
\quote{logical} \quote{disjunction} (\quote{or}) operators. It will turn 
out that while the principle of the excluded middle \emph{is} valid for 
any finitely defined collection of (non-)well-founded \lols, it need 
\emph{not} be valid for \quote{asymptotically} defined collection 
(non-)well-founded \lols. 

Classical logic, including propositional, first or higher order, concerns 
itself with \quote{properties} of a single \emph{fixed} \quote{unchanging} 
\quote{world}. Since there is no change, there is no corresponding sense 
of \quote{time}. Wee beasties, such as ourselves, experience a 
\quote{world} of constant change. While we will not (yet) explicitly 
define time, as wee beasties, we are naturally interested in what will 
happen to our world if we change it. The dynamic test combinators capture 
how given tests will change if a wee beastie changes its environment by 
first performing one of its possible actions on its environment. 

As mentioned above the pair of definitions, \in[beastieActions] and 
\in[beastieTests], are essentially Dynamic Logic, 
\cite{harelKozenTiuryn2000dynamicLogic}, augmented with the $\mu$ and 
$\nu$ operators taken from $\mu$-modal logic, 
\cite{demriGorankoLange2016temporalLogics}. 

\stopMMundi 

\subsection[title=Identity and Bisimulation]

When can a wee beastie know that two \lols, in its environment, are the 
\emph{same} \lol? Given the capabilities listed above, a wee beastie can 
never know if two \lols\ are \emph{identical}, the \emph{same} or 
\emph{equal}. The best a wee beastie can do is to compare a pair of \lol's 
using the same collection of \emph{tests}. 

\startDefinition[beastieIdentity]

Two \lols\ are \emph{Bisimilar} if they pass the same collection of tests.

\stopDefinition

\startMMundi This concept of Bisimulation has been identified by a number 
of Mathematicians and Computer Scientists. From the theory of 
non-well-founded sets, see \cite{aczel1988nonWellFoundedSets}. From the 
theory of co-algebras, see Chapter 3 of \cite{jacobs2017coalgebras}. 
\stopMMundi 

Finally, linking the above three definitions together we make the 
following assertion about any wee beastie and its environment of \lols: 

\begingroup\startAssertion

\startitemize[n]

\item \quote{\type{car}} of the \quote{\type{nil}} \lol\ is bisimular to 
\quote{\type{nil}}. \TODO{Is this correct?}

\item \quote{\type{cdr}} of the \quote{\type{nil}} \lol\ is bisimular to 
\quote{\type{nil}}. \TODO{Is this correct?}

\item \quote{\type{car}} of the \quote{\type{cons}} of two \lols\ is 
bisimular to the first \lol. 

\item \quote{\type{cdr}} of the \quote{\type{cons}} of two \lols\ is 
bisimular to the second \lol. 

\item \quote{\type{cons}} of the pair of \lols\ obtained by 
\quote{\type{car}} and \quote{\type{cdr}} of the same \lol, is bisimular 
to the original \lol. 

\stopAssertion\endgroup

As a result of these assertions, the only \quote{meaningful} tests are 
those which consist of collections of \quote{\type{car}}s and 
\quote{\type{cdr}}s. Assertions 3 and 4 imply that any 
\quote{\type{cons}}s can be undone by appropriate use of 
\quote{\type{car}}s and \quote{\type{cdr}}s. \TODO{Is this a meaningful 
observation?} 

\subsection[title=A Beastie's environment, reference=beastieReality]

\startMMundi

From a classical point of view, the atomic actions that a wee beastie can 
perform on its environment, defines a \quote{List of Lists} or 
\quote{Trees} \emph{endo-functor}, $T : \Set \rightarrow \Set$, from the 
Category of sets, \Set, to itself:

\placeformula[+]\startformula
  T : \Set \rightarrow \Set
\stopformula

\noindent This functor is defined by:

\placeformula[+]\startformula\startalign[n=3]
  \NC T : \NC X \mapsto \bold{1} + X \times X \NC \quad \text{(on objects)}   \NR
  \NC T : \NC f \mapsto \bold{1} + f \times f \NC \quad \text{(on morphisms)} \NR
\stopalign\stopformula

\noindent Let \wflol, denote the set of \emph{finite}, 
$\omega$-computationally\footnote{For a given ordinal, $\lambda$, the 
concept of \quote{$\lambda$-computational power}, is critical to our 
theory. In the next volume, once we have progressed far enough to be able 
to define the ordinals, we will define $\lambda$-computation as the 
computing \quote{power} associated with a wee beastie whose computational 
\quote{traces} are at most $\lambda$ in length. } well-founded, binary 
trees. Then there is an isomorphism:

\placesubformula\startformula
  \alpha : T(\wflol) \longRightIsoArrow \wflol
\stopformula

\noindent Recalling that $T(\wflol) = \bold{1} + \wflol \times \wflol$, 
then this isomorphism is defined by: 

\placeformula[+]\startformula\startalign
  \NC \alpha(\star) \NC = \type{nil}        \NR
  \NC \alpha(x, y)  \NC = \type{cons}(x, y) \NR
\stopalign\stopformula

\noindent This isomorphism makes \wflol\ an \emph{initial algebra}, 
that is an initial object in the category of $T$-algebras, 
$\bold{Alg}(T)$.

On \quote{paper}, any particular $\omega$-computationally well-founded 
binary tree can be denoted by either a balanced collection of left and 
right round brackets, \quote{(} and \quote{)} or a collection of the 
functions \type{cons} and \type{nil}\footnote{The function \type{nil}, 
being nullary, may be written with or without left, right pairs of round 
brackets.}. For example: 

\placeformula[+]\startformula\starttyping
((() ()) ())
\stoptyping\stopformula

\noindent or alternatively 

\placeformula[+]\startformula\starttyping
cons(cons(nil, nil), nil)
\stoptyping\stopformula

\noindent both denote the binary tree:

\placeformula[+]\startformula\startMPcode{commDiag}
  setupCommDiags ;
  
  addObject(1,4, "\type{cons}") ;
  addObject(2,2, "\type{cons}") ;
  addObject(3,1, "\type{nil}") ;
  addObject(3,3, "\type{nil}") ;
  addObject(2,5, "\type{nil}") ;
  
  drawObjects(1cm, 1cm) ;
  
  addArrow(1,4, 2,2, "-", 0)()()("", 0.5, );
  addArrow(1,4, 2,5, "-", 0)()()("", 0.5, );
  addArrow(2,2, 3,1, "-", 0)()()("", 0.5, );
  addArrow(2,2, 3,3, "-", 0)()()("", 0.5, );

\stopMPcode\stopformula

\noindent We will explicitly provide parsers for both notations below. 

Let \lol, denote the set of \emph{potentially} \emph{countably infinite}, 
$\omega$-computationally \emph{non-well-founded}, binary trees and maps 
between them. Then, there is an isomorphism: 

\placesubformula\startformula
  \zeta : \lol \longRightIsoArrow T(\lol)
\stopformula

\noindent Recalling that $T(\lol) = \bold{1} + \lol \times \lol$, 
then this isomorphism is defined by: 

\placeformula[+]\startformula\startalign
  \NC \zeta(\type{nil}) \NC = \star                          \NR
  \NC \zeta(\type{x})   \NC = (\type{car}(x), \type{cdr}(x)) \NR
\stopalign\stopformula

\noindent This isomorphism makes \lol\ a \emph{final coalgebra}, that is a 
final object in the category of $T$-coalgebras, $\bold{CoAlg}(T)$. 

Assuming a wee beastie has $\omega$-computational power, then clearly any 
\lol\ a wee beastie might \emph{create}, from scratch, must be 
well-founded. However, given any already \emph{existing} \lol, a wee 
beastie can not know if it is well-founded or not. Hence, for our work, 
the collection of \emph{non-well-founded} \lols\ is the \emph{primary} 
collection. The sub-collection of \emph{well-founded} \lols, is important 
but secondary. 

\stopMMundi 

\subsection[title=Barr's ladder]

\startMMundi

Because of its importance for our work we explicitly work out Barr's 
Theorem 3.2, \cite{barr1993terminalCoalgebrasWellFounded} and 
\cite{barr1994terminalCoalgebrasCorrection}, for the \quote{Lists of 
Lists} or \quote{Trees} endo-functor defined above. 

\blank[1ex] We begin by noting that:\blank[1ex]

\startitemize[n]

\item $\emptySet$ is the empty set in \Set. It is also the initial 
object of \Set\ (up to isomorphism). 

\item $\bold{1}$ is the singleton set in \Set. It is the final object of 
\Set\ (up to isomorphism). To be definite, we will use the $\bold{1} = \{ 
\type{nil} \}$ which is the set whose only \emph{element} is the 
\quote{\type{nil}}. 

\item Since $\emptySet$ and $\bold{1}$ are, respectively, the initial and 
final objects in \Set, we know that there is a unique morphism between 
them: $k : \emptySet \rightarrow \bold{1}$. \TODO{does $k$ need to be 
monic?} 

\item The set, $T(\emptySet) = \{ \type{nil} \}$.

\item The set, $T(\bold{1}) = \{ \type{nil}, (\type{nil}, \type{nil}) \}$. 

\stopitemize

\placeformula[+]\startformula \startMPcode{commDiag}
  setupCommDiags ; 
  
  addObject(1,1, "\emptySet");
  addObject(1,2, "T(\emptySet)");
  addObject(1,3, "\cdots");
  addObject(1,4, "T^n(\emptySet)");
  addObject(1,5, "T^{n+1}(\emptySet)");
  addObject(1,6, "\cdots");
  
  addObject(2,1, "\bold{1}");
  addObject(2,2, "T(\bold{1})");
  addObject(2,3, "\cdots");
  addObject(2,4, "T^n(\bold{1})");
  addObject(2,5, "T^{n+1}(\bold{1})");
  addObject(2,6, "\cdots");

  drawObjects(1.75cm, 2cm);

  addArrow(1,1, 1,2, ">", 0)()()("j",          0.5, top);
  addArrow(1,2, 1,3, ">", 0)()()("",           0.5, );
  addArrow(1,3, 1,4, ">", 0)()()("",           0.5, );
  addArrow(1,4, 1,5, ">", 0)()()("\vphantom\lgroup T^n(j)",     0.5, top);
  addArrow(1,5, 1,6, ">", 0)()()("",           0.5, );
  addArrow(1,1, 2,1, ">", 0)()()("k",          0.5, lft);
  addArrow(1,2, 2,2, ">", 0)()()("T(k)",       0.5, rt);
  addArrow(1,4, 2,4, ">", 0)()()("T^n(k)", 0.5, lft);
  addArrow(1,5, 2,5, ">", 0)()()("T^{n+1}(k)", 0.5, rt);
  addArrow(2,2, 2,1, ">", 0)()()("t",          0.5, bot);
  addArrow(2,3, 2,2, ">", 0)()()("",           0.5, );
  addArrow(2,4, 2,3, ">", 0)()()("",           0.5, );
  addArrow(2,5, 2,4, ">", 0)()()("\vphantom\lgroup T^{n+1}(t)", 0.5, bot);
  addArrow(2,6, 2,5, ">", 0)()()("",           0.5, );

\stopMPcode \stopformula

\stopMMundi

\startMMundi

Definitions \in[beastieActions] and \in[beastieTests] together define a 
pair of syntax, $\Sigma : \Set \rightarrow \Set$, and behaviour, $B : \Set 
\rightarrow \Set$, \emph{endo-functors} from the Category of Sets, \Set, 
to itself together with a collection of GSOS rules\footnote{See, for 
example, \cite{turiPlotkin1997operationalSemantics}, 
\cite{klin2011bialgebrasSOS}, and \cite{jacobs2017coalgebras}.}. 

The syntax functor: 

\placeformula[+]\startformula\startalign
  \NC \Sigma : \NC \Set \rightarrow \Set           \NR
  \NC \Sigma : \NC \bold{1} + X \times X \mapsto X \NR
\stopalign\stopformula

\noindent is defined by

\placeformula[+]\startformula\startalign
  \NC \Sigma(\star) \NC = \type{nil}        \NR
  \NC \Sigma(x, y)  \NC = \type{cons}(x, y) \NR
\stopalign\stopformula

\noindent Let \catWFLoL, denote the sub-category of \Set\ of 
\emph{finite}, $\omega$-computationally\footnote{For a given ordinal, 
$\lambda$, the concept of \quote{$\lambda$-computational power}, is 
critical to our theory. In the next volume, once we have progressed far 
enough to be able to define the ordinals, we will define 
$\lambda$-computation as the computing \quote{power} associated with a wee 
beastie whose computational \quote{traces} are at most $\lambda$ in 
length. } well-founded, binary trees, and maps between them. Then, 

\placesubformula\startformula\startalign
  \NC \Sigma : \catWFLoL \NC \rightarrow \catWFLoL        \NR
  \NC \Sigma(\catWFLoL)  \NC \longRightIsoArrow \catWFLoL \NR
\stopalign\stopformula

\noindent is an isomorphism making \catWFLoL\ an \emph{initial algebra}, 
that is an initial object in the category of $\Sigma$-algebras, 
$\bold{Alg}(\Sigma)$.

Similarly, the behaviour functor: 

\placeformula[+]\startformula\startalign
  \NC B : \NC \Set \rightarrow \Set           \NR
  \NC B : \NC X \mapsto \bold{1} + X \times X \NR
\stopalign\stopformula

\noindent is defined by

\placeformula[+]\startformula\startalign
  \NC B(\type{nil}) \NC = \star                          \NR
  \NC B(\type{x})   \NC = (\type{car}(x), \type{cdr}(x)) \NR
\stopalign\stopformula

\noindent Let \catLoL, denote the sub-category of \Set\ of 
\emph{potentially} \emph{countably infinite}, $\omega$-computationally 
\emph{non-well-founded}, binary trees and maps between them. Then, 

\placesubformula\startformula\startalign
  \NC \Sigma : \catLoL \NC \rightarrow \catLoL                \NR
  \NC \catLoL          \NC \longRightIsoArrow \Sigma(\catLoL) \NR
\stopalign\stopformula

\noindent is an isomorphism making \catLoL\ a \emph{final algebra}, 
that is an final object in the category of $\Sigma$-coalgebras, 
$\bold{CoAlg}(\Sigma)$.


\stopMMundi

\subsection[title=Lawvere's Theorem]

\section[title=Strategy]

This document will provide a rigorous \emph{computational} foundation of 
Mathematics, by... 

\TODO{Need to talk about lists of lists}

We will do this in a number of distinct steps. Firstly, by defining a 
computational langauge, JoyLoL (\quote{The Joy of Lists of 
Lists}\footnote{Or is it \quote{The Joy of Laughing out Loud}?}). JoyLoL 
is a functional \emph{concatenative} language based upon Manfred von 
Thun's language Joy, \cite{vonThun1994overview}. The critically important 
aspect of JoyLoL is that it is constructed to be a fixed point of the 
semantics functor. This means that JoyLoL provides its own denotation, 
operational and axiomatic semantics. JoyLoL does not rely upon any other 
\quote{pre-existing} structures or set theory to define its meaning. An 
other important aspect of JoyLoL is that it is a \emph{concatenative} 
function language. Almost all other functional programming languages are 
based upon Church's $\lambda$-calculus, importantly, this means that most 
such langauges are focused upon function evaluation and substitution. From 
a categorical point of view, this means that the collection of 
computational traces forms a Topos. Being \emph{concatenative}, the 
collection of JoyLoL computational traces forms a Category, which also 
happens to be a Topos. The distinction here is important. The requirements 
of being a Category are much simpler and valid of many more distinct 
sub-collections of computational traces. 

, we can construct the structure of all JoyLoL computational traces. We 
can define the collection of \emph{finite} substructures as those 
substructures for which a simple \emph{short} JoyLoL program \emph{halts}. 
These finite substructures 

Since there \emph{are} JoyLoL computations which do not halt, this 
structure 



%% Time 

\chapter{Time: Graphs, paths and refinements}

In anticipation of our future work, we take a slightly non-standard route towards defining
(trans-finite) computation. We leave it to the reader to understand both why we take
this route, and that we have, in fact, provided the same structures.

The classical theory of computation, with its (almost exclusive) focus on
\emph{terminating} computational processes, is largely focused on \emph{time} to the
exclusion of \emph{space}. For \emph{terminating} processes, any parallel (spacial)
process can be serialized. This serialisability \emph{is} equivalent to set theory's
Well-ordering Theorem and hence the Axiom of Choice. As we will see below,
\emph{non-terminating} processes need not be globally serializable. This is, again,
equivalent to the lack of global time in relativistic space-time.

Since we are not omni-potent beings, while we can prove theorems about non-terminating 
processes, we can not directly experience them. We can only access non-terminating processes 
via (terminating) ordinal approximations to the full non-terminating process. This means 
that, in a sense, ``all'' computation can \emph{effectively} be reduced to serializable 
terminating processes. 

In this section we explore the \emph{time} focused foundations of computations. In the next 
section we explore the addition of \emph{space-time} (and hence parallelism).

One of the central themes of this cycle of papers is the importance of (combinatorial)
graphs, paths and (co-)algebraic refinements. We start by defining (trans-finite) graphs.

There are essentially two distinct ways to define a mathematical structure: logically, by
explicitly assuming the existence of a set with specific (logically expressed) properties,
or computationally, in categorical language, as the fixed point of an endofunctor. From a
non-categorical inductive point of view, an early expression of this dichotomy can be
found in \cite[page 1]{moschovakis1974induction}. In this paper we follow the logical
approach. In subsequent papers in this cycle, we will use the computational approach.

There are relatively few discussions of non-finite graphs from which we mention Reinhard
Diestel's work, \cite{diestel2006graphTheory, diestel1990infiniteGraphTheory}.
Unfortunately, for trans-finite graphs, we must go, more or less, ``on beyond
zebra''\footnote{Growing up, one of my favourite books was Dr Seuss' ``On beyond Zebra'',
\cite{seuss1955onBeyondZebra}. \emph{Be warned}: One of my greatest faults is that I seem
congenitally incapable of staying within ``obvious'' bounds...}\footnote{Another of my
greatest faults is that my mind is a web of snippets of ideas. Linearizing this web into
single narrative, is \emph{very} hard for me to do, and may be hard to fully follow until 
you have heard ``the whole shaggy-dog story''.}.

Our definition of graph is inspired by the definition of metagraph in
\cite{macLane1971categoriesWorkingMathematician} section I.1. Note that our definition is
a generalisation of most typical definitions, for example \cite[section
1.1]{diestel2006graphTheory}. In particular, all of our graphs are \emph{explicilty}
\emph{directed}. Categorically, a graph is simply a presheaf on the category
$ \cdot \substack{\mathbf{\longrightarrow} \\[-0.7ex] \mathbf{\longrightarrow}} \cdot $

\begin{definition}
A \define{graph}{}, $G(V, E, \pi_0, \pi_1)$, is a set, $V$, of \define{verticies}{}, and a
set, $E$, of \define{edges}{}, together with two maps, \map{\pi_0}{E}{V} and
\map{\pi_1}{E}{V}, called the source and target maps.

A \define{labeled graph}{}, $lG(V, E, L_V, L_E, \pi_0, \pi_1, l_V, l_E)$, is a graph $G(V,
E, \pi_0, \pi_1)$ together with a set, $L_V$, of \define{vertex labels}{}, and a set,
$L_E$, of \define{edge labels}{}, and maps, \map{l_V}{V}{L_V}, and \map{l_E}{E}{L_E}.
Often the set of vertex labels, $L_V$, and vertex label map, $l_V$, are omitted (or
alternatively the set of vertex labels is assumed to be the trivial one point set).
 
Given a labeled graph, $lG(V, E, L_V, L_E, \pi_0, \pi_1, l_v, lE)$ its \define{underlying
graph}{}, is the graph, $G(V, E, \pi_0, \pi_1)$.
\end{definition}

\begin{definition}
A \define{graph morphism}{}, \map{\phi}{G_0}{G_1}, is a map between the graphs,
$G_0(V^{G_0}, E^{G_0}, \pi^{G_0}_0, \pi^{G_0}_1)$ and $G_1(V^{G_1}, E^{G_1}, \pi^{G_1}_0,
\pi^{G_1}_0)$ for which
\begin{enumerate}
\item $\phi(V^{G_0}) \subset V^{G_1}$
\item $\phi(E^{G_0}) \subset E^{G_1}$
\item $\phi \compose \pi^{G_0}_i = \pi^{G_1}_i \compose \phi$ for $i \in {0, 1}$
\end{enumerate}

A \define{labeled graph morphism}{}, \map{\phi}{lG_0}{lG_1}, between a pair of labeled
graphs, $lG_0(V^{lG_0}, E^{lG_0}, L^{lG_0}_V, L^{lG_0}_E, \pi^{lG_0}_0, \pi^{lG_0}_1,
l^{lG_0}_V, l^{lG_0}_V)$ and $lG_1(V^{lG_1}, E^{lG_1}, L^{lG_1}_V, L^{lG_1}_E,
\pi^{lG_1}_0, \pi^{lG_1}_1, l^{lG_1}_V, l^{lG_1}_V)$, is a graph morphism between the
underlying graphs, $G_0$ and $G_1$ for which
\begin{enumerate}
\item $\phi(L^{lG_0}_V) \subset L^{lG_1}_V$
\item $\phi(L^{lG_0}_E) \subset L^{lG_1}_E$
\item $\phi \compose l^{lG_0}_V = l^{lG_1}_V \compose \phi$
\item $\phi \compose l^{lG_0}_E = l^{lG_1}_E \compose \phi$
\end{enumerate}
\end{definition}

\begin{definition} \TODO{expand this to include isomorphism, monomorphism, epimorphisms}
A \define{graph embedding}{} is a graph morphism, \map{\phi}{G_0}{G_1}, respectively, a
labelled graph morphism, \map{\phi}{lG_0}{lG_1}, which is, categorically, a monomorphism.
\end{definition}

\begin{lemma}
Given a graph, $G(V, E, \pi_0, \pi_1)$, there exists a \define{dual graph}{},
$\tilde{G}(\tilde{V}, \tilde{E}, \tilde{\pi}_0, \tilde{\pi}_1)$, unique up to graph 
isomorphism, for which
\begin{enumerate}
\item $\tilde{V} = V$,
\item $\tilde{E} = \set{(v, u) \suchThat (u, v) \in E}$
\end{enumerate}
\end{lemma}

\begin{proof}
\TODO{PROVE THIS!}
\end{proof}

\TODO{Do we need the equivilant lemma for dual labeled graphs?}

Our next goal is to provide a definition of $1$-dimensional (time) ``$1$-paths''.
Intuitively a (directed) ``$1$-path'' \emph{in} a graph is an embedding of a ``totally
ordered'' directed graph.

\TODO{NOTE: the current use of $(x,y) \in E$ is not general enough for our purposes. It 
assumes that there is a unique edge between the vertices $x$ and $y$. For paths this is ok, 
for general graphs it is NOT. We need to rework this notation.}

\begin{definition}
A \define{(directed) 1-path}{}, $P(V, E, \pi_0, \pi_1)$, is a graph for which:

\begin{enumerate}
\item \define{Anti-symmetry:}{} for any pair of vertices, $x, y \in V$, if $(x,y) \in E$ and
$(y,x) \in E$ then $x = y$.
\item \define{Transitivity:}{} for any triple of verticies, $x, y, z \in V$, 
if $(x,y) \in E$ and $(y,z) \in E$ then $(x,z) \in E$.
\item \define{Total order:}{} for any pair of vertices, $x, y \in V$, if $x \neq y$ then
either $(x,y) \in E$ or $(y,x) \in E$ but not both.
\end{enumerate}

A \define{labelled 1-path}{}, $lP(V, E, L_V, L_E, \pi_0, \pi_1, l_V, l_E)$, is a labeled 
graph whose underlying graph is a 1-path.
\end{definition}

Given that $1$-paths are total orders (and hence are posets), we \emph{could} use the more
traditional, ``$\leq$'', notation for the edges in $1$-paths. However to ensure a more
obvious generalisation to $\lambda$-paths (for any ordinal, $\lambda$), we retain the edge
notation, ``$(x,y)$'', throughout.

\begin{definition}
For a path, $P(V, E, \pi_0, \pi_1)$, a vertex, $v \in V$, is \define{maximal}{} or
\define{final}{} (respectively \define{minimal}{} or \define{initial}{}) if for every
vertex, $x \in V$, there exists an edge $(x,v) \in E$ ($(v, x) \in E$).

A vertex is \define{internal}{} if it is neither initial or final. A vertex is on the  
\define{boundary}{} if it is either initial or final.

A \define{$1$-interval}{} is a $1$-path which has both maximal and minimal vertices. Given 
verticies, $x, y \in V$ which are minimal and maximal (respectively), we often denote the 
interval as, $[x,y]$.
\end{definition}

Note that by anti-symmetry, any maximal (minimal) elements of a path are unique. By total
order, if there exists two distinct vertices, $x,y \in V$, for which $(x,y) \in E$, then
the vertex $x$ can not be maximal, and conversely, the vertex $y$ can not be minimal.

\begin{definition}
A path, $P_0(V^{P_0}, E^{P_0}, \pi^{P_0}_0, \pi^{P_0}_1)$, is a \define{subpath}{} of a
path, $P_1(V^{P_1}, E^{P_1}, \pi^{P_1}_0, \pi^{P_1}_1)$, if there exists an embedding,
\map{\phi}{P_0}{P_1}.

Conversely, a path, $P_1(V^{P_1}, E^{P_1}, \pi^{P_1}_0, \pi^{P_1}_1)$, \define{extends}{}
a path, $P_0(V^{P_0}, E^{P_0}, \pi^{P_0}_0, \pi^{P_0}_1)$, if there exists an embedding,
\map{\phi}{P_0}{P_1}.
\end{definition}

Note that every edge in any path is trivially a subpath of the path. Conversely, a path
extends each of its edges. By abuse of notation, the edge, $(x,y)$, will often also denote
its corresponding sub\emph{path}, $P(\set{x,y}, \set{(x,y)}, \pi_0((x,y)):=x,
\pi_1((x,y)):=y)$.

\begin{definition}
Given paths, $P_0(V^{P_0}, E^{P_0}, \pi^{P_0}_0, \pi^{P_0}_1)$ and $P_1(V^{P_1}, E^{P_1},
\pi^{P_1}_0, \pi^{P_1}_1)$, an edge, $(x^{P_0}, y^{P_0}) \in E^{P_0}$ \define{covers}{} an
edge, $(x^{P_1}, y^{P_1}) \in E^{P_1}$, if there exists an subpath, $\hat{P_1}$ in $P_1$
which is both an interval, $[\hat{x^{P_1}}, \hat{y^{P_1}}]$, and an extension of
$(x^{P_1}, y^{P_1})$ for which there exists an embedding, \map{\phi}{(x^{P_0},
y^{P_0})}{\hat{P_1}} for which:
\begin{itemize}
\item $\phi(x^{P_0}) = \hat{x^{P_1}}$
\item $\phi(y^{P_0}) = \hat{y^{P_1}}$
\item $\phi( (x^{P_0}, y^{P_0}) ) = (\hat{x^{P_1}}, \hat{y^{P_1}})$
\end{itemize}
\end{definition}

\begin{definition}
A path, $P_1(V^{P_1}, E^{P_1}, \pi^{P_1}_0, \pi^{P_1}_1)$, is a \define{refinement}{} of a
path, $P_0(V^{P_0}, E^{P_0}, \pi^{P_0}_0, \pi^{P_0}_1)$, if every edge, $e^{P_1} \in 
E^{P_1}$ is covered by an edge $e^{P_0} \in E^{P_0}$.
\end{definition}

Having defined what a $1$-path is, we now need to explore the collection of $1$-paths.

\begin{theorem}\label{path:addFinal}
Given a path, $P(V, E, \pi_0, \pi_1)$, there exists a path, $P^+(V^+, E^+, \pi^+_0,
\pi^+_1)$,  unique up to graph isomorphism, for which
\begin{enumerate}
\item there exist a graph embedding, \map{\phi^+}{P}{P^+}
\item $V^+ = \phi^+(V) \union \set{V}$
\item for each $v \in V$ there exists an edge, $(\phi^+(v), \set{V}) \in E^+$.
\end{enumerate}
\end{theorem}

\begin{proof}
\TODO{PROVE THIS!}
\end{proof}

\begin{corollary}\label{path:addInitial}
Given a path, $P(V, E, \pi_0, \pi_1)$, there exists a path, $P^-(V^-, E^-, \pi^-_0,
\pi^-_1)$,  unique up to graph isomorphism, for which
\begin{enumerate}
\item there exist a graph embedding, \map{\phi^-}{P}{P^-}
\item $V^- = \phi^-(V) \union \set{V}$
\item for each $v \in V$ there exists an edge, $(\set{V}, \phi^-(v)) \in E^-$.
\end{enumerate}
\end{corollary}

\begin{proof}
Apply Theorem, \ref{path:addFinal}, to the dual path of $P$.
\end{proof}

\begin{corollary}
Given a path, $P(V, E, \pi_0, \pi_1)$, there exists a path, $P^{+-}(V^{+-}, E^{+-},
\pi^{+-}_0, \pi^{+-}_1)$,  unique up to graph isomorphism, for which
\begin{enumerate}
\item there exist a graph embedding, \map{\phi^{+-}}{P}{P^{+-}}
\item $V^{+-} = \phi^{+-}(V) \union \set{V} \union \set{V^+}$
\item for each $v \in V$ there exists an edge, $(\phi^{+-}(v), \set{V}) \in E^{+-}$,
\item for each $v \in V$ there exists an edge, $(\set{V^+}, \phi^{+-}(v)) \in E^{+-}$,
\item $(\set{V^+}, \set{V}) \in E^{+-}$.
\end{enumerate}
\end{corollary}

\begin{proof}
Apply Theorem, \ref{path:addFinal}, to obtain the path $P^+(V^+, E^+, \pi^+_0, \pi^+_1)$
which has a unique final vertex, and then Corollary, \ref{path:addInitial}, to obtain
$P^{+-}$.
\end{proof}

\begin{theorem}
Given a pair of paths, $P^-(V^-, E^-, \pi^-_0, \pi^-_0)$ and $P^+(V^+, E^+, \pi^+_0,
\pi^+_1)$, for which $P^-$ has a final vertex, $p^-$ and $P^+$ has an initial vertex,
$p^+$, there exists a path $P(V, E, \pi_0, \pi_1)$, unique up to graph isomorphism, for
which
\begin{enumerate}
\item there exists a graph embeddings, \map{\phi^-}{P^-}{P} and \map{\phi^+}{P^+}{P},
\item $\phi^-(p^-) = \phi^+(p^+)$,
\item for each $v^+ \in V^+$ and $v^+ \in V^+$ there exists an edge $(\phi^-(v^-), 
\phi^+(v^+)) \in E$.
\end{enumerate}
\end{theorem}

\begin{proof}
\TODO{PROVE THIS!}
\end{proof}

\TODO{Add associativity of path joining}

\begin{definition}
A labelled path, $lP(V, E, L_V, L_E, \pi_0, \pi_1, l_V, l_E)$, is \define{edge
consistent}{} if the set of edge labels, $L_E$, is equipped with a mapping,
\map{\compose}{L_E \times L_E}{L_E}, for which if $x, y, z \in V$ are three verticies for 
which $(x,y), (y,z) \in E$ (and hence $(x,z) \in E$), then $l_E((x,z)) = l_E((x,y)) \compose 
l_E((y,z))$.
\end{definition}

\begin{definition}
An graph embedding, \map{\phi}{P}{lG}, of a path, $P(V^P, E^P, \pi^P_0, \pi^P_1)$ into a 
labelled graph, $lG(V^{lG}, E,^{lG}, L_{V^{lG}}, L_{E^{lG}}, \pi^{lG}_0, \pi^{lG}_1, 
l_{V^{lG}}, l_{E^{lG}})$ induces a labelling on the path, $P$, via the pullback along 
$\phi$. Specifically, the \define{induced labelling}{} is the labelled path, $lP(V^P, E^P, 
L_{V^{lG}}, L_{E^{lG}}, \pi^P_0, \pi^P_1, l_{V^{lG}} \compose \phi, l_{E^{lG}} \compose 
\phi)$
\end{definition}

\begin{definition}
A labelled graph, $lG(V, E, L_V, L_E, \pi_0, \pi_1, l_V, l_E)$, has an \define{extensional
edge labelling}{} if two edges, $e_0, e_1 \in E$ are equal, $e_0 =_E e_1$ (as elements of
the set $E$) if and only if
\begin{enumerate}
\item $\pi_0(e_0) =_V \pi_0(e_1)$
\item $\pi_1(e_0) =_V \pi_1(e_1)$
\item $l_E(e_0) =_{L_E} l_E(e_1)$
\end{enumerate}
as elements of the respective sets.
\end{definition}

For at most countable graphs, defining ``linearity'' is relatively easily done by
``following'' single edges corresponding to the ``successor'' or ``next'' operation. For
trans-finite graphs we must add something more to be able to deal with ``limit points''.
Rather surprisingly, to deal with these ``limit points'' it is convenient to define
1-Categories, since a Category which is, sequentially complete as well as sequentially
cocomplete, is a graph which explicitly includes all of its ``paths''. Categorically, a
category is a presheaf on the category
$ \bullet \substack{\mathbf{\longrightarrow} \\[-0.7ex] \mathbf{\longleftarrow} 
\\[-0.7ex] \mathbf{\longrightarrow} } \bullet $
equipped with a consistent collection of 2-simplicies which are declared
``thin''\footnote{This way of looking at 1-categories is inspired by the work of Dominic
Verity, see, for example, \cite{verity2005complicialSets},
\cite{verity2006complicialSimplicialHomotopy} and
\cite{verity2006simplicialComplicialCategories}.}.

\begin{definition}
A \define{1-category}{}, $C(V, E, L_V, L_E, \pi_0, \pi_1, l_V, l_E, i)$, is a labelled graph 
with an extensional edge labelling, \map{l_E}{E}{L_E}, 
\end{definition}


\begin{definition}
A \define{1-category}{}, $C(E, V, \pi_0, \pi_1, i)$, is a graph with the addition of an 
\define{identity map}{} \map{i}{V}{E} for which $\pi_0 \compose i$ and $\pi_1 
\compose i$ are both the identity of the set $V$, and for which 
\begin{itemize}
\item If $f, g \in E$ and $\pi_1(f) = \pi_0(g)$ then there exists $f \compose g \in E$ for 
which $ \pi_0(f) = \pi_0(f \compose g)$ and $ \pi_1(f \compose g) = \pi_1(g)$
\item If $f \in E$ then $ f \compose i(\pi_0(f)) = f$ and $i(\pi_1(f)) \compose f = f$
\item If \map{f}{w}{x}, \map{g}{x}{y} and \map{h}{y}{z} then the graph, $hgf(\set{f,
g, h}, \set{w, x, y, z}, \pi_0, \pi_1)$ can be extended to a 1-path embedded in $C$ in only 
one way
\end{itemize}
\end{definition}


%% Space-Time

\chapter{Space-time: directed simplicies, refinements and space-time paths}

\section{Introduction}

In the previous section we explored the structures required to provide a \emph{time} focused 
foundations of computation. In this section we expand these time focused structures into 
their corresponding fully space-time structures.

We base our analysis of (directed) space on directed \emph{trans-finite} simplicies.
Recall that \emph{un-directed} simplicial structures are, categorically, defined as the
collection of pre-sheaves on the small category of finite (positive) ordinals. See for
example, the work of Grandis, \cite{grandis2001symSimpSets}, and
\cite{grandis2001fundamentalFunctorsSimplicial}. For a good \emph{concrete} introduction
to the categorical concept of pre-sheaves, see \cite{reyesReyesZolfaghari2004presheaves}.

We use \emph{directed} simplicies \emph{because} we wish to understand the ``higher
dimensional'' \emph{causal} structure of space-time.

Equally importantly, directed simplicies provide a higher dimensional generalisation of
the ubiquitously used concept of ``matrix'' outside of explicitly ``Euclidean vector
spaces''. Essentially a groupoid (that is a un-directed 1-category) \emph{is} a sparse
matrix. This analogy is critical when we want to study higher dimensional analogues of
``Markov Matrices'' and their corresponding ``Markov Chains''.

\TODO{word-smith these introductions together}

We are interested in \ndDeltaC{a}{b}, \sNdDeltaC{a}{b}, \DeltaC{a}{b}, \sDeltaC{a}{b}.

In this chapter, our aim is to define the discrete structures we will use to \emph{model}
our differential topologies.  These are simplicial \emph{structures} which are related to
the simplicial sets used by algebraic topologists.  However we \emph{explicitly} do not
limit ourselves to either finite dimensions \emph{or} to, potentially, assuming that these
structures are \emph{sets}.  With wild abandon, we will allow simplicial structures which
\emph{might} entail Russell's paradox.  Instead we will, essentially, use techniques from
Algebraic Set Theory (AST), to delineate those appropriately ``tame'' simplicial
sub-structures which prohibit Russell's paradox.  \TODO{DO we really need/mean this?
Doesn't category theory (see nLab discussion of set theory and ETCS) essentially
automatically tame Russell's paradox? Or does it just push his paradox into a slightly
more complex environment?  I suspect we could state everything as ``presheaves'' not into
\setC{} but into some larger category of classes/collections which \emph{should} be a
topos but I am unsure how that would change the overall theoretical tools that I want to
use. SO we will really only work with small categories and leave the full YACT theory for
later generations to finalize in all of its glory.}

\section{The small categories of \texorpdfstring{$\kappa$}{kappa}-Simplexes} 

In a Categorical setting, such as we are using, the most appropriate way to define
simplicial sets is via the Topos of functor presheaves, \opFuncCat{\DeltaC{}{}}{\setC{}}
(alternatively \altOpFuncCat{\DeltaC{}{}}{\setC{}}), see for example \cite[Chapter
1]{goerssJardin1999SimplicialHomotopyTh}, \cite[Section VII.5]{maclane1971a},
\cite[section 2.1]{arxiv2005math0410412v2}, \cite[Section
I.2]{may1967simplicialObjectsInAlgTopo}, \cite[Chapter 2]{gabrielZisman1967homotopyTh}
\TODO{add the numerous papers}.

We will actually base our transfinite generalization on Dominic Verity's exposition,
\cite[section 2.1]{arxiv2005math0410412v2}. The essential difference between the
expositions of Goerss and Jardin's versus that of Verity is that Goerss and Jardin's
exposition uses a more purely Categorical notation (which avoids the use of internal set
theoretic notation), while Verity's exposition uses a more classical notation by
essentially working in the internal set theory of \setC{}.  For those who are not familiar
with simplicial topology, Friedman's expository account of simplicial sets,
\cite{arxiv2008math0809.4221v1}, provides a highly recommended introduction to this theory
with a good number of visual explanations of what is going on both combinatorially as well
as geometrically.

It is significant that Goerss and Jardin's exposition makes frequent, mainly illustrative,
use of the adjunction between the categories of Simplicial Sets, \simpC{}, and Topology,
\topologyC{} (and in particular the topological space of ``Euclidean space'', $\Reals^{n}$
for some $n$).  At this point in our exposition (construction?) we have not defined
``Euclidean space''.  From our point of view, Euclidean spaces are particular types of
approximation structures in the category of Simplicity Sets, \simpC{}.  This is
\emph{because} we are primarily interested in how space-time is ``constructed'' for the
experience of a beastie, and do not assume its existence\footnote{We will however make use
of Euclidean spaces when we come to prove the \emph{existence} of sufficiently complex
simplicial sets}.

As stated in the introduction we are actually interested in including, at least, countably
infinite dimensional simplicies. To do this we need to generalize the usual definitions of
simplicial sets to that of \emph{transfinite} simplicial sets.  We do this by essentially
simply re-interpreting Dominc Verity's definition of \emph{finite} dimensional simplicial
sets in a transfinite setting.  Essentially all of the additional work is done by
generalizing the simplicial face and degeneracy maps to include these maps to and from
limit ordinals as well as ``just'' finite ordinals.

\section{Ordinals}

\begin{definition}
A pograph, \gC{}, is \define{totally ordered} if there is exactly one morphism between any
two objects. If \gC{} is a pograph, then any initial or terminal objects it might have are
unique. An initial object of \gC{} is the \define{least element} of \gC{}. A terminal
object of \gC{} is the \define{greatest element} of \gC{}.  A totally ordered graph,
\gC{}, is \define{well-ordered} if every non-empty full subgraph of \gC{}, has a least
element.  Given a totally-ordered graph, \gC{}, and an object $a \in \objects{\gC{}}$, the
\define{section}, $\gC{}_a$, is the full subgraph of \gC{}, who's objects are defined by
$\gC{}_a = \set{x \in \objects{\gC{}} \suchThat x \leq a \AND x \neq a}$.
\end{definition}

\TODO{We should probably be careful with our definition of totally ordered here.  We might
define totally ordered to be a pograph for which any \emph{distinct unordered} pair of
objects has a morphism.  If we drop the distinctness, then we essentially get a pocat. 
The critical distinction here is which types of graph maps are "natural"... if we disallow
morphisms between the same object, then we force all graph maps to be \emph{strict}, if we
allow morphisms between the same object, then we allow non-strict graph maps.  Which
definition do we really want?}

Category theory is all about relationships, in the form of morphisms, between objects.  In
categorical \emph{theory}, the objects themselves have no explicit structure other than
that detectable from the morphisms between the objects.  In \emph{applications} of
category theory, the objects of a category often have considerable explicit structure, for
example, the category of Groups, or the category of Topological Spaces.  One of the
recurring themes in our work is the interplay between relationships \emph{inside}
structured objects and the relationships \emph{between} these structured objects.  Our
first primary example of this is in our definition of ordinal graphs in the graph of all
totally ordered graphs.
\begin{definition}
The \define{graph of totally ordered graphs}, \totalOrderedC{}, is defined by
\begin{description}
\item[objects] are totally ordered graphs
\item[morphisms] are graph maps between totally ordered graphs.
\end{description}
\end{definition}
Note that graph maps between totally ordered graphs are the graphical form of the order
preserving maps of (set based) posets. We will study them in depth in the next section.

Before we can define the ordinals, we need to understand co-limits of objects in
\totalOrderedC{}.  This would be simpler if we already had the whole structure of Category
and Topos theory, something we won't have until much later in this mathematical narrative.
 However the actual construction is fairly simple and moreover adds to our understanding
of Category theory. The co-limits in \graphC{} are simply the ``disjoint union'' of an
appropriate indexing set of the component graphs.  Unfortunately, in \totalOrderedC{},
there is \emph{exactly} one morphism between any pair of objects, so the simple ``disjoint
union'' construction is insufficient for our needs.  Instead we can define ordinals if we
understood the successor construction, the order relation between ordinals, and the union
(directed co-limit) of predecessor ordinals.  \TODO{Point out that this is really the
definition of the von Neumann ordinals in a graphical context.}

\begin{definition}
Consider a totally ordered graph, \tC{}. The \define{successor graph}, \successor{\tC{}},
of \tC{}, is
\begin{description}
\item[objects] the objects of \tC{}, \objects{\tC{}}, together with \tC{},
\item[morphisms] the morphisms of \tC{} between any two objects of \tC{} , together with
the morphism $a \leq \tC{}$  for each object, $a \in \objects{\tC{}}$.
\end{description}
\end{definition}
By construction, \successor{\tC{}} is a totally ordered graph.

\begin{definition}
Consider a pair of graphs, \gC{} and \mathCategory{G'}.  Then \define{\gC{} is a
predecessor of \mathCategory{G'}}, denoted $\gC \leq \mathCategory{G'}$ iff \gC{} is a
full subgraph of \mathCategory{G'}.
\end{definition}
Again, by construction, we know that $\tC{} \leq \successor{\tC}$.  More importantly since
a full subgraph of a full subgraph is a full subgraph (via the composition of graph maps),
we know that $\tC{} \leq \successor{\successor{\cdots \successor{\tC{}}}}$.

\begin{definition}
Consider a totally ordered collection of totally ordered graphs. Then, (not surprisingly)
it is a totally ordered graph. \TODO{Is this not just a tautology? FIX THIS!  What we
really want to do here is to give an explicit construction of the directed co-limit.}
\end{definition}

\begin{definition}
The \define{graph of countable ordinals}, \omegaC{}, is defined by
\begin{description}
\item[objects] the totally ordered graphs, \successor{\successor{\cdots
\successor{\zeroC{}}}}, \item[morphisms] $a \leq b$ iff $a$ is a full subgraph of $b$ for
$a, b \in \objects{\omegaC{}}$.
\end{description}
\TODO{We really have not given a good description of the collection of objects (how many
s's are allowed?).}
\end{definition}
We can picture \omegaC{} as the following totally ordered graph:
\begin{cTikzPicture}
\coordinate (v0) at (0,0);
\coordinate (v1) at (2,0);
\coordinate (v2) at (4,0);
\coordinate (v-n-1) at (6,0);
\coordinate (v-n) at (8,0);
%
\node[above] at (v0)    {\zeroC{}};
\node[above] at (v1)    {\successor{\zeroC{}}};
\node[above] at (v2)    {\successor{\successor{\zeroC{}}}};
\node[above] at (v-n-1) {\successor{\successor{\cdots\successor{\zeroC{}}}}};
%\node[above] at (v-n)   {};
%
\fill (v0) circle[radius=1pt];
\fill (v1) circle[radius=1pt];
\fill (v2) circle[radius=1pt];
\fill (v-n-1) circle[radius=1pt];
%\fill (v-n) circle[radius=1pt];
%
\begin{scope}[->,shorten <=4pt,shorten >=4pt]
\path (v0) edge (v1);
\path (v1) edge (v2);
\path (v2) edge[dotted] (v-n-1);
\path (v-n-1) edge[dotted] (v-n);
\end{scope}
\end{cTikzPicture}

The following is a highly impredicative definition of the ordinals.  We could easily use
the ideas in \cite{joyalMoerdijk1995algSetTh} or even Martin-L\"{o}f's type theory, see
for example \cite{dybjer1991inductiveSets}, to provide a fully predicative definition for
at least the countable ordinals that we require.
\begin{definition}
(\cite{cameron1999a})  An \define{ordinal} is a well-ordered graph, \gC{}, for which
$\gC{}_a = a$ for all $a \in \objects{\gC{}}$.  \TODO{We need a better symbol for section
which we really should name tosection (total ordered section) so that we don't conflict
with the categorical notion of a section.} In other words each object in an ordinal,
\gC{}, is the totally ordered graph of all of its predecessors. \define{Zero}, \zeroC{},
which is defined to be the graph with one object, is the smallest ordinal.  An ordinal
which is the union (directed co-limit in \totalOrderedC{}) of all of its predecessors
$\lambda = \underset{\alpha < \lambda}{\union} \alpha$ is a \define{limit ordinal}. A
non-zero ordinal that is not a limit ordinal is a \define{successor ordinal}. A
\define{cardinal} is an ordinal $\alpha$ with the property that there is no bijective
graph map between $\alpha$ and any section of $\alpha$. \TODO{Need to define bijective
graph maps! Do we really need the definition of cardinal?!}
\end{definition}

\begin{definition}
For any pair of ordinals, $\alpha$ and $\beta$, \define{ordinal addition}, $\alpha +
\beta$, is recursively defined as follows:
\begin{enumerate}
\item $\alpha + \zeroC{} = \alpha$,
\item if $\beta$ is a successor ordinal, then $\alpha + \successor{\beta} =
\successor{\alpha + \beta}$,
\item if $\beta$ is a limit ordinal, then $\alpha + \beta = \underset{\eta <
\beta}{\union} \; \alpha + \eta$.
\end{enumerate}
\end{definition}
It is important to realize that ordinal addition is not commutative, in particular, note
that $n + \omegaC{} = \omegaC{}$ but that $\omegaC{} + n \neq \omegaC{}$. \TODO{We really
need to show that $\underset{\eta < \omegaC}{\union} \; n + \eta$ is isomorphic to
\omegaC{}. }

\TODO{ We need to show all of the following: ( \cite[Chapter 2: Ordinal
Numbers]{jech2003setTheory})
\begin{enumerate}
\item Any full subgraph of a well-ordered graph is well-ordered.
\item Any well-ordered graph, $X$, is isomorphic to a unique ordinal, $\ordinal{X}$.
\item For any pair of ordinals, $\alpha < \beta$ if and only if $\alpha \in \objects{\beta}$.
\item For any ordinal, $\objects{\alpha} = \set{ \beta \suchThat \beta < \alpha}$.
\item The successor of an ordinal, $\alpha$, is $\successor{\alpha} = \alpha + 1 = \alpha
\union \set{ \alpha }$. \TODO{This notation needs explaining or changing!}
\item If an ordinal $\alpha = \successor{\beta} = \beta + 1$ for some ordinal $\beta$ then
$\alpha$ is a \define{successor ordinal}.
\item If an ordinal $\lambda$ is not a successor ordinal for any $\beta < \lambda$, then
$\lambda$ is a \define{limit ordinal}.   (Note that it is convenient for us to define $0$
as a limit ordinal).
\item For any pair of ordinals, $\alpha$ and $\beta$, \define{ordinal addition} is defined 
as follows:
\begin{enumerate}
\item $\alpha + 0 = \alpha$
\item $\alpha + \successor{\beta} = \successor{\alpha + \beta}$
\item if $\beta$ is a limit ordinal, then $\alpha + \beta = \union_{\eta < \beta} \alpha +
\eta$
\TODO{use Jech's limit definitions -- actually we need to define our own graphical versions!}
\end{enumerate}
\end{enumerate}
}

Throughout this section we work with ordinals inside the Topos, \setC{}.  We note that
every surjection in \setC{} is a split epimorphism. This is equivalent to the internal set
theory of \setC{} allowing the Axiom of Choice, which similarly is equivalent to the
ordinals begin well ordered.  Since we are working in \setC{}, we may phrase our arguments
using classical set theory.

\begin{definition} (\cite[Definition 1, Section 2.1]{arxiv2005math0410412v2})
In \setC{}, fix $\kappa$ as a fixed \emph{small} cardinal, such as $\omega$.  
The \define{Category of symmetric $\kappa$-simplexes}, denoted \sDeltaC{}{}, is 
\begin{enumerate}
\item \textbf{objects} the set of ordinals, $0 \leq \alpha \leq \kappa$, (in \setC{})
considered as well-ordered sets, $\woSet{ \alpha } = \set{ 0 < 1 < \cdots < \alpha} =
s(\alpha)$,
\item \textbf{morphisms} the set of maps $\gamma : \woSet{ \alpha } \mapsTo \woSet{ \beta
}$ (in \setC{}), from the well-ordered set \woSet{\alpha} to the well-ordered set
\woSet{\beta}.
\end{enumerate}
The \define{Category of non-degenerate symmetric $\kappa$-simplexes}, denoted
\sNdDeltaC{}{}, is the subcategory of \sDeltaC{}{} consisting of all objects of
\sDeltaC{}{} together with only the monomorphisms of \sDeltaC{}{}.

The \define{Category of $\kappa$-simplexes}, denoted \DeltaC{}{}, is the subcategory of
\sDeltaC{}{} consisting of all of the objects of \sDeltaC{}{} together with only the order
preserving morphisms of \sDeltaC{}{}.

The \define{Category of non-degenerate $\kappa$-simplexes}, denoted \ndDeltaC{}{}, is the
subcategory of \DeltaC{}{} consisting of all of the objects of \DeltaC{}{} together with
only the \emph{strictly} order preserving morphisms of \DeltaC{}{}.
\end{definition}

For finite $n$, the graph \woSet{n} can be depicted as:
\begin{cTikzPicture}
\coordinate (v0) at (0,0);
\coordinate (v1) at (1.5,0);
\coordinate (v2) at (3,0);
\coordinate (v-n-1) at (4.5,0);
\coordinate (v-n) at (6,0);
%
\node[above] at (v0)    {$0$};
\node[above] at (v1)    {$1$};
\node[above] at (v2)    {$2$};
\node[above] at (v-n-1) {$n-1$};
\node[above] at (v-n)   {$n$};
%
\fill (v0) circle[radius=1pt];
\fill (v1) circle[radius=1pt];
\fill (v2) circle[radius=1pt];
\fill (v-n-1) circle[radius=1pt];
\fill (v-n) circle[radius=1pt];
%
\begin{scope}[->,shorten <=4pt,shorten >=4pt]
\path (v0) edge (v1);
\path (v1) edge (v2);
\path (v2) edge[dotted] (v-n-1);
\path (v-n-1) edge (v-n);
\end{scope}
\end{cTikzPicture}


The first and most important task is to understand the structure and compositional
decomposition of the hom sets of \sDeltaC{}{}, \DeltaC{}{} and \ndDeltaC{}{}.  We begin by
essentially following \cite[Notation 3 and Observations 7 and 8]{arxiv2005math0410412v2}. 
We note that, the usual simplicial face maps are not able to describe face maps which
``cross'' limit ordinals.  Instead of being able to define all face maps in terms of
``dropping'' a single vertex, we must work by ``dropping'' sets of verticies.

Before we begin, consider an orientation preserving map $m : \woSet{\alpha} \mapsTo
\woSet{\beta}$ then:
\begin{enumerate}
\item Let \image{m} denote the subset of \woSet{\beta} given by:
\begin{equation*}
	\image{m} = \set{ \gamma \in \woSet{\beta} \suchThat \thereExists \delta \in \woSet{\alpha} \forWhich m(\delta) = \gamma }
\end{equation*}
\item Let \cImage{m} denote the subset of \woSet{\beta} given by:
\begin{equation*}
	\cImage{m} = \woSet{\beta} \withOut \image{m}
\end{equation*}
Note that \cImage{m} is the subset of \woSet{\beta} which the morphism $m$ has ``dropped''.
\item Let \kernel{m} denote the subset of \woSet{\alpha} given by:
\begin{equation*}
	\kernel{m} = \set{ s(\gamma) \in \woSet{\alpha} \suchThat m(\gamma) = m(s(\gamma)) }
\end{equation*}
\end{enumerate}

\subsection{Properties of maps in \texorpdfstring{\sNdDeltaC{}{}}{sNdDelta} and
\texorpdfstring{\sDeltaC{}{}}{ndDelta}}

The following lemma is almost certainly in the category theory folklore\footnote{If I
could read French Mathematics, it is almost certainly to be found in SGA4
\cite{artinGrothendieckVerdierBourbakiDeligneSaintDonat1963SGA41},
\cite{artinGrothendieckVerdierDeligneSaintDonat1963SGA42} or
\cite{artinGrothendieckVerdierDeligneSaintDonat1963SGA43}, since I believe I have seen
references to Grothendieck making use of the category of symmetric simplexes,
\sDeltaC{}{}.}. The idea for the definitions of symmetric and non-degenerate symmetric
simplexes comes from the PhD thesis of John Shrimpton, \cite[Diagram on page
29]{shrimpton1989graphsSymmetryCatMethods}, elements of which were later published in
\cite{brownMorrisShrimptonWensley2008graphCat}. The importance of these categories of
symmetric and non-degenerate symmetric simplexes, other than possible references to their
use by Grothendieck, is that the ``standard'' Algebraic Topology definitions of
orientations of simplicial structures is wholly inadequate for our needs.

\begin{lemma} (Combing lemma)
Every morphism, $m : \woSet{\alpha} \mapsTo \woSet{\beta}$, in \sDeltaC{}{} can be
factored into an automorphism, $n$ of \woSet{\alpha} followed by an order preserving map,
$\hat{m} : \woSet{\alpha} \mapsTo \woSet{\beta}$ (i.e. a morphism of \DeltaC{}{}).
\end{lemma}
\begin{proof} 
Essentially we are combing a collection of threads one per element of \woSet{\alpha}.  As
such we can hold either end fixed and ``comb'' the threads straight. Since the kernel of
the morphism $m$, \kernel{m}, need not be empty it is prudent to hold the codomain
(\woSet{\beta}) fixed and comb back to the domain.

We will use (transfinite) induction on the elements of \woSet{\alpha}. The required
automorphism will be the product of a permutation for each element of \woSet{\alpha}. 
Since at each step of the induction, all previous ``threads'' have been ordered, each
subsequent permutation is independent of all previous permutations, so that no cycles are
longer than a two cycle.  This is important for the induction over any limit ordinals in
\woSet{\alpha}. the ordered product of the permutations at each induction step.

At each step, $0 \leq \gamma \leq \alpha$, we choose a permutation so that the appropriate
product of the inverses of the collection of permutations for all $0 \leq \lambda <
\gamma$ with the original morphism, $m$, is, when restricted to \set{\lambda \suchThat
\lambda < \gamma} is an order preserving map. We can then choose a permutation which
permutes $\gamma$ with the origin of the least element of $\tuple[]{m(\lambda)}_{\gamma
\leq \lambda \leq \alpha}$ (if there are multiple origins for such least elements choose
one).
\end{proof}

\begin{corollary}
Every morphism, $m : \woSet{\alpha} \mapsTo \woSet{\beta}$, in \sNdDeltaC{}{} can be
factored into an automorphism $n$ of \woSet{\alpha} followed by a \emph{strictly} order
preserving map, $\hat{m} : \woSet{\alpha} \mapsTo \woSet{\beta}$ (i.e. a morphism of
\ndDeltaC{}{}).
\end{corollary}
\begin{proof}
Since automorphisms are by definition monomorphisms, if $\hat{m}$ were not a monomorphism,
then the product, $m = \hat{m} \compose n$ would not be a monomorphism and hence not a
morphism of \sNdDeltaC{}{}.
\end{proof}

\begin{corollary}
Every morphism of either \sDeltaC{}{} or \sNdDeltaC{}{} (respectively) can be decomposed,
in possibly many ways, as the product of an automorphism of its domain followed by either
an order preserving or a strictly order preserving (respectively) map from its domain to
its codomain, followed by an automorphism of its codomain.
\end{corollary}

\subsection{Properties of coface maps in \texorpdfstring{\DeltaC{}{}}{Delta} and
\texorpdfstring{\ndDeltaC{}{}}{ndDelta}}

\begin{enumerate}
\item The injective maps in \DeltaC{}{} are referred to as \define{coface maps}.
\item Being injective, coface maps are \emph{strictly} order preserving and hence are also
morphisms in \ndDeltaC{}{}.  Conversely any \emph{strictly} order preserving map is
injective.  This means that \ndDeltaC{}{} only contains the injective or coface morphisms
of \DeltaC{}{}.
\item Consider $\alpha \leq \beta \leq \kappa$, and $U \subset \woSet{\beta}$ for which
$\woSet{\beta} \withOut U \isomorphic \woSet{\alpha}$. Note that $U$ and all of its
subsets are well-ordered since they are subsets of \woSet{\beta}, and hence are all
isomorphic to unique ordinals. We define the simplicial map $\coFace{\alpha}{\beta}{U} :
\woSet{\alpha} \mapsTo \woSet{\beta}$ by
	\begin{equation*}
		\coFace{\alpha}{\beta}{U}(i) = \ordinal{\set{ j \in U \suchThat j \leq i}} + i
	\end{equation*}
Note that since ordinal addition in \setC{} is noncommutative, the choice of order in the
above sum is significant.  We have chosen this order so that
$\image{\coFace{\alpha}{\beta}{U}} = \woSet{\beta}\withOut U$ (or alternatively,
$\cImage{\coFace{\alpha}{\beta}{U}} = U$) where the important case to check is, for
example, $\alpha < \omega \leq \beta$ with $\omega \notin U$.  Equally importantly, we
know that $\kernel{\coFace{\alpha}{\beta}{U}} = \emptySet{}$.
\item\label{property:unique.coface} If $D : \woSet{\alpha} \mapsTo \woSet{\beta}$ is an
injective order-preserving map, then $D = \coFace{\alpha}{\beta}{U}$ where $U = \cImage{D}
= \woSet{\beta} \withOut \image{D}$ (this generalizes \cite[page
4]{may1967simplicialObjectsInAlgTopo}).
\item For all $\alpha \leq \gamma \leq \beta \leq \kappa, \; U \subset \woSet{\beta}, \; W
\subset \woSet{\beta}$ and $V \subset \woSet{\gamma}$, if $\ordinal{\woSet{\beta} \withOut
W} = \ordinal{\woSet{\gamma}}, \; \ordinal{\woSet{\gamma} \withOut V} =
\ordinal{\woSet{\alpha}}$ and $U = \tuple[\coFace{\gamma}{\beta}{W}]{V} \union W$ then
$\coFace{\alpha}{\beta}{U} = \coFace{\gamma}{\beta}{W} \compose
\coFace{\alpha}{\gamma}{V}$.
\item It is instructive to note that, for finite $n$, $\coFace{n}{s(n)}{\set{j}}$ is the
usual coface map $D^n_j$.
\item For each ordinal, $\alpha$ and $\beta \in \woSet{\alpha}$ the simplicial map
$\epsilon^{\alpha}_{\beta} : \woSet{0} \mapsTo \woSet{\alpha}$ given by
$\epsilon^{\alpha}_{\beta}(0) = \beta$ is the $\beta$-vertex map of \woSet{\alpha}.
\end{enumerate}

\subsection{Properties of codegeneracy maps in \texorpdfstring{\DeltaC{}{}}{Delta}}

\begin{enumerate}
\item The surjective maps in \DeltaC{}{} are referred to as \define{codegeneracy maps}.

\item Consider $\beta \leq \alpha \leq \kappa$, and $V \subset \woSet{\alpha}$ for which
$\woSet{\alpha} \withOut V \isomorphic \woSet{\beta}$. Note that $\woSet{\alpha}\withOut
V$ and all of its subsets are well-ordered since they are subsets of \woSet{\alpha}, and
hence they are all isopmorphic to unique ordinals. We define the simplicial map
$\coDegeneracy{\alpha}{\beta}{V} : \woSet{\alpha} \mapsTo \woSet{\beta}$ by
	\begin{equation*}
		\coDegeneracy{\alpha}{\beta}{V}(i) = \ordinal{\set{ j \in \woSet{\alpha}\withOut V
		\suchThat j < i}}
	\end{equation*}
We note that $\kernel{\coDegeneracy{\alpha}{\beta}{V}} = V$ and that
$\image{\coDegeneracy{\alpha}{\beta}{V}} = \woSet{\beta}$.

\item\label{property:unique.codegeneracy} If $S : \woSet{\alpha} \mapsTo \woSet{\beta}$ is
a surjective order-preserving map, then $S = \coDegeneracy{\alpha}{\beta}{\kernel{S}}$
(this generalizes \cite[page 4]{may1967simplicialObjectsInAlgTopo}).

\item For all $\beta \leq \gamma \leq \alpha \leq \kappa, \; U \subset \woSet{\gamma}, \;
W \subset \woSet{\alpha}$ and $V \subset \woSet{\alpha}$ if $\ordinal{\woSet{\alpha}
\withOut W} = \ordinal{\woSet{\gamma}}, \; \ordinal{\woSet{\gamma} \withOut U} =
\ordinal{\woSet{\beta}}$ and $V = \tuple[\tuple{\coDegeneracy{\alpha}{\gamma}{W}}^{-1}]{U}
\union W$, then $\coDegeneracy{\alpha}{\beta}{V} = \coDegeneracy{\gamma}{\beta}{U}
\compose \coDegeneracy{\alpha}{\gamma}{W}$.

\item It is instructive to note that, for finite $n$, $\coDegeneracy{n}{s(n)}{\set{j}}$ is
the usual codegeneracy map $S^n_j$.

\item For each ordinal, $\alpha$ the simplicial map $\eta^{\alpha} : \woSet{\alpha}
\mapsTo \woSet{0}$ given by $\eta^{\alpha}(\beta) = 0$ is the $\alpha$-degeneracy map of
\woSet{\alpha}.
\end{enumerate}

\subsection{Properties of general maps in \texorpdfstring{\DeltaC{}{}}{Delta} and
\texorpdfstring{\ndDeltaC{}{}}{ndDelta}}

\begin{lemma}\label{lemma:factorization}
Every order preserving map, $m : \woSet{\alpha} \mapsTo \woSet{\beta}$, factors into a
unique composite $m = m^f \compose m^d$.
\end{lemma}
\begin{proof}
Let $\gamma = \ordinal{\image{m}} = \ordinal{\woSet{\alpha}\withOut\kernel{m}}$ then
$\gamma \leq \alpha$ and $\gamma \leq \beta$.  Define the map $m^d : \woSet{\alpha}
\mapsTo \woSet{\gamma}$ by $m^d = \coDegeneracy{\alpha}{\gamma}{\kernel{m}}$.  Similarly
define the map $m^f : \woSet{\gamma} \mapsTo \woSet{\beta}$ by $m^f =
\coFace{\gamma}{\beta}{\woSet{\beta}\withOut\image{m}}$.  Then we have $m = m^f \compose
m^d$.
 
By properties \ref{property:unique.coface} and \ref{property:unique.codegeneracy} of
respectively the coface and codegeneracy maps, these are the only maps, $m^f$ and $m^d$
with the respectively required image, $\image{m}$, and kernel, $\kernel{m}$.  Hence this
decomposition is unique.
\end{proof}

\begin{corollary}
If $\alpha \leq \beta \leq \kappa$, every strictly order preserving map, $m :
\woSet{\alpha} \mapsTo \woSet{\beta}$ is uniquely $m =
\coFace{\alpha}{\beta}{\woSet{\beta}\withOut\image{m}}$.
\end{corollary}

In order to understand the combinatorial structure of
\homomorphisms{\DeltaC{}{}}{\woSet{\alpha}, \woSet{\beta}}, which is the set of all order
preserving maps between a pair of objects, \woSet{\alpha} and \woSet{\beta}, in
\DeltaC{}{}, we need to add a bit more notation. Let \powerSet[c]{\woSet{\alpha}} denote
the set of complements, with respect to \woSet{\alpha}, of subsets of \woSet{\alpha}. 
Since we are working in \setC{}, we know that \powerSet[c]{\woSet{\alpha}} is trivially
isomorphic to \powerSet{\woSet{\alpha}} since every subset in \setC{} is complemented. 
Note that this is only true in a Boolean Topos such as \setC{}.

Consider $\alpha, \beta \leq \kappa$.  We note that as sets of ordinals, \woSet{\alpha}
and \woSet{\beta} are subsets of \woSet{\kappa}.  Let
\kappaPullBackCoPowerSet{\alpha}{\beta} denote the pullback of $\ordinal{\cdot} :
\powerSet[c]{\woSet{\alpha}} \mapsTo \woSet{\kappa}$ and $\ordinal{\cdot} :
\powerSet[c]{\woSet{\beta}} \mapsTo \woSet{\kappa}$.  That is,
\kappaPullBackCoPowerSet{\alpha}{\beta} is the unique object of \setC{} which makes the
following diagram commute:
\begin{cTikzPicture}
\matrix (m) [comDiagM]
{ \kappaPullBackCoPowerSet{\alpha}{\beta} & \powerSet[c]{\woSet{\alpha}} \\
  \powerSet[c]{\woSet{\beta}}             & \woSet{\kappa} \\ };
\path[comDiagP]
(m-1-1) edge node[above] {$ \pi_{\alpha} $} (m-1-2)
(m-1-1) edge node[left]      {$ \pi_{\beta} $}   (m-2-1)
(m-1-2) edge node[right]   { \ordinal{\cdot} } (m-2-2)
(m-2-1) edge node[below] { \ordinal{\cdot} } (m-2-2);
\end{cTikzPicture}
Alternatively in the internal set theory of \setC{} we have
$$ \kappaPullBackCoPowerSet{\alpha}{\beta} = \set{ (V, U) \in \powerSet{\alpha} \times
\powerSet{\beta} \suchThat \ordinal{\woSet{\alpha} \withOut V} = \ordinal{\woSet{\beta}
\withOut U}} $$

\begin{lemma}\label{lemma:combStructDeltaHom}
$$\homomorphisms{\DeltaC{}{}}{\woSet{\alpha}, \woSet{\beta}} \isomorphic_{\setC}
\kappaPullBackCoPowerSet{\alpha}{\beta}$$
\end{lemma}\begin{proof}
The proof of lemma \ref{lemma:factorization} provides the forward mapping, in \setC{},
from \homomorphisms{\DeltaC{}{}}{\woSet{\alpha}, \woSet{\beta}} to
\kappaPullBackCoPowerSet{\alpha}{\beta}.

Conversely, for each pair of ordinals, $\alpha$ and $\beta$, and pairs of sets, $U \subset
\woSet{\beta}$ and $V \subset \woSet{\alpha}$ for which $\ordinal{\woSet{\beta}\withOut U}
= \ordinal{\woSet{\alpha}\withOut V} = \gamma$ we have a unique order preserving map $m :
\woSet{\alpha} \mapsTo \woSet{\beta}$ defined by $m = d^{\gamma,\beta}_U \compose
s^{\alpha,\gamma}_V$ for which $\image{m} = \woSet{\beta}\withOut U$ and $\kernel{m} = V$.
\end{proof}

\begin{corollary}
$$\homomorphisms{\ndDeltaC{}{}}{\woSet{\alpha}, \woSet{\beta}} \isomorphic_{\setC} 
\tuple[\pi^{-1}_{\alpha}]{\emptySet} \isomorphic_{\setC}
\ndKappaPullBackCoPowerSet{\alpha}{\beta}$$
\end{corollary}
\begin{proof}
Since any morphism, $m \in \homomorphisms{\ndDeltaC{}{}}{\woSet{\alpha}, \woSet{\beta}}$,
is strictly order preserving and hence injective, we know that $\kernel{m} = \emptySet$. 
Tracing through the proof of lemma \ref{lemma:combStructDeltaHom} as well as the commuting
diagram in the definition of \kappaPullBackCoPowerSet{\alpha}{\beta} with $\kernel{m} = V
= \emptySet \subset \woSet{\alpha}$ provides that required isomorphisms in \setC{}.
\end{proof}

If we are given a map which is the composition of a coface followed by a codegeneracy, that 
is the ``wrong way around'', what is its unique coface/codegeneracy factorization given by 
lemma \ref{lemma:factorization}?  Consider $\alpha, \beta, \gamma \leq \kappa, \; U \subset 
\woSet{\gamma}$ and $V \subset \woSet{\gamma}$ for which $\alpha \leq \gamma, \; 
\ordinal{\woSet{\alpha}} = \ordinal{\woSet{\gamma}\withOut U}$ and $\beta \leq \gamma, \; 
\ordinal{\woSet{\beta}} = \ordinal{\woSet{\gamma}\withOut V}$, then the morphism 
$$m^{\alpha,\gamma,\beta}_{U,U} = \coDegeneracy{\gamma}{\beta}{V} \compose 
\coFace{\alpha}{\gamma}{U} : \woSet{\alpha} \mapsTo \woSet{\beta}$$ is well defined. 

Let $\tilde{U} = \image{m^{\alpha,\gamma,\beta}_{V,U}}$, then:
\begin{align*}
\tilde{U} & = \image{m^{\alpha,\gamma,\beta}_{V,U}} \\
  & = \image{ \coDegeneracy{\gamma}{\beta}{V} \compose \coFace{\alpha}{\gamma}{U}} \\
  & = \tuple[\coDegeneracy{\gamma}{\beta}{V}]{\image{\coFace{\alpha}{\gamma}{U}}} \\
  & = \tuple[\coDegeneracy{\gamma}{\beta}{V}]{\woSet{\gamma} \withOut U}
\end{align*}

Similarly, let $\tilde{V} = \kernel{m^{\alpha,\gamma,\beta}_{V,U}}$, then
\begin{align*}
\tilde{V} & = \kernel{m^{\alpha,\gamma,\beta}_{V,U}} \\
  & = \kernel{ \coDegeneracy{\gamma}{\beta}{V} \compose \coFace{\alpha}{\gamma}{U}} \\
  & = 
  \tuple[\tuple{\coFace{\alpha}{\gamma}{U}}^{-1}]{\kernel{\coDegeneracy{\gamma}{\beta}{V}}} 
  \\
  & = \tuple[\tuple{\coFace{\alpha}{\gamma}{U}}^{-1}]{V}
\end{align*}

Finally let, $\tilde{\gamma} = \ordinal{\image{m^{\alpha,\gamma,\beta}_{V,U}}} = 
\ordinal{\woSet{\alpha}\withOut \kernel{m^{\alpha,\gamma,\beta}_{V,U}}}$, then
\begin{equation*}
\coDegeneracy{\gamma}{\beta}{V} \compose \coFace{\alpha}{\gamma}{U} = 
\coFace{\tilde{\gamma}}{\beta}{\tilde{U}} \compose 
\coDegeneracy{\alpha}{\tilde{\gamma}}{\tilde{V}}
\end{equation*}

Consider a morphism, $m : \woSet{\alpha} \mapsTo \woSet{\beta}$, of \DeltaC{}{}.
\begin{enumerate}
\item If $\beta < \alpha \leq \kappa$ then \kernel{m} is non empty.  This means that there 
are no injective order preserving maps from \woSet{\alpha} to \woSet{\beta}, and hence 
\homomorphisms{\ndDeltaC{}{}}{\woSet{\alpha}, \woSet{\beta}} is empty.
%
\item If $\alpha = \beta$ then $m = \identity{\woSet{\alpha}}$ is the only \emph{strictly} 
order preserving morphism from \woSet{\alpha} to itself, and hence 
$\homomorphisms{\ndDeltaC{}{}}{\woSet{\alpha},\woSet{\alpha}} = 
\set{\identity{\woSet{\alpha}}}$
%
\item If $\alpha < \beta \leq \kappa$ then \cImage{m} is non empty and hence there are no 
surjective order preserving maps from \woSet{\alpha} to \woSet{\beta}.
\end{enumerate}

\subsection{Idempotents of \ndDeltaC{}{}, \DeltaC{}{}, \sDeltaC{}{}}

Following \cite[Chapter 5]{reyesReyesZolfaghari2004presheaves} we are interested in the 
generic figures of each of \ndDeltaC{}{}, \DeltaC{}{} and \sDeltaC{}{}.  These generic 
figures are essentially defined by the idempotents of each category. \TODO{The only 
idempotents of \ndDeltaC{}{} are the identity morphisms of each \woSet{\alpha}.  The 
idempotents of \DeltaC{}{} are the different ways that simplexes of higher dimension can be 
(singularly) embedded in simplexes of lower dimension, while those of \sDeltaC{}{} are the 
symmetrization of the idempotents of \DeltaC{}{}. (See 
\cite[page90]{reyesReyesZolfaghari2004presheaves}).  For both \DeltaC{}{} and \sDeltaC{}{}, 
cannonical inclusion of the original category in its respective Cauchy completion is full 
and faithful and essentially surjective by arguments similar to that given in the discussion 
to \cite[Propostion 5.2.1]{reyesReyesZolfaghari2004presheaves} and just before \cite[Remark 
5.2.5]{reyesReyesZolfaghari2004presheaves}.  This means that both categories and their 
respective Cauchy completions are equivalent as categories.}

\TODO{Introduce and define the Cauchy completion of a small category using ``splitting'' of 
idempotents as discussed in \cite[Section 5.2]{reyesReyesZolfaghari2004presheaves}.}

\section{The graph of automorphisms of \texorpdfstring{$\Delta$}{Delta}}

Our aim in this section is to follow the work of John Shrimpton (\cite[Chapter 
5]{shrimpton1989graphsSymmetryCatMethods}, \cite[Sections 3, 
4]{brownMorrisShrimptonWensley2008graphCat}, see also \cite[Section 2]{brown1994symmetry}) 
to define the group-graph of automorphisms a graph, in our case the group-graph of 
automorphisms of \DeltaC{}{}.

\section{The graph of ordinals}

The definition of the collection of test-$1$-paths is fairly straight forward. Fix the
first (non-zero) limit ordinal $\omega$. For any $0 \leq n \leq \omega$ let \woSet{n}
denote the totally ordered graph defined as
\begin{description}
\item[objects] there is an object for each $0 \leq i < 1+n$,
\item[morphisms] for each object $0 \leq i < 1+n$ there is the morphism $\hat{m}_i : i 
\mapsTo i$, if $0 \leq 1+i < 1+n$ then there is  also the morphism $m_i : i \mapsTo i+1$.
\end{description}
together with the obvious \domain{\cdot}, \coDomain{\cdot} and \Identity{\cdot} mappings.  
It is important to note that ordinal addition is not commutative, see Appendix 
\ref{chap:ordinals}.  In particular $1 + \omega = \omega$.  This means that our choice of 
order in the condition, $0 \leq i < 1+n$ is significant. For finite $n$, the graph \woSet{n} 
can be depicted as:
\begin{cTikzPicture}
\coordinate (v0) at (0,0);
\coordinate (v1) at (1.5,0);
\coordinate (v2) at (3,0);
\coordinate (v-n-1) at (4.5,0);
\coordinate (v-n) at (6,0);
%
\node[above] at (v0)    {$0$};
\node[above] at (v1)    {$1$};
\node[above] at (v2)    {$2$};
\node[above] at (v-n-1) {$n-1$};
\node[above] at (v-n)   {$n$};
%
\fill (v0) circle[radius=1pt];
\fill (v1) circle[radius=1pt];
\fill (v2) circle[radius=1pt];
\fill (v-n-1) circle[radius=1pt];
\fill (v-n) circle[radius=1pt];
%
\begin{scope}[->,shorten <=4pt,shorten >=4pt]
\path (v0) edge (v1);
\path (v1) edge (v2);
\path (v2) edge[dotted] (v-n-1);
\path (v-n-1) edge (v-n);
\end{scope}
\end{cTikzPicture}

Consider a pair of test paths, \woSet{\alpha} and \woSet{\beta} for $\alpha, \beta \leq 
\omega$.  With out loss of generality we can assume that $\alpha \leq \beta$. \TODO{There is 
are $\beta - \alpha + 1$ inclusion graph maps from \woSet{\alpha} into \woSet{\beta}.  Since 
we have included the degenerate identity morphism, there are also all of the surjections 
contained in \DeltaC{}{}.  This asymmetry shows that our current definition of Graph is 
incomplete. The more convenient definition of graph is as a category, or alternatively we 
are too permissive in our definition of simplicial set and really only want delta sets.  }

\TODO{Show that any two test paths can be combined into a composite test path.  Show that 
this composition is associative. Show that all test paths can be generated by composition of 
the test path \woSet{1}.}

The graph, \mathCategory{T1P}, of test-$1$-paths is defined as
\begin{description}
\item[objects] for any $0 \leq n \leq \omega$, the graphs \woSet{n} defined above,
\item[morphisms] any graph map between a pair of objects, \woSet{n} and \woSet{m},
\end{description}
with the obvious \domain{\cdot}, \coDomain{\cdot} and \Identity{\cdot} mappings.  

While this is an explicit construction, when we get far enough in our exposition to be able 
to define the following concepts, we will see that the important properties of 
\mathCategory{T1P} are
\begin{enumerate}
\item for each object, \woSet{n}, considered as a graph, is connected,
\item for each object, \woSet{n}, considered as a graph, the fundmental groupoid, 
\fundGroupoid{\woSet{n}}, is trivial,
\item for each object, \woSet{n}, considered as a graph, is unbranched,
\item for each object, \woSet{n}, can always be extended (i.e. can be embedded in a 
``larger'' object \woSet{m}), 
\item \mathCategory{T1P} is a \emph{poset},  (in fact it is a Domain),
\item (the located \mathCategory{T1P} is shift invariant \TODO{define this shift map}),
\item if there is a morphism $f : \woSet{n} \mapsTo \woSet{n}$ then there exists a unique 
morphism, $f' : \woSet{n} \mapsTo \woSet{m}$ (\woSet{n} factors through ?? (conversely 
factors through ??) \TODO{this is really not complete but I am not yet sure how best to say 
it},  \TODO{This effectively show that all objects are ``flat'' --- that is there is only 
one way to complete/extend them},
\item if there are morphisms $f : \woSet{n} \mapsTo \woSet{m}$ and $ f' : \woSet{m} \mapsTo 
\woSet{m'}$ then there is a unique composition \TODO{again I do not (yet) know how to say 
this correctly.} \TODO{similarly this shows, in some sense, the ``flatness'' of the objects}
\end{enumerate}
\TODO{explicitly name this graph, as a Symbolic Dynamics.}

\TODO{QUESTION: how do we define the Symbolic Dynamics notion of overlapping words and or 
magical words.}

\TODO{We want to stress the collections of paths rather than simply composition.  The 
example of Symbolic Dynamics non-subshifts of finite type such as minimal shifts, 
\cite{lindMarcus1995a}, is key here.  There is unlikely to be \emph{any} ``composition'' 
operator for minimal shifts.... however they \emph{are} interesting collections of paths.  
Composition operators will be definable for the equivalent of ``subshifts of finite type'' 
which are essentially algebraic algebras defined by generators (words) and relations 
(constraints on which words can be composed).}

\section{Introduction: The category of simplicial sets}

\TODO{We need a better intro here!}

\TODO{Discuss the fact that working in \setC{} means we can use the local set theory with 
\emph{just} points (i.e. \setC{} has enough points). Should also discuss subsets in Topos 
and \setC{}.}

\section{The category of simplicial sets}

\begin{definition}
A \define{(transfinite) simplicial set}, $X$, is an object in the functor category, 
\opFuncCat{\DeltaC{}{}}{\setC{}}, or alternatively a covariant functor $X : 
\opposite{\DeltaC{}{}} \mapsTo{} \setC{}$ (or a contravariant functor $X : \DeltaC{}{} 
\mapsTo \setC{}$).  Explicitly it is defined by:
\begin{enumerate}
\item For each object, \woSet{\alpha} in \DeltaC{}{}, there is an associated set 
$\tuple[X]{\woSet{\alpha}} = X_{\woSet{\alpha}} = X_{\alpha}$ in \setC{}.
\item For each morphism, $m : \woSet{\alpha} \mapsTo \woSet{\beta}$, in \DeltaC{}{}, there 
is an associated map $\tuple[X]{m} = X_m : X_{\beta} \mapsTo X_{\alpha}$ in \setC{} for 
which the following diagram commutes:
%
\begin{cTikzPicture}
\matrix (m) [comDiagM]
{ \woSet{\alpha} & X_{\alpha} \\
  \woSet{\beta}  & X_{\beta} \\ };
\path[comDiagP]
(m-1-1) edge node[above] { $X$ }   (m-1-2)
(m-1-1) edge node[left]  { $m$ }   (m-2-1)
(m-2-2) edge node[right] { $X_m$ } (m-1-2)
(m-2-1) edge node[below] { $X$ }   (m-2-2);
\end{cTikzPicture}
%
\item If $m : \woSet{\alpha} \mapsTo \woSet{\beta}$ and $m' : \woSet{\beta} \mapsTo 
\woSet{\gamma}$ are composable morphisms in \DeltaC{}{} then $X_m : X_{\beta} \mapsTo 
X_{\alpha}$ and $X_{m'} : X_{\gamma} \mapsTo X_{\beta}$ are composable maps and 
$X_{m\compose m'} = X_{m'} \compose X_m$ in \setC{}. This means that the following diagram 
commutes:
%
\begin{cTikzPicture}
\matrix (m) [comDiagM]
{ \woSet{\alpha} & X_{\alpha} \\
  \woSet{\beta}  & X_{\beta}  \\ 
  \woSet{\gamma} & X_{\gamma} \\ 
};
\path[comDiagP]
(m-1-1) edge node[above] { $X$ }      (m-1-2)
(m-2-1) edge node[above] { $X$ }      (m-2-2)
(m-3-1) edge node[below] { $X$ }      (m-3-2)
%
(m-1-1) edge                node[right]  { $m$ }             (m-2-1)
(m-2-1) edge                node[right]  { $m'$ }            (m-3-1)
(m-1-1) edge[bend right=60] node[left]   { $m' \compose m$ } (m-3-1)
%
(m-2-2) edge                node[left]  { $X_m$ }                 (m-1-2)
(m-3-2) edge                node[left]  { $X_{m'}$ }              (m-2-2)
(m-3-2) edge[bend right=60] node[right] { $X_m \compose X_{m'}$ } (m-1-2);
\end{cTikzPicture}
%
\item For each coface map, \coFace{\alpha}{\beta}{U} in \DeltaC{}{}, the image via the 
functor $X$, $\tuple[X]{\coFace{\alpha}{\beta}{U}} = X_{\coFace{\alpha}{\beta}{U}} = 
\face{\alpha}{\beta}{U}$, is a \define{face} map in $X$.
\item For each codegeneracy map, \coDegeneracy{\alpha}{\beta}{V} in \DeltaC{}{}, the image 
via the functor $X$, $\tuple[X]{\coDegeneracy{\alpha}{\beta}{V}} = 
X_{\coDegeneracy{\alpha}{\beta}{V}} = \degeneracy{\alpha}{\beta}{V}$, is a 
\define{degeneracy} map in $X$.
\item For each morphism, $m : \woSet{\alpha} \mapsTo \woSet{\beta}$, with its unique 
coface/codegeneracy factorization, $m = 
\coFace{\gamma}{\beta}{\woSet{\beta}\withOut\image{m}} \compose 
\coDegeneracy{\alpha}{\gamma}{\kernel{m}}$ we have a correspondingly unique degeneracy/face 
factorization, 
\begin{align*}
\tuple[X]{m} & = \tuple[X]{\coFace{\gamma}{\beta}{\woSet{\beta}\withOut\image{m}} \compose 
                 \coDegeneracy{\alpha}{\gamma}{\kernel{m}}} \\
             & = \tuple[X]{\coDegeneracy{\alpha}{\gamma}{\kernel{m}}} \compose 
                 \tuple[X]{\coFace{\gamma}{\beta}{\woSet{\beta}\withOut\image{m}}}  \\
             & = \degeneracy{\alpha}{\gamma}{\kernel{m}} \compose 
                 \face{\gamma}{\beta}{\woSet{\beta}\withOut \image{m}}
\end{align*}
\end{enumerate}
For each $\alpha \leq \kappa$, the elements of $X_{\alpha}$ are 
\define{$\alpha$-simplicies}. \TODO{Define dimension of a simplex to be the $\alpha$ for 
which it is contained in $X_{\alpha}$.}
\end{definition}

\begin{definition} \TODO{Comment on naturality and refer to \cite[Chapter 
7]{awodey2006catTh}}
A \define{(transfinite) simplicial map}, $f : X \mapsTo Y$ is defined by:
\begin{enumerate}
\item For each object, \woSet{\alpha}, in \DeltaC{}{}, there is an associated map 
$f_{\woSet{\alpha}} = f_{\alpha} : X_{\alpha} \mapsTo Y_{\alpha}$.
\item If $m : \woSet{\alpha} \mapsTo \woSet{\beta}$ is a morphism (order preserving map) in 
\DeltaC{}{} then $f_{\alpha} \compose X_m = Y_m \compose f_{\beta}$, that is the diagram on 
the right commutes:
%
\begin{cTikzPicture}
\matrix (m) [comDiagM]
{ \woSet{\alpha} & & X_{\alpha} & Y_{\alpha} \\
  \woSet{\beta}  & & X_{\beta}  & Y_{\beta}  \\ };
\begin{scope}[comDiagP]
\path
(m-1-1) node (delta)   {} (m-2-1)
(m-1-3) node (simplex) {} (m-2-3);
\path
(m-1-1) edge node (delta)   {}               (m-2-1)
%
(m-1-3) edge node[above]    { $f_{\alpha}$ } (m-1-4)
(m-2-3) edge node (simplex) {}               (m-1-3)
(m-2-4) edge node[right]    { $Y_m$ }        (m-1-4)
(m-2-3) edge node[below]    { $f_{\beta}$ }  (m-2-4)
%
(delta) edge[densely dotted, shorten <=5pt, shorten >=20pt] (simplex);
%
\node[left] at (delta)   { $m$ };
\node[left] at (simplex) { $X_m$ };
\end{scope}
\end{cTikzPicture}
%
\end{enumerate}
\end{definition}

\begin{lemma}
The composition of two simplicial maps is a simplicial map.
\end{lemma}
\begin{proof}
\TODO{complete this proof! DO WE REALLY NEED THIS?}
\end{proof}

\TODO{We need to discuss the corresponding functor category of non-degenerate simplicial 
sets (Freidman calls Delta sets, \cite[Section 2.3]{arxiv2008math0809.4221v1}).  Morphisms 
of \funcCat{\ndDeltaC{}{}}{\setC{}} are not as ``flexible'' as morphisms of 
\funcCat{\DeltaC{}{}}{\setC{}}. The map identify ``whole'' simplicies but they may not 
identify subsimplicies (faces) unless the pair of containing simplicies are also identified. 
Certainly there is an adjoint pair of functors between the categories of simplicial sets and 
non-degenerate simplicial sets, and that, on objects, this adjoint pair is an isomorphism, 
and that every morphism of non-degenerate simplicial sets extends (freely) to a morphism of 
simplicial sets.  However on morphisms this mapping is injective but not surjective.}

\begin{definition}
The category of \define{simplicial sets}, \simpC{}, is the functor category, 
\funcCat{\opposite{\DeltaC{}{}}}{\setC{}}.  Explicitly, the objects of \simpC{} are 
(covariant) functors $X : \opposite{\DeltaC{}{}} \mapsTo \setC{}$ and the morphisms of 
\simpC{} are natural transformations between two functors, $X, Y : \opposite{\DeltaC{}{}} 
\mapsTo \setC{}$.
\end{definition}

\begin{definition}
Consider a simplicial map, $f : X \mapsTo Y$.  The \define{image of a simplicial map}, 
\image{f}, is defined by:
\begin{enumerate}
\item For each object, \woSet{\alpha}, in \DeltaC{}{}, there is a set 
$f_{\alpha}(X_{\alpha}) \subset Y_{\alpha}$.
\item For each morphism, $m : \woSet{\alpha} \mapsTo \woSet{\beta}$, in \DeltaC{}{}, there 
is a restriction map, $\restrictedTo{Y_m}{f_{\beta}(X_{\beta})} : f_{\beta}(X_{\beta}) 
\mapsTo f_{\alpha}\compose X_m (X_{\beta}) \subset f_{\alpha}(X_{\alpha})$ 
\end{enumerate}
\end{definition}

\begin{lemma}
The image of a simplicial map is a simplicial set.
\end{lemma}
\begin{proof} Consider a simplicial map, $f : X \mapsTo Y$.
\begin{enumerate}
\item For each $0$-simplex, \woSet{\alpha}, in \DeltaC{}{}, let $\image{f}_{\alpha} = 
f_{\alpha}(X_{\alpha})$.
\item For each $1$-simplex, $m : \woSet{\alpha} \mapsTo \woSet{\beta}$, in \DeltaC{}{}, let 
$\image{f}_m = \restrictedTo{Y_m}{f_{\beta}(X_{\beta})}$.
\end{enumerate}
All we need to show is that composable $m$ and $m'$ are still composable... \TODO{show this!}
\end{proof}

\section{The Yoneda embedding: the standard simplicies}

\TODO{We need to define the subset of non-degenerate morphisms between simplicial sets. We 
need to note that for \DeltaC{\alpha}{} for, $\alpha < \beta \leq \kappa$, \emph{all} 
morphisms in \homomorphisms{\DeltaC{}{}}{\woSet{\beta}, \woSet{\alpha}} are 
\emph{degenerate}. For $\alpha \leq \kappa$, a simplicial set is an 
\define{$\alpha$-simplicial set}, if $X_{\beta} = \emptySet$ for all $\alpha < \beta$.  This 
last statement is incorrect.  We need to define non-degenerate simplicies and then define an 
$\alpha$-simplicial set to be one for which all (sub)$\beta$-simplicies for $\alpha < \beta 
\leq \kappa$ are degenerate. Define dimension of a non-degenerate simplex to be the $\alpha$ 
for which...}

\TODO{Need to work in Goerss and Jardine's Example 1.7 at this point.  And provide explicit 
definitions of degenerate and non-degenerate and elemental simplicies.}  \TODO{The concept 
of collapsing scheme looks useful to our work, see \cite{citterio2001a}} \TODO{Allegretti's 
work, \cite{allegretti2008simplicialSets} provides an interesting link between Van Kempen's 
theorem, groupoids and classification in the context of our work.}

Since it is key to most of our subsequent work, we now work through Goerss and Jardine's 
Example 1.7, \cite[page 6]{goerssJardin1999SimplicialHomotopyTh}, in some detail.  This is 
essentially a standard application of the Categorical concepts of representable Functors, 
the Yoneda Lemma, and the Yoneda Embedding.  A good exposition of these ideas can be found 
in \cite[Section 4.5]{barrWells1995catTh} or \cite[Chapter 8]{awodey2006catTh}.  
\TODO{Provide other expositions.}

Consider $\alpha \leq \kappa$.  Since \woSet{\alpha} \emph{is} a set in \setC{}, we know 
that \homomorphisms{\DeltaC{}{}}{\cdot, \woSet{\alpha}} is a (covariant) functor from 
\opposite{\DeltaC{}{}} to \setC{}, and hence is an object of 
\opFuncCat{\DeltaC{}{}}{\setC{}}, which in turn means that it is a simplicial set. We define 
$\DeltaC{\alpha}{} = \homomorphisms{\DeltaC{}{}}{\cdot, \woSet{\alpha}}$ to be the 
\define{$\alpha$-standard simplex}. 

In Categorical terms $ \DeltaC{\alpha}{} = \tuple[y]{\woSet{\alpha}}= 
\homomorphisms{\DeltaC{}{}}{\cdot, \woSet{\alpha}} $ is the image of the Yondea 
\emph{Embedding}, $ y : \DeltaC{}{} \mapsTo \opFuncCat{\DeltaC{}{}}{\setC{}} $.  It is a 
representable functor which is represented by the object \woSet{\alpha} in \setC{}.  
Explicitly:
\begin{enumerate}
\item For an object, \woSet{\beta} in \DeltaC{}{},  
$\tuple[\homomorphisms{\DeltaC{}{}}{\cdot, \woSet{\alpha}}]{\woSet{\beta}} = 
\homomorphisms{\DeltaC{}{}}{\woSet{\beta}, \woSet{\alpha}}$ which since \setC{} is a locally 
small category, is a set in \setC{}. 

In particular, we know from our work in \DeltaC{}{}, this is the \emph{set} of order 
preserving morphisms from the \emph{set} \woSet{\beta} to the \emph{set} \woSet{\alpha} 
which is isomorphic, in \setC{}, to the pullback, \kappaPullBackCoPowerSet{\beta}{\alpha}. 

\item For a morphism, $m : \woSet{\beta} \mapsTo \woSet{\delta}$ in \DeltaC{}{}, 
$\tuple[\homomorphisms{\DeltaC{}{}}{\cdot, \woSet{\alpha}}]{m} = 
\homomorphisms{\DeltaC{}{}}{m, \woSet{\alpha}} : \homomorphisms{\DeltaC{}{}}{\woSet{\delta}, 
\woSet{\alpha}} \mapsTo \homomorphisms{\DeltaC{}{}}{\woSet{\beta}, \woSet{\alpha}}$. 

In particular for a morphism $m' : \woSet{\delta} \mapsTo \woSet{\alpha}$, we have that 
$\tuple[\homomorphisms{\DeltaC{}{}}{m, \woSet{\alpha}}]{m'} = m' \compose m$ which is a 
morphism in \homomorphisms{\Delta}{\woSet{\beta}, \woSet{\alpha}} as required.

Combinatorially, we can also consider \homomorphisms{\DeltaC{}{}}{m, \woSet{\alpha}} as a 
map between the pullbacks, \kappaPullBackCoPowerSet{\delta}{\alpha}, and 
\kappaPullBackCoPowerSet{\beta}{\alpha}.
If the above $m$ and $m'$ have their unique coface/codegeneracy factorizations, $$m = 
\coFace{\gamma}{\beta}{\woSet{\beta}\withOut\image{m}} \compose 
\coDegeneracy{\delta}{\gamma}{\kernel{m}}$$ and $$m' = 
\coFace{\gamma'}{\delta}{\woSet{\delta}\withOut\image{m'}} \compose 
\coDegeneracy{\alpha}{\gamma'}{\kernel{m'}}$$ respectively, then, working in \DeltaC{}{}, we 
have 
\begin{align*}
m'' & = m' \compose m \\
      & = \coFace{\gamma'}{\delta}{\woSet{\delta}\withOut\image{m'}} \compose 
      \coDegeneracy{\alpha}{\gamma'}{\kernel{m'}}
            \compose 
            \coFace{\gamma}{\beta}{\woSet{\beta}\withOut\image{m}} \compose 
            \coDegeneracy{\delta}{\gamma}{\kernel{m}}
\end{align*}
\TODO{SORT THIS OUT!  -- It should be ``easy'', but my brain is dead!  To do this correctly 
and to raise the combinatorial profile of this work we need a notation for these pullbacks!}
\end{enumerate}

Similarly we can define $ \ndDeltaC{\alpha}{} = \homomorphisms{\ndDeltaC{}{}}{\cdot, 
\woSet{\alpha}}$ to be the \define{$\alpha$-standard non-degenerate simplex}. Again, in 
Categorical terms $ \ndDeltaC{\alpha}{} = \tuple[y_{nd}]{\woSet{\alpha}}= 
\homomorphisms{\ndDeltaC{}{}}{\cdot, \woSet{\alpha}} $ is the image of the Yondea 
\emph{Embedding}, $ y_{nd} : \ndDeltaC{}{} \mapsTo \opFuncCat{\ndDeltaC{}{}}{\setC{}} $.  It 
is a representable functor which is represented by the object \woSet{\alpha} in \setC{}.  
Explicitly:
\begin{enumerate}
\item For an object, \woSet{\beta} in \ndDeltaC{}{},  
$\tuple[\homomorphisms{\ndDeltaC{}{}}{\cdot, \woSet{\alpha}}]{\woSet{\beta}} = 
\homomorphisms{\ndDeltaC{}{}}{\woSet{\beta}, \woSet{\alpha}}$ which since \setC{} is a 
locally small category, is a set in \setC{}. 

In particular, we know from our work in \ndDeltaC{}{}, this is the \emph{set} of 
\emph{strictly} order preserving morphisms from the \emph{set} \woSet{\beta} to the 
\emph{set} \woSet{\alpha} which is isomorphic, in \setC{}, to the pullback, 
\ndKappaPullBackCoPowerSet{\beta}{\alpha}. 
\item For a morphism, $m : \woSet{\beta} \mapsTo \woSet{\delta}$ in \ndDeltaC{}{}, 
$\tuple[\homomorphisms{\ndDeltaC{}{}}{\cdot, \woSet{\alpha}}]{m} = 
\homomorphisms{\ndDeltaC{}{}}{m, \woSet{\alpha}} : 
\homomorphisms{\ndDeltaC{}{}}{\woSet{\delta}, \woSet{\alpha}} \mapsTo 
\homomorphisms{\ndDeltaC{}{}}{\woSet{\beta}, \woSet{\alpha}}$. 

In particular for a morphism $m' : \woSet{\delta} \mapsTo \woSet{\alpha}$,  we have that 
$\tuple[\homomorphisms{\ndDeltaC{}{}}{m, \woSet{\alpha}}]{m'} = m' \compose m$ which is a 
morphism in \homomorphisms{\ndDeltaC{}{}}{\woSet{\beta}, \woSet{\alpha}} as required.

Combinatorially, we can also consider \homomorphisms{\ndDeltaC{}{}}{m, \woSet{\alpha}} as a 
map between the pullbacks, \ndKappaPullBackCoPowerSet{\delta}{\alpha}, and 
\ndKappaPullBackCoPowerSet{\beta}{\alpha}. If the above $m$ and $m'$ have their unique 
coface/codegeneracy factorizations, $$m = 
\coFace{\gamma}{\beta}{\woSet{\beta}\withOut\image{m}}$$ and $$m' = 
\coFace{\gamma'}{\delta}{\woSet{\delta}\withOut\image{m'}}$$ respectively, then, working in 
\DeltaC{}{}, we have 
\begin{align*}
m'' & = m' \compose m \\
     & = \coFace{\gamma'}{\delta}{\woSet{\delta}\withOut\image{m'}}
            \compose 
            \coFace{\gamma}{\beta}{\woSet{\beta}\withOut\image{m}} \\
       & = \coFace{?}{?}{?}
\end{align*}
\TODO{SORT THIS OUT!  -- It should be ``easy'', but my brain is dead!  To do this correctly 
and to raise the combinatorial profile of this work we need a notation for these pullbacks!}
\end{enumerate}

\begin{definition}
A \define{simplex category} (respectively a \define{non-degenerate simplex category}) is 
given by
\begin{enumerate}
\item Objects: 
\end{enumerate}
\end{definition}

\TODO{We need to expand upon Goerss and Jardine's classifying maps example.  Then note 
define a simplical set to be an $\alpha$-simplicial set if, for $\alpha < \beta \leq 
\kappa$, the classifying maps, \ndHomomorphisms{\simpC{}}{\DeltaC{\beta}{}, X}, is empty.}

For a given simplicial set, $F$ in \simpC{}, the Yoneda \emph{Lemma} implies that, for each 
$\alpha \leq \kappa$, the simplicial maps whose domains are the $\alpha$-standard simplex, 
$\DeltaC{\alpha}{} \mapsTo F$, classifies the $\alpha$-simplicies of $F$ in that there is a 
natural isomorphism between $$ \homomorphisms{\simpC{}}{\DeltaC{\alpha}{}, F} \isomorphic 
F_{\alpha} $$ Given its importance to our subsequent work, we explicitly expand out what is 
essentially the proof of the Yondea Lemma in our specific case as follows. See \cite[Lemma 
8.2]{awodey2006catTh} for a good detailed general proof of the Yondea Lemma.

We are interested in defining a mapping $\eta_{\woSet{\alpha},F} : 
\homomorphisms{\DeltaC{}{}}{\woSet{\alpha},F} \mapsTo \tuple[F]{\woSet{\alpha}}$. Fix both a 
simplicial set $F$ and $\alpha \leq \kappa$. We begin by considering an arbitrary simplicial 
map, $\theta : \DeltaC{\alpha}{} \mapsTo F$ in \homomorphisms{\simpC{}}{\DeltaC{\alpha}{}, 
F}. That is, $\theta$ is a natural transformation from the functor $\DeltaC{\alpha}{} = 
\homomorphisms{\DeltaC{}{}}{\cdot, \woSet{\alpha}} : \opposite{\DeltaC{}{}} \mapsTo 
\woSet{\alpha}$ \emph{to} the functor $F : \opposite{\DeltaC{}{}} \mapsTo \setC{}$. Since we 
are interested in showing a mapping into the set $\tuple[F]{\woSet{\alpha}} = F_{\alpha}$ we 
should consider $\tuple[\DeltaC{\alpha}{}]{\woSet{\alpha}} = 
\homomorphisms{\DeltaC{}{}}{\woSet{\alpha},\woSet{\alpha}}$ which is the set of order 
preserving maps from \woSet{\alpha} to itself.  However the identity map, 
\identity{\woSet{\alpha}}, \emph{is} the only one such order preserving mapping. 

\TODO{Complete this discussion using \cite[Lemma 8.2]{awodey2006catTh}.}
\TODO{We need to work out in detail the work of \cite[Lemma 
2.1]{goerssJardin1999SimplicialHomotopyTh} which shows essentially that a simplicial set, 
$X$, is the ``sum'' of its $\DeltaC{\alpha}{} \mapsTo X$ ``parts'' (see also 
\cite[Proposition 8.10]{awodey2006catTh}).} \TODO{nLab:subfunctor 

A subfunctor of a functor G:C?D is a pair (F,i) where F:C?D is a functor and i:F?G is a 
natural transformation such that its components i M:F(M)?G(M) are monic. In fact one often 
by a subfunctor means just an equivalence class of such monic natural transformations; 
compare subobject.

In a concrete category with images one can choose a representative of a subfunctor where the 
components of i are genuine inclusions of the underlying sets; then a subfunctor is just a 
natural transformation whose components are inclusions. The naturality in terms of concrete 
inclusions just says that for all f:c?d, F(f)=G(f)? F(c). If the set-theoretic circumstances 
allow consideration of a category of functors, then a subfunctor is a subobject in such a 
category.

A subfunctor (F,i) of the identity id C:C?C in a category with images is an often used case: 
it amounts to a natural assignment c?F(c)?ic of a subobject to each object c in C. For 
concrete categories with images then F(f)=f? F(c).

A subfunctor of a representable functor Hom(?,x) is precisely a sieve over the representing 
object x.

  Revised on August 29, 2009 20:32:10 
  by Toby Bartels}
  
\section{A bestiary of simplicial sets}  

With this accumulated knowledge, we can provide the following ``picture'' of \DeltaC{2}{}

\begin{cTikzPicture}
\coordinate (O) at (0,0);

\coordinate (A0) at (0,2);
\coordinate (A1) at (0,4);
\coordinate (A2) at ($ (A0) + (O)!2cm!60:(A0) $);
\coordinate (A3) at ($ (A0) + (O)!2cm!-60:(A0) $);

\coordinate (B0) at ($ (O)!2cm!120:(A0) $);
\coordinate (B1) at ($ (O)!4cm!120:(A0) $);
\coordinate (B2) at ($ (B0) + (O)!2cm!60:(A0) $);
\coordinate (B3) at ($ (B0) - (O)!2cm!(A0) $);

\coordinate (C0) at ($ (O)!2cm!-120:(A0) $);
\coordinate (C1) at ($ (O)!4cm!-120:(A0) $);
\coordinate (C2) at ($ (C0) + (O)!2cm!-60:(A0) $);
\coordinate (C3) at ($ (C0) - (O)!2cm!(A0) $);

\draw (A0) edge node (AB0) {} (B0);
\draw (B0) edge node (BC0) {} (C0);
\draw (C0) edge node (CA0) {} (A0);
\draw (A2) edge node (AB1) {} (B2);
\draw (B3) edge node (BC1) {} (C3);
\draw (C2) edge node (CA1) {} (A3);

\fill[color=black!5] (A0) -- (B0) -- (C0) -- cycle;

\node[above]       at (A1) {\set{a}};
\node[below left]  at (B1) {\set{b}};
\node[below right] at (C1) {\set{c}};

\node[above left]  at (AB1) {\set{a,b}};
\node[above right] at (CA1) {\set{a,c}};
\node[below]       at (BC1) {\set{b,c}};

\node at (O) {\set{a,b,c}};

\foreach \x in { (A0), (A1), (A2), (A3), (B0), (B1), (B2), (B3), (C0), (C1), (C2), (C3) } {
  \fill \x circle[radius=1pt];
}

\begin{scope}[->, shorten >=4pt, shorten <=4pt] 
\path (A0) edge node[auto]      {\face{0}{2}{\set{b,c}}} (A1);
\path (B0) edge node[auto]      {\face{0}{2}{\set{a,c}}} (B1);
\path (C0) edge node[auto,swap] {\face{0}{2}{\set{a,b}}} (C1);
\path (A2) edge node[auto]      {\face{0}{1}{\set{b}}} (A1);
\path (A3) edge node[auto,swap] {\face{0}{1}{\set{c}}} (A1);
\path (B2) edge node[auto,swap] {\face{0}{1}{\set{a}}} (B1);
\path (B3) edge node[auto]      {\face{0}{1}{\set{c}}} (B1);
\path (C2) edge node[auto]      {\face{0}{1}{\set{a}}} (C1);
\path (C3) edge node[auto,swap] {\face{0}{1}{\set{b}}} (C1);
\end{scope}
\begin{scope}[<-, shorten >=2pt, shorten <=3pt] 
\path (AB1) edge node[auto] {\face{1}{2}{\set{c}}} (AB0);
\path (BC1) edge node[auto] {\face{1}{2}{\set{a}}} (BC0);
\path (CA1) edge node[auto] {\face{1}{2}{\set{b}}} (CA0);
\end{scope}

\end{cTikzPicture}

As we will see below the boundary of \DeltaC{2}{}, denoted $\boundary{\DeltaC{2}{}}$, is

\begin{cTikzPicture}
\coordinate (O) at (0,0);

\coordinate (A0) at (0,2);
\coordinate (A1) at (0,4);

\coordinate (B0) at ($ (O)!2cm!120:(A0) $);
\coordinate (B1) at ($ (O)!4cm!120:(A0) $);

\coordinate (C0) at ($ (O)!2cm!-120:(A0) $);
\coordinate (C1) at ($ (O)!4cm!-120:(A0) $);

\draw (A0) edge node[auto,swap] {\set{a,b}} (B0);
\draw (B0) edge node[auto,swap] {\set{b,c}} (C0);
\draw (C0) edge node[auto,swap] {\set{a,c}} (A0);

\node[above]       at (A1) {\set{a}};
\node[below left]  at (B1) {\set{b}};
\node[below right] at (C1) {\set{c}};

\foreach \x in { (A0), (A1), (B0), (B1), (C0), (C1) } {
  \fill \x circle[radius=1pt];
}

\begin{scope}[->, shorten >=4pt, shorten <=4pt] 
\path (A0) edge[bend left=30]  node[auto]      {\face{0}{1}{\set{b}}} (A1);
\path (A0) edge[bend right=30] node[auto,swap] {\face{0}{1}{\set{c}}} (A1);
\path (B0) edge[bend left=30]  node[auto]      {\face{0}{1}{\set{c}}} (B1);
\path (B0) edge[bend right=30] node[auto,swap] {\face{0}{1}{\set{a}}} (B1);
\path (C0) edge[bend left=30]  node[auto]      {\face{0}{1}{\set{a}}} (C1);
\path (C0) edge[bend right=30] node[auto,swap] {\face{0}{1}{\set{b}}} (C1);
\end{scope}

\end{cTikzPicture}

Similarly, we will see that, the following simplicial set, whose vertices can be labelled by 
the integer pairs in $ \Integers{} \times \Integers{}$, has no boundary
\begin{cTikzPicture}

\foreach \x in {-2, -1, 0, 1} {
  \foreach \y in {-2, -1, 0, 1} {
    \fill (\x, \y) circle[radius=1pt];
    \draw (\x, \y) -- ($ (\x, \y) + (1,0) $);
    \draw (\x, \y) -- ($ (\x, \y) + (0,1) $);
    \draw (\x, \y) -- ($ (\x, \y) + (1,1) $);
  }
  
  \fill (\x, 2) circle[radius=1pt];
  \draw (\x, 2) -- ($ (\x, 2) + (1,0) $);
  \draw[densely dotted] (\x, 2) -- ($ (\x, 2) + (0,0.5) $);
  \draw[densely dotted] (\x, 2) -- ($ (\x, 2) + (0.5,0.5) $);
  \draw[densely dotted] ($ (\x, -2) + (1, -0.5) $) -- ($ (\x, -2) + (1, 0) $);
  \draw[densely dotted] ($ (\x, -2) + (0.5, -0.5) $) -- ($ (\x, -2) + (1, 0) $);
}

\foreach \y in {-2, -1, 0, 1} {
  \fill (2, \y) circle[radius=1pt];
  \draw (2, \y) -- ($ (2, \y) + (0,1) $);
  \draw[densely dotted] (2, \y) -- ($ (2, \y) + (0.5,0) $);
  \draw[densely dotted] (2, \y) -- ($ (2, \y) + (0.5,0.5) $);
  \draw[densely dotted] ($ (-2, \y) - (0.5, 0) $) -- (-2, \y);
  \draw[densely dotted] ($ (-2, \y) - (0.5, 0.5) $) -- (-2, \y); 
}

\fill (2, 2) circle[radius=1pt];
\draw[densely dotted] (2,2) -- (2.5,2.5);
\draw[densely dotted] (2,2) -- (2.5,2);
\draw[densely dotted] (2,2) -- (2,2.5);

\draw[densely dotted] (-2,2) -- (-2.5,2);
\draw[densely dotted] (-2,2) -- (-2.5,1.5);

\draw[densely dotted] (-2,-2) -- (-2,-2.5);
\end{cTikzPicture}

Finally, to remind ourselves that simplical sets need \emph{not} be nice regular 
``manifolds'', consider the following simplicial set formed by joining two copies of 
\DeltaC{2}{} with one copy of each of \DeltaC{3}{}, \DeltaC{1}{} and \DeltaC{0}{} together 
with a copy of \boundary{\DeltaC{2}{}} (the boundary of \DeltaC{2}{}).

\begin{cTikzPicture}
\coordinate (A) at (-0.8, 0);
\coordinate (B) at (0,-1);
\coordinate (C) at (0,1);
\coordinate (D) at (0.8, 0);
\coordinate (E) at (-1,-1);
\coordinate (F) at (2,0);
\coordinate (G) at (-1,1);
\coordinate (H) at (1,1);
\coordinate (I) at (-2,1);
\coordinate (J) at (-1,2);

\foreach \x in { (A), (B), (C), (D), (E), (F), (G), (H), (I), (J) } {
  \fill \x circle[radius=1pt];
}

\draw[fill=black!5] (A) -- (B) -- (C) -- cycle;
\draw[fill=black!5] (B) -- (D) -- (C) -- cycle;
\draw[densely dotted] (A) -- (D);
\draw[fill=black!5] (A) -- (E) -- (B) -- cycle;
\draw (A) -- (G) -- (C);
\draw (D) -- (F);
\draw[fill=black!5] (G) -- (I) -- (J) -- cycle;
\end{cTikzPicture}
where the tetrahedron in the center, because it is a copy of \DeltaC{3}{}, is ``filled''.

Again we will see that the boundary of this simplicial set is

\begin{cTikzPicture}
\coordinate (A) at (-0.8, 0);
\coordinate (B) at (0,-1);
\coordinate (C) at (0,1);
\coordinate (D) at (0.8, 0);
\coordinate (E) at (-1,-1);
\coordinate (F) at (2,0);
\coordinate (G) at (-1,1);
\coordinate (H) at (1,1);
\coordinate (I) at (-2,1);
\coordinate (J) at (-1,2);

\foreach \x in { (A), (B), (C), (D), (E), (F), (G), (I), (J) } {
  \fill \x circle[radius=1pt];
}

\draw[fill=black!5] (A) -- (B) -- (C) -- cycle;
\draw[fill=black!5] (B) -- (D) -- (C) -- cycle;
\draw[densely dotted] (A) -- (D);
\draw (A) -- (E) -- (B) -- cycle;
\draw (G) -- (I) -- (J) -- cycle;
\end{cTikzPicture}
where the tetrahedron in the center, which is now a copy of \boundary{\DeltaC{3}{}}, is 
``hollow''.

As another example of a simplicial set consider
\begin{cTikzPicture}
\coordinate (A) at (-1,0);
\coordinate (B) at (1,0);
\fill (A) circle[radius=1pt];
\fill (B) circle[radius=1pt];
\path (A) edge[bend left=30] (B);
\path (A) edge[bend right=30] (B);
\end{cTikzPicture}
This example shows that we can not consider each simplex as a subset of the $0$-simplicies.

\TODO{We now need to define the sets of degenerate and non-degenerate simplicies. Which 
allows us to define a simplicial set to be an $\alpha$-simplicial set if all simplicies of 
dimension $\alpha < \beta \leq \kappa$ or greater are degenerate. Define the dimension of a 
simplical set to be the largest $\alpha$ for which there exists a non-degenerate 
$\alpha$-simplex.} 
\begin{remark}
Not surprisingly, \DeltaC{}{} is a $1$-simplicial set.  Explicitly, \DeltaC{}{0} is the set 
of $0$-simplicies, \set{\woSet{\alpha} \suchThat \alpha \leq \kappa}.  \DeltaC{}{1} is the 
set of $1$-simplicies, or order preserving maps, between pairs of $0$-simplicies, 
\woSet{\alpha} and \woSet{\beta}. Finally $\DeltaC{}{\gamma} = \emptySet$ for all $1 < 
\gamma \leq \kappa$.  Since there are no $\gamma$-simplicies for $1 < \gamma \leq \kappa$, 
we only need to exhibit the simplicial maps corresponding to $\coFace{0}{1}{\set{0}}, 
\coFace{0}{1}{\set{1}}, \coDegeneracy{1}{0}{\set{1}}$.  For which we have:
\begin{enumerate}
\item $\DeltaC{}{\coFace{0}{1}{\set{0}}}(m:\woSet{\alpha} \mapsTo \woSet{\beta}) = 
\woSet{\beta}$ (``domain'')
\item $\DeltaC{}{\coFace{0}{1}{\set{1}}}(m:\woSet{\alpha} \mapsTo \woSet{\beta}) = 
\woSet{\alpha}$ (``codomain'')
\item $\DeltaC{}{\coDegeneracy{1}{0}{\set{1}}}(\woSet{\alpha}) = \identity{\woSet{\alpha}}$ 
(``identity map'')
\end{enumerate}
\TODO{is this complete?!  It is no longer correct... correct this for our current notion of 
$1$-simplicial set.}
\end{remark}

\TODO{We want to introduce the fundamental group of a simplical set.  We want to introduce 
the covering space of a simplicial set.  We eventually want to introduce the (co)homology of 
a simplicial set.}

\section{Closure operators}

\section{Boundary operators}

\TODO{We need to define the boundary operator $\boundary : \simpC{} \mapsTo \simpC{}$.  
Really this definition boils down to the question ``how are we going to \emph{use} the 
boundary operator?''  A simplex is \emph{not} on the boundary of a simplicial set \emph{if} 
it is surrounded by other simplicies.  If it is internal to a collapsible sub-simplicial 
set. If it is internal to a locally contractible sub-simplicial set.}



\TODO{We need to find an (finite) algebra free way of defining the boundary operators.  We 
can do this by defining subfunctors and faces to be injective and surjective subfunctors.  
We can then discuss the poset of subfunctors and those subfunctors that are ``shared'' are 
``internal'' and hence not part of the boundary.  We may be able to define free words as a 
logic of relationships and all of the different possible models which respect those 
relationships.}
\TODO{We really want to get rid of this section entirely!  We really want to define the 
boundary operator without the use of abelian groups so that we do not need strictly finite 
sums!  This SHOULD be do able!}

\begin{definition}
\TODO{Define simplicial abelian group as the free group over X... al la 
\cite{goerssJardin1999SimplicialHomotopyTh} and then embedd X into this abelian group.  
Probably really want $\Integers_3$ rather than \Integers{} or $\Integers_2$. Define grade as 
}.
The \define{$n$-simplicial boundary operator}, $\partial_n : X_{n+1} \mapsTo X_n$, is 
\begin{equation}
\partial = \partial_n = \sum_{i=0}^n (-1)^i d_i
\end{equation}
\end{definition}

\begin{definition}
An $n$-simplex, $x \in X_n$, is a \define{face} if it is the image, $x = d_i(x')$, of an 
$(n+1)$-simplex, $x' in X_{n+1}$ for some $0 \leq i \leq n+1$.  An $n$-simplex, $x \in X_n$, 
is \define{degenerate} if it is the image, $x = s_i(x')$, of an $(n-1)$-simplex, $x' \in 
X_{n-1}$ for some $0 \leq i \leq n+1$. An $n$-simplex, $x \in X_n$, is a \define{boundary} 
if it is the image, $x = \partial_n(x')$, of an $(n+1)$-simplex, $x' \in X_{n+1}$.
\end{definition}

\begin{definition}
Fix a nondegenerate simplex, $x \in X$.  The \define{Hom-set for $\partial x$ with respect 
to $X$} is $\homomorphisms{X}{\partial x} = \set{ y \in X \suchThat \partial y = \partial 
x}$.
\end{definition}

\begin{remark}
A Hom-set, \homomorphisms{X}{\partial x}, is the set of all ways, within the simplicial set 
$X$, in which the boundary $\partial x$ can be ``filled in''. The reason for calling this 
the Hom-set will become clear when we come to consider the simplicial generalization of 
Category Theory in the next section. Essentially Theorem \TODO{which Theorem} will show that 
a Category is the ``ghost'' of a completed simplicial set.
\end{remark}

\TODO{need to define and provide lemmas for intersection, union of simplicies}
\TODO{need to define proto hom set of simplicies which have the same boundary}

\TODO{show that any $\gamma$-simplicial set can be embedded into a complete free 
(transfinite) simplicial set.  However what is the free set built out of a simplicial set 
which has simplicies of all orders?  Should it just be the free simplicial set on the $X_0$? 
Or should it respect the lack of simplicies at various orders? If we do not respect the lack 
of simplicies than any free simplicial set is just a free simplicial set on the 
$0$-simplicies.... is this what we want? The free/completion of a simplicial set is just the 
adjoint to the forgetful functor which forgets some of the simplicial sets structure (for 
example above some ``level''). This is in fact incorrect by in principle correct.  We want 
to define free simplicial sets using the universal algebraic approach and in topos theory 
this would be done with a monad operator from \simpC{} to itself which is a closure/interior 
operator.  This is NOT what the boundary operator is ;-( }

\begin{definition}
Given a set $X_0$, the free complete simplicitial set generated by $X_0$ is....
\end{definition}

\begin{lemma}
Given a simplicial set, X, its free completion is....
\end{lemma}

\begin{example}
A graph as a simplicial set.... \TODO{there is a minimal and a maximal version given by 
whether or not a given proto-$n$-simplex is an $n$-simplex for $1 < n$.  This suggests that 
we should have a ``Free'' functor associated to the forgetful functor which forgets all 
$n$-simplicies for $1 < n$}.
\end{example}

\section{Simplicial Approximation Structures}

\TODO{Need to discuss the limit (or is it colimit?) structure in \simpC{}.  In particular we 
want to understand how one ``simpler'' simplicial set embeds into another more ``detailed'' 
simplicial set. We are probably leading up to the concept of bisimulation and so will need 
to define open maps similarly to \cite{joyalNielsenWinskel1996bisimulation} and 
\cite{joyalMoerdijk1995algSetTh}.  In particular it will be the ideal structures associated 
to SAS's that will be interesting.}

\TODO{We can define a canonical closure operator (and hence a canonical topology on a given 
simplicial set) by including all non-degenerate simplicies above the set of verticies of a 
given simplicial set (and then filling in all requierd degenerate simplicies -- is this just 
the free completion?).}

\TODO{With SAS's we can define dimensioned interior operators (roughly due to the closure 
operators discussed above) which map \simpC{} into itself.  However these interior operators 
are really naturally defined as an operator from  \simpApproxC{} to itself.  At a specific 
approximation (this is really a Topos concept in the Topos of \symDynC{}), the interior 
simplical set is a retract of the simplicies which ``touch'' the boundary of the image of 
the closure operator. Or alternatively a sub-simplicial set is open if its closure is a 
proper sub-simplicial set (or is it distinct from the boundary operator?)}

\section{Categorical Actions}

\begin{definition}
A functor, $C \mapsTo S$, is \define{functorial} if for each commutative diagram
\begin{cTikzPicture}
\matrix (m) [comDiagM]
{ \DeltaC{k}{} & S \\
   \DeltaC{n}{} & C \\ };
\path[comDiagP, injection]
(m-1-1) edge (m-1-2)
             edge (m-2-1)
(m-2-1) edge (m-2-2)
(m-2-2) edge[map] node[right] {$w$} (m-1-2);
\end{cTikzPicture}
there exists a unique monic functor, $h : \Delta^n \mapsTo S$ which makes the diagram
\begin{cTikzPicture}
\matrix (m) [comDiagM]
{ \DeltaC{k}{} & S \\
   \DeltaC{n}{} & C \\ };
\path[comDiagP, injection]
(m-1-1) edge (m-1-2)
             edge (m-2-1)
(m-2-1) edge[densely dotted] node[inFront] {$h$} (m-1-2)
             edge node[below] {$i$} (m-2-2)
(m-2-2) edge[map] node[right] {$w$} (m-1-2);
\end{cTikzPicture}
commute.  \TODO{This is too strong.  We only want the top triangle to commute and in the 
bottom triangle for the injection of $h$ to be into the image of $w \compose i$}.
\end{definition}

\begin{lemma}
A functor, $C \mapsTo S$, is functorial if it is functorial with respect to $k = 0$ (in the 
above definition).
\end{lemma}
\begin{proof}
\TODO{prove this!}
\end{proof}

\begin{conjecture}
Every functor is functorial.
\end{conjecture}
\begin{proof}
\TODO{prove this!}
\end{proof}

\begin{definition}
The \define{action of a category}, $C$, on a set, $S$, is a functorial functor $a : C 
\mapsTo S$ \cite[Action of a Category on a set, version 2]{nLab}. An \define{action is free} 
if for each $c \in \objects{C}$, and $f \in \homomorphisms{C}{\partial c}$ then $a(f) = 
a(\identity{c})$ implies $f = \identity{c}$.
\end{definition}

\TODO{What we are calling an action should really be called a selection (or something 
similar).  We \emph{can} define an action as the collection of ``pointed'' simplicies of the 
selection.  That is for each simplex of the selection there are the $0$-simplicies of the 
selection simplex which represents the action.  In the $1$-category case we can (and do) 
identify the $0$-simplicies with the selected $1$-simplex since there is only one ``other'' 
$0$-simplex.  This then gives us a more traditional action on the $0$-simplicies}

\TODO{provide a more catgorical definition of free -- not so evil}

\TODO{We can actually provide a definition of action which is closer to the current 
definition of action by considering the ``set of actions'' ``mapping'' from the current 
$0$-simplex to each of the ``other'' $0$-simplicies of the ``selected'' $n$-simplex. This 
then gives us an ``action'' on the set of $0$-simplicies which is very similar to the 
current definition of action.}

\section{Category of Words}

\TODO{We really have to narrow a definition of word.  In reality words need not be free 
(i.e. homotopic to a point) but rather may have non-trivial homotopy groups.  Do we know 
that transfinite simplicial sets always have simple covers?  If so then we really have a 
category of covered words.  We then want to understand how to join words in the cover and 
obtain an meaningful joined word in the ``base''. }

\TODO{Define cartesian fibration see \cite[Section7.2]{brown2006TopologyGroupoids} and 
\cite[Definition 2.1]{streicher2008fibredCat}}

\begin{definition} \cite[??]{brown2006TopologyGroupoids} \\
Fix a pair of (small) categories, $C$ and $W$.  The category $W$ \define{covers}{} $C$ if 
there exists a functor $w : W \mapsTo C$ for which $w$ has the right lifting property with 
respect to the monic inclusion of $\Delta^0$ into $\Delta^n$ for any $n$, given by the 
diagram:
\begin{cTikzPicture}
\matrix (m) [comDiagM]
{ \DeltaC{0}{} & W \\
   \DeltaC{n}{} & C \\ };
\path[comDiagP, injection]
(m-1-1) edge (m-1-2)
             edge (m-2-1)
(m-2-1) edge[map,densely dotted] node[inFront] {$h$} (m-1-2)
             edge (m-2-2)
(m-1-2) edge[map] node[right] {$w$} (m-2-2);
\end{cTikzPicture}
\TODO{do we want $h$ monic?}
\end{definition}

\begin{lemma}
Every cover has an associate action on its domain induced by its image.  Every action on a 
set has an associated cover by the set on the action's domain.
\end{lemma}
\begin{proof}
\TODO{prove this!}
\TODO{does this exhibit categorical action of functor image on the domain of a functor}
\end{proof}

\TODO{Show that every cover is a Kan fibration  -- do this by showing that if you have 
commuting diagrams (as above) with standard simplicies which intersect then the total 
diagram commutes}

\begin{lemma}
Fix a triple of (small) categories, $C$, $W$ and $W'$.  Assume that $W$ covers $C$ via the 
functor $w : W \mapsTo C$.  If there exists a functor $w' : W' \mapsTo C$, then there exists 
a functor $h : W' \mapsTo W$ which makes the following diagram commute:
\begin{cTikzPicture}
\matrix (m) [comDiagM]
{      & W \\
   W' & C \\ };
\path[comDiagP, map]
(m-2-1) edge node[above left] {$h$} (m-1-2)
(m-1-2) edge node[right] {$w$} (m-2-2)
(m-2-1) edge node[below]  {$w'$} (m-2-2);
\end{cTikzPicture}
If $W'$ is connected then \image{w} is connected.
\end{lemma}
\begin{proof}
This follows \cite[Lemma 10.3.1]{brown2006TopologyGroupoids}. \TODO{do we need $W'$ to cover 
its $w'$-image?} \TODO{can we state that the $h$-image of $W'$ is connected in $W$? -- Is 
this the reason we need connectedness in $W'$?} \TODO{Do we really want to state that $h$ is 
monic?}
\end{proof}

\TODO{Do we need to state and prove lemmas similar to \cite[Lemmas 10.3.2 and 
10.3.3]{brown2006TopologyGroupoids}?}

\begin{lemma}
If $W$ is a cover of $C$ via $w : W \mapsTo C$, then there exists an action of $C$ on $W_0$.
\end{lemma}
\begin{proof}
\TODO{we need to show that there exists a functor $a : C \mapsTo \powerSet{W_0}$, in this 
case we follow \cite[Section 10.4]{brown2006TopologyGroupoids}}
\end{proof}

\TODO{Do we need to state and prove lemmas similar to \cite[Lemma 
10.4.2]{brown2006TopologyGroupoids} (which proves the existance of covers and in particular 
free covers)?}

\begin{definition}
Fix a (small) category, $C$.  A \define{bare word} over $C$, is a (small) category, $W$, 
together with a functor, $w : W \mapsTo C$ which is a free cover of the $w$-image of $W$ in 
$C$.  \TODO{define subword (do I need universality?)}. A \define{specific located word} over 
$C$, is a triple, \tuple{p, w, w'}, of bare words over $C$ for which $p$ and $w$ are 
subwords of $w'$ and $p$ is isomorphic to $\Delta^0$.  A \define{located word} is an 
equivilance class of specific located words where \tuple{p, w, w'} is equivilant to 
\tuple{\hat{p}, \hat{w}, \hat{w'}} iff $p = \hat{p}, w = \hat{w}$ and there exists a bare 
word, $w''$ for which $w$ and $\hat{w'}$ are intersecting subwords of $w''$.  \TODO{this is 
too strict -- we probably only care about $p = p'$ and $w = w'$ up to some equivilance or 
isomorphism not strict identity!  -- this is realated to the problem that we do not have a 
nice ``lattice'' structure to the ``points'' underlying this ``space'' so we really do not 
have a nice ``bundle'' structure.}
\end{definition}

\TODO{\cite[Section 10.2]{brown2006TopologyGroupoids} defines a groupoid cover as a 
bijection of the local star at an object.  A cartesian fibration \cite[Definition 
2.1]{streicher2008fibredCat} is defined in such a way that all compositions in the base lift 
to compositions in the fiber.  Are these two definitions equivilant in our case?}

\TODO{comment/prove that covers are cartesian fibrations which are unique}.


\chapter{State machines}

\begin{definition}

A state machine... A finite state machine is a state machine ...

\end{definition}

\subsection{$\aleph_0$-non-deterministic Turing Machine}

\subsection{o-machines}

\section{Trans-finite defintions}

\section{Relationship with o-machines}

\begin{theorem}

Given a set, $\gamma$, there exists a trans-finite Turing machine, $T$, corresponding to
the $\gamma$-o-machine, $O$.

\end{theorem}


\begin{theorem}

Given a trans-finite Turing machine, $T$, there exists a set, $\gamma$, and a
$\gamma$-o-machine, $O$, that simulates $T$.

\end{theorem}

\begin{proof}
something proving the previous theorem.
\end{proof}

\chapter{Trans-finite computability theory}

\chapter{Reflections}

Tukey's lemma, \cite[Maximal Principle II, page 10]{jech1973axiomOfChoice}, seems rather
anomalous since, of the equivalent forms of the Axiom of Choice, it is the only one to
\emph{explicitly} mention the finite/infinite boundary. In fact if we look carefully at
the \emph{proof} of the equivalents, \cite[Theorem 2.1, page 10]{jech1973axiomOfChoice},
trans-finite induction is used at least twice.  A deep question is: ``would an infinite
being have a more powerful, from our point of view, concept of the finite-infinite
boundary and hence a more powerful trans-finite induction?'' In other words would an 
infinite being have a different concept of what ``finite'' means?

For us, as finite beings, what the Tukey's lemma anomaly is ``saying'' is that, finite
beings can only ever obtain a \emph{finite}, and hence limited, description of the
``full'' mathematics of an infinite being. That is the finite/infinite boundary of 
trans-finite induction provides a fundamental boundary for any finite being.

\printbibliography
\end{document}

