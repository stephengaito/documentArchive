% LaTeX source for the constructiveFoundations document
%

\documentclass[a4paper,openany]{amsbook}
\usepackage{disitt}
\usepackage{disitt-symbols}
\usepackage{mdframed}
\newmdenv[linecolor=white,backgroundcolor=gray!10]{infobox}
\newenvironment{myQuote}{\begin{quotation}}{\end{quotation}}
\surroundwithmdframed[linecolor=white,backgroundcolor=gray!10]{myQuote}

\begin{document}
\frontmatter
\sloppy

\title[Constructive Foundations]{Reimplementing Raymond Turner's ``Constructive Foundations 
for Functional Languages'' in diSimplicial Theory}
%% author: stg
\author{Stephen Gaito}
\address{PerceptiSys Ltd, 21 Gregory Ave, Coventry, CV3 6DJ, United Kingdom}%
\email{stephen@perceptisys.co.uk}%
\urladdr{http://www.perceptisys.co.uk}


%% version summary
\thanks{Created: 2016-10-12}
\thanks{Git commit \gitReferences{} (\gitAbbrevHash{}) commited on \gitAuthorDate{} by \gitAuthorName{}}
\thanks{AMS-\LaTeX{}'ed on \today{}.}

%% Copyrights
\thanks{\textbf{Copyright: \copyright{} Stephen Gaito, PerceptiSys Ltd \the\year{}; Some rights reserved}}
\thanks{\textbf{This work is licensed under a Creative Commons Attribution-ShareAlike 4.0 International License.}}

\subjclass[2010]{Primary unknown; Secondary unknown} %
\keywords{Keyword one, keyword two etc.}%

\begin{abstract}
We ``reimplement'' Raymond Turner's \emph{Constructive Foundations for Functional
Languages} in diSimplicial Theory.
\end{abstract} 
\maketitle 
\tableofcontents 
\mainmatter

\section{Introduction}

The objective of this paper is to work through Dr Raymond Turner's book,
\cite{turner1991constructiveFoundations}, \emph{Constructive Foundations for
Functional Languages}. In particular, our goal is to implement the tools and
theory required to state and check the proof of the well-founded inductive
version of the quick sort algorithm (Theorem 13.5 in section 13.3 on page 244)
in the context of diSimplicial theory.

Our objective goal is to take the proof of the quick sort algorithm and compile
an executable suitable for us on a Linux machine which can load a finite
diSimplicial structure which represents a List of orderable ``things'' and
returns a diSimplicial structure which represents the order List of these
``things''.

To reach this goal there are a number of important subtasks. Firstly we need to
develop enough of the theory to provide the predicate calculus required to state
the sepcification of the quick sort algorithm itself. We also need to define
Lists, as well as orderable objects. Finally, we need to provide tools to assist
in the management of the conceptual complexity of the proof.

\section{End goal}

\subsection{Explicit logics based proof theory proof}

We begin by reproducing Dr Turner's Theorem 13.5.

%\newcommand{\and}{\ensuremath{\wedge}}
%\newcommand{\or}{\ensuremath{\vee}}

\begin{theorem}
$ \forall x \in List [N] \exists y \in List [N] {O(y) \and P(x,y)} $
\end{theorem}

\begin{proof}

By well-founded induction with induction formual set to $\delta[x] = \exists y \in List[N]{O(y) \nd P(x,y)}.$ We have to prove that $$ \forall x \in List[N]{\forall z(z<<x \implies \delta[z]) \implies \delta[x]}. $$

\begin{flalign*}
%
[x \in L]                                           &\quad (1)  & \text{assumption} \\
x = [ ] or \exists n \in N \exists u in L (x = n:u) &\quad (2)  & \text{theorem} \\
[x = [ ]]                                           &\quad (3)  & \text{assumption} \\
O([ ]) \and P([ ], [ ])                             &\quad (4)  & \text{P and O} \\
\exists y \in L {O(y) and P(x,y)}                   &\quad (5)  & (4), \exists i \\
\forall z(z<<x \implies \delta[x]) \implies \delta[x] &\quad (6) & \forall i, (5), \implies i \\
[\exists n \in N \exists u \in L (x = n:u)]         &\quad (7)  & \text{assumption} \\
[\forall z(z << x \implies \delta[z])]              &\quad (8)  & \text{assumption} \\
[ x = n:u ]                                         &\quad (9)  & \text{assumption} \\
Lnu << x                                            &\quad (10) & \gamma \\
Lnu << x \implies \delta[Lnu]                       &\quad (11) & \forall e, (8) \\
\delta[Lnu]                                         &\quad (12) & \implies e, (11), (10) \\
Rnu << x                                            &\quad (13) & \gamma \\
Rnu << x \implies \delta[Rnu]                       &\quad (14) & \forall e, (8) \\
\delta[Rnu]                                         &\quad (15) & \implies e, (13), (14) \\
[O[w] and P(Lnu, w)]                                &\quad (16) & \text{assumption} \\
[O[w] and P(Rnu, v)]                                &\quad (17) & \text{assumption} \\
O(w *[n]*v) and P(w*[n]*v,x)                        &\quad (18) & P and O, \theta, \gamma, \phi \\
\delta[x]                                           &\quad (19) & \exists i, (18) \\
\delta[x]                                           &\quad (20) & \exists e, (19), (17), (15) \\
\delta[x]                                           &\quad (21) & \exists e, (20), (16), (12) \\
\delta[x]                                           &\quad (22) & \exists e (\text{twice}), (7), (9) \\
\forall z(z<<x \implies \delta[z]) \implies \delta[x] &\quad (23) & \implies i, (8), (22) \\
\forall z(z<<x \implies \delta[z]) \implies \delta[x] &\quad (24) & or e, (3), (7), (2) \\
x \in L \implies \forall z(z<<x \implies \delta[z]) \implies \delta[x] &\quad (25) & \implies i, (1), (24) \\
\forall x \in L(\forall z(z<<x \implies \delta[z]) \implies \delta[x]) &\quad (26) & \forall i, (25)
%
\end{flalign*}

\end{proof}

The problem with this proof, for the average mathematician, is that, 
while the proof is correct, its essential structure is hidden by far too 
much ``extraneous'' detail.

\subsection{More traditional social proof}

Both mathematicians and computer programmers, are adept at managing this 
``extraneous'' detail. Unfortunately the tools either group use, are 
very very different.

\begin{theorem}
%
Given a List of natural numbers, $x \in List[N]$, the quick sort algorithm:
%
\begin{code}
  
\end{code} 
%
produces a List, $y \in List[N]$, which is ordered, $O(y)$, 
and is a permutation of $x$, that is $P(x,y)$.
%
\end{theorem}

\begin{proof}
The proof proceeds via well-founded induction using the induction ...

\textbf{base case}

\textbf{induction step}

\end{proof}

\bibliographystyle{amsalpha}
\bibliography{constructiveFoundations}

\end{document}

