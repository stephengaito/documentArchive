% A ConTeXt document [master document: jPeg.tex]

\section[title=Overview]

We develop a \joylol\ Parsing Expression Grammar (jPeg) modelled upon the 
virtual machine implementation of the Lua Parsing Expression Grammar 
(LPeg). The Lua version upon which we base our implementation can be found 
in \cite{ierusalimschy2008lpegArticle} and 
\cite{medeirosIerusalimschy2008parsingMachinePEGs}. Particularly important 
for our work are the extensive correctness proofs provided by 
\cite{medeirosIerusalimschy2008parsingMachinePEGs}. 

An important difference is that while LPeg's implementation is based upon 
an interpreter specific to the LPeg virtual machine code, we will use 
\joylol's interpreter directly. The backtracking stack used by LPeg is 
replaced by a careful use of \joylol's process stack. By implementing jPeg 
directly in \joylol, jPeg programs will benefit from any optimizations 
provided by the \joylol\ compiler. 

We begin by looking at the PEG description of Peg's grammar, taken from 
\cite{ford2004parsingExpressiongGrammars}. 

\starttyping
# Hierarchical syntax
Grammar    <- Spacing Definition+ EndOfFile
Definition <- Identifier LEFTARROW Expression
Expression <- Sequence (SLASH Sequence)*
Sequence   <- Prefix*
Prefix     <- (AND / NOT)? Suffix
Suffix     <- Primary (QUESTION / STAR / PLUS)?
Primary    <- Identifier !LEFTARROW
             / OPEN Expression CLOSE
             / Literal / Class / DOT

# Lexical syntax
Identifier <- IdentStart IdentCont* Spacing
IdentStart <- [a-zA-Z_]
IdentCont  <- IdentStart / [0-9]

Literal    <- [’] (![’] Char)* [’] Spacing
             / ["] (!["] Char)* ["] Spacing
Class      <- ’[’ (!’]’ Range)* ’]’ Spacing
Range      <- Char ’-’ Char / Char
Char       <- ’\\’ [nrt’"\[\]\\]
             / ’\\’ [0-2][0-7][0-7]
             / ’\\’ [0-7][0-7]?
             / !’\\’ .

LEFTARROW  <- ’<-’ Spacing
SLASH      <- ’/’ Spacing
AND        <- ’&’ Spacing
NOT        <- ’!’ Spacing
QUESTION   <- ’?’ Spacing
STAR       <- ’*’ Spacing
PLUS       <- ’+’ Spacing
OPEN       <- ’(’ Spacing
CLOSE      <- ’)’ Spacing
DOT        <- ’.’ Spacing

Spacing    <- (Space / Comment)*
Comment    <- ’#’ (!EndOfLine .)* EndOfLine
Space      <- ’ ’ / ’\t’ / EndOfLine
EndOfLine  <- ’\r\n’ / ’\n’ / ’\r’
EndOfFile  <- !.
\stoptyping

We now look at the same PEG description of Peg's grammar, taken from 
\cite{ierusalimschy2008lpegArticle}. 

\starttyping
pattern     <- grammar / simplepatt
grammar     <- (nonterminal ’<-’ sp simplepatt)+
simplepatt  <- alternative (’/’ sp alternative)*
alternative <- ([!&]? sp suffix)+
suffix      <- primary ([*+?] sp)*
primary     <- 
  ’(’ sp pattern ’)’ sp / ’.’ sp / literal /
  charclass / nonterminal !’<-’
literal     <- [’] (![’] .)* [’] sp
charclass   <- ’[’ (!’]’ (. ’-’ . / .))* ’]’ sp
nonterminal <- [a-zA-Z]+ sp
sp          <- [ \t\n]*
\stoptyping

Translated into LPeg:

\starttyping
local lp = require 'lpeg';
local P, S, R, V = lp.P, lp.S, lp.R, lp.V;

local lpeg = P {
  pattern     = V"grammar" + V"simplepatt" ;
  grammar     = ( V"nonterminal" * S’<-’ * V"sp" * V"simplepatt" )^1 ;
  simplepatt  = V"alternative" * ( S’/’ * V"sp" * V"alternative" )^0 ;
  alternative = ( (S'!' + S"&")^-1 * V"sp" * V"suffix" )^1 ;
  suffix      = V"primary" * ([*+?] * V"sp")^0 ;
  primary     = 
    S’(’ * V"sp" * V"pattern" * S’)’ * V"sp" +
    S’.’ * V"sp" +
    V"literal" +
    V"charclass" +
    V"nonterminal" * -S’<-’ ;
  literal     = S"’" * (-S"’" * .)^0 * S"’" * V"sp" ;
  charclass   = S’[’ * (-S']' * (. S’-’ . + .))^0 * S’]’ * V"sp" ;
  nonterminal = R("az", "AZ")^1 * V"sp" ;
  sp          = (S" " + S"\t" + S"\n")^0 ;
};
\stoptyping

We need to implement the following \joylol\ words:

\startitemize[1]

\item \bold{\type{Fail}}

\item \bold{\type{Commit}}

\item \bold{\type{Choose}}

\item \bold{\type{RepeatAtLeast}}

\item \bold{\type{RepeatAtMost}}

\item \bold{\type{Char}}

\item \bold{\type{CharSet}}

\item \bold{\type{Any}}

\stopitemize

\TODO{We have not covered \bold{\type{not}}, \bold{\type{and}}, 
CharSet difference, captures, or actions on captures.} 

As suggested in \cite{ierusalimschy2008lpegArticle} we might implement 
captures using a \type{Capture} call with three distinct arguments:

\startitemize[n]

\item \type{begin n} where the actual capture begins $n$ characters before 
the current character. This capture method should be called \emph{as soon 
as} it is known that the current path will succeed. 

\item \type{end} this marks the end of a capture

\item \type{full n} this is the same as a \type{begin n} immediately 
followed by an \type{end}. 

\stopitemize 

For all of the above \joylol\ words, the data stack must include both the 
current text structure (which must include a indication of the current 
character) as well as the current collection of captures.

A compiler is a pipeline of co-routines. The parser might itself be a 
pipeline of co-routines, one for the lexer, one for the ultimate parser, 
but there could be numerous intermediary co-routines parsing more complex 
syntactic structures. 

