% A ConTeXt document [master document: hilbertsProgram.tex]

\chapter[title=Formal System]

By its very nature, the process of founding Mathematics is circular. All 
previous attempts at founding Mathematics, have quietly assumed 
\emph{Mathematical} Logic, in one form or another, as a given. Less 
obviously, all previous foundations of Mathematics, have also implicitly 
assumed that \emph{something} can \emph{compute} the logical validity of a 
proof, and hence have implicitly assumed the \emph{Mathematical} theory of 
computation. 

Since we can, \quotation{from moment to moment, see that we have made 
marks in the sand}, we will simply assume that we can \emph{compute}. But 
\emph{if} we are going to \emph{found} Mathematics by \emph{Computing} 
things, how do we know that what we have computed is really what we set 
out to compute? How do we compute the validity of what we have computed? 
What do we mean by the validity of what we have computed? 

Previous attempts to found Mathematics answered their equivilant questions 
using Logic. 

Talk about Mathematicians as (Rigorous) (Computer) Language designers... 
mathematicians design languages with which to computer interesting 
\quote{results} in a rigorous way. 


Talk about intuition, classical proofs, computational proofs.

Talk about proof relevant mathematics and proof existent mathematics.

Talk about how classical/intuitionistic constructive/non-constructive 
debate misses the point. All of mathematics is one.. the real question is 
which "collections" of mathematical objects admit proofs of the various 
types. Recursive sets are classical, Recursively enumerable sets are 
intuitionistic. non-constructive proofs require unbounded search using 
axiom of choice. 

Talk about proofs as computation traces. Proof sketches provide way-points 
in any given computation traces. Existence of a proof-computation-trace 
"proves" a theorem, but "having" computational traces, or various 
way-points, provide far more information for the purposes of both 
"understanding" and "generalization". 

The "base space" of computational traces form a category, while the fibers 
of logic or type theory, form a topos. Essentially this is because JoyLoL 
computational traces can be concatenated but the logic "about" the 
computational traces, must identify "things" "inside" each computation. 
Hence we do not use variables in the base space, but we do in the fibres. 

non-total computation is formally based upon the ideas from 
lindstrom1989nonWellFoundedSetsMartinLofTypeTheory NOTE that the 
uniformity of the computation from step to step is critical to providing a 
well defined computation at infinity. 

Object Language

Meta-Language

We need the ability to specify (sub)collections of \quote{computational 
results}. 

\subject[title=A Tale of many languages]

talk about joylol and joylol0

talk about conservative extensions

talk about mathematicians/programmers versus sceptical 
philosophers/business analysts

talk about intuition being operationalized... how do we tell if we have 
operationalized correctly? 

talk about intuitive "ordinals", ""

Need to "prove" something simple


Define DictNode:

\starttyping
typedef struct joylol_object_struct {
  JClass    *type;
  TagType    tag;
  FlagsType  flags; // an arbitrary collection of bits
} JObj;

typedef struct dictNode_object_struct {
  JObj         super;
  Symbol      *symbol;
  JObj        *preObs;
  JObj        *value;
  JObj        *postObs;
  DictNodeObj *left;
  DictNodeObj *right;
  DictNodeObj *previous;
  DictNodeObj *next;
  size_t       height;
  long         balance;
} DictNodeObj;
\stoptyping

\section[title=Mappae Mundi]

Our foundation of mathematics is best understood as an, alas, fairly 
complex multi-induction of a Categorical natural transformation between a 
pair of functors. While we do not require Category theory for this 
induction, using ideas from Category theory helps us navigate through this 
complexity. In effect Category theory provides one (of many) mappa mundi. 

The ideas required to navigate through this multi-induction, comes from 
Bart Jacob's pair of books \cite{jacobs1999categoricalLogicTypeTheory} and 
\cite{jacobs2017coalgebras} combined with Harel, Kozen and Tiuryn's 
\cite{harelKozenTiuryn2000dynamicLogic} as well as Demri, Goranko and 
Lange's \cite{demriGorankoLange2016temporalLogics}. 

Coalgebras, bisimulations, modal logic, invariants and liftings of both 
relations and predicates, permeate everything we do, 
\cite{jacobs2017coalgebras}. In particular Modal logics via the predicate 
lifting approach using the natural transformation between a pair of 
functors, see Definition 6.5.1 \cite{jacobs2017coalgebras}\footnote{See 
also \cite{kupkePattinson2011coalgModalLogics} discussion of the same 
ideas.}, is particularly important. However \emph{the} modal logic that we 
are after is Dynamic Logic, see sections 5.2 and 11.3 
\cite{harelKozenTiuryn2000dynamicLogic}. This \quote{modal logic} actually 
requires the lifting of \emph{both} predicates and relations.

The critical complication comes from our need to \quote{prove} that the 
natural transformation between the pair of functors, which is a 
\emph{computation}, actually computes what we say it computes. To do this 
we will make use of the tools found in 
\cite{demriGorankoLange2016temporalLogics} culminating in the 
\quote{Satisfiability-Checking Game for ${L}_\mu$} found in section 
15.4.1. Central to this is the use of the least and greatest fixed point 
operators, $\mu$ and $\nu$, in a $\mu$-modal logic variant of Dynamic 
Logic. We will use \quote{Lawvere's mating rule} definition of $\forall$ 
and $\exists$ as described in Jacob's Lemma 4.1.8 
\cite{jacobs1999categoricalLogicTypeTheory}. However, since we are 
defining a concatenative computational language, instead of forming fibres 
over \emph{sets} of variables we fibre over lists. The fibres are then all 
the possible \quote{valuations} of a given list. 

In all of this, the critically important thing is that the \emph{syntax} 
of the Dynamic (modal) logic is interpreted in the \emph{semantics} of the 
coalgebra of lists of lists. In the literature\footnote{See for example, 
\cite{jacobs2017coalgebras} section 6.5 and 
\cite{kupkePattinson2011coalgModalLogics}.}, the \quote{predicate lifting} 
approach, using a natural transformation between a pair of functors, while 
syntactically more concrete, is seen as slightly more arbitrary in that 
there are potentially more choices of syntactical functors. While strictly 
not necessary, we will follow the more concrete syntactical 
\quote{predicate lifting} approach to make it easier for mere mortals, 
such as ourselves, to follow the construction. Part of this construction 
will however emphasize how each particular syntactical structure \emph{is 
written} in a more formal list of lists format. 

In all cases, the functors we construct, are computable functions written 
in the \joylolZero\ language which is the chosen syntax. We 
\quote{implement} this \joylolZero\ language in two ways. Firstly the 
natural transformation \emph{is} the implementation (interpreter) of the 
syntax, \joylolZero, in the semantics of list of lists. However, to 
provide a portable interpreter of the \joylolZero\ language, we annotate 
the natural transformation with small chuncks of ANSI-C code, which when 
collectively compiled using an ANSI-C compiler, provide an executable 
which can interpret any \joylolZero\ code on any modern computer. In 
particular, this means that \joylolZero\ interprets itself. 

As well as annotating the natural transformation with ANSI-C code, an 
integral part of the \joylolZero\ code which implements the natural 
transformation, is a running Hoare logic description of the pre and post 
conditions of each computational step. In order to keep the complexity of 
this construction to a minimum, we keep the initial \joylolZero\ 
interpreter (the natural transformation) as simple as possible. However, 
this means that the \joylolZero\ interpreter is not able to make even 
obvious inferences. Once \joylolZero\ has been accepted as correct, 
future, more sophisticated, versions of \joylol\ can be built based upon 
\joylolZero. In particular, these more sophisticated versions of \joylol\ 
can be given greater inferential power with which to make the proofs of 
computational correctness simpler. 

For the rest of this subsection we sketch, using a high-level, classical, 
Categorical point of view, the construction of the \joylolZero\ 
interpreter. Being mappae mundi, each of the classical parts of this 
sketch are merely suggestive of the actual construction we will need. 
However, in each case, these mappae mundi provide useful points of 
reference with which to orient ourselves as we build our actual 
construction. To help keep the parallels to mind we will keep the 
\quote{names} of each \quote{component} part we sketch here, in the final 
construction below. 

We begin by taking six \quote{component} parts from Definition 6.5.1 of 
\cite{jacobs2017coalgebras}: 

\startitemize[n]

\item A functor, $F : \lol\ \rightarrow \lol$, which \quote{defines} the 
Coalgebra of Lists of Lists. This coalgebra of Lists of Lists, \emph{is} 
the semantics of the \joylolZero\ computational language. 

\item The terminal $F$-Coalgebra $(\lol, c:\lol \rightarrow F(\lol)$. We 
assert that this terminal coalgebra actually exists. 

\item A functor, $L : \lol\ \rightarrow \lol$, which \quote{defines} the 
Algebra of the $\mu$-modal Dynamic Logic which we call \joylolZero. This 
algebra \emph{is} the syntax of the \joylolZero\ computational language. 

\item The initial $L$-Algebra $(\text{L}, \gamma:L(\text{L}) \rightarrow 
\text{L})$. 

\item A functor, $P = 2^{(-)} : \lol^{\text{op}} \rightarrow \lol$, which 
is the contravariant \quote{powerset} functor. 

\item A natural transformation, $\delta : LP \Rightarrow PF$, which is the 
\emph{interpreter} of the \joylolZero\ computational language. 

\stopitemize

\ \par

Following Chapter 2 of \cite{jacobs1999categoricalLogicTypeTheory}, we can 
view \joylolZero\ as essentially a \quote{simple type theory} with the 
single basic type, \lol. Contexts are important. For us a context is 
actually a (data) stack description, see 
\cite{diggins2008simpleTypeInference}. This means that the \quote{rules} 
of context (stack) manipulation will be slightly different for our case. 

\TODO{describe the classifying category Definition 2.1.1 
\cite{jacobs1999categoricalLogicTypeTheory}... is essentially \joylolZero\ 
\quote{programs}. The arrows are the process stack while the objects are 
the data stack.} 

\TODO{explore the $\lambda 1$ and $\lambda 1_\times$-calculi described in 
section 2.3 \cite{jacobs1999categoricalLogicTypeTheory}. Comment on the 
fact that propositions as types are \joylolZero\ programs. Where do the 
type formers fit in to our work?} 

\TODO{In the terminology of \cite{jacobs1999categoricalLogicTypeTheory}, 
the $\mu$-modal Dynamic Logic is really the \emph{simultaneous} addition 
of a Simple Type Theory (the program part) and a First-order predicate 
logic (the predicate part) at the same time. See Chapters 2 and 4 
\cite{jacobs1999categoricalLogicTypeTheory}. In particular see sections 
4.2.4 through 4.2.7 of \cite{jacobs1999categoricalLogicTypeTheory}. We 
will retain Jabob's structured contexts by having a stack description as 
well as a separate collection of predicate in every context.} 

We will also be interested in the functorial semantics described in 
section 2.2 of \cite{jacobs1999categoricalLogicTypeTheory}. Since we are 
building a fixed point of the functorial semantics operator ...

\TODO{describe the structure of our Dynamic Logic sections 5.1, 5.2, 11.1, 
11.2, and 11.3 in \cite{harelKozenTiuryn2000dynamicLogic}} 

What is a set? Computationally there are many ways to \emph{implement} a 
\quote{set}. The easiest is to simply list all of the elements, 
\emph{once} for each element. For \emph{data} it is easy to ensure a given 
element is contained in the list exactly once. While we will show that it 
is possible to \emph{sort} a \emph{process}, and hence remove duplicates, 
it is equally easy to realize that one can only remove all duplicates in 
the infinitely sorted limit. Any particular \emph{finite} sorted 
approximation to the infinite sort, may always have multiple copies of any 
given element. This suggests that \quote{mulit-sets} or \quote{bags} are 
more natural \emph{computataional} equivalent to the \emph{axiomatic} 
concept of \quote{set}. \TODO{Find references for multi-sets.}

What then is the \emph{computational} equivalent of the \quote{power-set}. 
It should be the \quote{power-multi-set}. \TODO{While the collection of 
all sub-multi-sets of a given multi-set can include arbitrary 
multiplicities, we \emph{should} define a \emph{sub}-multi-set to only 
have multiplicities \emph{less than or equal to} the multiplicities of the 
parent multi-set. We can then define a characteristic function to 
choose/reject a given multi-set element. Then the power-multi-set is the 
collection of all of these characteristic functions.} 

\TODO{discuss functions based upon interpreted code OR lists.} 


Finally, at the moment, we will \emph{only} provide the base case of 
$\omega$-computational foundations. The additional computational 
\quote{axiom} required to get us from the base case to all of mathematics 
is simple and from a classical compuational point of view, fairly evident, 
but will require some philosophical reflection. We leave this 
computational \quote{axiom} until much later in this document where we can 
give it the careful consideration it requires. 

\TODO{Provide a selection of useful references to Category/Topos theory} 

\section[title=Semantic Functor]

\startMMundi We begin by building the Semantic Functor, $F$, together with 
its associated $F$-Coalgebra map, $c : \lol \rightarrow F(\lol)$. Using 
the notation from Section 2.2.2 in \cite{jacobs2017coalgebras} on 
\quote{Binary $A$-Trees}, our $F$ functor will be essentially $F : \lol 
\rightarrow \lol \times \lol$.\stopMMundi 

\section[title=Syntaxtic Functor]

We now proceed by building the Syntaxtic Functor, $L$, together with its 
associated $L$-Algebra map, $\gamma : L(\text{L}) \rightarrow \text{L}$. 

\TODO{While at the same time we will build the natural transformation, 
$\delta : LP \Rightarrow PF$.} 

\TODO{We need to build the injection from $\text{L} \rightarrow \lol$.} 

