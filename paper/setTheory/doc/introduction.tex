% A ConTeXt document [master document: setTheory.tex]

\chapter[title=Introduction]



We limber up by computing, as a process, the \emph{universe} of all 0-1 
sequences. 

To do this we essentially need to walk a growing tree of 0-1 sequences. At 
the tip of each branch we then add two new tips, one for 0 and one for 1. 
However, to ensure that we build the whole tree of 0-1 sequences, we must 
pursue a \emph{breadth first} walk of the existing tree at any 
\quote{point of time}. 

In a register machine, we might keep a pointer based \quote{in memory} 
structure to capture the tree of 0-1 sequences. In such a situation, we 
could \quote{simply} \quote{walk} the pointers in this in memory tree. 
However, \emph{any} particular register machine has a correspondingly 
fixed memory size. Any such fixed memory size is small relative to the 
\emph{universe} of 0-1 sequences. This means that we can \emph{not} use a 
register machine to compute the universe of 0-1 sequences. 

With any non-register machine based program written in \joylol, the 
fundamental problem is that we can only work at the top of the data and 
process stacks. This means that we need to keep track of a \emph{stack} of 
past left-right tree walking choices, as well as a \emph{stack} of tree 
parts obtained while \quote{walking} \quote{up} along a given branch. 

Since the stack of left-right tree walking choices represent what we need 
to do next, we shall keep them, in reverse order of \quote{execution}, on 
the process stack. Similarly, the stack of tree parts represent the parts 
of the tree that we have dismantled as we \quote{walk} \quote{up} to the 
tip of a given branch. As we back-track along the stack of choices, we 
must put these tree parts back onto the \quote{underlying} \quote{tree}. 
This suggests that we keep the stack of tree parts on the data stack. 

Since we are programming a \emph{process}, we know that it will never 
complete its computation. This means that while our specification of this 
process may have a useful precondition, since the process never completes, 
our specification has no (particularly) meaningful post-condition. How do 
we ensure that our program computes what we say it computes? We do this by 
ensuring an invariant is satisfied at regular points in overall 
computation. 

Our precondition is simply that the \quote{tree} is empty. Our regular 
invariant, is that, at regular points in the computation, we have a 
\emph{complete} tree of 0-1 sequences of size $n$. In order to show that 
our process computes the \quote{whole} universe of 0-1 sequences, we 
\quote{simply} have to show that the over all computation will pass the 
computational point at which the tree of depth $n$ has been constructed 
for every $n$ we might choose. 

Our invariant should be true when ever the stack of choices is empty, as 
this is the point at which a particular breadth sweep has finished and the 
next breadth sweep is about to begin. In particular we must show two 
things, (1) that this end of breadth sweep occurs infinitely often in the 
computation, and (2) each successive end of breadth sweep has built a 
successively larger portion of the tree of 0-1 sequences. 

We could place the tree parts onto the process stack near the next choice 
operator. 

We need a \quote{build at tip} word which places a tree of depth one onto 
the top of the data stack. 

We need a \quote{take left branch} word. If there is no left branch, then 
we \quote{build at the tip} leaving the result on the data stack for the 
corresponding \quote{rebuild tree} to build it into a new sub-tree.

We need a \quote{dismantle tree} word which places the left and right 
branches on to the top two elements of the data stack, and places a 
\quote{rebuild tree}, \quote{take right branch} and, \quote{take left 
branch} onto the top of the process stack (in that order). 

We need a \quote{rebuild tree} word which builds the sub-trees on the top 
two elements of the data stack into one tree on the top of the data stack. 

We need a \quote{take right branch} which swaps the top two elements of 
the data stack. If there is no right branch, then we build at the tip 
leaving it on the top of the data stack.

